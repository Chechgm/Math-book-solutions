\section*{Chapter 7.D. Polar decomposition and singular value decomposition}
\addcontentsline{toc}{section}{Chapter 7.D. Polar decomposition and singular value decomposition}

16 When do two maps have the same singular values?

\begin{exercise}{1}
  Fix $u,x\in V$ with $u\neq 0$. Define $T\in\LLL(V)$ by $Tv =\brackets{v,u}x$ for every $v\in V$. Prove that
  \[
  \sqrt{T^\ast T}v=\frac{\norm{x}}{\norm{u}}\brackets{v,u}u
  \]
  for every $v\in V$.
\end{exercise}
\begin{proof}
 We first have to prove that $S=\sqrt{T^\ast T}$ is a positive operator.
 
 Self-adjoint: We have 
 \begin{align*}
     \brackets{v, S^\ast w} =& \brackets{Sv, w}\\
     =& \brackets{\frac{\norm{x}}{\norm{u}}\brackets{v,u}u, w}\\
     =& \frac{\norm{x}}{\norm{u}}\brackets{v,u}\brackets{u, w}\\
     =& \frac{\norm{x}}{\norm{u}}\brackets{v,u}\overline{\brackets{w, u}}\\
     =& \brackets{v,\frac{\norm{x}}{\norm{u}}\brackets{w, u}u},
 \end{align*}
 where the last inequality follows because $\norm{x}\in\R$ for all $x$. Then, $S$ is self-adjoint.

 Positivity: We have
 \begin{align*}
     \brackets{Tv,v} =& \brackets{\frac{\norm{x}}{\norm{u}}\brackets{v,u}u,v}\\
     =& \frac{\norm{x}}{\norm{u}}\brackets{\brackets{v,u}u,v}\\
     =& \frac{\norm{x}}{\norm{u}}\brackets{u,v}\brackets{v,u}\\
     =& \frac{\norm{x}}{\norm{u}}\overline{\brackets{v,u}}\brackets{v,u}>0,
 \end{align*}
 because norms are positive and $\overline{\brackets{v,u}}\brackets{v,u}$, unless $v=0$. Then, S is positive.
 
 To finish the proof we need to show that $\sqrt{T^\ast T}^2$ as defined above equals $T^\ast T$. We have:
 \begin{align*}
     \sqrt{T^\ast T}\sqrt{T^\ast T}v =& \sqrt{T^\ast T}\left[\frac{\norm{x}}{\norm{u}}\brackets{v,u}u\right]\\
     =& \frac{\norm{x}}{\norm{u}}\brackets{v,u}\sqrt{T^\ast T}u\\
     =& \frac{\norm{x}}{\norm{u}}\brackets{v,u}\frac{\norm{x}}{\norm{u}}\brackets{u,u}u\\
     =& \frac{\norm{x}^2}{\norm{u}^2}\brackets{v,u}\brackets{u,u}u\\
     =& \norm{x}^2\brackets{v,u}u\\
     =& \brackets{x,x}\brackets{v,u}u\\
     =& \brackets{\brackets{v,u}x,x}u\\
     =& \brackets{Tv,x}u.
 \end{align*}
 On the other hand, and using Example 7.4, where we proved $T^\ast v=\brackets{v,x}u$,
 \begin{align*}
     T^\ast Tv =& T^\ast(\brackets{v,u}x)\\
     =& \brackets{\brackets{v,u}x,x}u\\
     =& \brackets{Tv,x}u,
 \end{align*}
 as required.
\end{proof}

\begin{exercise}{3}
  Suppose $T\in\LLL(V)$. Prove that there exists an isometry $S\in\LLL(V)$ such that $T =\sqrt{TT^\ast}S$.
\end{exercise}
\begin{proof}
 Consider the polar decomposition of $T^\ast=S\sqrt{TT^\ast}$, and take adjoints on both sides of the equality: $(T^\ast)^\ast = T =(S\sqrt{TT^\ast})^\ast =(\sqrt{TT^\ast})^\ast S^\ast =\sqrt{TT^\ast}S^\ast$, where the last inequality holds because $\sqrt{TT^\ast}$ is a positive operator. From 7.42, we know that $S^\ast$ is an isometry, giving us what we wanted to prove.
\end{proof}

\begin{exercise}{8}
  Suppose $T\in\LLL(V),S\in\LLL(V)$ is an isometry, and $R\in\LLL(V)$ is a positive operator such that $T=SR$. Prove that $R =\sqrt{T^\ast T}$. [The exercise above shows that if we write $T$ as the product of an isometry and a positive operator (as in the Polar Decomposition 7.45), then the positive operator equals $\sqrt{T^\ast T}$].
\end{exercise}
\begin{proof}
 It is already given to us that $R$ is positive. We have $T^\ast T =(SR)^\ast (SR) =R^\ast S^\ast S R =R^\ast R =R^2$. The second to last equality follows from 7.42 where we concluded that $S^{-1} =S^\ast$. The last equality follows from the fact that $R$ is positive, so it is self-adjoint, meaning $R^\ast = R$. Thus we have that $R^2 =T^\ast T$, so that $R =\sqrt{T^\ast T}$.
\end{proof}

\begin{exercise}{9}
  Suppose $T\in\LLL(V)$. Prove that $T$ is invertible if and only if there exists a unique isometry $S\in\LLL(V)$ such that $T =S\sqrt{T^\ast T}$.
\end{exercise}
\begin{proof}
 ($\Rightarrow$) Suppose, for the sake of contradiction, that $T$ is invertible but $S$ is not unique. Hence, there exists an isometry $S'\in\LLL(V)$ so that $T=S'\sqrt{T^\ast T}$. We then have $S\sqrt{T^\ast T} =S'\sqrt{T^\ast T}$. Since $T$ is invertible, by Exercise 3.D.9 $\sqrt{T^\ast T}$ is also invertible. Hence, $TT^{-1} =S\sqrt{T^\ast T}(S'\sqrt{T^\ast T})^{-1} =S\sqrt{T^\ast T}\sqrt{T^\ast T}^{-1}(S')^{-1} =S(S')^{-1} \neq I$. The last inequality follows from 7.42 where we proved that $S^{-1}=S^\ast$.

 ($\Leftarrow$) Suppose $S$ is unique and, for the sake of contradiction, $T$ is not invertible. Because $S$ is an isometry, $S$ is invertible by 7.42. From 3.D.9, the product $SR$ is invertible if and only if both $S$ and $R$ are invertible, hence $R =\sqrt{T^\ast T}$ is not invertible, so that $\sqrt{T^\ast T}$ is not surjective. 

 Let $e_1\dots,e_n$ be an orthonormal basis of $V$. And because $\sqrt{T^\ast T}$ is not surjective, we assume that the first $k$ vectors of such basis span $\range\sqrt{T^\ast T}$. Now let $S'e_{k+1}=-e_{k+1}$ and $S'e_{k+1}=-e_{k+1}$ and $S'=S$ otherwise. We then have that $T=S\sqrt{T^\ast T}=S'\sqrt{T^\ast T}$ (simply evaluate at vectors spanned by $e_{k+1}$. 
 
 To see $S'$ is an isometry, consider $\brackets{S'e_{k+1},S'e_i} =\brackets{-Se_{k+1},Se_i} =-\brackets{Se_{k+1},Se_i} =0$ if $i\neq k+1$. Otherwise, $\brackets{S'e_{k+1}, S'e_{k+1}} =\brackets{-Se_{k+1}, -Se_{k+1}} =\brackets{Se_{k+1}, Se_{k+1}} =1$. So that by 7.42, $S'$ is an isometry, giving us the desired result.
\end{proof}

\begin{exercise}{10}
  Suppose $T\in\LLL(V)$ is self-adjoint. Prove that the singular values of $T$ equal the absolute values of the eigenvalues of $T$, repeated appropriately.
\end{exercise}
\begin{proof}
 We have $T^\ast =T$, let $v$ be an eigenvector of $T$ with eigenvalue $\lambda$. We have $T^\ast Tv =TTv =T\lambda v =\lambda Tv =\lambda^2v$. From 7.52, we know that the singular values of $T$ are the nonnegative square roots of the eigenvalues of $T^\ast T$. That is, the singular values of $T$ are $\sqrt{\lambda^2} =\absoluteValue{\lambda}$, as desired.
\end{proof}

\begin{exercise}{11}
  Suppose $T\in\LLL(V)$. Prove that $T$ and $T^\ast$ have the same singular values.
\end{exercise}
\begin{proof}
 We defined (in conjunction with 7.52) the singular values of $T$ as the nonnegative square roots of the eigenvalues of $T^\ast T$, so that the singular values of $T^\ast$ are the square roots of the eigenvalues of $(T^\ast)^\ast T^\ast =TT^\ast$. Then we just need to prove that $T^\ast T$ and $TT^\ast$ have the same eigenvalues. But we know this is true from exercise 5.A.23, as desired.
\end{proof}

\begin{exercise}{13}
  Suppose $T\in\LLL(V)$. Prove that $T$ is invertible if and only if 0 is not a singular value of $T$.
\end{exercise}
\begin{proof}
 ($\Rightarrow$) We will prove this by contrapositive. Suppose 0 is a singular value of $T$. Then by the definition of singular values, 7.52 and Exercise 5.A.23, we have that 0 is an eigenvalue of $TT^\ast$. That is, for some $v\neq 0$, we have $TT^\ast v =Tw =0$. Here we have two options. First, if $w=0$, then $T^\ast v=0$ and $T$ is not invertible because $T^\ast$ is not invertible (see Exercise 7.A.5). Second, $w\neq 0$ and $\nullspace T\neq\set{0}$ so that $T$ is not invertible.

 ($\Leftarrow$) We will prove this by contrapositive. Suppose $T$ is not invertible. Then $\nullspace T\neq\set{0}$. That is, there exists $v\in V$ so that $Tv=0$. We have $T^\ast Tv =0$ so that $0$ is an eigenvalue of $T^\ast T$ and by 7.52, 0 is a singular value of $T$.
\end{proof}

\begin{exercise}{14}
  Suppose $T\in\LLL(V)$. Prove that $\dim\range T$ equals the number of nonzero singular values of $T$.
\end{exercise}
\begin{proof}
 We know that $\sqrt{T^\ast T}$ is a positive operator, so that $\sqrt{T^\ast T}$ is self-adjoint. By the Complex Spectral Theorem, we can find a basis $e_1,\dots,e_n$ of $V$ consisting of eigenvectors of $\sqrt{T^\ast T}$. We have that $\dim\range \sqrt{T^\ast T}$ equals the number of non-zero eigenvalues of $\sqrt{T^\ast T}$; this because $\sqrt{T^\ast T}e_i =\lambda_ie_i$. Furthermore, the eigenvectors corresponding to non-zero eigenvalues are orthonormal. From 7.42, we know that $Se_1,\dots,Se_n$ is a list of orthonormal vectors for any isometry $S$. Recall that the polar decomposition of a $T$ is given by $T =S\sqrt{T^\ast T}$ for an isometry $S$, hence, the arguments above imply that $\dim\range T$ is the number of nonzero singular values of $T$, as required.
\end{proof}

\begin{exercise}{15}
  Suppose $S\in\LLL(V)$. Prove that $S$ is an isometry if and only if all  the singular values of $S$ equal 1.
\end{exercise}
\begin{proof}
 ($\Rightarrow$) Suppose $S$ is an isometry. From 7.42 we have that $S^\ast S =I$. Furthermore, from 7.52 we know that the singular values of $S$ are the nonnegative square roots of the eigenvalues of $S^\ast S =I$, which we know all are 1, giving us the desired result.

 ($\Leftarrow$) Suppose all singular values of $S$ equal 1. Then by 7.52 the eigenvalues of $S^\ast S$ are 1 (the squares of the singular values of $S$). Since $S^\ast S$ is normal, we can use the Spectral Theorem to find an orthonormal basis of $V$ consisting of eigenvectors of $S^\ast S$. For arbitrary $v,w\in V$, we have $\brackets{Sv,Sw} =\brackets{S^\ast Sv,w} =\brackets{v,w}$ where the last equality follows from writing $v$ as a linear combination of the basis of eigenvectors of $S^\ast S$, and using the fact that all eigenvalues of $S^\ast S$ are 1. Thus, $S$ is an isometry,
\end{proof}

\begin{exercise}{16}
  Suppose $T_1,T_2\in\LLL(V)$. Prove that $T_1$ and $T_2$ have the same singular values if and only if there exists isometries $S_1,S_2\in\LLL(V)$ such that $T_1 =S_1T_2S_2$.
\end{exercise}
\begin{proof}
We will use the following lemmas: 

(i) If $S,T\in\LLL(V)$ and $S$ is an isometry, and $T$ has eigenvalues $\lambda_i$ with corresponding eigenvectors $v_i$, then the map $STS^\ast$ has eigenvalues $\lambda_i$ with eigenvectors $Sv_i$. To see this, notice that $ST^\ast S^\ast Sv_i =ST^\ast v_i =S\lambda_iv_i =\lambda_iSv_i$, giving us the desired result.

(ii) Let $R,T\in\LLL(V)$ be self-adjoint. $R$ and $T$ have the same diaongalised matrix if and only if there is an isometry $S$ such that $R = STS^\ast$.

For the forward direction, suppose $R$ and $T$ are self-adjoint and have the same diagonalised matrix consisting of eigenvalues $\lambda_1,\dots,\lambda_n$. By the Spectral Theorem, there are orthonormal bases of $V$ consisting of eigenvectors of $R:e_1,\dots,e_n$ and $T:f_1,\dots,f_n$. Consider the linear map $S$ defined as $Sf_i=e_i$ for all $i$. By 7.42, $S$ is an isometry. Furthermore, let $v\in V$. We have $STS^\ast v =STS^\ast (a_1e_1+\dots+a_ne_n) =ST (a_1f_1+\dots+a_nf_n) =S(a_1\lambda_1f_1+\dots+a_n\lambda_nf_n) =a_1\lambda_1e_1+\dots+a_n\lambda_ne_n =a_1Re_1+\dots+a_nRe_n = R(a_1e_1+\dots+a_ne_n) =Rv$, as required.

For the converse, suppose there exists an isometry $S$ so that $R=STS^\ast$ From the Spectral Theorem, there is an orthonormal basis of $V$ consisting of eigenvectors of $T$. By Lemma (i), we have that the eigenvalues of $R$ are the same of those of $STS^\ast$, so that their diagonal matrices are the same.

 ($\Rightarrow$) Suppose $T_1$ and $T_2$ have the same singular values. Since singular values of $T$ are the eigenvalues of $\sqrt{T^\ast T}$, then $\sqrt{T_1^\ast T_1}$ and $\sqrt{T_2^\ast T_2}$ have the same diagonalised matrix. By Lemma (ii), there exist an isometry $S$ so that $\sqrt{T_1^\ast T_1} = S\sqrt{T_2^\ast T_2}S^\ast$. Finally, by the polar decomposition of $T_1$ and $T_2$, there exist isometries $S_1$ and $S_2$ so that $S_1T_1 =\sqrt{T_1^\ast T_1}$ and $S_2T_2 =\sqrt{T_2^\ast T_2}$. Replacing this in our previous equality, gives us: $S_1T_1 = SS_2T_2S^\ast$, which implies $T_1 = S_1^\ast SS_2T_2S^\ast$. Since the product of isometries is an isometry (by definition, $\norm{S_1S_2v} =\norm{S_2v} =\norm{v}$), we get the desired result.

 ($\Leftarrow$) For the proof of the converse, we will first prove the following lemma: 
 
 If $S, T\in\LLL(V)$ and $S$ is an isometry, then $\sqrt{S^\ast T^\ast TS}=S^\ast\sqrt{T^\ast T}S$. First, notice that squaring both sides of the equality gives us $S^\ast T^\ast TS =S^\ast \sqrt{T^\ast T}SS^\ast\sqrt{T^\ast T}S =S^\ast T^\ast TS$, because $S^{-1}=S^\ast$ and $\sqrt{T^\ast T}^2 =T^\ast T$.

 Now we have to prove that $S^\ast\sqrt{T^\ast T}S$ is positive.

 Self-adjoint: For $v,w\in V$, we have $\brackets{S^\ast\sqrt{T^\ast T}Sv, w} =\brackets{v, (S^\ast\sqrt{T^\ast T}S)^\ast w} =\brackets{v, S^\ast\sqrt{T^\ast T}^\ast Sw} =\brackets{v, S^\ast\sqrt{T^\ast T}Sw}$, where the last equality follows because $\sqrt{T^\ast T}$ is positive (and hence, self-adjoint).

 Positivity: We have $\brackets{S^\ast\sqrt{T^\ast T}Sv, v} =\brackets{\sqrt{T^\ast T}Sv, Sv} =\brackets{\sqrt{T^\ast T}w, w} \geq 0$, where the last inequality follows from $\sqrt{T^\ast T}$ being positive.

 Now the proof of the main result. We have that the singular values of $T_1$ are the eigenvalues of $\sqrt{T_1^\ast T_1} =\sqrt{(S_1T_2S_2)^\ast (S_1T_2S_2)} =\sqrt{S_2^\ast T_2^\ast S_1^\ast (S_1T_2S_2)} =\sqrt{S_2^\ast T_2^\ast T_2S_2} =S_2^\ast\sqrt{T_2^\ast T_2}S_2$. Where the last equality follows from the Lemma above. Using Lemma (ii), we conclude that $\sqrt{T_1^\ast T_1}$ has the same eigenvalues as $\sqrt{T_2^\ast T_2}$ so that $T_1$ and $T_2$ have the same singular values, as desired.
\end{proof}

\begin{exercise}{17}
  Suppose $T\in\LLL(V)$ has singular value decomposition given by 
  \[
  Tv =s_1\brackets{v,e_1}f_1+\dots+s_n\brackets{v,e_n}f_n
  \]
  for every $v\in V$, where $s_1,\dots,s_n$ are the singular values of $T$. and $e_1,\dots,e_n$ and $f_1,\dots,f_n$ are orthonormal bases of $V$.
  \begin{enumerate}
      \item Prove that if $v\in V$, then
      \[
      T^\ast v =s_1\brackets{v,f_1}e_1+\dots+s_n\brackets{v,f_n}e_n.
      \]
      \item Prove that $v\in V$, then
      \[
      T^\ast Tv =s_1^2\brackets{v,e_1}e_1+\dots+s_n^2\brackets{v,e_n}e_n.
      \]
      \item Prove that if $v\in V$, then
      \[
      \sqrt{T^\ast T}v =s_1\brackets{v,e_1}e_1+\dots+s_n\brackets{v,e_n}e_n.
      \]
      \item Suppose $T$ is invertible. Prove that if $v\in V$, then 
      \[
      T^{-1}v =\frac{\brackets{v,f_1}e_1}{s_1}+\dots+\frac{\brackets{v,f_n}e_n}{s_n}
      \]
      for every $v\in V$.
  \end{enumerate}
\end{exercise}
\begin{proof}
 \begin{enumerate}
     \item Let $v,w\in V$. We have 
     \begin{align*}
         \brackets{v,&T^\ast w} 
         = \brackets{Tv, w}\\
         =& \brackets{s_1\brackets{v,e_1}f_1+\dots+s_n\brackets{v,e_n}f_n, w}\\
         =& \brackets{s_1\brackets{v,e_1}f_1+\dots+s_n\brackets{v,e_n}f_n, a_1e_1+\dots+a_ne_n}\\
         =& \brackets{s_1\brackets{v,e_1}f_1, a_1e_1}
         +\dots+\brackets{s_i\brackets{v,e_i}f_i, a_je_j}
         +\dots+\brackets{s_n\brackets{v,e_n}f_n, a_ne_n}\\
         =& s_1\brackets{v,e_1}\brackets{f_1, a_1e_1}
         +\dots+s_i\brackets{v,e_i}\brackets{f_i, a_je_j}
         +\dots+s_n\brackets{v,e_n}\brackets{f_n, a_ne_n}\\
         =& \brackets{v,s_1\overline{\brackets{f_1, a_1e_1}}e_1}
         +\dots+\brackets{v,s_i\overline{\brackets{f_i, a_je_j}}e_i}
         +\dots+\brackets{v,s_n\overline{\brackets{f_n, a_ne_n}}e_n}\\
         =& \brackets{v,s_1\brackets{a_1e_1, f_1}e_1}
         +\dots+\brackets{v,s_i\brackets{a_je_j, f_i}e_i}
         +\dots+\brackets{v,s_n\brackets{a_ne_n, f_n}e_n}\\
         =& \sum_j\brackets{v,s_1\brackets{a_je_j, f_1}e_1}
         +\dots+\sum_j\brackets{v,s_i\brackets{a_je_j, f_i}e_i}
         +\dots+\sum_j\brackets{v,s_n\brackets{a_ne_n, f_n}e_n}\\
         \begin{split}
         =& \brackets{v,s_1\brackets{a_1e_1+\dots+a_ne_n, f_1}e_1}\\
         &+\dots+\brackets{v,s_i\brackets{a_1e_1+\dots+a_ne_n, f_i}e_i}\\
         &+\dots+\brackets{v,s_n\brackets{a_1e_1+\dots+a_ne_n, f_n}e_n}
         \end{split}\\
         =& \brackets{v,s_1\brackets{w, f_1}e_1}
         +\dots+\brackets{v,s_i\brackets{w, f_i}e_i}
         +\dots+\brackets{v,s_n\brackets{w, f_n}e_n}.
     \end{align*}
     Where we used that $\bar{s_i}=s_i$ because eigenvalues of positive operators are nonnegative. Thus, $T^\ast v =s_1\brackets{v,f_1}e_1+\dots+s_n\brackets{v,f_n}e_n$, as required.
     \item Let $v\in V$. We have
     \begin{align*}
         T^\ast Tv 
         =& s_1\brackets{Tv,f_1}e_1+\dots+s_n\brackets{Tv,f_n}e_n\\
        \begin{split}
             =& s_1\brackets{s_1\brackets{v,e_1}f_1+\dots+s_n\brackets{v,e_n}f_n,f_1}e_1\\
             &+\dots+s_n\brackets{s_1\brackets{v,e_1}f_1+\dots+s_n\brackets{v,e_n}f_n,f_n}e_n
        \end{split}\\
        =& s_1\brackets{s_1\brackets{v,e_1}f_1,f_1}e_1+\dots+s_n\brackets{s_n\brackets{v,e_n}f_n,f_n}e_n\\
        =& s_1^2\brackets{v,e_1}\brackets{f_1,f_1}e_1+\dots+s_n^2\brackets{v,e_n}\brackets{f_n,f_n}e_n\\
        =& s_1^2\brackets{v,e_1}e_1+\dots+s_n^2\brackets{v,e_n}e_n.
     \end{align*}
     Where we used that $\brackets{s_i\brackets{v,e_j}f_j,f_i}e_i =s_i\brackets{v,e_j}\brackets{f_j,f_i}e_i =0$ if $i\neq j$, because $f_1,\dots,f_n$ is an orthonormal basis.
     \item Let $v\in V$. We have
     \begin{align*}
         \sqrt{T^\ast T}^2v =& s_1\brackets{\sqrt{T^\ast T}v,e_1}e_1+\dots+s_n\brackets{\sqrt{T^\ast T}v,e_n}e_n\\
         \begin{split}
             =& s_1\brackets{s_1\brackets{v,e_1}e_1+\dots+s_n\brackets{v,e_n}e_n,e_1}e_1\\
             &+\dots+ s_n\brackets{s_1\brackets{v,e_1}e_1+\dots+s_n\brackets{v,e_n}e_n,e_n}e_n
         \end{split}\\
        =& s_1\brackets{s_1\brackets{v,e_1}e_1,e_1}e_1 +\dots+ s_n\brackets{s_n\brackets{v,e_n}e_n,e_n}e_n\\
        =& s_1^2\brackets{v,e_1}\brackets{e_1,e_1}e_1 +\dots+ s_n^2\brackets{v,e_n}\brackets{e_n,e_n}e_n\\
        =& s_1^2\brackets{v,e_1}e_1 +\dots+ s_n^2\brackets{v,e_n}e_n =T^\ast Tv.
     \end{align*}
     Where we used that $\brackets{s_i\brackets{v,e_j}e_j,e_i} =s_i\brackets{v,e_j}\brackets{e_j,e_i} =0$ if $i\neq j$ because $e_1,\dots,e_n$ is an orthonormal basis.

     To finish the proof, we have to prove that $\sqrt{T^\ast T}$ is positive. 

     First,
     \begin{align*}
         \brackets{\sqrt{T^\ast T}v,v} =& \brackets{s_1\brackets{v,e_1}e_1 +\dots+ s_n\brackets{v,e_n}e_n,v}\\
         =& \brackets{s_1\brackets{v,e_1}e_1,v} +\dots+ \brackets{s_n\brackets{v,e_n}e_n,v}\\
         =& s_1\brackets{v,e_1}\brackets{e_1, v} +\dots+ s_n\brackets{v,e_n}\brackets{e_n,v}\\
         =& s_1\brackets{v,e_1}\overline{\brackets{v,e_1}} +\dots+ s_n\brackets{v,e_n}\overline{\brackets{v,e_n}}\geq 0.
     \end{align*}
     The last inequality follows because $z\overline{z}\geq 0$ for any complex number (see 4.5).

     Second, let $w\in V$,
     \begin{align*}
         \brackets{v,\sqrt{T^\ast T}^\ast w} =& \brackets{\sqrt{T^\ast T}v,w}\\
         =& \brackets{s_1\brackets{v,e_1}e_1,w} +\dots+ \brackets{s_n\brackets{v,e_n}e_n,w}\\
         =& s_1\brackets{v,e_1}\brackets{e_1, w} +\dots+ s_n\brackets{v,e_n}\brackets{e_n,w}\\
         =& \brackets{v,s_1\overline{\brackets{e_1, w}}e_1} +\dots+ \brackets{v,s_n\overline{\brackets{e_n,w}}e_n}\\
         =& \brackets{v,s_1\brackets{w, e_1}e_1} +\dots+ \brackets{v,s_n\brackets{w,e_n}e_n}\\
         =& \brackets{v,s_1\brackets{w, e_1}e_1 +\dots+ s_n\brackets{w,e_n}e_n} = \brackets{v,\sqrt{T^\ast T}},
     \end{align*}
     so that $\sqrt{T^\ast T}$ is self-adjoint, finalising our proof.
     \item Let $v\in V$. We have 
     \begin{align*}
         T^{-1}Tv =& \frac{\brackets{Tv,f_1}e_1}{s_1}+\dots+\frac{\brackets{Tv,f_n}e_n}{s_n}\\
         \begin{split}
             =& \frac{\brackets{s_1\brackets{v,e_1}f_1+\dots+s_n\brackets{v,e_n}f_n,f_1}e_1}{s_1}\\
             &+\dots+\frac{\brackets{s_1\brackets{v,e_1}f_1+\dots+s_n\brackets{v,e_n}f_n,f_n}e_n}{s_n}
         \end{split}\\
         =& \frac{\brackets{s_1\brackets{v,e_1}f_1, f_1}e_1}
         {s_1}+\dots+\frac{\brackets{s_n\brackets{v,e_n}f_n,f_n}e_n}{s_n}\\
         =& \frac{s_1\brackets{v,e_1}\brackets{f_1, f_1}e_1}{s_1}+\dots+\frac{s_n\brackets{v,e_n}\brackets{f_n,f_n}e_n}{s_n}\\
         =& \brackets{v,e_1}\brackets{f_1, f_1}e_1+\dots+\brackets{v,e_n}\brackets{f_n,f_n}e_n\\
         =& \brackets{v,e_1}e_1+\dots+\brackets{v,e_n}e_n =v.
     \end{align*}
     Where the last equality follows from 6.30. Furthermore, we used $\brackets{s_i\brackets{v,e_j}f_j,f_i}e_i =0$ if $i\neq j$, as in 2. We can prove that $TT^{-1}v=v$ in a similar way, arriving to the desired result.
 \end{enumerate}
\end{proof}

\begin{exercise}{18}
  Suppose $T\in\LLL(V)$. Let $\hat{s}$ denote the smallest singular value of $T$, and let $s$ denote the largest singular value of $T$.
  \begin{enumerate}
      \item Prove that $\hat{s}\norm{v} \leq\norm{Tv} \leq s\norm{v}$ for every $v\in V$.
      \item Suppose $\lambda$ is an eigenvalue of $T$. Prove that $\hat{s} \leq\absoluteValue{\lambda} \leq s$.
  \end{enumerate}
\end{exercise}
\begin{proof}
 \begin{enumerate}
     \item Since $T^\ast T$ is self-adjoint, by the Spectral Theorem, we can find an orthonormal basis, $e_1,\dots,e_n$, of $V$ consisting of eigenvectors of $T^\ast T$. Furthermore, let $\lambda_1,\dots,\lambda_n$ be their corresponding eigenvalues. For an arbitrary $v\in V$, which we can represent as a linear combination of $e_1,\dots,e_n$. We have
     \begin{align*}
         \norm{Tv}^2
         =& \brackets{Tv,Tv} \\
         =& \brackets{T^\ast Tv,v}\\
         =& \brackets{T^\ast T(a_1e_1+\dots+a_ne_n),v}\\
         =& \brackets{a_1\lambda_1e_1+\dots+a_n\lambda_ne_n,v}\\
         =& \brackets{a_1\lambda_1e_1,v}+\dots+\brackets{a_n\lambda_ne_n,v}\\
         =& \lambda_1\brackets{a_1e_1,v}+\dots+\lambda_n\brackets{a_ne_n,v}\\
         \leq& \lambda\brackets{a_1e_1,v}+\dots+\lambda\brackets{a_ne_n,v}\\
         =& \lambda\brackets{a_1e_1+\dots+a_ne_n,v}\\
         =& \lambda\brackets{v,v} =\lambda\norm{v}^2.
     \end{align*}
     Where we took $\lambda$ to be the greatest eigenvalue of $T^\ast T$. In particular, we know that the inequality holds because $\brackets{a_ie_i,v}=\brackets{a_ie_i,a_ie_i}\geq 0$, given that $e_1,\dots,e_n$ are orthonormal and because by 7.35 the eigenvalues of a positive operator are nonnegative. We can find a similar result (but reversing the inequality) if we consider the smallest eigenvalue, $\hat{\lambda}$, of $T^\ast T$. Hence $\hat{\lambda}\norm{v}^2 \leq\norm{Tv}^2 \leq \lambda\norm{v}^2$. Taking square root on all elements of the inequality and 7.52, we get the desired inequality.
     \item In 1, take $v$ to be an eigenvector of $T$. This gives us the required result by recalling that $\norm{v}=1$ and $\norm{Tv} =\norm{\lambda v} =\absoluteValue{\lambda}\norm{v} =\absoluteValue{\lambda}$.
 \end{enumerate}
\end{proof}