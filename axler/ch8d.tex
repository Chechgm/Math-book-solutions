\subsection*{Chapter 8.D. Jordan form}
\addcontentsline{toc}{subsection}{Chapter 8.D. Jordan form}


\begin{exercise}{3}
  Suppose $N\in\LLL(V)$ is nilpotent. Prove that the minimal polynomial of $N$ is $z^{m+1}$, where $m$ is the length of the longest consecutive string of 1's that appears on the line directly above the diagonal in the matrix of $N$ with respect to any Jordan basis for $N$.
\end{exercise}
\begin{proof}
 The minimal polynomial of $N$ is the unique monic polynomial, $p(z)$, of smallest degree such that $p(N)=0$. Furthermore, by 8.49 we know that the eigenvalues of $N$ are the zeros of the minimal polynomial. In the case of a nilpotent operator, this has two implications, first, the minimal polynomial must be simply a power of $z$ (since the roots of such polynomial are 0 and nilpotent operators have only 0 eigenvalues, as in exercise 8.A.7). Second, the power of $z$ in the polynomial must be the largest power, $j$, for the generalised eigenvector of $N$, say $v$, such that $N^jv\neq 0$. If this wasn't the case, then $p(N)v\neq 0$. Now, since the 1's above the diagonal in the Jordan form of $N$ are precisely the powers of the generalised eigenvectors that compose the Jordan basis (by 8.60) of $N$, then the result follows. 

 XXX fill
\end{proof}

\begin{exercise}{6}
  Suppose $N\in\LLL(V)$ is nilpotent and $v_1,\dots,v_n$ and $m_1,\dots,m_n$ are as in 8.55. Prove that $N^{m_1}v_1\dots,N^{m_n}v_n$ is a basis of $\nullspace N$. [The exercise above implies that $n$, which equals $\dim\nullspace N$, depends only on $N$ and not on the specific Jordan basis chosen for $N$].
\end{exercise}
\begin{proof}
 Notice that in 8.55, we said that $N^{m_1}v_1,\dots,v_1,\dots,N^{m_n}v_n,\dots,v_n$ is a basis of $V$, and $N^{m_1+1}v_1=\dots=N^{m_n+1}v_n=0$. We will prove \\$\vecspan(N^{m_1}v_1\dots,N^{m_n}v_n) =\nullspace N$ by double containment.

 ($\subseteq$) Suppose $v\in\vecspan(N^{m_1}v_1\dots,N^{m_n}v_n)$. Then $Nv =N[a_1(N^{m_1}v_1)+\dots+a_n(N^{m_n}v_n)] =a_1(N^{m_1+1}v_1)+\dots+a_n(N^{m_n+1}v_n) =0$, so that $v\in\nullspace V$. 
 
 ($\supseteq$) Suppose, for the sake of contradiction, it is not the case that $\nullspace N\subseteq\vecspan(N^{m_1}v_1\dots,N^{m_n}v_n)$. Then there exists $v\in\nullspace N$ with\\ $v\notin\vecspan(N^{m_1}v_1\dots,N^{m_n}v_n)$, but then $0 =Nv = N[a_1(N^{m_1}v_1)+\dots+a_n(N^{m_n}v_n)+a_{n+1}N^{m_1-1}v_1+\dots+a_Nv_n]$, with some $a_{n+1},\dots,a_N$ not equal to 0. However, this implies that $N^{m_1}v_1,\dots,Nv_1,\dots,N^{m_n}v_n,\dots,Nv_n$ are not linearly independent. Since we assumed that $N^{m_1}v_1,\dots,v_1,\dots,N^{m_n}v_n,\dots,v_n$ is a basis of $V$, then it must be the case that $\nullspace N\subseteq\vecspan(N^{m_1}v_1\dots,N^{m_n}v_n)$
\end{proof}