\subsection*{Chapter 7.D. Polar decomposition and singular value decomposition}
\addcontentsline{toc}{subsection}{Chapter 7.D. Polar decomposition and singular value decomposition}

9 Nice criterion of invertibility in terms of the SVD 
14 Other nice interpretation of singular values 
16 When do two maps have the same singular values?
17 Writing generic maps of T in terms of the SVD
18 Too bad we haven't done too much Carothers yet, cause this ties in perfectly. Still it's nice to see singular values kind of form a lower and upper bound of many things, like matrix norm and eigenvalues. Friedberg continues this discussion nicely and shows how this is actually useful in practice.


\begin{exercise}{1}
  Fix $u,x\in V$ with $u\neq 0$. Define $T\in\LLL(V)$ by $Tv =\brackets{v,u}x$ for every $v\in V$. Prove that
  \[
  \sqrt{T^\ast T}v=\frac{\norm{x}}{\norm{u}}\brackets{v,u}u
  \]
  for every $v\in V$.
\end{exercise}
\begin{proof}
 We first have to prove that $S=\sqrt{T^\ast T}$ is a positive operator.
 
 Self-adjoint: We have 
 \begin{align*}
     \brackets{v, S^\ast w} =& \brackets{Sv, w}\\
     =& \brackets{\frac{\norm{x}}{\norm{u}}\brackets{v,u}u, w}\\
     =& \frac{\norm{x}}{\norm{u}}\brackets{v,u}\brackets{u, w}\\
     =& \frac{\norm{x}}{\norm{u}}\brackets{v,u}\overline{\brackets{w, u}}\\
     =& \brackets{v,\frac{\norm{x}}{\norm{u}}\brackets{w, u}u},
 \end{align*}
 where the last inequality follows because $\norm{x}\in\R$ for all $x$. Then, $S$ is self-adjoint.

 Positivity: We have
 \begin{align*}
     \brackets{Tv,v} =& \brackets{\frac{\norm{x}}{\norm{u}}\brackets{v,u}u,v}\\
     =& \frac{\norm{x}}{\norm{u}}\brackets{\brackets{v,u}u,v}\\
     =& \frac{\norm{x}}{\norm{u}}\brackets{u,v}\brackets{v,u}\\
     =& \frac{\norm{x}}{\norm{u}}\overline{\brackets{v,u}}\brackets{v,u}>0,
 \end{align*}
 because norms are positive and $\overline{\brackets{v,u}}\brackets{v,u}$, unless $v=0$. Then, S is positive.
 
 To finish the proof we need to show that $\sqrt{T^\ast T}^2$ as defined above equals $T^\ast T$. We have:
 \begin{align*}
     \sqrt{T^\ast T}\sqrt{T^\ast T}v =& \sqrt{T^\ast T}\left[\frac{\norm{x}}{\norm{u}}\brackets{v,u}u\right]\\
     =& \frac{\norm{x}}{\norm{u}}\brackets{v,u}\sqrt{T^\ast T}u\\
     =& \frac{\norm{x}}{\norm{u}}\brackets{v,u}\frac{\norm{x}}{\norm{u}}\brackets{u,u}u\\
     =& \frac{\norm{x}^2}{\norm{u}^2}\brackets{v,u}\brackets{u,u}u\\
     =& \norm{x}^2\brackets{v,u}u\\
     =& \brackets{x,x}\brackets{v,u}u\\
     =& \brackets{\brackets{v,u}x,x}u\\
     =& \brackets{Tv,x}u.
 \end{align*}
 On the other hand, and using Example 7.4, where we proved $T^\ast v=\brackets{v,x}u$,
 \begin{align*}
     T^\ast Tv =& T^\ast(\brackets{v,u}x)\\
     =& \brackets{\brackets{v,u}x,x}u\\
     =& \brackets{Tv,x}u,
 \end{align*}
 as required.
\end{proof}

\begin{exercise}{3}
  Suppose $T\in\LLL(V)$. Prove that there exists an isometry $S\in\LLL(V)$ such that $T =\sqrt{TT^\ast}S$.
\end{exercise}
\begin{proof}
 Consider the polar decomposition of $T^\ast=S\sqrt{TT^\ast}$, and take adjoints on both sides of the equality: $(T^\ast)^\ast = T =(S\sqrt{TT^\ast})^\ast =(\sqrt{TT^\ast})^\ast S^\ast =\sqrt{TT^\ast}S^\ast$, where the last inequality holds because $\sqrt{TT^\ast}$ is a positive operator. From 7.42, we know that $S^\ast$ is an isometry, giving us what we wanted to prove.
\end{proof}

\begin{exercise}{8}
  Suppose $T\in\LLL(V),S\in\LLL(V)$ is an isometry, and $R\in\LLL(V)$ is a positive operator such that $T=SR$. Prove that $R =\sqrt{T^\ast T}$. [The exercise above shows that if we write $T$ as the product of an isometry and a positive operator (as in the Polar Decomposition 7.45), then the positive operator equals $\sqrt{T^\ast T}$].
\end{exercise}
\begin{proof}
 It is already given to us that $R$ is positive. We have $T^\ast T =(SR)^\ast (SR) =R^\ast S^\ast S R =R^\ast R =R^2$. The second to last equality follows from 7.42 where we concluded that $S^{-1} =S^\ast$. The last equality follows from the fact that $R$ is positive, so it is self-adjoint, meaning $R^\ast = R$. Thus we have that $R^2 =T^\ast T$, so that $R =\sqrt{T^\ast T}$.
\end{proof}

\begin{exercise}{9}
  Suppose $T\in\LLL(V)$. Prove that $T$ is invertible if and only if there exists a unique isometry $S\in\LLL(V)$ such that $T =S\sqrt{T^\ast T}$.
\end{exercise}
\begin{proof}
 ($\Rightarrow$) %Suppose, for the sake of contradiction, that $T$ is invertible but $S$ is not unique. Hence, there exists an isometry $S'\in\LLL(V)$ so that $T=S'\sqrt{T^\ast T}$. We then have $S\sqrt{T^\ast T} =S'\sqrt{T^\ast T}$. Since $\sqrt{T^\ast T}$ is a positive operator, then by Exercise 7.C.7, it is invertible. Hence, $S =S\sqrt{T^\ast T}\sqrt{T^\ast T}^{-1} =S'\sqrt{T^\ast T}\sqrt{T^\ast T}^{-1} =S'$, as required.

 ($\Leftarrow$) fill%Consider the linear map $T^{-1}=\sqrt$
\end{proof}

\begin{exercise}{10}
  Suppose $T\in\LLL(V)$ is self-adjoint. Prove that the singular values of $T$ equal the absolute values of the eigenvalues of $T$, repeated appropriately.
\end{exercise}
\begin{proof}
 We have $T^\ast =T$, let $v$ be an eigenvector of $T$ with eigenvalue $\lambda$. We have $T^\ast Tv =TTv =T\lambda v =\lambda Tv =\lambda^2v$. From 7.52, we know that the singular values of $T$ are the nonnegative square roots of the eigenvalues of $T^\ast T$. That is, the singular values of $T$ are $\sqrt{\lambda^2} =\absoluteValue{\lambda}$, as desired.
\end{proof}

\begin{exercise}{11}
  Suppose $T\in\LLL(V)$. Prove that $T$ and $T^\ast$ have the same singular values.
\end{exercise}
\begin{proof}
 We defined (in conjunction with 7.52) the singular values of $T$ as the nonnegative square roots of the eigenvalues of $T^\ast T$, so that the singular values of $T^\ast$ are the square roots of the eigenvalues of $(T^\ast)^\ast T^\ast =TT^\ast$. Then we just need to prove that $T^\ast T$ and $TT^\ast$ have the same eigenvalues. But we know this is true from exercise 5.A.23, as desired.
\end{proof}

\begin{exercise}{13}
  Suppose $T\in\LLL(V)$. Prove that $T$ is invertible if and only if 0 is not a singular value of $T$.
\end{exercise}
\begin{proof}
 ($\Rightarrow$) We will prove this by contrapositive. Suppose 0 is a singular value of $T$. Then by the definition of singular values, 7.52 and Exercise 5.A.23, we have that 0 is an eigenvalue of $TT^\ast$. That is, for some $v\neq 0$, we have $TT^\ast v =Tw =0$. Here we have two options. First, if $w=0$, then $T^\ast v=0$ and $T$ is not invertible because $T^\ast$ is not invertible (see Exercise 7.A.5). Second, $w\neq 0$ and $\nullspace T\neq\set{0}$ so that $T$ is not invertible.

 ($\Leftarrow$) We will prove this by contrapositive. Suppose $T$ is not invertible. Then $\nullspace T\neq\set{0}$. That is, there exists $v\in V$ so that $Tv=0$. We have $T^\ast Tv =0$ so that $0$ is an eigenvalue of $T^\ast T$ and by 7.52, 0 is a singular value of $T$.
\end{proof}

\begin{exercise}{14}
  Suppose $T\in\LLL(V)$. Prove that $\dim\range T$ equals the number of nonzero singular values of $T$.
\end{exercise}
\begin{proof}
 Fill
\end{proof}

\begin{exercise}{15}
  Suppose $S\in\LLL(V)$. Prove that $S$ is an isometry if and only if all  the singular values of $S$ equal 1.
\end{exercise}
\begin{proof}
 ($\Rightarrow$) Suppose $S$ is an isometry. From 7.42 we have that $S^\ast S =I$. Furthermore, from 7.52 we know that the singular values of $S$ are the nonnegative square roots of the eigenvalues of $S^\ast S =I$, which we know all are 1, giving us the desired result.

 ($\Leftarrow$) Suppose all singular values of $S$ equal 1. Then by 7.52 the eigenvalues of $S^\ast S$ are 1 (the squares of the singular values of $S$). Since $S^\ast S$ is normal, we can use the Spectral Theorem to find an orthonormal basis of $V$ consisting of eigenvectors of $S^\ast S$. For arbitrary $v,w\in V$, we have $\brackets{Sv,Sw} =\brackets{S^\ast Sv,w} =\brackets{v,w}$ where the last equality follows from writing $v$ as a linear combination of the basis of eigenvectors of $S^\ast S$, and using the fact that all eigenvalues of $S^\ast S$ are 1. Thus, $S$ is an isometry,
\end{proof}

\begin{exercise}{16}
  Suppose $T_1,T_2\in\LLL(V)$. Prove that $T_1$ and $T_2$ have the same singular values if and only if there exists isometries $S_1,S_2\in\LLL(V)$ such that $T_1 =S_1T_2S_2$.
\end{exercise}
\begin{proof}
 ($\Rightarrow$)

 ($\Leftarrow$) Suppose there exist isometries $S_1,S_2\in\LLL(V)$ such that $T_1 =S_1T_2S_2$. Then by 7.52, the singular values of $T_1$ are the nonnegative square roots of $\sqrt{(S_1T_2S_2)^\ast(S_1T_2S_2)} =\sqrt{S_2^\ast T_2^\ast S_1^\ast S_1T_2S_2} =\sqrt{S_2^\ast T_2^\ast IT_2S_2}$
\end{proof}

\begin{exercise}{17}
  Suppose $T\in\LLL(V)$ has singular value decomposition given by 
  \[
  Tv =s_1\brackets{v,e_1}f_1+\dots+s_n\brackets{v,e_n}f_n
  \]
  for every $v\in V$, where $s_1,\dots,s_n$ are the singular values of $T$. and $e_1,\dots,e_n$ and $f_1,\dots,f_n$ are orthonormal bases of $V$.
  \begin{enumerate}
      \item Prove that if $v\in V$, then
      \[
      T^\ast v =s_1\brackets{v,f_1}e_1+\dots+s_n\brackets{v,f_n}e_n.
      \]
      \item Prove that $v\in V$, then
      \[
      T^\ast Tv =s_1^2\brackets{v,e_1}e_1+\dots+s_n^2\brackets{v,e_n}e_n.
      \]
      \item Prove that if $v\in V$, then
      \[
      \sqrt{T^\ast T}v =s_1\brackets{v,e_1}e_1+\dots+s_n\brackets{v,e_n}e_n.
      \]
      \item Suppose $T$ is invertible. Prove that if $v\in V$, then 
      \[
      T^{-1}v =\frac{\brackets{v,f_1}e_1}{s_1}+\dots+\frac{\brackets{v,f_n}e_n}{s_n}
      \]
      for every $v\in V$.
  \end{enumerate}
\end{exercise}
\begin{proof}
 \begin{enumerate}
     \item fill
     \item fill
     \item fill
     \item fill
 \end{enumerate}
\end{proof}

\begin{exercise}{18}
  Suppose $T\in\LLL(V)$. Let $\hat{s}$ denote the smallest singular value of $T$, and let $s$ denote the largest singular value of $T$.
  \begin{enumerate}
      \item Prove that $\hat{s}\norm{v} \leq\norm{Tv} \leq s\norm{v}$ for every $v\in V$.
      \item Suppose $\lambda$ is an eigenvalue of $T$. Prove that $\hat{s} \leq\absoluteValue{\lambda} \leq s$.
  \end{enumerate}
\end{exercise}
\begin{proof}
 Fill
\end{proof}