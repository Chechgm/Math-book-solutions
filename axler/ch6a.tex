\subsection*{Chapter 6.A. Inner Products and Norms}
\addcontentsline{toc}{subsection}{Chapter 6.A. Inner Products and Norms}


\begin{exercise}{3}
  Suppose $\bF=\R$ and $V\neq\set{0}$. Replace the positivity condition (which states that $\langle v, v\rangle\geq 0$ for all $v\in V$) in the definition of an inner product (6.3) with the condition that $\langle v, v\rangle>0$ for some $v\in V$. Show that this change in the definition does not change the set of functions from $V\times V$ to $\R$ that are inner products on $V$.
\end{exercise}
\begin{proof}
 What we need to prove here is that there is no $v\in V$ for which $\langle v, v\rangle<0$. 
 
 Suppose, for the sake of contradiction, that there exists $w\in V$ such that $\langle w, w\rangle<0$. Notice that inner products are continuous functions on their arguments, as $\langle w,w\rangle =\sum_{i,j}w_iw_j\langle e_i, e_j\rangle$. and $\langle e_i,e_j\rangle$ are constants for all standard basis vectors  of $V$, $e_i,e_j$. Then we would have, by the intermediate value theorem, that there is a vector $x$ with $\langle x,x\rangle =0$ in the segment $w,v$. 
 
 But because of the definiteness condition, $x=0$ so that $\alpha v +(1-\alpha) w=0$ for $\alpha\in [0,1]$. This implies $\alpha(v_1e_1+\dots+v_ne_n)+(1-\alpha)(w_1e_1+\dots+w_ne_n)=0$ and $\alpha v_ie_i=-(1-\alpha)w_ie_i$ which is $-\alpha/(1-\alpha)v_i=cv_i=w_i$, where we defined $c=-\alpha/(1-\alpha)$.

 However, this implies that $\langle w,w\rangle = \langle cv,cv\rangle = c^2\langle v,v\rangle > 0$, which contradicts our initial hypothesis. Hence it must be the case that $\langle w,w\rangle>0$.
\end{proof}

\begin{exercise}{4}
  Suppose $V$ is a real inner product space.
    \begin{enumerate}
        \item Show that $\langle u+v,u-v\rangle=\lVert u\rVert^2-\lVert v\rVert^2$ for every $u,v\in V$.
        \item Show that if $u,v\in V$ have the same norm, then $u+v$ is orthogonal to $u-v$.
        \item Use part 2 to show that the diagonals of a rhombus are perpendicular to each other.
    \end{enumerate}  
\end{exercise}
\begin{proof}
 \begin{enumerate}
     \item We have 
     \begin{align*}
        \langle u+v,u-v\rangle &= \langle u,u\rangle-\langle u,v\rangle+\langle v,u\rangle-\langle v,v\rangle\\
        &= \lVert u\rVert^2-\lVert v\rVert^2 -\langle u,v\rangle+\overline{\langle u,v\rangle}
        &= \lVert u\rVert^2-\lVert v\rVert^2.
     \end{align*}
     The second to third inequality follows because in a real vector space the complex conjugate of a number is itself.
     \item If $\lVert u\rVert^2=\lVert v\rVert^2$, then $\langle u+v,u-v\rangle=\lVert u\rVert^2-\lVert v\rVert^2=0$, so that $u+v$ is orthogonal to $u-v$.
     \item Since a rhombus is a special case of a parallelogram where all sides have the same length, then the conditions for exercise 2 hold, so that we can conclude that the diagonals of the rhombus are perpendicular.
 \end{enumerate}
\end{proof}

\begin{exercise}{13}
  Suppose $u,v$ are nonzero vectors in $\R^2$. Prove that $\langle u,v\rangle= \lVert u\rVert\lVert v\rVert\cos\theta$, where $\theta$ is the angle between $u$ and $v$ (thinking of $u$ and $v$ as arrows with initial point at the origin). Hint: Draw the triangle formed by $u,v$ and $u-v$; then use the law of cosines.
\end{exercise}
\begin{proof}
 In the text we saw we can think of squared norm as the length of a particular vector. That is, $\lVert u\rVert^2$ is the length of $u$ from the origin. First, we have
 \begin{align*}
     \langle u-v, u-v\rangle =& \langle u, u-v\rangle-\langle v,u-v\rangle\\
     =& \langle u-v,u\rangle-\langle u-v,v\rangle\\
     =& \langle u,u\rangle-\langle v,u\rangle-\langle u,v\rangle +\langle v,v\rangle\\
     =& \lVert u\rVert^2+\lVert v\rVert^2-2\langle u,v\rangle
 \end{align*}
 Now from the law of cosines we have that
 \begin{align*}
     \langle u-v,u-v\rangle =& \lVert u\rVert^2+\lVert v\rVert^2-2\langle u,v\rangle\\
     =& \lVert u\rVert^2 +\lVert v\rVert^2 -2\lVert u\rVert\lVert v\rVert\cos\theta &&\iff\\
     -2\langle u,v\rangle =& -2\lVert u\rVert\lVert v\rVert\cos\theta &&\iff\\
     \langle u,v\rangle =& \lVert u\rVert\lVert v\rVert\cos\theta,
 \end{align*}
 as required.
\end{proof}

\begin{exercise}{14}
  The angle between two vectors (thought of as arrows with initial point at the origin) in $\R^2$ or $\R^3$ can be defined geometrically. However, geometry is not as clear in $\R^n$ for $n>3$. Thus the angle between two nonzero vectors $x,y\in\R^n$ is defined to be $\arccos\frac{\langle x,y\rangle}{\lVert x\rVert\lVert y\rVert}$, where the motivation for this definition comes from the previous exercise. Explain why the Cauchy-Schwarz inequality is needed to show that this definition makes sense.
\end{exercise}
\begin{proof}
 The function $\arccos$ is defined in the following domain and codomain $\arccos:[-1,1]\to[0,\pi]$. Then the Cauchy-Schwarz inequality $\lvert\langle x, y\rvert\rangle\leq\lVert x\rVert\lVert y\rVert$ guarantees that $\lvert \langle x,y\rangle\rvert/\lVert x\rVert\lVert y\rVert\leq 1$, as it is required by the function definition.
\end{proof}

\begin{exercise}{17}
  Prove or disprove: there is an inner product on $\R^2$ such that the associated norm is given by $\lVert (x,y)\rVert=\max\{x,y\}$ for all $(x,y)\in\R^2$.
\end{exercise}
\begin{proof}
 Consider $(-1,0)\in\R^2$, which would give us $\lVert (-1,0)\rVert=0$. However, by 6.10, we know that $\lVert v\rVert=0$ if and only if $v=0$ which is not the case here. Hence, $\max\set{x,y}$ is not a valid norm and hence there cannot be an inner product that induces such function
\end{proof}

\begin{exercise}{18}
  Suppose $p>0$. Prove that there is an inner product on $\R^2$ such that the associated norm is given by $\lVert (x,y)\rVert =(x^p+y^p)^{1/p}$ for all $(x,y)\in\R^2$ if an only if $p=2$.
\end{exercise}
\begin{proof}
 ($\Rightarrow$) Suppose there is an inner product on $\R^2$ with the mentioned characteristics. Then it must be the case that the square  of the norm of a vector with itself gives us a valid inner product of the same element with itself. That is, $\langle (x,y), (x,y)\rangle=\lVert (x,y)\rVert^2= (x^p+y^p)^{2/p}$. Now take $\lambda\in\R$. We have $\lVert\lambda(x,y)\rVert= ((\lambda x)^p+(\lambda y)^p)^{2/p}=(\lambda^p(x^p+y^p))^{2/p}$ this equals $\lambda^2(x^p+y^p)^{2/p}$ if and only if $p=2$.

 ($\Leftarrow$) Suppose $p=2$. Then the norm is defined as $\lVert (x,y)\rVert =(x^2+y^2)^{1/2}$. We have that the dot product as in definition 6.2 for $\R^2$ is given by $\langle (x_1,y_1), (x_2,y_2)\rangle= x_1x_2+y_1y_2$. Considering the dot product of a vector in $\R^2$ with respect to itself and taking the square root of the resulting value gives us the norm $\lVert (x,y)\rVert$ as defined above.
\end{proof}

\begin{exercise}{19}
  Suppose $V$ is a real inner product space. Prove that 
  \[
  \langle u,v\rangle=\frac{\lVert u+v\rVert^2-\lVert u-v\rVert^2}{4}
  \]
  for all $u,v\in V$.
\end{exercise}
\begin{proof}
 We have
 \begin{align*}
     \lVert u+v\rVert^2-\lVert u-v\rVert^2 =& \langle u+v,u+v\rangle - \langle u-v,u-v\rangle\\
     =&\lVert u\rVert^2+\lVert v\rVert^2+\langle u,v\rangle +\langle v,u\rangle\\
     &-[\lVert u\rVert^2+\lVert v\rVert^2+\langle u,-v\rangle+\langle -v,u\rangle]\\
     =& 2[\langle u,v\rangle+\langle v, u\rangle]\\
     =& 4\langle u,v\rangle.
 \end{align*}
Where the last two inequalities follow from the fact that, in a real inner product space, $\langle u,v\rangle=\langle v,u\rangle$, and $\langle -v,u\rangle=\langle u,-v\rangle=-\langle u,v\rangle$.
\end{proof}

\begin{exercise}{20}
  Suppose $V$ is a complex inner product space. Prove that
  \[
  \langle u,v\rangle =\frac{\lVert u+v\rVert^2-\lVert u-v\rVert^2+\lVert u+iv\rVert^2i-\lVert u-iv\rVert^2i}{4}
  \]
  for all $u,v\in V$.
\end{exercise}
\begin{proof}
 Notice that in exercise 19 we worked out $\lVert u+v\rVert^2-\lVert u-v\rVert^2$ and used a property of real inner product spaces only on the last two steps. Now we will work out the rest of the numerator. We have 
 \begin{align*}
     \lVert u+iv\rVert^2-\lVert u-iv\rVert^2 =& \langle u+iv, u+iv\rangle- \langle u-iv,u-iv\rangle\\
     =& \lVert u\rVert^2 +\langle u, iv\rangle +\langle iv,u\rangle +\lVert iv\rVert^2\\
     &-[\lVert u\rVert^2+\langle u,-iv\rangle+\langle -iv,u\rangle+\lVert -iv\rVert^2]\\
     =& \langle u, iv\rangle +\langle iv,u\rangle +\lvert i\rvert^2\lVert v\rVert^2\\
     &-[\langle u,-iv\rangle+\langle -iv,u\rangle+\lvert -i\rvert^2\lVert v\rVert^2]\\
     =& \langle u, iv\rangle +\langle iv,u\rangle -\langle u,-iv\rangle-\langle -iv,u\rangle\\
     =& -i\langle u,v\rangle +i\langle v, u\rangle -i\langle u,v\rangle +i\langle v,u\rangle\\
     =& 2i[\langle v,u\rangle-\langle u,v\rangle]
 \end{align*}
 Now we replace this in the total equation to obtain the following
 \begin{align*}
     \lVert u+v\rVert^2-&\lVert u-v\rVert^2+\lVert u+iv\rVert^2i-\lVert u-iv\rVert^2i\\ &= 2[\langle u,v\rangle +\langle v,u\rangle] + 2i^2[\langle v,u\rangle-\langle u,v\rangle]\\
     &= 4\langle u,v\rangle,
 \end{align*}
 as required.
\end{proof}

\begin{exercise}{21}
  A norm on a vector space $U$ is a function $\lVert\,\rVert:U\to[0,\infty)$ such that $\lVert u\rVert=0$ if and only if $u=0$, $\lVert\alpha u\rVert= \lvert\alpha \rvert\lVert u\rVert$ for all $\alpha\in\bF$ and all $u\in U$, and $\lVert u+v\rVert\leq \lVert u\rVert +\lVert v\rVert$ for all $u,v\in U$. Prove that a norm satisfying the parallelogram equality comes from an inner product (in other words, show that if $\lVert\,\rVert$ is a norm on $U$ satisfying the parallelogram equality, then there is an inner product $\langle\, ,\,\rangle$ on $U$ such that $\lVert u\rVert=\langle u,u\rangle^{1/2}$ for all $u\in U$).
\end{exercise}
\begin{proof}
 We will divide the solution to this exercise in several parts.

 First we will prove the result for $\bF=\R^n$. Let 
 \[ \langle x,y\rangle =\frac{1}{4}[\lVert x+y\rVert^2- \lVert x-y\rVert^2]. \]

 We have to prove all the properties in definition 6.3, which we will do in steps.

 Positivity: we have $\langle v,v\rangle = (1/4)\lVert 2v\rVert^2= (1/4)\lvert 2\rvert^2\lVert v\rVert^2= \lVert v\rVert^2\geq 0$, as required.

 Definiteness: we have $\langle v,v\rangle=0$, if and only if $\lVert 2v\rVert^2=0$ which is equivalent to $\lVert v\rVert^2=0$, that is $\lVert v\rVert =0$ which we know is only true if and only if $v=0$.

 Additivity in first slot: 

 Homogeneity in the first slot:

 Conjugate symmetry: 
\end{proof}