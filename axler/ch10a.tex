\subsection*{Chapter 10.A. Trace}
\addcontentsline{toc}{subsection}{Chapter 10.A. Trace}


\begin{exercise}{1}
  Suppose $T\in\LLL(V)$ and $v_1,\dots,v_n$ is a basis of $V$. Prove that the matrix $\MMM(T, (v_1,\dots,v_n))$ is invertible if and only if $T$ is invertible.
\end{exercise}
\begin{proof}
 We have:
 \begin{align*}
     &TT^{-1}=T^{-1}T=I &&\iff\\
     &\MMM(TT^{-1})=\MMM(T^{-1}T)=I &&\iff\\
     &\MMM(T)\MMM(T^{-1})=\MMM(T^{-1})\MMM(T)=I.&&
 \end{align*}
 Where the product of matrices of linear maps follows from 3.42. Thus, there exists an inverse of $\MMM(T)$, say $\MMM(T^{-1})$, if and only if there exists an inverse of the operator $T$, $T^{-1}$.
\end{proof}

\begin{exercise}{8}
  Suppose $V$ is an inner product space and $v,w\in V$. Define $T\in\LLL(V)$ by $Tu=\brackets{u,v}w$. Find a formula for $\trace T$.
\end{exercise}
\begin{proof}
 We will first derive the eigenvalues of $T$. By the definition of eigenvalues and eigenvectors, an eigenvalue $\lambda$ with corresponding eigenvector $u$ must have the following property: $Tu =\brackets{u,v}w =\lambda u$. However, this implies $u$ must be a multiple of $w$, without loss of generality, let $u=w$, so that $\brackets{w,v}w=\lambda w$ and $\lambda=\brackets{w,v}$. 
 
 Now, the above derivation is true for all eigenvalues of $T$, so that $\brackets{w,v}$ is the only eigenvalue of $T$. Furthermore, we know by 8.26 that the sum of multiplicities of eigenvalues must be $\dim V$, then we get the following expression for the trace of $T$: $\trace T=(\dim V)\brackets{w,v}$.
\end{proof}

\begin{exercise}{10}
  Suppose $V$ is an inner product space and $T\in\LLL(V)$. Prove that $\trace T^\ast =\overline{\trace T}$.
\end{exercise}
\begin{proof}
 Let $\overline{\MMM(T)}$ be the conjugate transpose of $\MMM(T)$. We have:
 \begin{align*}
     \trace T^\ast =& \trace\MMM(T^\ast)\\
     =& \trace\overline{\MMM(T)}\\
     =& \overline{\trace T}.
 \end{align*}
 Where the third equality follows from the fact that the trace is simply the sum of the diagonal elements of $\overline{\MMM(T)}$, which is simply the sum of the complex conjugate of the diagonal elements of $\MMM(T)$, as required.
\end{proof}

\begin{exercise}{14}
  Suppose $T\in\LLL(V)$ and $c\in\bF$. Prove that $\trace(cT)=c\trace T$.
\end{exercise}
\begin{proof}
 Let $A_{i,j}$ be the element on the $i$'th row and $j$'th column of $\MMM(T)$. We have 
 \begin{align*}
     \trace(cT) =& \sum_{j=1}^ncA_{j,j}\\
     =& c\sum_{j=1}^nA_{j,j}\\
     =& c\trace T.
 \end{align*}
 Where the second equality follows from the definition of scalar multiplication of a matrix (3.37).
\end{proof}

\begin{exercise}{15}
  Suppose $S,T\in\LLL(V)$. Prove that $\trace(ST)=\trace(TS)$.
\end{exercise}
\begin{proof}
 This is exactly the content of 10.14, taking into account that we can represent $T$ and $S$ as matrices, and the fact that the trace of on operator does not depend on the basis chosen to represent its matrix (10.15).
\end{proof}

\begin{exercise}{19}
  Suppose $V$ is an inner product space. Prove that $\brackets{S,T}=\trace(ST^\ast)$ defines an inner product on $\LLL(V)$.
\end{exercise}
\begin{proof}
 Let $S,T,R\in\LLL(V)$ and $\lambda\in\bF$.

 Positivity and definiteness: We have
 \begin{align*}
     \brackets{S,S} =& \trace(SS^\ast)\\
     =& \trace(\MMM(SS^\ast))\\
     =& \sum_{j=1}^n S_{j,j}\overline{S_{j,j}}\\
     =& \sum_{j=1}^n\absoluteValue{S_{j,j}}^2 \geq 0,
 \end{align*}
 with equality if and only if $S_{j,j}=0$ for all $j$. Since this computation is basis independent, taking the standard basis of $V$ shows us that $S$ itself is the 0 operator.

 Additivity in first slot: We have
 \begin{align*}
     \brackets{S+T,R} =& \trace((S+T)R^\ast)\\
     =& \trace(\MMM((S+T)R^\ast))\\
     =& \trace(\MMM(SR^\ast+TR^\ast))\\
     =& \trace(SR^\ast+TR^\ast)\\
     =& \trace(SR^\ast)+\trace(TR^\ast)\\
     =& \brackets{S,R}+\brackets{T,R}
 \end{align*}
 Where the third equality follows from the distributive property for matrix multiplication (exercise 3.C.13).

 Homogeneity in first slot: We have
 \begin{align*}
     \brackets{\lambda S,T} =& \trace(\lambda ST^\ast)\\
     =& \trace(\MMM(\lambda ST^\ast))\\
     =& \lambda\trace(\MMM(ST^\ast))\\
     =& \lambda\brackets{S,T}.
 \end{align*}

 Conjugate symmetry: We have
 \begin{align*}
     \overline{\brackets{S,T}} =& \overline{\trace(ST^\ast)}\\
     =& \trace((ST^\ast)^\ast)\\
     =& \trace(TS^\ast)\\
     =& \brackets{T,S}.
 \end{align*}
\end{proof}