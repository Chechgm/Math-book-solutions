\subsection*{Chapter 10.A. Trace}
\addcontentsline{toc}{subsection}{Chapter 10.A. Trace}

1 Basic correspondence between matrix and operator, very important
8 Easy application of the concept of a trace 
10 Linking trace with adjoints
14 Elementary property
15 Elementary property
19 So matrices also form an inner product space! Wow!

\begin{exercise}{1}
  Suppose $T\in\LLL(V)$ and $v_1,\dots,v_n$ is a basis of $V$. Prove that the matrix $\MMM(T, (v_1,\dots,v_n))$ is invertible if and only if $T$ is invertible.
\end{exercise}
\begin{proof}
 Fill
\end{proof}

\begin{exercise}{8}
  Suppose $V$ is an inner product space and $v,w\in V$. Define $T\in\LLL(V)$ by $Tu=\brackets{u,v}w$. Find a formula for $\trace T$.
\end{exercise}
\begin{proof}
 Fill
\end{proof}

\begin{exercise}{10}
  Suppose $V$ is an inner product space and $T\in\LLL(V)$. Prove that $\trace T^\ast =\overline{\trace T}$.
\end{exercise}
\begin{proof}
 Fill
\end{proof}

\begin{exercise}{14}
  Suppose $T\in\LLL(V)$ and $c\in\bF$. Prove that $\trace(cT)=c\trace T$.
\end{exercise}
\begin{proof}
 Fill
\end{proof}

\begin{exercise}{15}
  Suppose $S,T\in\LLL(V)$. Prove that $\trace(ST)=\trace(TS)$.
\end{exercise}
\begin{proof}
 Fill
\end{proof}

\begin{exercise}{19}
  Suppose $V$ is an inner product space. Prove that $\brackets{S,T}=\trace(ST^\ast)$ defines an inner product on $\LLL(V)$.
\end{exercise}
\begin{proof}
 Fill
\end{proof}