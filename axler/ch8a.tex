\subsection*{Chapter 8.A. Generalized eigenvectors and nilpotent operators}
\addcontentsline{toc}{subsection}{Chapter 8.A. Generalized eigenvectors and nilpotent operators}

1+2 Some computations to get used to the concepts
3 Nice properties how generalized eigenspaces behave under inverses
4 Compare with eigenspaces of different eigenvectors
6 An example of an operator without a square root, contrast this with square roots of positive operators!
7 Eigenvalues of nilpotent operators
8 Properties of nilpotent operators 
9 Properties of nilpotent operators 
12 This gives a nice characterization of nilpotent operators since all nilpotent operators have this form 
13 Nilpotent operators can (almost) never be normal! 
14 Real version of Schur's theorem

\begin{exercise}{1}
  Fill
\end{exercise}
\begin{proof}
 Fill
\end{proof}

\begin{exercise}{2}
  Fill
\end{exercise}
\begin{proof}
 Fill
\end{proof}

\begin{exercise}{3}
  Fill
\end{exercise}
\begin{proof}
 Fill
\end{proof}

\begin{exercise}{4}
  Fill
\end{exercise}
\begin{proof}
 Fill
\end{proof}

\begin{exercise}{6}
  Fill
\end{exercise}
\begin{proof}
 Fill
\end{proof}

\begin{exercise}{7}
  Fill
\end{exercise}
\begin{proof}
 Fill
\end{proof}

\begin{exercise}{8}
  Fill
\end{exercise}
\begin{proof}
 Fill
\end{proof}

\begin{exercise}{9}
  Fill
\end{exercise}
\begin{proof}
 Fill
\end{proof}

\begin{exercise}{12}
  Fill
\end{exercise}
\begin{proof}
 Fill
\end{proof}

\begin{exercise}{13}
  Fill
\end{exercise}
\begin{proof}
 Fill
\end{proof}

\begin{exercise}{14}
  Fill
\end{exercise}
\begin{proof}
 Fill
\end{proof}
