\subsection*{Chapter 7.A. }
\addcontentsline{toc}{subsection}{Chapter }

Axler 7: Operators on Inner Product Spaces
A Self-Adjoint and Normal Operators
1 Good exercise to get used to concepts
2 Relationship of eigenvalues between adjointness, what does it mean for self-adjoint operators? This falls in the theme that adjoint operators are "alike" complex connjugate and self-adjoint is "alike" real numbers
4 Practice with concepts 
7 Real numbers commute, but self-adjoint operators don't always
8+9 Remember R forms a real subspace of C
11 What "numbers" do projection operators correspond to
15 simple example
21 Derivatives are not self-adjoint but normal! Important for QM

\begin{exercise}{1}
  Suppose $n$ is a positive integer. Define $T\in\LLL(\bF^n)$ by 
  \[T(z_1,\dots,z_n)=(0,z_1,\dots,z_{n-1}\].
  Find a formula for $T^\ast(z_1,\dots,z_n)$.
\end{exercise}
\begin{proof}
 Fill
\end{proof}

\begin{exercise}{1}
  Fill
\end{exercise}
\begin{proof}
 Fill
\end{proof}

\begin{exercise}{1}
  Fill
\end{exercise}
\begin{proof}
 Fill
\end{proof}

\begin{exercise}{1}
  Fill
\end{exercise}
\begin{proof}
 Fill
\end{proof}

\begin{exercise}{1}
  Fill
\end{exercise}
\begin{proof}
 Fill
\end{proof}

\begin{exercise}{1}
  Fill
\end{exercise}
\begin{proof}
 Fill
\end{proof}

\begin{exercise}{1}
  Fill
\end{exercise}
\begin{proof}
 Fill
\end{proof}

\begin{exercise}{1}
  Fill
\end{exercise}
\begin{proof}
 Fill
\end{proof}

\begin{exercise}{1}
  Fill
\end{exercise}
\begin{proof}
 Fill
\end{proof}