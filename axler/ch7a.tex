\subsection*{Chapter 7.A. Self-Adjoint and normal operators}
\addcontentsline{toc}{subsection}{Chapter 7.A. Self-Adjoint and normal operators}


\begin{exercise}{1}
  Suppose $n$ is a positive integer. Define $T\in\LLL(\bF^n)$ by 
  \[T(z_1,\dots,z_n)=(0,z_1,\dots,z_{n-1}).\]
  Find a formula for $T^\ast(z_1,\dots,z_n)$.
\end{exercise}
\begin{proof}
 Notice that the matrix representing $T$ with respect to the standard basis of $\bF^n$ is given by the matrix with 1 in the $i$-th row and $(i-1)$-th column for $i=2,\dots,n$ and 0 everywhere else. By 7.10, the matrix of $T^\ast$ (with respect to the same basis) is the conjugate transpose of the previously described matrix, that is, the matrix with 1 in the $i$-th column and $(i-1)$-th row and 0 everywhere else. This matrix represents the linear transformation given by $T^\ast(z_1,\dots,z_n)=(z_2,\dots,z_{n},0)$, as desired.
\end{proof}

\begin{exercise}{2}
  Suppose $T\in\LLL(V)$ and $\lambda\in\bF$. Prove that $\lambda$ is an eigenvalue of $T$ if and only if $\bar{\lambda}$ is an eigenvalue of $T^\ast$.
\end{exercise}
\begin{proof}
 We will prove the first direction, and the second follows from a symmetric argument. Consider the linear map given by $(T-\lambda I)$. Since $\lambda$ is an eigenvalue of $T$, we know this map is not injective (by 5.6). By exercise 4, we have that $(T-\lambda I)^\ast=(T^\ast-\bar{\lambda}I)$ is not surjective. Using 5.6 again we conclude that $\bar{\lambda}$ is an eigenvalue of $T^\ast$, as required.
\end{proof}

\begin{exercise}{4}
  Suppose $T\in\LLL(V,W)$. Prove that
  \begin{enumerate}
      \item $T$ is injective if and only if $T^\ast$ is surjective;
      \item $T$ is surjective if and only if $T^\ast$ is injective.
  \end{enumerate}
\end{exercise}
\begin{proof}
\begin{enumerate}
    \item We will prove both directions at the same time. Suppose $T\in\LLL(V,W)$. We have,
    \begin{align*}
        T\text{ is injective }&\iff\nullspace T=\set{0}\\
        &\iff\set{0}=(\range T^\ast)^\perp\\
        &\iff(\set{0})^\perp=((\range T^\ast)^\perp)^\perp\\
        &\iff W=\range T^\ast\\
        &\iff T^\ast\text{ is surjective,}
    \end{align*}
    as required. The first equivalence follows from 3.10, the second equivalence follows from 7.7 (c), the fourth equivalence follows from 6.46 (b) and 6.51.
    \item We will prove both directions in one, as above. Suppose $T\in\LLL(V,W)$. We have
    \begin{align*}
        T\text{ is surjective }&\iff\range T=W\\
        &\iff W=(\nullspace T^\ast)^\perp\\
        &\iff W^\perp=((\nullspace T^\ast)^\perp)^\perp\\
        &\iff \set{0}=\nullspace T^\ast\\
        &\iff T^\ast\text{ is injective,}
    \end{align*}
    as required.
\end{enumerate}
\end{proof}

\begin{exercise}{7}
  Suppose $S,T\in\LLL(V)$ are self-adjoint. Prove that $ST$ is self-adjoint if and only if $ST=TS$.
\end{exercise}
\begin{proof}
 ($\Rightarrow$) Suppose $ST$ is self adjoint. Then, $ST = (ST)^\ast = T^\ast S^\ast = TS$, as required.

 ($\Leftarrow$) Suppose $ST=TS$. Consider the adjoint of $ST$: $(ST)^\ast = T^\ast S^\ast = TS = ST$, so that $ST$ is self-adjoint.
\end{proof}

\begin{exercise}{8}
  Suppose $V$ is a real inner product space. Show that the set of self-adjoint operators on $V$ is a subspace of $\LLL(V)$.
\end{exercise}
\begin{proof}
 Let $S,T\in\LLL(V)$ be self-adjoint and $\lambda\in\bF$.

 Addition: We have $(S+T)^\ast =S^\ast+T^\ast =S+T$, where the first inequality follows from 7.6 (a), so that $S+T$ is also self-adjoint.

 Scalar multiplication: We have $(\lambda S)^\ast =\bar{\lambda}S^\ast =\lambda S$, where the first inequality follows from 7.6 (b) and the last inequality follows from the fact that $\lambda\in\R$, hence, $\lambda S$ is self-adjoint.

 Hence, the set of self-adjoint operators is a subspace of $\LLL(V)$, as required.
\end{proof}

\begin{exercise}{9}
  Suppose $V$ is a complex inner product space with $V\neq\set{0}$. Show that the set of self-adjoint operators on $V$ is not a subspace of $\LLL(V)$.
\end{exercise}
\begin{proof}
 Let $T\in\LLL(V)$ be self-adjoint and $T\neq 0$. We have $(i T)^\ast =\bar{i}T^\ast =-iT\neq iT$ so that $iT$ is not self-adjoint and hence the set of self-adjoint operators on a complex inner product space is not a subspace, as required.
\end{proof}

\begin{exercise}{11}
  Suppose $P\in\LLL(V)$ is such that $P^2=P$. Prove that there is a subspace $U$ of $V$ such that $P=P_U$ if and only if $P$ is self-adjoint.
\end{exercise}
\begin{proof}
 ($\Rightarrow$) Suppose there is a subspace $U$ of $V$ such that $P=P_U$. Then for $v,v'\in V$ we can write $v=u+w$ and $v'=u'+w'$ where $u,u'\in U$ and $w,w'\in U^\perp$ and $Pv=u, Pv'=u'$. We have $\brackets{Pv,v'} =\brackets{u, v'} =\brackets{u, u'+w'} =\brackets{u,u'}$, because $\brackets{u,w'}=0$. Likewise, $\brackets{v,P v'} =\brackets{u,u'}$, so that $\brackets{Pv,v'}=\brackets{v,Pv'}$, as required.

 ($\Leftarrow$) Suppose $P$ is self-adjoint. Then for all $v,v'\in V$, $\brackets{Pv,v'} =\brackets{v,Pv'}$. Let $U=\range P$, we can write $v=u+w$. Moreover, by 7.7 (c), we know that $\nullspace P =(\range P)^\perp =U^\perp$. Then, $Pv =P(u+w) =Pu +Pw =Pu$, as required.
\end{proof}

\begin{exercise}{15}
  Fix $u,x\in V$. Define $T\in\LLL(V)$ by $Tv=\brackets{v,u}x$ for every $v\in V$.
  \begin{enumerate}
      \item Suppose $\bF=\R$. Prove that $T$ is self-adjoint if and only if $u, x$ is linearly dependent.
      \item Prove that $T$ is normal if and only if $u, x$ is linearly dependent.
  \end{enumerate}
\end{exercise}
\begin{proof}
On example 7.6 we showed that the adjoint of $T$ is $T^\ast v =\brackets{v,x}u$.
 \begin{enumerate}
    \item ($\Rightarrow$) Suppose $T$ is self-adjoint. Then $\brackets{v,u}x=Tv =T^\ast v =\brackets{v,x}u$. For $u,x\neq 0$, and letting $v=x$, we can write $u=(\brackets{x,u}/\brackets{x,x})x$ so that $u$ is a scalar multiple of $x$, as desired.
    
     ($\Leftarrow$) Suppose $u$ and $x$ are linearly dependent, then we can write $u=ax$ for some $a\in\R$. We have $T^\ast v =\brackets{v,x}u =\brackets{v,u/a}ax =(1/\bar{a})\brackets{v,u}ax =\brackets{v,u}x =Tv$, where $1/\bar{a}=1/a$ because $a\in\R$, as required.
     \item ($\Rightarrow$) Suppose $T$ is normal so that $TT^\ast v =\brackets{\brackets{v,x}u, u}x =\brackets{\brackets{v,u}x, x}u=T^\ast T$, then if $u,x\neq 0$, and letting $v=u$, we have $u =(\brackets{\brackets{u,x}u, u}/\brackets{\brackets{u,u}x, x})x$ so that $u$ is a scalar multiple of $x$.
     
     ($\Leftarrow$) Suppose $u$ and $x$ are linearly dependent. Then, for $a\in\bF$, we can write $u=ax$. We have $TT^\ast v =\brackets{\brackets{v,x}u, u}x =\brackets{\brackets{v,(1/a)u}ax, ax}(1/a)u =\brackets{\brackets{v,u}(a/\bar{a})x, x}(\bar{a}/a)u =(a/\bar{a})\brackets{\brackets{v,u}x, x}(\bar{a}/a)u =\brackets{\brackets{v,u}x, x}u=T^\ast T$
 \end{enumerate}
\end{proof}

\begin{exercise}{21}
  Fix a positive integer $n$. In the inner product space of continuous real-valued functions on $[-\pi,\pi]$ with inner product 
  \[\brackets{f,g}=\int_{-\pi}^\pi f(x)g(x)dx,\]
  let $V=\vecspan(1,\cos x,\cos 2x,\dots,\cos nx,\sin x,\sin 2x,\dots,\sin nx)$.
  \begin{enumerate}
      \item Define $D\in\LLL(V)$ by $Df=f'$. Show that $D^\ast = -D$. Conclude that D is normal but not self-adjoint.
      \item Define $T\in\LLL(V)$ by $Tf=f''$. Show that $T$ is self-adjoint.
  \end{enumerate}
\end{exercise}
\begin{proof}
 \begin{enumerate}
     \item Using $V$, we represent any function as a linear combination of the basis vectors $f(x)=a_0+a_1\cos x+a_2\cos 2x+\dots+a_n\cos nx+a_{n+1}\sin x+\dots+a_{2n}\sin nx$ for $a_i\in\R$. We have
     \begin{align*}
         \brackets{Df,g} =& \int_{-\pi}^\pi f'(x)g(x)dx\\
         =& \int_{-\pi}^\pi D(a_0+a_1\cos x+\dots+a_{2n}\sin nx)\\
         &(b_0+b_1\cos x+\dots+b_{2n}\sin nx dx\\
         =& \int_{-\pi}^\pi (-a_1\sin x-2a_2\sin 2x+\dots+na_{2n}\cos nx)\\
         &(b_0+b_1\cos x+\dots+b_{2n}\sin nx dx\\
         =& \int_{-\pi}^\pi -a_1b_0\sin x-2a_2b_0\sin 2x+\dots+na_{2n}b_0\cos nx\\
         &-a_1(\cos x) b_1\sin x-2a_2(\cos x)b_1\sin 2x+\dots+na_{2n}(\cos x)b_1\cos nx\\
         &+\dots\\
         &-a_1(\sin nx)b_2n\sin x-2a_2(\sin nx)b_2n\sin 2x+\dots+na_{2n}(\sin nx)b_2n\cos nx dx
     \end{align*}

     Since $D^\ast=-D$, then we have that $D^\ast D =-DD =D-D =DD^\ast$, so that $D$ is normal.
     \item We can represent any function using a linear combination of vectors in $V$ as above. We have
     \begin{align*}
         \brackets{Tf,g} =& \int_{-\pi}^\pi f''(x)g(x)dx\\
         =& \int_{-\pi}^\pi T(a_0+a_1\cos x+\dots+a_{2n}\sin nx)\\
         &(b_0+b_1\cos x+\dots+b_{2n}\sin nx dx\\
         =& \int_{-\pi}^\pi (-a_1\cos x+\dots-n^2a_{2n}\cos nx)\\
         &(b_0+b_1\cos x+\dots+b_{2n}\sin nx) dx\\
     \end{align*}
 \end{enumerate}

 The derivative of $(cos nx)(sin mx)$ is $-(m cos(m x) cos(n x) + n sen(m x) sen(n x))/(m^2 - n^2)$
 The integral of $(sin nx)(sin mx)$ is $(n cos(n x) sen(m x) - m cos(m x) sen(n x))/(m^2 - n^2)$
 The integral of $(cos nx)(cos mx)$ is $(m cos(n x) sen(m x) - n cos(m x) sen(n x))/(m^2 - n^2)$
\end{proof}