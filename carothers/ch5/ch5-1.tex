\section{Continuous functions}


\begin{exercise}{1}
Given a function $f:S\to T$ and sets $A,B\subseteq S$ and $C,D\subset T$ establish the following:
\begin{enumerate}
    \item $A \subseteq f^{-1}(f(A))$, with equality for all $A$ if and only if $A$ is one-to-one.
    \item $f(f^{-1}(C))\subseteq C$, with equality for all $C$ if and only if $f$ is onto.
    \item $f(A\cup B)= f(A)\cup f(B)$.
    \item $f^{-1}(C\cup D)=f^{-1}(C)\cup f^{-1}(D)$.
    \item $f(A\cap B)\subseteq f(A)\cap f(B)$, with equality for all $A$ and $B$ if and only if $f$ is one-to-one.
    \item $f^{-1}(C\cap D)=f^{-1}(C)\cap f^{-1}(D)$.
    \item $f(A)\setminus f(B)\subseteq f(A\setminus B)$.
    \item $f^{-1}(C\setminus D)= f^{-1}(C)\setminus f^{-1}(D)$.
\end{enumerate}
Generalise, whenever possible, to arbitrary unions and intersections.
\end{exercise}
\begin{proof}
\begin{enumerate}
    \item We have $f^{-1}(f(A))=\set{y\in S: f(y)\in f(A)}$, thus all $y\in A$ are in  $f^{-1}(f(A))$, since they are mapped to an element in $f(A)$.
    Suppose $f$ is one-to-one, then whenever $f(x)=f(y)$, we have $x=y$, so that each distinct pairs of elements in $A$ are mapped to distinct elements in $f(A)$, and thus the preimage of any pair of elements of $f(A)$ does not map to a single element;
    in other words, $f^{-1}(f(A))=A$.
    
    \item We have $f(f^{-1}(C))=\set{f(x)\in T: x\in f^{-1}(C)}$.
    Since $f^{-1}(C)=\set{x\in S: f(x)\in C}$, then the elements of $f^{-1}(C)$ will be mapped to elements in $C$ by $f$, as required.
    Now suppose $f$ is onto.
    Then, for all $y\in C$, there is an $x\in S$ with $y=f(x)$.
    Thus, $x\in f^{-1}(C)$, and $y\in f(f^{-1}(C))$, as required.
    
    \item ($\subseteq$)
    Let $y\in f(A\cup B)$, then there exists an $x\in A$ or $x\in B$ so that $y=f(x)$.
    But this implies that $y=f(x)\in f(A)$ or $y=f(x)\in f(B)$;
    that is $f(A\cup B)\subseteq f(A)\cup f(B)$.

    ($\supseteq$)
    Let $y\in f(A)\cup f(B)$.
    Then $y\in f(A)$ or $y\in f(B)$.
    As above, this implies there is a $x\in A$ or a $x\in B$ with $y=f(x)$ which in turn implies $y=f(x)\in f(A\cup B)$;
    that is $f(A)\cup f(B)\subseteq f(A\cup B)$
    
    \item ($\subseteq$)
    Suppose $x\in f^{-1}(C\cup D)$, thus, $f(x)\in C$ or $f(x)\in\cup D$.
    But this implies $x\in f^{-1}(C)$ or $x\in f^{-1}(D)$;
    that is, $x\in f^{-1}(C)\cup f^{-1}(D)$.

    ($\supseteq$) 
    Suppose $x\in f^{-1}(C)\cup f^{-1}(D)$.
    Then, $f(x)\in C$ or $f(x)\in D$.
    But this implies $x\in f^{-1}(C\cup D)$.
    
    \item ($\subseteq$)
    Suppose $f(x)\in f(A\cap B)$.
    Then there is a $y$ so that $y\in A$ and $y\in B$, and $f(x)=f(y)$ therefore $f(x)=f(y)\in f(A)$ and $f(x)=f(y)\in f(B)$.
    
    ($\supseteq$, with one-to-one)
    Now suppose $f$ is one-to-one and let $f(x)\in f(A)\cap f(B)$.
    Then there exists $y\in A\cap B$ such that $f(x)=f(y)$.
    To see this, suppose by contradiction this wasn't the case;
    thus there is a $z\neq x$ so that $f(x)=f(z)$.
    But this contradicts $f$ being one-to-one.
    Hence, $f(x)=f(y)\in f(A\cap B)$.
    
    \item ($\subseteq$) 
    Suppose $x\in f^{-1}(C\cap D)$.
    Thus $f(x)\in C$ and $f(x)\in D$;
    that is $x\in f^{-1}(C)\cap f^{-1}(D)$.

    ($\supseteq$)
    Suppose $x\in f^{-1}(C)\cap f^{-1}(D)$.
    Then $f(x)\in C$ and $f(x)\in D$;
    that is, $f(x)\in C\cap D$ which implies $x\in f^{-1}(C\cap D)$.
    
    \item Suppose $f(x)\in f(A)\setminus f(B)$.
    Then $f(x)\in f(A)$ but $f(x)\notin f(B)$.
    That is, $x\in A$ but $x\notin B$ and thus $f(x)\in f(A\setminus B)$.
    
    \item ($\subseteq$)
    Suppose $x\in f^{-1}(C\setminus D)$.
    Then $f(x)\in C\setminus D$, so $f(x)\in C$ but $f(x)\notin D$;
    that is, $x\in f^{-1}(C)$ and $x\notin f^{-1}(D)$, which is the same as $x\in f^{-1}(C)\setminus f^{-1}(D)$.

    ($\supseteq$)
    Suppose $x\in f^{-1}(C)\setminus f^{-1}(D)$.
    Then $f(x)\in C$ but $f(x)\notin D$;
    that is $f(x)\in C\setminus D$ which implies $x\in f^{-1}(C\setminus D)$.
\end{enumerate}
\end{proof} 

\begin{exercise}{2}
Given a subset $A$ of some ``universal'' set $S$, we define $\chi_A:S\to\R$, the characteristic function of $A$, by $\chi_A(x)=1$ if $x\in A$ and $\chi_A(x)=0$ if $x\notin A$.
Prove of disprove the following formulas:
\begin{enumerate}
    \item $\chi_{A\cup B}=\chi_A +\chi_B$,
    \item $\chi_{A\cap B}=\chi_A \chi_B$,
    \item $\chi_{A\setminus B}=\chi_A-\chi_B$.
\end{enumerate}
What corrections are necessary?
\end{exercise}
\begin{proof}
\begin{enumerate}
    \item This is not correct.
    Consider the sets $A=B=[0,1]$.
    Then $\chi_{A\cup B}(1)=1$ and $\chi_A(1)+\chi_B(1)=2$.

    The way to make this a correct statement is "correcting" by the intersection.
    That is, $\chi_{A\cup B}=\chi_A+\chi_B-\chi_{A\cap B}$.
    To see this is the correct formula, let $x\in A\cup B$, then there are two options:
    either $x\in A$ and $x\notin B$, in which case $\chi_{A\cup B}(x)=\chi_A(x)=1$ (we can reason analogously for $x\notin A$ but $x\in B$);
    or, $x\in A$ and $x\in B$, in which case $\chi_{A\cup B}(x)=\chi_A(x)+\chi_B(x)-\chi_{A\cap B}(x)=1$.
    \item This is correct.
    To see this, consider $x\in S$.
    If $x\in A$ but $x\notin B$, then $\chi_{A\cap B}(x)=\chi_A(x)\chi_B(x)=0$ (we can reason analogously for the case when $x\notin A$ but $x\in B$).
    If $x\in A$ and $x\in B$, then $\chi_{A\cap B}(x)=\chi_A(x)\chi_B(x)=1$.
    \item This is not correct.
    The correct version is $\chi_{A\setminus B}= \chi_{A}-\chi_{B}+\chi_{B\setminus A}$.
    To see this consider the following cases:
    i) $x\notin A,B$, then both sides of the equality are 0. 
    ii) $x\in A,B$, then the left hand side of the equation is 0, and on the right hand side we have 1-1+0=0.
    iii) $x\in A$ but $x\notin B$, then on the left hand side of the equation we have 1 and on the right hand side we have 1-0+0=1.
    iv) Finally, $x\in B$ but $x\notin A$, then on the left hand side we have 0 and on the right hand side we have 0-1+1=0, as required.
\end{enumerate}
\end{proof} 

\begin{exercise}{5}
Is there a continuous characteristic function on $\R$?
If $A\subseteq\R$, show that $\chi_A$ is continuous at each point of $\text{int}(A)$.
Are there any other points of continuity?
\end{exercise}
\begin{proof}
(Continuous characteristic function on $\R$)
Yes, the trivial one $\chi_\R$.

($\chi_A$ is continuous on $\text{int}(A)$)
$\text{int}(A)$ is defined to be the largest open set contained in $A$, so that every element in $a\in\text{int}(A)$ has an open ball around it, say $B_\epsilon(a)$ contained in $\text{int}(A)$.
Thus $\chi_A$ is 1 for all elements in $B_\epsilon(a)$ and so for any sequence $(a_n)$ and $a_n\to a$, we have that $\chi_A(a_n)=1$, so that $\chi_A$ is continuous in $\text{int}(A)$.

(Other points of continuity)
Not generally.
Consider $A=[0,1]\subseteq\R$ and $\chi_A$.
Then $\chi_A$ is continuous in $\text{int}(A)=(0,1)$, but certainly not continuous at 0 or 1.
To see this consider the sequence $a_n=(-1)^n(1/n)\to 0$.
\end{proof} 

\begin{exercise}{6}
Let $f:\R\to\R$ be continuous.
Show that $\set{x:f(x)>0}$ is an open subset of $\R$ and that $\set{x:f(x)=0}$ is a closed subset of $\R$.
If $f(x)=0$ whenever $x$ is rational, show that $f(x)=0$ for every real $x$.
\end{exercise}
\begin{proof}
($\set{x:f(x)>0}$ is open)
We have that $(0,\infty)\subset\R$ is open.
The preimage of an open set of a continuous function is continuous.

($\set{x:f(x)=0}$ is closed)
Consider a sequence $(x_n)$ in $X=\set{x:f(x)=0}$ with $x_n\to x$.
Then $f(x_n)=0$ for all $n$, but since $f$ is continuous, we also have that $0=\lim_nf(x_n) =f(\lim x_n)=f(x)$, so that $x\in X$ and thus $X$ is closed.

(If $f(x)=0$ for $x\in\Q$ then $f(x)=0$ for all $x$)
Suppose, for the sake of contradiction, there exists $x\in\R\setminus\Q$ with $f(x)=1$.
Consider any sequence of rationals $(x_n)$ with $x_n\to x$ whose existence is guaranteed by the density of the rationals in the reals.
For all $n$ we have that $f(x_n)=0$.
From the continuity of $f$ we have $0 =\lim_n f(x_n) =f(\lim_n x_n) =f(x) =1$, which is a contradiction.
Thus $f(x)$ must be 0 too.
\end{proof} 

\begin{exercise}{8}
Let $f:\R\to\R$ be continuous.
\begin{enumerate}
    \item If $f(0)>0$, show that $f(x)>0$ for all $x$ in some interval $(-a,a)$.
    \item If $f(x)\geq 0$ for every rational $x$, show that $f(x)\geq 0$ for all real $x$.
    Will this results hold with $\geq0$ replaced by $>0$? 
    Explain
\end{enumerate}
\end{exercise}
\begin{proof}
\begin{enumerate}
    \item Suppose, for the sake of contradiction, this is not the case.
    Then for all $n$, the interval $(-1/n, 1/n)$ contains an $x_n$ with $f(x_n)\leq 0$.
    So let $(x_n)$ be the sequence composed of the $x_n$, and notice that $x_n \to 0$.
    By the order limit Theorem, we have that $\lim_n f(x_n) \leq 0$.
    By the continuity of $f$ we have that $\lim_n f(x_n) = f(\lim_n x_n) = f(x) = f(0) > 0$,  which contradicts our previous statement.
    \item Yes.
    We will prove this by contradiction.
    Suppose there exists an irrational $x$ with $f(x)<0$.
    Now consider a sequence of rationals $(x_n)$ with $x_n\to x$, whose existence is guaranteed by the density of the rationals in the reals.
    We have that for all $n$, $f(x_n)\geq 0$ and so by the order limit Theorem, $\lim_n f(x_n)\geq 0$.
    However, since $f$ is continuous, $\lim_n f(x_n) = f(\lim_n x_n) = f(x) < 0$, which is a contradiction.

    If $\geq 0$ is replaced by $>0$ the result does not follow. 
    A simple function where this does not happen is $f(x) = \absoluteValue{x-\pi}$, which maps every rational to something greater than 0 but $\pi$ to 0 and yet is continuous.
\end{enumerate}
\end{proof} 

\begin{exercise}{9}
Let $A\subseteq M$.
Show that $f:(A,d)\to (N,\rho)$ is continuous at $a\in A$ if and only if, given $\epsilon>0$, there is a $\delta>0$ such that $\rho(f(x),f(a))<\epsilon$ whenever $d(x,a)<\delta$ and $x\in A$.
We paraphrase that statement by saying that ``$f$ has a point of continuity relative to $A$''.
\end{exercise}
\begin{proof}
This is precisely the definition of continuity.
The only difference here is that $A$ is a subset of $M$ where the metric space is defined.
However, as we have seen in example 3.1.c any subset of a metric space is itself a metric space, so that there is no essential difference when defining a function on the whole metric space or a subspace thereof.
\end{proof} 

\begin{exercise}{10}
Let $A=(0,1]\cup \set{2}$, considered as a subset of $\R$.
Show that every function $f:A\to\R$ is continuous relative to $A$, at 2.
\end{exercise}
\begin{proof}
The statement of continuity relative to $A$ holds trivially at 2 for any $f:A\to\R$ since for any $0<\delta<1$ there exists no element of $A$ with $\absoluteValue{x-a}<1$.
\end{proof} 

\begin{exercise}{14}
A continuous function on $\R$ is completely determined by its values on $\Q$.
Use this to ``count'' the continuous functions $f:\R\to\R$.
\end{exercise}
\begin{proof}
To see this, consider any continuous function $f:\R^\R$, and consider its restriction to $\Q$: $f|_\Q = g$.
For the sake of contradiction, suppose there is another continuous function $f'$, with $f'|_\Q = g$ and for some $x \in \R\setminus\Q$, $f(x)\neq f'(x)$.
Let $(x_n)$ be a sequence of rational numbers $(x_n)$ with $x_n \to x$, it holds that $f(x_n)=f'(x_n)$ for all $n$.
Taking the limits on both sides of the equality gives us 
\begin{align*}
    f(x) = f(\lim_n x_n) = \lim_n f(x_n) =
    \lim_n f'(x_n) =  f'(\lim_n x_n) = f'(x)  
\end{align*}
which results in a contradiction.

We just found that the map that restricts a continuous function to the rationals is injective from $\R^\R$ to $\R^\Q=\R$.
We will now use a Cantor-Schr\"oder-Bernstein argument, and find an injective map from $\R$ to the set of continuous functions to show that the cardinality of the continuous functions in $\R$ is equal to the cardinality of $\R$.
This map is simple.
Consider $a \mapsto f(x)=a$ for any $a \in\R$.
The constant function is continuous and, moreover, it is different for all $a\in\R$.
\end{proof} 

\begin{exercise}{17}
Let $f,g:(M,d)\to(N,\rho)$ be continuous, and let $D$ be a dense subset of $M$.
If $f(x)=g(x)$ for all $x\in D$, show that $f(x)=g(x)$ for all $x\in M$.
If $f$ is onto, show that $f(D)$ is dense in $N$.
\end{exercise}
\begin{proof}
($f(x) =g(x)$ for all $x\in M$)
Since $D$ is dense in $M$, then $\bar{D} =M$.
So take $x\in M\setminus D$, and take a sequence $(x_n)$ in $D$ with $x_n\to x$.
We have $f(x_n)=g(x_n)$ for all $n$.
Since $f$ and $g$ are continuous, we also have that $\lim_n f(x_n) = f(\lim_n x_n) = f(x)$, and furthermore the limit of sequences is unique;
hence $(f(x_n))$ and $(g(x_n))$ converge to the same limit $f(x)=g(x)$.

(Density of $f(D)$)
Suppose in addition $f$ is onto and consider $f(x)\in N\setminus f(D)$, so that $x\notin D$.
Since $D$ is dense in $M$, then there exists a sequence $(x_n)$ with $x_n\to x$.
As above, $f$ is continuous so we know that $\lim_n f(x_n) = f(\lim_n x_n) = f(x)$, and thus $f(D)$ is dense in $N$, as $(f(x_n))$ is fully contained in $f(D)$.
\end{proof} 

\begin{exercise}{19}
A function $f:\R\to\R$ is said to satisfy a Lipschitz condition if there is a constant $K<\infty$ such that $\absoluteValue{f(x)-f(y)} <K\absoluteValue{x-y}$ for all $x,y\in\R$.
More economically, we say that $f$ is Lipschitz (or Lipschitz with constant $K$ if a particular constant seems to matter).
Show that $\sin x$ is Lipschitz with constant $K=1$.
Prove that a Lipschitz function is (uniformly) continuous.
\end{exercise}
\begin{proof}
($\sin x$ is Lipschitz with $K=1$)
Let $x,y\in\R$ and assume $x<y$.
By the mean value Theorem, there exists $z\in(x,y)$ such that
\begin{align*}
    \sin'(z)
    = \cos(z)
    = \frac{\sin(x)-\sin(y)}{x-y}.
\end{align*}
Take absolute value in both sides of the equality, and notice that $\cos(z)\leq 1$ for all $z$.
Then we have that
$1\geq\absoluteValue{\sin(x)-\sin(y)}/\absoluteValue{x-y}$, so that
$\absoluteValue{x-y}>\absoluteValue{\sin(x)-\sin(y)}$, as required.

(Lipschitz is uniformly continuous)
This is a Corollary to exercise 25, noticing that the choice of $\delta$ was not dependent on the point of continuity.
\end{proof} 

\begin{exercise}{20}
If $d$ is a metric on $M$, show that $\absoluteValue{d(x,z)-d(y,z)} \leq d(x,y)$ and conclude that the function $f(x)=d(x,z)$ is continuous on $M$ for any fixed $z\in M$.
This says that $d(x,y)$ is separately continuous;
that is, continuous on each variable separately.
\end{exercise}
\begin{proof}
By the triangle inequality: 
$d(x,z) \leq d(x,y) + d(y,z)$, so that 
$d(x,z) - d(y,z) \leq d(x,y)$.
Likewise,
$d(y,z) \leq d(y,x) + d(x,z)$, and
$d(y,z) - d(x,z) = -(d(x,z) - d(y,z)) \leq d(x,y)$,
giving us the desired result.

To see $f(x)$ is continuous, fix $\epsilon>0$ and choose $\delta=\epsilon$.
We then obtain $\absoluteValue{d(x,z)-d(y,z)} \leq d(x,y) < \delta = \epsilon$, as required.
\end{proof} 

\begin{exercise}{22}
Define $E:\N\to l_1$ by $E(n)=(1,\dots,1,0,\dots)$ where the first $n$ entries are 1 and the rest are 0.
Show that $E$ is an isometry (into).
\end{exercise}
\begin{proof}
Let $n,m\in\N$ and assume, without loss of generality, that $n\geq m$, we then have that $E(n)-E(m)=(0,\dots,0,1\dots,1,0,\dots)$, where the first $m$ and every element after the $n+1$-th element are 0.
Looking at the distances between both elements:
$\sum\absoluteValue{E(n)-E(m)} =n-m$, the right hand side being equal to the distance in $\N$, so that $E(n)$ is an isometry.
\end{proof} 

\begin{exercise}{24}
Let $V$ be a normed vector space.
If $y \in V$ is fixed, show that the maps $\alpha\to\alpha y$, from $\R$ into $V$, and $x\to x+y$, from $V$ into $V$, are continuous.
\end{exercise}
\begin{proof}
($\alpha\to\alpha y$)
Let $a,b\in \R$.
We have $\norm{ay -by} =\absoluteValue{a-b}\norm{y}$.
Thus, by choosing $\delta\norm{y}<\epsilon$ we obtain the continuity of the map.

($x\to x+y$)
Let $x,z\in V$.
We have $\norm{x+y -z+y}=\norm{x-z}$.
Thus, choosing $\delta=\epsilon$ we get the result.
\end{proof} 

\begin{exercise}{25}
A function $f:(M,d)\to(N,\rho)$ is called Lipschitz if there is a constant $K<\infty$ such $\rho(f(x),f(y))\leq Kd(x,y)$ for all $x,y\in M$.
Prove that a Lipschitz mapping is continuous
\end{exercise}
\begin{proof}
Let $\epsilon>0$.
Since $f$ is Lipschitz, then we have that for some $K<\infty$, $\rho(f(x),f(y)) \leq Kd(x,y)$ holds.
Thus, choose $\epsilon>K\delta$, so that $\rho(f(x),f(y)) \leq Kd(x,y) < K\delta < \epsilon$, giving us the continuitiy of $f$.
\end{proof} 

\begin{exercise}{26}
Provide the answer to a question raised in chapter 3 by showing that integration is continuous.
Specifically, show that the map $L(f)=\int_a^b f(t) dt$ is Lipschitz with constant $K =b-a$ for all $f\in C[a,b]$.
\end{exercise}
\begin{proof}
Let $f,g\in C[a,b]$.
We have 
\begin{align*}
    d(L(f),L(g))
    =& \absoluteValue{\int^b_a f(t)dt - \int^b_a g(t)dt}\\
    =& \absoluteValue{\int^b_a f(t) - g(t)dt}\\
    \leq& \absoluteValue{\int^b_a \sup_t \absoluteValue{f(t)-g(t)} dt}\\
    =& \absoluteValue{(b-a) \sup_t \absoluteValue{f(t)-g(t)}}\\
    =& (b-a) \sup_t \absoluteValue{f(t)-g(t)}
    = d_\infty(f,g),
\end{align*}
as required.
\end{proof} 

\begin{exercise}{27}
Fix $k\geq 1$ and define $f:l_\infty \to \R$ by $f(x) =x_k$.
Is $f$ continuous?
[Hint: $f$ is Lipschitz].
\end{exercise}
\begin{proof}
Let $(x_n),(y_n) \in l_\infty$.
For a fixed $k$, we have 
$d(f((x_n)),f((y_n))) = \absoluteValue{x_k-y_k} \leq \sup_k\absoluteValue{x_k-y_k} = d_\infty((x_n),(y_n))$.
The inequality following because any difference between two $k$-th elements of the sequence is less than or equal to the largest such difference.
Since $f$ is Lipschitz with $K=1$, $f$ is continuous.
\end{proof} 

\begin{exercise}{29}
Fix $y\in l_\infty$ and define $h: l_1\to l_1$ by $h(x)=(x_ny_n)^\infty_{n=1}$.
Show that $h$ is continuous.
\end{exercise}
\begin{proof}
Let $x,x'\in l_1$.
We have 
\begin{align*}
    d_1(h(x),h(x'))
    =& \sum_{i=1}^\infty \absoluteValue{x_iy_i-x'_iy_i}\\
    =& \sum_{i=1}^\infty \absoluteValue{y_i}\absoluteValue{x_i-x'_i}\\
    \leq& \sum_{i=1}^\infty \norm{y}_\infty\absoluteValue{x_i-x'_i}\\
    =& \norm{y}_\infty\sum_{i=1}^\infty \absoluteValue{x_i-x'_i}\\
    =& \norm{y}_\infty d_1(x,x'),
\end{align*}
so that $h$ is Lipschitz with $K=\norm{y}_\infty$, and hence it is continuous.
\end{proof} 

\begin{exercise}{30}
Let $f: (M,d) \to (N,\rho)$.
Prove that $f$ is continuous if and only if $f(\bar{A})\subseteq \overline{f(A)}$ for every $A \subseteq M$ if and only if $f^{-1}(B^\circ) \subseteq (f^{-1}(B))^\circ$ for every $B \subseteq N$.
Give an example of a continuous function such that $f(\bar{A}) \neq \overline{f(A)}$ for some $A \subseteq M$.
\end{exercise}
\begin{proof}
($f$ is continuous if and only if $f(\bar{A})\subseteq \overline{f(A)}$)

($\Rightarrow$)
Let $A$ be arbitrary and let $f(x)\in f(\bar{A})$.
If $x\in A$, then $f(x)\in\overline{f(A)}$, so assume $x\notin A$ and is a limit point of $A$.
Thus, there exists a sequence $(x_n)$ in $A$ so that $x_n\to x$, likewise, $f(x_n)$ is a sequence in $f(A)$.
Since $x$ is continuous, we have that $f(x_n)\to f(x)$, so that $f(x)$ is a limit point of $f(A)$, and $f(x)\in\overline{f(A)}$, as required.

($\Leftarrow$)
We will prove this by contrapositive.
Thus, suppose $f$ is not continuous, we want to prove that $f(\bar{A}) \not\subseteq \overline{f(A)}$.
Since $f$ is not continuous, then there exists an $x\in M$ such that for any sequence $(x_n)$ with $x_n\to x$ then $f(x_n)\not\to f(x') \neq f(x)$.
Let $A$ be an open ball around $x$ with $x$ (and potentially $x'$, if originally in that open ball) removed.
Thus, we have that $x\in\bar{A}$ and $f(x) \in f(\bar{A})$, however, $f(x)\notin \overline{f(A)}$.

($f$ is continuous if and only if $f^{-1}(B^\circ)\subseteq (f^{-1}(B))^{-1}$)

($\Rightarrow$)
Since $B^\circ\subseteq B$, then $f^{-1}(B^\circ) \subseteq f^{-1}(B)$.
From the continuity of $f$ and the fact that $B^\circ$ is open, then $f^{-1}(B^\circ)$ is open too.
Finally, $(f^{-1}(B))^\circ$ is the greatest open set contained in $f^{-1}(B)$ so that $f^{-1}(B^\circ)$ (which we said is open) must be a subset of this greatest open set, as required.

($\Leftarrow$)
Suppose $f^{-1}(B^\circ)\subseteq (f^{-1}(B))^\circ$.
Let $A$ be an open set, so that $A=A^\circ$.
Thus, by our assumption, $f^{-1}(A)\subseteq (f^{-1}(A))^\circ$.
However, $(f^{-1}(A))^\circ\subseteq f^{-1}(A)$, since the interior of a set is contained in the set itself. 
Putting the two containments together, we get that $f^{-1}(A) = (f^{-1}(A))^\circ$, so that $f^{-1}(A)$ is open and thus $f$ is continuous.

(A counterexample for $f(\bar{A}) \neq \overline{f(A)}$)
Let $f(x)=\arctan(x)$ and let $A=\R$.
Then $f(\bar{A})=f(A)=(-\pi/2,\pi/2)$, whereas $\overline{f(A)}=\overline{(-\pi/2,\pi/2)}=[-\pi/2,\pi/2]$.
\end{proof} 

\begin{exercise}{31}
Let $f: (M,d) \to (N,\rho)$.
\begin{enumerate}
    \item If $M =\bigcup^\infty_{n=1} U_n$ where each $U_n$ is an open set in $M$, and if $f$ is continuous on each $U_n$, show that $f$ is continuous on $M$.
    \item If $M =\bigcup^N_{n=1} E_n$ and if each $E_n$ is a closed set in $M$, and if $f$ is continuous on each $E_n$, show that $f$ is continuous on $M$.
    \item Give an example showing that $f$ can fail to be continuous on all of $M$ if, instead, we use a countably infinite union of closed sets $M = \bigcup^\infty_{n=1} E_n$ on 2.
\end{enumerate}
\end{exercise}
\begin{proof}
\begin{enumerate}
    \item We know $f|_{U_n}$ is continuous, so that for an open set $V\in M$.
    We have $f|_{U_n}^{-1}(V) = \set{x: f|_{U_n}(x)\in V} = f^{-1}(V)\cap U_n$ is open (since it is the intersection of open sets).
    Since $M=\bigcup_{n=1}^\infty U_n$, then $f(x)=\bigcup_{n=1}^\infty f|_{U_n}(x)$, thus for any open set in $V$, we have
    \begin{align*}
        f^{-1}(V) = \bigcup_{n=1}^\infty f^{-1}|_{U_n}(V) = \bigcup_{n=1}^\infty f^{-1}(V)\cap U_n.
    \end{align*}
    Since the last equality is the arbitrary union of open sets, it is open, so that $f$ is continuous.
    \item This proof is essentially the same as above.
    From Theorem 5.3 we know that the preimage of a closed set for a continuous function is closed;
    and from Theorem 4.4 we know that the finite union of closed sets is closed (by taking complements and applying deMorgan).
    With these two theorems and the technique as above we get the desired result.
    \item Let $M = \set{0} \cup \set{1/n: n \in N}$, which is a countable union of singletons, each of which is a closed set.
    Now consider the function $f(x)=0$ if $x\neq 0$ and $f(x)=1$ otherwise.
    In such case, $f$ is continuous on each $E_n$, but $f$ is not a continuous function, as any sequence $x_n\to 0$ will produce $1 =f(0) =f(\lim x_n) \neq \lim f(x_n) =0$.
\end{enumerate}
\end{proof} 

\begin{exercise}{34}
Show that $d$ is continuous in $M \times M$, where $M \times M$ is supplied with ``the'' product metric (see exercise 3.46).
This says that $d$ is jointly continuous, that is, continuous as a function of two variables.
[Hint: If $x_n \to x$ and $y_n \to y$, show that $d(x_n,y_n) \to d(x,y)$].
\end{exercise}
\begin{proof}
We will ignore the hint.
Fix $\epsilon>0$, and choose $\delta=\epsilon$.
Let $(x,y),(x',y')\in M\times M$ with the $d_1$ metric.
We have 
\begin{align*}
    \absoluteValue{d(x,y)-d(x',y')}
    \leq d(x,x')+d(y,y') 
    = d_1((x,y),(x',y')) < \delta = \epsilon.
\end{align*}
Were the first inequality follows from exercise 1.1.12 in Kreyszig.
\end{proof} 

\begin{exercise}{35}
Show that a set $U$ is open in $M$ if and only if $U=f^{-1}(V)$ for some continuous function $f: M \to \R$ and some open set $V$ in $\R$.
\end{exercise}
\begin{proof}
($\Rightarrow$)
Suppose $U$ is open.
We have to find a continuous function $f:M\to\R$ so that for some open set $V$, $f^{-1}(V)=U$.
Let $f(x) = d(x,M\setminus U) = \inf_{y\in M\setminus U}d(x,y)$.
We have that $f^{-1}(V) = \set{x\in M: \inf_{y\in M\setminus U}d(x,y)\in V}$.
Consider $V=(-\infty,0)\cup(0,\infty)$, which is open.
Since $M\setminus U$ is a closed set (because $U$ is open), then it contains all its limit points.
Thus, $f(x)=d(x,y)=0$ only if $x\in M\setminus U$ so that $f(x)\neq 0$ if $x\in U$ giving us the desired result.

($\Leftarrow$)
This is essentially the definition of a continuous function through open sets.
\end{proof} 

\begin{exercise}{36}
Suppose that we are given a point $x$ and a sequence $(x_n)$ in a metric space $M$, and suppose that $f(x_n) \to f(x)$ for every continuous, real-valued function $f$ on $M$.
Does it follow that $x_n \to x$ in $M$?
Explain.
\end{exercise}
\begin{proof}
Yes.
Consider the function $f(y)=d(x,y)$ which is continuous and real valued.
Thus for a given sequence $x_n\to x$, the assumption implies $f(x_n)\to f(x)$ or $d(x_n,x) \to d(x,x) =0$, giving us the convergence of the sequence.
\end{proof} 
