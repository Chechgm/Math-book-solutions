\subsection{Continuous functions}

1*
2*
5C
6C
8*
9*
10*
14*
17*
19*
20*
22C
24*
25*
26*
27C
29C
30*
31*

34*
35*
36*

\begin{exercise}{1}
Given a function $f:S\to T$ and sets $A,B\subseteq S$ and $C,D\subset T$ establish the following:
\begin{enumerate}
    \item $A \subseteq f^{-1}(f(A))$, with equality for all $A$ if and only if $A$ is one-to-one.
    \item $f(f^{-1}(C))\subseteq C$, with equality for all $C$ if and only if $f$ is onto.
    \item $f(A\cup B)= f(A)\cup f(B)$.
    \item $f^{-1}(C\cup D)=f^{-1}(C)\cup f^{-1}(D)$.
    \item $f(A\cap B)\subseteq f(A)\cap f(B)$, with equality for all $A$ and $B$ if and only if $f$ is one-to-one.
    \item $f^{-1}(C\cap D)=f^{-1}(C)\cap f^{-1}(D)$.
    \item $f(A)\setminus f(B)\subseteq f(A\setminus B)$.
    \item $f^{-1}(C\setminus D)= f^{-1}(C)\setminus f^{-1}(D)$.
\end{enumerate}
Generalise, whenever possible, to arbitrary unions and intersections.
\end{exercise}
\begin{proof}
\begin{enumerate}
    \item We have $f^{-1}(f(A))=\set{y\in S: f(y)\in f(A)}$, thus all $y\in f^{-1}(f(A))$ are in $A$.
    Suppose $f$ is one-to-one, then whenever $f(x)=f(y)$, we have $x=y$, so that each distinct pairs of elements in $A$ are mapped to distinct elements in $f(A)$, and thus the preimage of any pair of elements of $f(A)$ does not map to a single element;
    in other words, $f^{-1}(f(A))=A$.
    
    \item We have $f(f^{-1}(C))=\set{f(x)\in T: x\in f^{-1}(C)}$.
    Since $f^{-1}(C)=\set{x\in S: f(x)\in C}$, then the elements of $f^{-1}(C)$ will be mapped to elements in $C$ by $f$, as required.
    Now suppose $f$ is onto.
    Then, for all $y\in C$, there is an $x\in S$ with $y=f(x)$.
    Thus, $x\in f^{-1}(C)$, and $y\in f(f^{-1}(C))$, as required.
    
    \item ($\subseteq$)
    Let $f(x)\in f(A\cup B)$, then $x\in A$ or $x\in B$.
    But this implies that $f(x)\in f(A)\cup f(B)$;
    that is $f(A\cup B)\subseteq f(A)\cup f(B)$.

    ($\supseteq$)
    Let $f(x)\in f(A)\cup f(B)$.
    Then $f(x)\in f(A)$ or $f(x)\in f(B)$.
    But this implies $x\in A$ or $x\in B$, which in turn implies $f(x)\in f(A\cup B)$;
    that is $f(A)\cup f(B)\subseteq f(A\cup B)$
    
    \item
    \item 
    \item 
    \item 
    \item
\end{enumerate}
\end{proof} 

\begin{exercise}{2}
Given a subset $A$ of some ``universal'' set $S$, we define $\chi_A:S\to\R$, the characteristic function of $A$, by $\chi_A(x)=1$ if $x\in A$ and $\chi_A(x)=0$ if $x\notin A$.
Prove of disprove the following formulas:
\begin{enumerate}
    \item $\chi_{A\cup B}=\chi_A +\chi_B$,
    \item $\chi_{A\cap B}=\chi_A \chi_B$,
    \item $\chi_{A\setminus B}=\chi_A-\chi_B$.
\end{enumerate}
What corrections are necessary?
\end{exercise}
\begin{proof}
fill
\end{proof} 

\begin{exercise}{5}
Is there a continuous characteristic function on $\R$?
If $A\subseteq\R$, show that $\chi_A$ is continuous at each point of $\text{int}(A)$.
Are there any other points of continuity?
\end{exercise}
\begin{proof}
fill
\end{proof} 

\begin{exercise}{6}
Let $f:\R\to\R$ be continuous.
Show that $\set{x:f(x)>0}$ is an open subset of $\R$ and that $\set{x:f(x)=0}$ is a closed subset of $\R$.
If $f(x)=0$ whenever $x$ is rational, show that $f(x)=0$ for every real $x$.
\end{exercise}
\begin{proof}
fill
\end{proof} 

\begin{exercise}{8}
Let $f:\R\to\R$ be continuous.
\begin{enumerate}
    \item If $f(0)>0$, show that $f(x)>0$ for all $x$ in some interval $(-a,a)$.
    \item If $f(x)\geq 0$ for every rational $x$, show that $f(x)\geq 0$ for all real $x$.
    Will this results hold with $\geq0$ replaced by $>0$? 
    Explain
\end{enumerate}
\end{exercise}
\begin{proof}
fill
\end{proof} 

\begin{exercise}{9}
fill
\end{exercise}
\begin{proof}
fill
\end{proof} 

\begin{exercise}{10}
fill
\end{exercise}
\begin{proof}
fill
\end{proof} 

\begin{exercise}{14}
fill
\end{exercise}
\begin{proof}
fill
\end{proof} 

\begin{exercise}{17}
fill
\end{exercise}
\begin{proof}
fill
\end{proof} 

\begin{exercise}{19}
fill
\end{exercise}
\begin{proof}
fill
\end{proof} 

\begin{exercise}{20}
fill
\end{exercise}
\begin{proof}
fill
\end{proof} 

\begin{exercise}{22}
fill
\end{exercise}
\begin{proof}
fill
\end{proof} 

\begin{exercise}{24}
fill
\end{exercise}
\begin{proof}
fill
\end{proof} 

\begin{exercise}{25}
fill
\end{exercise}
\begin{proof}
fill
\end{proof} 

\begin{exercise}{26}
fill
\end{exercise}
\begin{proof}
fill
\end{proof} 

\begin{exercise}{27}
fill
\end{exercise}
\begin{proof}
fill
\end{proof} 

\begin{exercise}{29}
fill
\end{exercise}
\begin{proof}
fill
\end{proof} 

\begin{exercise}{30}
fill
\end{exercise}
\begin{proof}
fill
\end{proof} 

\begin{exercise}{31}
fill
\end{exercise}
\begin{proof}
fill
\end{proof} 

\begin{exercise}{34}
fill
\end{exercise}
\begin{proof}
fill
\end{proof} 

\begin{exercise}{35}
fill
\end{exercise}
\begin{proof}
fill
\end{proof} 

\begin{exercise}{36}
fill
\end{exercise}
\begin{proof}
fill
\end{proof} 
