\section{The space of continuous functions}


\begin{exercise}{63}
Let $[a,b]$ be any closed, bounded interval in $\R$, and let $\sigma:[0,1]\to[a,b]$ be defined by $\sigma(t)= a + t(b-a)$.
Prove that:
\begin{enumerate}
    \item $\sigma$ is a homeomorphism.
    \item $f\in C[a,b]$ if and only if $f\circ \sigma\in C[0,1]$.
    \item The map $f\mapsto f \circ \sigma$ is an isometry from $C[a,b]$ onto $C[0,1]$.
    The map $T(f) = f \circ \sigma$ actually does much more;
    it is both an algebra and a lattice isomorphism.
    That is, it also preserves the algebraic and the order structures.
    Specifically, given any $f,g\in C[a,b]$, check that:
    \item $T(\alpha f + \beta g) = \alpha T(f) + \beta T(g)$, for all $\alpha,\beta\in\R$.
    \item $T(fg)=T(f)T(g)$.
    \item $T(f) \leq T(g)$ if and only if $f\leq g$.
\end{enumerate}
Thus for all practical purposes, $C[a,b]$ and $C[0,1]$ are identical.
\end{exercise}
\begin{proof}
\begin{enumerate}
    \item Let $\sigma^{-1}(x) = (x-a)/(b-a)$.
    We have that $\sigma \circ \sigma^{-1} (x) = a + [(x-a)/(b-a)](b-a) = a + x - a = x$, and likewise $\sigma^{-1}\circ\sigma(x) = x$, so that $\sigma^{-1}$ is the inverse of $\sigma$ and thus it is a bijection.
    Furthermore, a simple $\epsilon-\delta$ argument gives us that both $\sigma$ and $\sigma^{-1}$ are continuous, so that $\sigma$ is a homeomorphism.
    \item 
    ($\Rightarrow$)
    If $f:[a,b]\to\R$ is continuous, then $f\circ \sigma:[0,1]\to\R$ is continuous because the composition of continuous functions is continuous.

    ($\Leftarrow$)
    Suppose $f\circ\sigma \in C[0,1]$ is continuous.
    We have $(f\circ\sigma)\circ\sigma^{-1}=f$ is continuous because it is the composition of two continuous functions.

    \item 
    For the next 4 points, let $f,g\in C[a,b]$.
    We have 
    \begin{align*}
    \sup_{x\in[0,1]}\absoluteValue{T(f)(x)-T(g)(x)} 
    =& \sup_{x\in[0,1]}\absoluteValue{f\circ\sigma(x) - g\circ\sigma(x)}\\
    =& \sup_{x\in[0,1]}\absoluteValue{(f-g)\circ\sigma(x)}\\
    =& \sup_{\substack{y=\sigma(x);\,\\ x\in[0,1]}}\absoluteValue{(f-g)(y)}\\
    =& \sup_{y\in[a,b]}\absoluteValue{f(y)-g(y)},
    \end{align*}
    so that $T$ is an isometry.
    
    \item 
    Let  $f,g \in C[a,b]$ and $h(x) = \alpha f(x) + \beta g(x)$, then,
    \begin{align*}
        T(\alpha f + \beta g)(x)
        =& T(h)(x)\\
        =& (h\circ\sigma)(x)\\
        =& h(\sigma(x))\\
        =& \alpha f(\sigma(x)) + \beta g(\sigma(x))\\
        =& \alpha(f\circ\sigma)(x) 
            + \beta (g\circ\sigma)(x)\\
        =& \alpha T(f) + \beta T(g),
    \end{align*}
    as required.
    
    \item
    Now let $h(x) = f(x)g(x)$, we then have 
    \begin{align*}
        T(fg)(x)
        =& T(h)(x)\\
        =& (h\circ \sigma)(x)\\
        =& h(\sigma(x))\\
        =& f(\sigma(x))g(\sigma(x))\\
        =& (f\circ\sigma)(x)(g\circ\sigma)(x)\\
        =& T(f)T(g)(x),
    \end{align*}
    as required.
    
    \item $T(f) \leq T(g)$ if and only if $f\leq g$.
    We have
    \begin{align*}
        T(f)(x) \leq& T(g)(x) &&\iff\\
        (f \circ \sigma)(x) \leq& (g \circ \sigma)(x) &&\iff\\
        (f-g) \circ \sigma(x) \leq& 0 &&\iff\\
        (f-g) \leq& 0 &&\iff\\
        f \leq& g
    \end{align*}
    Where the second to last line follows from the fact that the inequality most hold for all $y\in[a,b]$, and so it must also be the case for all $\sigma(x)\in[a,b]$ and $x\in[0,1]$.
\end{enumerate}
\end{proof} 
