\section{Limits and continuity}


\begin{exercise}{40}
Prove Theorem 1.17. 

Theorem 1.17. Let $f$ be a real valued function defined in some punctured neighborhood of $a\in\R$. Then the following are equivalent:
\begin{enumerate}
    \item There exists a number $L$ such that $\lim_{x\to a}f(x)=L$ (by the $\varepsilon-\delta$ definition).
    \item There exists a number $L$ such that $f(x_n)\to L$ whenever $x_n\to a$, where $x_n\neq a$ for all $n$.
    \item $(f(x_n))$ converges (to something) whenever $x_n\to a$, where $x_n\neq a$ for all $n$.
\end{enumerate}
\end{exercise}
\begin{proof}
($1\Rightarrow 3$) Suppose there exists a number $L$ such that $\lim_{x\to a}f(x) =L$ (by the $\varepsilon-\delta$ definition). Let $\varepsilon>0$ and let $x_n\to a$. by the $\varepsilon-\delta$ definition, there exists a $\delta$ such that if $\absoluteValue{x_n-a}<\delta$, then it is the case that $\absoluteValue{f(x_n)-L}<\varepsilon$. Furthermore, by the definition of $\lim_{n\to\infty}x_n=a$, we know there exists an $N$ so that if $n\geq N$, it must be the case that $\absoluteValue{x_n-a}<\delta$. Hence, $(f(x_n))\to L$, as required.

($3\Rightarrow 2$) Suppose $(f(x_n))$ converges (to something) whenever $x_n\to a$. Then let $x_n\to a$ and $y_n\to a$, and consider the sequence given by $x_1,y_1,\dots$ (this is called interleaving), which converges to $a$. We have that $f(x_1),f(y_1),f(x_2),\dots$ must converge to something by assumption. Hence, $(f(x_n))$ and $f(y_n))$ must converge to the same number. If we call this number $L$, we get the desired result.

($2\Rightarrow 1$) Suppose, for the sake of contradiction, that $f$ does not converge to $L$ under the $\varepsilon-\delta$ definition but there exists a number $L$ such that $f(x_n)\to L$ whenever $x_n\to a$. If we say that $f(x)$ does not converge under the $\varepsilon-\delta$ definition, then we say that there exists a $\varepsilon>0$ so that for all $\delta$, it is the case that $\absoluteValue{x-a}<\delta$ and $\absoluteValue{f(x)-L}\geq L$. Let $x_n\to a$, and choose $\delta =1/n$. Then we can find an $N$ such that whenever $n>N$ it holds that $\absoluteValue{x_n-a}<1/n$ but putting it together with the non-convergence of $f$ we get $\absoluteValue{f(x_n)-L}\geq \varepsilon$, so that $f(x_n)\not\to L$ as $x_n\to a$, giving us a contradiction.
\end{proof} 

\begin{exercise}{41}
Prove Theorem 1.18, including 1.18(iv) as one of the equivalent conditions

Theorem 1.18. Let $f$ be a real-valued function defined in some neighborhood of $a\in\R$. Then, the following are equivalent:
\begin{enumerate}
    \item $f$ is continuous at $a$ (by the $\varepsilon-\delta$ definition).
    \item $f(x_n)\to f(a)$ whenever $x_n\to a$.
    \item $(f(x_n))$ converges (to something) whenever $x_n\to a$.
    \item $f(a-)$ and $f(a+)$ both exist, and both are equal to $f(a).$
\end{enumerate}
\end{exercise}
\begin{proof}
($1\Rightarrow 4$) Suppose $f$ is continuous at $a$, by the $\varepsilon-\delta$ definition. Then for all $\varepsilon>0$, there exists a $\delta >0$ such that if $\absoluteValue{x-a}\leq\delta$, then it is the case that $\absoluteValue{f(x)-f(a)}<\varepsilon$. Fix $\varepsilon>0$. If $x\to a+$, we can find $\delta>0$ so that $a-x\leq\delta$, so that $\absoluteValue{x-a}\leq\delta$ and $\absoluteValue{f(x)-f(a)}<\varepsilon$ holds (from the definition of continuity). Hence, $f(x+)=\lim_{x\to a+}f(x)=f(a)$. The argument works mutatis mutandis for $f(a-)$, giving us the desired result.

($4\Rightarrow 3$) We will first prove that if $\lim_{x\to a-}f(x) =\lim_{x\to a+}f(x) =f(a)$, then $\lim_{x\to a}f(x)= f(a)$. To see this, notice that $\lim_{x\to a-}f(x)=f(a)$ means that for every $\varepsilon>0$, there exists a $\delta'>0$ so that whenever $a-x<\delta'$ we have that $\absoluteValue{f(x)-f(a)}<\varepsilon$. Likewise, because $\lim_{x\to a+}f(x)=f(a)$, there exists $\delta''>0$ such that whenever $a-x<\delta''$ it holds that $\absoluteValue{f(x)-f(a)}<\varepsilon$. Take $\delta =\min\set{\delta',\delta''}$, then we have that whenever $\absoluteValue{x-a}<\delta$ it holds that $\absoluteValue{f(x)-f(a)}<\varepsilon$; that is, $\lim_{x\to a}f(x)=f(a)$.

Now we prove the main result. Suppose $f(a-)=f(a+)=f(a)$. Fix $\varepsilon>0$ and let $x_n\to a$. Given that $\lim_{x\to a}f(x)=f(a)$, then for all $\varepsilon>0$, there exists a $\delta>0$ so that whenever $\absoluteValue{x-a}<\delta$ it also holds that $\absoluteValue{f(x)-f(a)}<\varepsilon$. Since $x_n\to a$, then for all $\delta>0$, there exists an $N$ such that whenever $n> N$, it is the case that $\absoluteValue{x_n-a}<\delta$. Since the existence of such $N$ is guaranteed by the convergence of $x_n$, we get that $(f(x_n))\to L$, whenever $x_n\to a$, as required.

($3\Rightarrow 2$) As in exercise 40.

($2\Rightarrow 1$) As in exercise 40.
\end{proof}

\begin{exercise}{45}
Let $f:[a,b]\to\R$ be continuous and suppose that $f(x)=0$ whenever $x$ is rational. Show that $f(x)=0$ for every $x\in[a,b]$.
\end{exercise}
\begin{proof}
For the sake of contradiction, suppose $f$ is continuous and $f(x)=0$ for all $x\in\Q$, but there exists $x'\in\R\setminus\Q$ such that $f(x')\neq 0$. By continuity, we have that for all sequences $(x_n)$ with $x_n\to x'$, it must be the case that $f(x_n)\to f(x')$. Consider the sequence $(x_n)\to x'$ composed of $x_n\in\Q$ for all $n$ (whose existence is guaranteed by Theorem 1.3, for example), then $f(x_n)=0$ for all $n$ and hence the sequence does not converge to $f(x')\neq 0$, as required.
\end{proof}

\begin{exercise}{46}
Let $f:\R\to\R$ be continuous.
\begin{enumerate}
    \item If $f(0)>0$, show that $f(x)>0$ for all $x$ in some open interval $(-a,a)$.
    \item If $f(x)\geq 0$ for every rational $x$, show that $f(x)\geq 0$ for all real $x$. Will this result hold with `$\geq 0$' replaced by `$>0$'? Explain.
\end{enumerate}
\end{exercise}
\begin{proof}
\begin{enumerate}
    \item For the sake of contradiction, suppose that for all open intervals $(-a,a)$ there exists an $x$ such that $f(x)\leq 0$. Consider the following sequence. For each $n$, consider the intervals given by $(-1/n,1/n)$ and choose $x_n$ to be the one with $f(x_n)\leq 0$. Certainly $(x_n)\to 0$, however, $f$ is not continuous as $f(0)=\varepsilon$ is enough to break the limit $f(x_n)\to f(0)$, so that by Theorem 1.18 $f$ is not continuous, giving us a contradiction. 
    \item Suppose, $f$ is continuous and $f(x)\geq 0$ for every rational $x$. Furthermore, for the sake of contradiction, suppose there exists an irrational number, $y$ such that $f(y)<0$.

    By Theorem 1.18, we must have that $f(y_n)\to f(y)$ whenever $y_n\to y$. Consider a sequence $y_n$ where all $y_n$ are rational but $y_n$ gets arbitrarily close to $y$ (the existence of such sequence is guaranteed by Theorem 1.3). Then $-f(y)=\varepsilon$ breaks the limit $f(y_n)\to f(y)$, so that by Theorem 1.18 $f$ is not continuous, giving us a contradiction.

    The result changes if we replace $\geq 0$ with $>0$. To see this, consider the function $f(x)=\absoluteValue{x-\sqrt{2}}$. This function is greater than 0 for all rationals but 0 whenever $x=\sqrt{2}$. Furthermore, $f$ is continuous, hence $f$ does not satisfy the alternative statement.
\end{enumerate}
\end{proof}

\begin{exercise}{49}
Let $f:(a,b)\to\R$ be monotone and let $a<x<b$. Show that $f$ is continuous at $x$ if and only if $f(x-)=f(x+)$.
\end{exercise}
\begin{proof}
($\Rightarrow$) Suppose $f$ is continuous at $x$. Then by Theorem 1.18 both $f(x+)$ and $f(x-)$ exist and are equal to $f(x)$.

($\Leftarrow$) Suppose $f(x-)=f(x+)$ and, for the sake of contradiction, $f(x)\neq f(x-)=f(x+)$. Without loss of generality, suppose $f(x)>f(x+)$, so that $f(x)-f(x+) >0$. 

Here we have two cases. First, either $f$ is not monotonic, as we can find $y>x$ with $f(x)-f(y) >0$ giving us a contradiction. Second, if we could not find such $y$, then $\lim_{z\to x+}f(z) \neq f(x+)$, by noticing that $f(x)-f(x+) =\varepsilon$ is enough to break the limit given by $f(z_n)\to f(x+)$ for a sequence $z_n\to x$ with $z_n>x$, for all $n$. This is also a contradiction. Hence, it must be the case that $f(x-)=f(x+)=f(x)$, which by Theorem 1.18 implies $f$ is continuous.
\end{proof}