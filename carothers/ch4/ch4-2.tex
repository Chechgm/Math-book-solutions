\section{Closed sets}


\begin{exercise}{1}
Show that an ``open rectangle'' $(a,b)\times (c,d)$ is an open set in $\R^2$. More generally, if $A$ and $B$ are open in $\R$, show that $A\times B$ is open in $\R^2$. 
If $A$ and $B$ are closed in $\R$, show that $A\times B$ is closed in $\R^2$. 
This also holds in an arbitrary metric space with the product metric (see exercise 3.46).
\end{exercise}
\begin{proof}
We will prove the more general case for the product of two arbitrary metric spaces and then the results for $\R^2$ will follow as a corollary.

(Open sets) Let $A$ and $B$ be open sets in $(X,d)$ and $(M,d')$. 
Then for all $x\in A$ and $y\in B$, there exists an $\epsilon>0$ so that $B_{\epsilon/2}(x)=\set{a:d(x,a)<\epsilon/2}\subseteq A$ and $B_{\epsilon/2}(y)=\set{b:d'(x,b)<\epsilon/2}\subseteq B$. 
Now consider $(x,y)\in A\times B$. 
For all $(a,b)\in A\times B$, we have that $\rho((x,y),(a,b))=d(x,a)+d'(y,b)<2\epsilon/2=\epsilon$, so that $B_\epsilon(x,y)=\set{a,b:\rho((x,y),(a,b))=d(x,a)+d'(y,b)<\epsilon}\subseteq A\times B$, as required.

(Closed sets) Let $A$ and $B$ be closed sets in $(X,d)$ and $(Y,d')$. 
By Theorem 4.9, if $(x_n)\in A$ converges to $x\in X$, then $x\in A$, and likewise if $(y_n)$ converges to $y\in Y$, then $y\in B$. 
Consider the space $(X\times Y, \rho)$. 
Then a convergent sequence in $X\times Y$ converges to the product of each of the coordinate limits. 
Since these limits belong to $X$ and $Y$, by Theorem 4.9, then its product belongs to $X\times Y$, so that $X\times Y$ is closed by the same Theorem.
\end{proof} 

\begin{exercise}{2}
If $F$ is a closed set and $G$ is an open set in a metric space $M$, show that $F\setminus G$ is closed and that $G\setminus F$ is open.
\end{exercise}
\begin{proof}
We have that $F\setminus G= F\cap G^C$. Since $G$ is open, $G^C$ is closed. 
Theorem 4.3 says that an arbitrary union of open sets is again open. 
Applying De Morgan's laws to this Theorem tells us that an arbitrary intersection of closed sets is closed, so that $F\setminus G$ is closed.

We can prove mutatis mutandis that $G\setminus F$ is open.
\end{proof} 

\begin{exercise}{3}
Some authors say that two metrics $d$ and $\rho$ on a set $M$ are equivalent if they generate the same open sets. 
Prove this. (Recall that we have defined equivalence to mean that $d$ and $\rho$ generate the same convergent sequences. 
See exercise 3.42.)
\end{exercise}
\begin{proof}
We will prove this via closed sets and Theorem 4.9. 
Let $A\subset M$ be closed (under d), so that by Theorem 4.9, if $(x_n)\in A$ converges (under d) to $x\in M$, then $x\in A$. 
Because $d$ and $\rho$ are equivalent, then $x_n\to x$ under $\rho$ so $A$ is closed under $\rho$ too. 
By Theorem 4.9, $A^C$ is open, and since we constructed $A$ arbitrarily (and it is closed under both $d$ and $\rho$), then $A^C$ is open under $d$ if and only if it is open under $\rho$, as desired.
\end{proof} 

\begin{exercise}{5}
Let $f:\R\to\R$ be continuous. 
Show that $\set{x:f(x)>0}$ is an open subset of $\R$ and that $\set{x:f(x)=0}$ is a closed subset of $\R$. 
Prove this for a general metric space too.
\end{exercise}
\begin{proof}
Notice that the following proofs are not dependent on any metric, and so the results for $\R$ follow as a Corollary.

($A=\set{x:f(x)>0}$) 
Consider $A^C=\set{x:f(x)\leq 0}$. 
Let $(x_n)\in A^C$, and $x_n\to x$. 
Thus, $f(x_n)\leq 0$ for all $n$, and because $f$ is continuous, $f(x_n)\to f(x)$. 
By the order limit Theorem (see, for example, Theorem 2.3.4 in Abbott) $f(x)\leq 0$, so that $x\in A^C$, $A^C$ is closed, and by Theorem 4.9, $A$ is open.

($A=\set{x:f(x)=0}$) 
Suppose $(x_n)\in A\subseteq X$ for an arbitrary metric space $(X,d)$ and $x_n\to x$. 
Then $f(x_n)=0$ for all $n$. 
Since $f$ is continuous, then $f(x)=0$ and so $x\in A$. 
By Theorem 4.9, $A$ is closed.
\end{proof} 

\begin{exercise}{7}
Show that every open set in $\R$ is the union of (countably many) open intervals with rational endpoints. 
Use this to show that the collection $\cU$ of all open subset of $\R$ has the same cardinality as $\R$ itself.
\end{exercise}
\begin{proof}
We will prove that we can construct any open interval as the countable union of open intervals with rational endponints. 
Then we will use Theorem 4.6 which states that any open subset of $\R$ can be written as a countable union of disjoint open intervals.
Consider $(a,b)$ if $a$ and $b$ are rational then we are done, so suppose that $a$ is not rational.
We know there exists a sequence $(a_n)$ with each $a_n\in Q$, $a<a_n$ and $a_n\to a$.
Thus, we have that $\cup_n^\infty (a_n,b) =(a,b)$.
If $b$, or $a$ and $b$ were not rational, we can find a countable union of open intervals mutatis mutandis.
From Theorem 4.6 we know that any open subset of $\R$ can be written as a countable union of disjoint open intervals, say $U=\cup_m^\infty (a_m,b_m)$, but from our previous conclusion, we have $U =\cup_m^\infty (a_m,b_m) =\cup_m^\infty \cup_n^\infty(a_{m_n},b_{m_n})$ which is a countable union, as required.

To prove that $\cU$ and $\R$ have the same cardinality, we will use a Cantor-Schr\"oder-Bernstein argument.

For the injective function, recall that $\Q^\N\cong\R\setminus\set{0}$, and the function $r\in\R$ given by $r\mapsto (0,r)$ if $r>0$ and $(r,0)$ otherwise is injective.

Second, to find an injection from $\Q^\N$ to $\cU$ we will first proof the following Lemma: 
for any two sets $X$ and $Y$, the existence of a surjection from $X$ to $Y$ guarantees the existence of an injection from $Y$ to $X$.
\textit{Proof:} Let $f:X\to Y$ be surjective.
For $y\in Y$, take any $x\in X$ so that $f(x)=y$ and label such $x=g(y)$, then $g:Y\to X$ is injective.

A surjective function is given by the following mapping $(a_1,b_1,\dots,a_n,b_n,\dots)\mapsto \cup_n (a_n,b_n)$.
So that by our lemma, there exists an injective function from $\cU$ to $\Q^\N$, and by the Cantor-Schr\"oder-Bernstein Theorem, $\Q^\N\cong\R\cong\cU$.
\end{proof} 

\begin{exercise}{8}
Show that every open interval (and hence every open set) in $\R$ is a countable union of closed intervals and that every closed interval in $\R$ is a countable intersection of open intervals.
\end{exercise}
\begin{proof}
To construct $(a,b)$ as a countable union of closed intervals, consider $\cup_n [a+1/n, b-1/n]$. 
On the other hand, to construct $[a,b]$ as a countable intersection of open intervals consider $\cap_n (a-1/n,b+1/n)$.
\end{proof} 

\begin{exercise}{11}
Let $e^{(k)}=(0,\dots,0,1,0,\dots)$, where the $k$-th entry is 1 and the rest are 0s. 
Show that $\set{e^{(k)}:k\geq 1}$ is closed as a subset of $l_1$.
\end{exercise}
\begin{proof}
Let $(x_n)$ be a sequence in $A=\set{e^{(k)}:k\geq 1}$, notice that in this subspace the metric induced by $\norm{\cdot}_1$ is the same as the discrete metric (because $\norm{x-y}=\sum_i\absoluteValue{x^i-y^i}=0$ if and only if $x^i=y^i$ and 1 otherwise). 
As we argued on exercise 3.43, under the discrete metric only sequences that are eventually constant converge. 
Knowing this, let $(x_n)$ converge, and hence be eventually constant. 
Thus $x_n=x$ for all $n$ whenever $n>N$ for some $N$, and thus $x\in A$ giving us that $A$ is closed by Theorem 4.9.
\end{proof} 

\begin{exercise}{12}
Let $F$ be the set of all $x\in l_\infty$ such that $x_n=0$ for all but finitely many $n$. 
Is $F$ closed? open? neither? explain.
\end{exercise}
\begin{proof}
($F$ not closed)
To see this, consider the sequence (of sequences) $x^{(k)}$, where for a specific $k$, $(x^k_n)=(1,1/2,\dots,1/k,0,0,\dots)$. 
For all $k$, $(x^k_n)\in l_\infty$, and $x^k_n=0$ for all but finitely many $n$.
We have that $(x^k_n)\to (x_n)$ where $(x_n)=(1,1/2,\dots,1/n,1/(n+1),\dots)$, which contains no zeros, so it does not belong to the original set.

($F$ is not open)
Let $\epsilon>0$. 
Consider the open ball around $0\in F$ given by $B_\epsilon(0)=\set{y\in l_\infty:d_\infty(0,y)=\norm{y}_\infty<\epsilon}$, and the sequence $y_n=\epsilon/2$ for all $n$.
We have that $\norm{y}_\infty<\epsilon$, though $y\notin F$.
Since this is true for all $\epsilon>0$, then it is not possible to have an open ball around $0\in F$ contained in $F$, and thus $F$ is not open.
\end{proof} 

\begin{exercise}{14}
Show that the set $A=\set{x\in l_2:\absoluteValue{x_n}\leq 1/n, n =1,2,\dots}$ is a closed set in $l_2$ but that $B=\set{x\in l_2:\absoluteValue{x_n}<1/n, n =1,2,\dots}$ is not an open set.
[Hint: Does $B\supseteq B_\epsilon(0)$?].
\end{exercise}
\begin{proof}
($A$ is closed)
Let $(x^{(n)})$ be a sequence (of sequences) in $A$ such that $x^{(n)}\to x$.
We will prove $x\in A$.
For all $\epsilon>0$ there exists an $N\in\N$ such that whenever $n>N$, $d(x^{(n)},x)=\sum_i^\infty \absoluteValue{x^{(n)}_i-x_i}^2<\epsilon$.
This implies that for all $i$, $\absoluteValue{x^{(n)}_i-x_i}^2<\epsilon$ so that the sequence converges coordinate-wise.
Since for each coordinate we have $\absoluteValue{x^{(n)}_i}\leq 1/i$, then by the order limit Theorem (see Abbott Theorem 2.3.4.) then $\absoluteValue{x_i}\leq 1/i$ for all $i$, and $x\in A$, as required.

($B$ is not open)
Consider the sequence (of sequences) $(x^{(n)})$ given by $x^{n}=(0,0,\dots,1/n,1/(n+1),1/(n+2),\dots$.
For all $n$, $x^{(n)}$ is not in $B$ (since $x^{(n)}_n=1/n$.
However, $x^{(n)}\to x=0$, since $d(x^{(n)},0) =\sum_i^\infty\absoluteValue{x^{(n)}_i}^2$, which we know we can get arbitrarily close to 0.
Thus, $B^C$ is not closed (because there is a sequence in it that doesn't converge to an element in $B^C$), and thus $B$ is not open.
\end{proof} 

\begin{exercise}{16}
If $(V,\norm{\cdot})$ is any normed space, prove that the closed ball $\bar{B}_1(0)=\set{x\in V:\norm{x}\leq 1}$ is always the closure of the open ball $B_1(0)=\set{x\in V:\norm{x}< 1}$.
\end{exercise}
\begin{proof}
Let $x\in\bar{B}_1(0)$.
Consider the sequence $(x_n)$ with $x_n\in B_1(0)$ given by $x_n=x(1-1/n)$.
We have that $x_n\to x$ so that $x$ is a limit point of $B_1(0)$.
Thus $\bar{B}_1(0)$ is the closure of $B_1(0)$.
\end{proof} 

\begin{exercise}{17}
Show that $A$ is open if and only if $A^\text{\degree}=A$ and that $A$ is closed if and only if $\bar{A}=A$.
\end{exercise}
\begin{proof}
($A$ is open if and only if $A^\text{\degree}=A$)

($\Rightarrow$) 
Suppose $A$ is open. 
Since $A^\text{\degree}$ is the largest open set contained in $A$, the largest set contained in $A$ is $A$ itself, and since $A$ is open, then the largest open set contained in $A$ is $A$ itself. 
Thus $A^\text{\degree}=A$.

($\Leftarrow$) 
Let $A^\text{\degree}=A$. 
By definition, $A^\text{\degree}$ is the largest open set contained in $A$, so that $A^\text{\degree}$ is open and hence $A$ is open.

($A$ is closed if and only if $\bar{A}=A$)

($\Rightarrow$) 
Suppose $A$ is closed. 
$\bar{A}$ is the smallest closed set containing $A$, so that $A\subseteq\bar{A}$. 
Now let $a\in\bar{A}$. 
Then either $a\in A$ or $a$ is a limit point of $A$. 
If $a\in A$, then $\bar{A}\subseteq A$ finishing the proof. 
On the other hand, if $a$ is a limit point of $A$, and $A$ is closed, then $a\in A$ so that in any case $\bar{A}\subseteq A$. 
Thus $\bar{A}=A$.

($\Leftarrow$) 
Suppose $\bar{A}=A$. 
By definition, $\bar{A}$ is closed, in particular, the smallest closed set containing $A$. 
Thus $A$ is closed.
\end{proof} 

\begin{exercise}{18}
Given a nonempty bounded subset $E$ of $\R$, show that $\sup E$ and $\inf E$ are elements of $\bar{E}$. 
Thus $\sup E$ and $\inf E$ are elements of $E$ whenever $E$ is closed.
\end{exercise}
\begin{proof}
We know from exercise 1.4 that a nonempty bounded subset of $\R$ has sequence $(x_n)$ converging to $x=\sup E$. 
By Corollary 4.11, we know that $x\in E$ if and only if there exists a sequence that converges to $x$, so that $\sup E\in E$. 
We can reason in an analogous way to conclude that $\inf E\in E$. 
\end{proof} 

\begin{exercise}{19}
Show that $\text{diam}(A)=\text{diam}(\bar{A})$.
\end{exercise}
\begin{proof}
Let $x,y\in\bar{A}$ and $\epsilon>0$.
Then there exist sequences $(x_n)$ and $(y_n)$ in $A$ so that $x_n\to x$ and $y_n\to y$.
That is, there exist $N,M\in\N$ so that whenever $n>N$ and $m>M$ it holds that $d(x_n,x)<\epsilon/2$ and $d(y_m,y)<\epsilon/2$. 
Let $n>\max\set{N,N}$.
We have that 
\[
d(x,y) 
\leq d(x,x_n)+d(x_n,y_n)+d(y_n,y) 
< d(x_n,y_n)+\epsilon.
\]
Since we can do this for all $x,y\in\bar{A}$ and $\epsilon>0$, then it is also the case that $\text{diam}(\bar{A})<\text{diam}(A)+\epsilon$.
To see this, notice that there exist $x,y\in\bar{A}$ that can get $\epsilon>0$ close to $\text{diam}(\bar{A})$ 
(otherwise the supremum would be another one).
That is, we can find the sequences as above.
It is also the case that for all $x_n,y_n\in A$ (the names of the elements here are suggestive), $d(x_n,y_n)\leq\text{diam}(A)$.
Now, since $A\subseteq\bar{A}$, it is also true that $\text{diam}(A)<\text{diam}(\bar{A})+\epsilon$, so that $\absoluteValue{\text{diam}(\bar{A})-\text{diam}(A)}<\epsilon$, as desired.
\end{proof} 

\begin{exercise}{21}
If $A$ and $B$ are any sets in $M$, show that $\overline{A\cup B} =\bar{A}\cup\bar{B}$ and $\overline{A\cap B} \subseteq\bar{A}\cap\bar{B}$. 
Give an example showing that this last inclusion can be proper.
\end{exercise}
\begin{proof}
($\overline{A\cup B} =\bar{A}\cup\bar{B}$)

($\subseteq$) 
Suppose $x\in\overline{A\cup B}$. 
Then $x\in A\cup B$ or $x$ is a limit of $A\cup B$.
If $x\in A\cup B$ then we are done because that implies $x\in A$ or $x\in B$ so that $x\in\bar{A}$ or $x\in\bar{B}$ which implies $x\in\bar{A}\cup\bar{B}$.
So suppose $x\notin\overline{A\cup B}$ but $x$ is a limit point of $\overline{A\cup B}$.
Thus, there exists a sequence $(x_n)$ in $A\cup B$ so that $x_n\to x$.
Notice, however, that this implies there is a subsequence of $(x_n)$ consisting of only elements of $A$ or elements of $B$ that converges to $x$ (because subsequences of convergent sequences converge to the same limit), and thus $x$ is a limit of $A$ or $B$ (depending on the set containing the sequence). 
Thus, $x\in\bar{A}\cup\bar{B}$.

($\supseteq$) 
Suppose $x\in\bar{A}\cup\bar{B}$, then $x\in A$, $x\in B$ or $x$ is a limit point of either.
If $x\in A$ or $x\in B$, we are done because then $x\in\overline{A\cup B}$, so suppose, without loss of generality, $x$ is a limit point of $A$.
This implies there exists a sequence $(x_n)$ of elements in $A$ so that $x_n\to x$.
However, this sequence is also contained in $A\cup B$, so that $x$ is also a limit point of $A\cup B$ and thus $x\in\overline{A\cup B}$, as desired.

($\overline{A\cap B} \subseteq\bar{A}\cap\bar{B}$)

($\subseteq$)
Suppose $x\in\overline{A\cap B}$.
Then either $x\in A\cap B$ or $x$ is a limit point of the intersection.
In the former case, we are done because then $x\in A$ and $x\in B$ which implies $x\in\bar{A}$ and $x\in\bar{B}$ so that $x\in\bar{A}\cap\bar{B}$.
So suppose that $x$ is a limit point of $\overline{A\cap B}$.
This implies there exists a sequence $(x_n)$ of elements in $A\cap B$ so that $x_n\to x$.
Thus, $(x_n)$ is both in $A$ and $B$ meaning that $x$ is a limit point of both $A$ and $B$;
that is, $x\in\bar{A}\cap\bar{B}$, as required.

For the example, consider the following two open intervals, $A=(0,1)$ and $B=(1,2)$.
We have that $\overline{A\cap B}=\emptyset$ and $\bar{A}\cap\bar{B}=[0,1]\cap[1,2]=\set{1}$.
\end{proof}

\begin{exercise}{24}
Show that $\bar{A} =(\text{int}(A^C))^C$ and that $A^{\text{\degree}} =(\text{cl}(A^C))^C$.
\end{exercise}
\begin{proof}
($\bar{A} =(\text{int}(A^C))^C$)
Consider the following definition given by Carothers:
\begin{align*}
    (\text{int}(A^C))^C 
    =& \set{x\in A^C:B_\epsilon(x)\subseteq A^C\text{ for some }\epsilon>0}^C
\end{align*}
That is, $(\text{int}(A^C))^C$ is the set of $x$ so that either $x\in A$ or the set of $x$ so that for all $\epsilon>0$, there exists a $b\in B_\epsilon(x)$ such that $b\in A$, which is the same as saying that $x$ is a limit point of $A$. 
These elements together form precisely $\bar{A}$.

($A^{\text{\degree}} =(\text{cl}(A^C))^C$)
We can use the results from the previous proof.
We have that $(\text{cl}(A^C))^C$ consists of all the elements of $A^C$ or the $x$ so that for all $\epsilon>0$ so that there exists a $b\in B_\epsilon(x)$ with $b\in A^C$.
Negating this statement is the same as saying that the set is composed of $x\in A$ and for $x\in A$ there exists an $\epsilon>0$ so that for all $b\in B_\epsilon(x)$ $b\notin A^C$; 
that is, $B_\epsilon(x)\subseteq A$.
Notice, however, that this coincides with Carothers definition:
\begin{align*}
    \text{int}(A)=A^{\text{\degree}}=\set{x\in A:B_\epsilon(x)\subseteq A\text{ for some }\epsilon>0},
\end{align*}
as required.
\end{proof} 

\begin{exercise}{26}
We define the distance from a point $x\in M$ to a nonempty set $A$ in $M$ by $d(x,A)=\inf\set{d(x,a):a\in A}$.
Prove that $d(x,A)=0$ if and only if $x\in \bar{A}$.
\end{exercise}
\begin{proof}
($\Rightarrow$) 
Suppose, by contrapositive, that $x\notin\bar{A}$.
Then $x\notin A$, but furthermore, there exists an $\epsilon>0$ so that $B_\epsilon(x)\subseteq A^C$.
We have that $d(x,A)\geq\epsilon$, as required.

($\Leftarrow$) 
Suppose $x\in\bar{A}$, then either $x\in A$ in which case $0\in\set{d(x,a):a\in A}$, as $d(x,x)=0$, since $d(x,y)\geq 0$, or $x$ is a limit point of $A$.
If $A$ is a limit point of $A$, then there exists a sequence $(x_n)$ in $A$ so that $x_n\to x$.
Thus $d(x_n,x)\to d(x,x)=0$, and $d(x,A)=0$, as required.
\end{proof} 

\begin{exercise}{28}
Given a set $A$ in $M$ and $\epsilon>0$, show that $\set{x\in M:d(x,A)<\epsilon}$ is an open set and that $\set{x\in M:d(x,A)\leq\epsilon}$ is a closed set (and each contains $A$).
\end{exercise}
\begin{proof}
(Open)
Let $x\in X=\set{x\in M:d(x,A)<\epsilon}$, so that $d(x,A)\leq\epsilon-\epsilon'$ for some $\epsilon'>0$.
Now consider the open ball $B_{\epsilon'}(x)=\set{y:d(x,y)<\epsilon'}$.
Let $x'\in B_{\epsilon'}(x)$, so that $d(x,x')<\epsilon'$.
From exercise 27 we know that $\absoluteValue{d(x',A)-d(x,A)} \leq d(x',x)<\epsilon'$.
The first inequality also implies that $d(x',A)\leq d(x',x)+d(x,A)<\epsilon$, so that $x'\in X$, and $B_{\epsilon'}(x)\subseteq X$, as required.

(Closed)
From exercise 27, we know that $f(x)=d(x,A)$ is continuous.
We also know that for continuous functions, $f(x_n)\to x$ holds for a sequence $(x_n)$ with $x_n\to x$.
Thus, if $(x_n)$ is a sequence in $X'=\set{x\in M:d(x,A)\leq\epsilon}$, then for each $x_n$ we must have $f(x_n)=d(x_n,A)\leq\epsilon$, and by the order limit Theorem, then $f(x)=d(x,A)\leq\epsilon$, giving us that $x\in X'$ so that $X'$ is closed, as required.
\end{proof} 

\begin{exercise}{32}
We define the distance between two (nonempty subset $A$ and $B$ of $M$ by $d(A,D)=\inf\set{d(a,b):a\in A,b\in B}$.
Given an example of two disjoint closed sets $A$ and $B$ in $\R^2$ with $d(A,B)=0$.
\end{exercise}
\begin{proof}
Consider the two following sets $A=\set{(x,0):x\in\R}$ and $B=\set{(x,1/x):x>0}$.
Both sets are closed;
$A$ is closed as we can identify it with $\R$ and if we consider a convergent sequence in $B$, then the limit will belong to $B$.
Furthermore, $A$ and $B$ are disjoint, as for any $(x,y)\in B$, $y\neq 0$.
Finally, notice that for all $\epsilon>0$, we can find an $x\in\R$ so that for $(x,0)\in A$ and $(x,1/x)\in B$, we have $d(a,b)=\max\set{\absoluteValue{x-x}=0,\absoluteValue{1/x}}<\epsilon$, so that $d(A,B)=0$, as required.
\end{proof} 

\begin{exercise}{33}
Let $A$ be a subset of $M$. A point $x\in M$ is called a limit point of $A$ if every neighborhood of $x$ contains a point of $A$ that is different from $x$ itself, that is, if $(B_\epsilon(x)\setminus \set{x})\cap A\neq\emptyset$ for every $\epsilon>0$. If $x$ is a limit point of $A$, show that every neighborhood of $x$ contains infinitely many points of $A$.
\end{exercise}
\begin{proof}
Suppose $x$ is a limit point of $A$. 
Then for all $n\in\N$, there exists at least one $a_n\in A\setminus\set{x}$ so that $a\in B_{1/n}(x)$. 
Thus there is an infinite sequence $(a_n)$ of distinct elements of $A$ inside $B_1(x)$. 
Now take any $\epsilon>0$, it is the case that for some $N\in\N$, $\epsilon>1/N$, since the set $A'=\set{a_1,\dots,a_N}$ is finite, then the set $\set{a_n: a_n\in (a_n)}\setminus A'$ is infinite, giving us the desired result.
\end{proof} 

\begin{exercise}{34}
Show that $x$ is a limit point of $A$ if and only if there is a sequence $(x_n)$ in $A$ such that $x_n\to x$ and $x_n\neq x$ for all $n$.
\end{exercise}
\begin{proof}
($\Rightarrow$) Suppose $x$ is a limit point of $A$. 
Then for all $\epsilon=1/n>0$, we have $B_\epsilon(x)\cap A\setminus\set{x}\neq\emptyset$.
Let $x_n\in B_\epsilon(x)\cap A\setminus\set{x}$, then $x_n\to x$, and $x_n\neq x$ for all $n$.

($\Leftarrow$) Suppose $(x_n)$ with $x_n\to x$ and $x_n\neq x$ for all $n$. 
Then for all $\epsilon>0$, there exists an $N\in\N$ so that if $n>N$ it holds that $d(x_n,x)<\epsilon$. 
But since $x_n\neq x$ for all $n$, that's exactly the same as saying that $B_\epsilon(x) =\set{y:d(x,y)<\epsilon}\cap A\setminus\set{x} \neq\emptyset$ for all $\epsilon$, as required.
\end{proof} 

\begin{exercise}{37}
Prove the Bolzano-Weierstrass Theorem:
Every bounded infinite subset of $\R$ has a limit point.
[Hint: Use the nested interval Theorem.
If $A$ is a bounded infinite subset of $\R$, then $A$ is contained in some closed bounded interval $I_1$.
At least one of the left or right halves of $I_1$ contains infinitely many points of $A$.
Call this new closed interval $I_2$.
Continue].
\end{exercise}
\begin{proof}
We will follow the hint to construct a sequence with a limit.
First, we notice that $A$ is contained in a closed and bounded interval, which we call $I_1$.
This interval contains infinite points.
Now split $I_1$ into two (almost) halves, by choosing a point in $A$, consisting of closed and bounded intervals, at least one these will have an infinite number of points.
Let $I_2$ be one (or the) interval with infinite points.
Additionally, $I_2\subseteq I_1$.
We do this countably infinite times to obtain a sequence of closed and bounded nested intervals $I_1\supseteq I_2\supseteq \dots$.
Now we can apply the nested interval theorem to conclude that the intersection of such intervals is non empty: $x=\cap_n^\infty I_n$, and moreover $x$ is unique since the length of $I_n$ tends to 0 as $n\to\infty$.
$x$ is a limit point in $A$.
To see this, define the sequence given by the cutting point of the intervals, so that for $I_n=[a_n,b_n]$, we define a sequence $(x_n)$ with $x_n=a_n$ if $a_n$ is the point in $A$ used to cut $I_{n-1}$ and $x_n=b_n$ if $b_n$ was used. 
We have $x_n\to x$.
\end{proof} 

\begin{exercise}{40}
If $x\in A$ and $x$ is not a limit point of $A$, then $x$ is called an isolated point of $A$. 
Show that a point $x\in A$ is an isolated point of $A$ if and only if $(B_\epsilon(x)\setminus\set{x})\cap A=\emptyset$ for some $\epsilon>0$. 
Prove that a subset of $\R$ can have at most countably many isolated points, thus showing that every uncountable subset of $\R$ has a limit point.
\end{exercise}
\begin{proof}
Using the definition in exercise 33, a point $x\in M$ is a limit point of $A$ if $(B_\epsilon(x)\setminus\set{x})\cap A\neq\emptyset$, for all $\epsilon>0$.
Thus, negation of this statement yields that $x$ is an isolated point of $A$ if there exists an $\epsilon>0$ so that $(B_\epsilon(x)\setminus\set{x})\cap A=\emptyset$, which is precisely what we wanted to prove.

Before proving that a subset, say $M$, of $\R$ can have at most countably many isolated points, we will first prove that there exists a closed and bounded subset of $M$ that contains uncountable many points.
To see this, notice that $M=\cup_{n\in\N} (M\cap [-n,n])$.
If there was no $a$ such that $[-a,a]$ contains uncountable many points of $M$, then $M$ would be the countable union of countable sets, and thus itself countable.
Knowing about the existence of such subset $[-a,a]\subseteq M$ with uncountable many points, then we can use the Bolzano-Weierstrass Theorem which states that any closed and bounded set contains a convergent sequence.
The limit of such convergent sequence is in $[-a,a]\subseteq M$, giving us the desired result.
\end{proof} 

\begin{exercise}{41}
Related to the notion of limit points and isolated points are boundary points. 
A point $x\in M$ is said to be a boundary point of $A$ if each neighborhood of $x$ hits both $A$ and $A^C$. 
In symbols, $x$ is a boundary point of $A$ if and only if $B_\epsilon(x)\cap A \neq\emptyset$ and $B_\epsilon(x)\cap A^C \neq\emptyset$ for every $\epsilon>0$. 
Verify each of the following formulas, where $\text{bdry}(A)$ denotes the set of boundary points of $A$:
\begin{enumerate}
    \item $\text{bdry}(A)=\text{bdry}(A^C)$,
    \item $\text{cl}(A)=\text{bdry}(A)\cup\text{int}(A)$,
    \item $M=\text{int}(A)\cup\text{bdry}(A)\cup\text{int}(A^C)$.
\end{enumerate}
Notice that the first and last equations tell us that each set $A$ partitions $M$ into three regions: 
the points ``well inside'' $A$, the points ``well outside'' $A$ and the points on the common boundary of $A$ and $A^C$.
\end{exercise}
\begin{proof}
\begin{enumerate}
    \item Since the logical operator `and' is commutative, then the definition of boundary applies to both the boundary of $A$ and the boundary of $A^C$.
    \item Proposition 4.10 states that $x\in\bar{A}$ if and only if for all $\epsilon>0$, it holds that $B_\epsilon(x)\cap A \neq\emptyset$.
    This definition gives us two type of elements in $A$:
    those that have an open ball contained in $A$ around them (which would conform $\text{int}(A)$,
    and those that don't, in which case for all $\epsilon>0$, it will holds that $B_\epsilon(x)\cap A^C \neq\emptyset$, which would conform the boundary of $A$, giving us the desired result.
    \item From two previous points, we can conclude that $\text{int}(A)\cup\text{bdry}(A)\cup\text{int}(A^C)=\bar{A}\cup \bar{A^C}$ which includes all points in $M$, as required.
\end{enumerate}
\end{proof} 

\begin{exercise}{42}
If $E$ is a nonempty bounded subset of $\R$, show that $\sup E$ and $\inf E$ are both boundary points of $E$. Hence, if $E$ is also closed, then $\sup E$ and $\inf E$ are elements of $E$.
\end{exercise}
\begin{proof}
We will prove the result for $s=\sup E$.
The proof for $\inf E$ will follow using a similar strategy.
On exercise 1.3 we gave an alternative characterisation of the supremum of $E$ by saying that $s=\sup E$ if s is an upper bound of $E$, and for all $\epsilon>0$, there exists $e\in E$ so that $e>s-\epsilon$;
that is, for all $\epsilon>0$, there exists $e\in B_\epsilon(s)$ and so $B_\epsilon(s)\cap E\neq\emptyset$.
Furthermore, by exercise 1.2 if $B$ is the set of all upper bounds of $E$, then $B$ is non empty, bounded below and $\inf B=\sup E$.
Using the same reasoning as above (with a definition of $\inf B$ similar to that of $\sup E$), we can conclude that $B_\epsilon(s)\cap B\neq\emptyset$.
However, notice that $B\subseteq E^C$, thus, $B_\epsilon(s)\cap E^C\neq\emptyset$, and $s\in\text{bdry}(E)$, as required.
\end{proof} 

\begin{exercise}{43}
Show that $\text{bdry}(A)$ is always a closed set; 
in fact, $\text{bdry}(A)=\bar{A}\setminus A^\text{\degree}$.
\end{exercise}
\begin{proof}
This follows directly from the definition of boundary and the equivalence between $A$ being closed, and $x\in A$ whenever $B_\epsilon(x)\cap A \neq\emptyset$ for all $\epsilon>0$, from Theorem 4.9.
\end{proof} 

\begin{exercise}{44}
Show that $A$ is closed if and only if $\text{bdry}(A)\subseteq A$.
\end{exercise}
\begin{proof}
($\Rightarrow$)
Suppose $A$ is closed.
Let $x\in\text{bdry}(A)$.
Then for all $\epsilon=1/n>0$, it is true that $B_\epsilon(x)\cap A\neq\emptyset$.
Thus, consider the sequence $(x_n)$ where $x_n$ is an element from $B_\epsilon(x)\cap A$.
We have that $x_n\to x$, and since $A$ is closed, it contains its limit points so that $x\in A$;
that is, $\text{bdry}(A)\subseteq A$.

($\Leftarrow$) 
We will prove this by contrapositive.
Thus, suppose $A$ is not closed.
Then there exists a sequence $(x_n)$ in $A$ with $x_n\to x$, but $x\notin A$.
By the definition of limit, we have that for all $\epsilon>0$, $B_\epsilon(x)\cap A\neq\emptyset$.
Furthermore, we assumed $x\notin A$, so that $B_\epsilon(x)\cap A\neq\emptyset$.
Putting these two statements together, we have that $x\in\text{bdry}(A)$ and $\text{bdry}(A)\not\subseteq A$, as required.
\end{proof} 

\begin{exercise}{45}
Give examples showing that $\text{bdry}(A)=\emptyset$ and $\text{bdry}(A)=M$ are both possible.
\end{exercise}
\begin{proof}
($\text{bdry}(A)=\emptyset$)
Either the reals or the empty set.

($\text{bdry}(A)=M$)
The rationals or irrationals.
\end{proof} 

\begin{exercise}{46}
A set $A$ is said to be dense in $M$ (or, as some authors say, everywhere dense) if $\bar{A}=M$. For example, both $\Q$ and $\R\setminus\Q$ are dense in $\R$. Show that $A$ is dense in $M$ if and only if any of the following hold:
\begin{enumerate}
    \item Every point in $M$ is the limit of a sequence from $A$.
    \item $B_\epsilon(x)\cap A\neq\emptyset$ for every $x\in M$ and every $\epsilon>0$.
    \item $U\cap A\neq\emptyset$ for every nonempty open set $U$.
    \item $A^C$ has empty interior.
\end{enumerate}
\end{exercise}
\begin{proof}
(Dense $\Rightarrow 1$) 
Suppose $\bar{A}=M$, then all the elements of $M$ are either in $A$ or they are limit points of $A$. 
In both cases, they are limit point of sequences in $A$.

($1\Rightarrow 2$)
Suppose every point in $M$ is the limit of a sequence from $A$.
Let $x\in M$, then there exists a sequence $(x_n)$ with $x_n\in A$ for all $n$ so that $x_n\to x$.
But this means that for all $\epsilon>0$, there exists an $N\in\N$ so that $d(x_n,x)<\epsilon$ for all $n>N$.
That is, $x_n\in B_\epsilon(x)$, so that $B_\epsilon(x)\cap A \neq\emptyset$ for all $\epsilon>0$.

($2\Rightarrow 3$)
Suppose $B_\epsilon(x)\cap A \neq\emptyset$ for every $x\in M$ and every $\epsilon>0$.
Since the previous statement is true for all $x\in M$, we can construct $U$ in $M$ as an arbitrary union of open sets $U=\cup_{i\in\cI} B_{\epsilon_i}(x_i)$.
However, we assume that for any of those such open balls $B_{\epsilon_i}(x_i)\cap A\neq\emptyset$, thus $U\cap A\neq\emptyset$.

($3\Rightarrow 4$)
Suppose $U\cap A\neq\emptyset$ for every nonempty open set $U$ and, for the sake of contradiction, suppose $\text{int}(A^C) =M\setminus A \neq\emptyset$.
Since $\text{int}(A^C)$ is not empty, then there exists $x\in A^C$ so that $B_\epsilon(x)\subseteq A^C$ for some $\epsilon>0$.
But this is a contradiction, because $B_\epsilon(x)\cap A\neq \emptyset$ (by noticing $B_\epsilon(x)=U$ is an open set) which is inconsistent with $B_\epsilon(x)$ being contained in $A^C$.

($4\Rightarrow$ dense)
Suppose $A^C$ has empty interior.
Thus, for all $x\in A^C$ and all $\epsilon>0$, it is the case that $B_\epsilon(x)\cap A\neq\emptyset$ (this is exactly what the definition of having an empty interior means, from Carothers definition).
But notice that this is exactly to say that for all $x\in M\setminus A$, $x$ is a limit point of $A$.
Thus $\bar{A}=M$.
\end{proof} 

\begin{exercise}{48}
A metric space is called separable if it contains a countable dense subset. 
Find examples of countable dense sets in $\R$, in $\R^2$ and in $\R^n$.
\end{exercise}
\begin{proof}
In $\R^n$ the set $\Q^n$ countable (because it is the finite product of countable sets) and dense (because we can approximate any irrational and thus, any tuple of irrationals, with rational numbers).
Countable dense sets in $\R$ and $\R^2$ follow as corollaries.
\end{proof} 

\begin{exercise}{50}
Show that $l_\infty$ is not separable.
[Hint: Consider the set $2^\N$, consisting of all sequences of 0s and 1s, as a subset of $l_\infty$.
We know that $2^\N$ is uncountable.
Now what?].
\end{exercise}
\begin{proof}
This proof comes from 1.3-9 in Kreyszig's Functional analysis.
Let $y=(y_1,y_2,\dots)$ be a sequence of zeros and ones.
Then $y\in l^\infty$.
With $y$ we associate the real number $\hat{y}$ whose binary representation is $y_1/2 + y_2/2^2 + y_3/2^3 +\dots$.
We now use the facts that the set of points in the interval $[0,1]$ is uncountable, each $\hat{y}$ has a binary representation, and different $\hat{y}$ have different binary representations.
Hence there are uncountable many sequences of zeros and ones.
The metric on $l^\infty$ show that any two of them which are not equal must be of distance 1 apart.
If we let each of these sequences be the center of a small ball, say of radius 1/3, these balls do not intersect and we have uncountably many of them.
If $M$ is any dense set in $l^\infty$, each of these nonintersecting balls must contain an element of $M$.
Hence $M$ cannot be countable.
Since $M$ was an abitrary dense set, this show that $l^\infty$ cannot have dense subsets that are uncountable.
Consequently, $l^\infty$ is not separable.
\end{proof} 

\begin{exercise}{52}
If $M$ is separable, show that any collection of disjoint open sets in $M$ is at most countable.
\end{exercise}
\begin{proof}
If $M$ is separable, then there exists a countable subset of $A\subseteq M$ so that $A$ is dense in $M$, mathematically $\bar{A}=M$.
Thus, for all $m\in M$, either $m\in A$, or for all $\epsilon>0$, $B_\epsilon(b)\cap A\neq\emptyset$.
Consider a collection $\cB$ of disjoint open sets in $M$.
Then for any $B\in \cB$, it is the case that $B\cap A\neq\emptyset$. 
Furthermore, since $\cB$ consists of disjoint sets, then there are at most countable number of elements in $\cB$, since i) each one of them has to contain at least one element of $A$, ii) no two sets can contain the same elements, and iii) $A$ is countable.
\end{proof} 

\begin{exercise}{54}
A set $A$ is said to be nowhere dense in $M$ if $\operatorname{int}(\operatorname{cl}(A)) = \emptyset$.
Show that $\set{x}$ is nowhere dense in $M$ if and only if $x$ is not an isolated point of $M$.
\end{exercise}
\begin{proof}
($\Rightarrow$)
We will prove this by contrapositive.
Suppose $x$ is an isolated point, so that by exercise 4.40, there exists an $\epsilon >0$, so that $B_\epsilon(x)\setminus \set{x} \cap A = \emptyset$.
But we have that $\overline{\set{x}} = \set{x}$, so that $\overline{\set{x}}$ contains no nonempty open set.

($\Leftarrow$)
Suppose $\set{x}$ is nowhere dense in $M$.
Thus, $\emptyset = \operatorname{int}(\operatorname{cl}(\set{x})) = \operatorname{int}(\overline{\set{x}}) = \operatorname{int}(\set{x})$.
Which implies, that for any $\epsilon>0$, $B_\epsilon(x) \setminus \set{x} \cap A =\emptyset$;
thus $x$ is an isolated point.
\end{proof} 

\begin{exercise}{60}
Show that each of the following is equivalent to the statement ``$A$ is nowhere dense'':
\begin{enumerate}
    \item $\bar{A}$ contains no nonempty open set.
    \item Each nonempty open set in $M$ contains a nonempty open subset that is disjoint from $A$.
    \item Each nonempty open set in $M$ contains an open ball that is disjoint from $A$.
\end{enumerate}
\end{exercise}
\begin{proof}
($1 \Rightarrow 2$)
Let $G$ be a nonempty subset of $M$ and consider $G \setminus \bar{A} = G \cap \bar{A}^C$.
We have that $\bar{A}^C$ is open (because $\bar{A}$ is closed by definition, so that $G \cap \bar{A}^C$ itself is open because it is the intersection of two open sets.
We have that $G \setminus \bar{A} \subseteq G$ and we have that it is disjoint from $\bar{A}$.
Notice that if $G\setminus \bar{A}$ is nonempty, it would mean that $G\subseteq \bar{A}$ so that $\bar{A}$ contains a nonempty open set which contradicts our hypothesis.

($2 \Rightarrow 3$)
This is fairly clear, as if each nonempty open set in $M$ contains a nonempty open subset that is disjoint from $A$, then this open subset contains an open ball, so that this open ball is also disjoint from $A$.

($3 \Rightarrow 1$)
By contradiction suppose $\bar{A}$ contains a nonempty open set, say $B$, but each nonempty subset of $M$ contains an open ball that is disjoint from $A$. 
This produces an instantaneous contradiction, $B \subseteq \bar{A} \subseteq M$, but by hypothesis, there should be an open ball in $B$ disjoint of $A$ which is not possible, since $B$ is fully contained in $\bar{A}$, and $\operatorname{bdry}(A) = \bar{A}\setminus A$ contains no open ball.
\end{proof} 