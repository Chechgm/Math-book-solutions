\subsection{The real numbers}


\begin{exercise}{1}
If $A$ is a nonempty subset of $\R$ that is bounded below, show that $A$ has a greatest lower bound. That is, show that there is a number $m\in\R$ satisfying (i) $m$ is a lower bound for $A$; and (ii) if $x$ is a lower bound for $A$, then $x\leq m$. [Hint: Consider the set $-A=\{-a:a\in A\}$ and show that $m=-\sup(-A)$ works].
\end{exercise}
\begin{proof}
Consider $-A$ as above. Because $A$ is bounded below, then it must be the case $-A$ is bounded above (suppose it is not bounded above, then for all $-k\in\R$, there exists $-a\in -A$ with $-k\leq -a$ but this is the same as $k\geq a$ but this would imply $A$ is not bounded below). Since $A$ is bounded above, then by the Least Upper Bound Axiom, then there exists a number $-m\in\R$ such that for all $-a\in -A$, it holds that $-a\leq -m$ and for all other upper bound $-x\in\R$, $-m<-x$. Multiplying everything by $-1$ we get that $m\leq a$ for all $a\in A$ and for all other bounds $x\in A$, $m\geq x$, as required.
\end{proof}

\begin{exercise}{3}
Establish the following apparently different (but ``fancier'') characterization of the supremum. Let $A$ be a nonempty subset of $\R$ that is bounded above. Prove that $s=\sup A$ if and only if (i) $s$ is an upper bound for $A$, and (ii) for every $\varepsilon>0$, there is an $a\in A$ such that $a>s-\varepsilon$. State and prove the corresponding result for the infimum of a nonempty subset of $\R$ that is bounded below.
\end{exercise}
\begin{proof}
fill
\end{proof}

\begin{exercise}{4}
fill
\end{exercise}
\begin{proof}
fill
\end{proof}

\begin{exercise}{6}
Prove that every convergent sequence of real numbers is bounded. Moreover, if $(a_n)$ is convergent, show that $\inf_na_n\leq \lim_{n\to\infty}a_n\leq\sup_na_n$.
\end{exercise}
\begin{proof}
fill
\end{proof}
