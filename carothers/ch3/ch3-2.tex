\subsection{Normed vector spaces}

16*
18*
19*
21
22
23


\begin{exercise}{16}
Let $V$ be a vector space, and let $d$ be a metric on $V$ satisfying $d(x,y)=d(x-y,0)$ and $d(\alpha x,\alpha y)=\absoluteValue{\alpha}d(x,y)$ for every $x,y\in V$ and every scalar $\alpha$. Show that $\norm{x}=d(x,0)$ defines a norm on $V$ (that has $d$ as its ``usual'' metric). Give an example of a metric on the vector space $\R$ that fails to be associated with a norm in this way.
\end{exercise}
\begin{proof}
fill
\end{proof} 

\begin{exercise}{18}
Show that $\norm{x}_\infty\leq \norm{x}_2\leq\norm{x}_1$ for any $x\in \R^n$. Also check that $\norm{x}_1\leq n\norm{x}_\infty$ and $\norm_1\leq\sqrt{n}\norm{x}_2$.
\end{exercise}
\begin{proof}
fill
\end{proof} 

\begin{exercise}{19}
Show that we have $\sum_{i=1}^nx_iy_i=\norm{x}_2\norm_2$ (quality in the Cauchy-Shwarz inequality) if and only if $x$ and $y$ are proportional, that is if and only if either $x=\alpha y$ or $y=\alpha x$ for some $\alpha\geq 0$.
\end{exercise}
\begin{proof}
fill
\end{proof} 

\begin{exercise}{x}
fill
\end{exercise}
\begin{proof}
fill
\end{proof} 

\begin{exercise}{x}
fill
\end{exercise}
\begin{proof}
fill
\end{proof} 

\begin{exercise}{x}
fill
\end{exercise}
\begin{proof}
fill
\end{proof} 
