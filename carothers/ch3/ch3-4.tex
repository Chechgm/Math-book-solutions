\subsection{Limits in metric spaces}


\begin{exercise}{27}
Show that $\text{diam}(B_r(x))\leq 2r$, and give an example where strict inequality holds.
\end{exercise}
\begin{proof}
Let $y,z\in B_r(x)$ be arbitrary. Then $d(z,y)\leq d(z,x)+d(x,y)<2r$, so that $\text{diam}(B_r(x))\leq 2r$. 

Example: let $X$ be a discrete metric space. Then $B_1(x)=\set{x}$. In this case, $2\neq 0=\sup\set{d(a,b):a,b\in B_1(x)}$.
\end{proof} 

\begin{exercise}{29}
Prove that $A$ is bounded if and only if $\text{diam}(A)<\infty$.
\end{exercise}
\begin{proof}
($\Rightarrow$) Suppose $A$ is bounded. Thus, there exists $a\in M$ and $r>0$ so that $A\subseteq B_r(a)$. Let $b,c\in A$ be arbitrary. We have $d(b,c)\leq d(b,a)+d(a,c)\leq 2r$. Thus, $\text{diam}(A)\leq 2r<\infty$, as required.

($\Leftarrow$) Suppose $\text{diam}(A)<\infty$. Then there exists a real, say $r$, so that $\text{diam}(A) =\sup\set{d(a,b):a,b\in A} =r$. That is, for all $a,b\in A$, we have $d(a,b)<r$, hence we can find a point in $a\in A\subseteq M$ so that $A\subseteq B_r(a)=\set{y\in M:d(a,y)<r}$. In words, $A$ is bounded.
\end{proof} 

\begin{exercise}{30}
If $A\subseteq B$, show that $\text{diam}(A)\leq\text{diam}(B)$.
\end{exercise}
\begin{proof}
Notice that $\text{diam}(A) =\sup\set{d(a,b):a,b\in A}$. However, since $A\subseteq B$, then we have $\set{d(a,b):a,b\in A} \subseteq\set{d(a,b):a,b\in B}$. By exercise 1.2.b we have that $\sup\set{d(a,b):a,b\in A} \leq\sup\set{d(a,b):a,b\in B}$, that is $\text{diam}(A) \leq\text{diam}(B)$.
\end{proof} 

\begin{exercise}{32}
In a normed vector space $(V,\norm{\cdot})$ show that $B_r(x)=x+B_r(0)=\set{x+y:\norm{y}<r}$ and that $B_r(0)=rB_1(0)=\set{rx:\norm{x}<1}$.
\end{exercise}
\begin{proof}
Take any element of $x+B_r(0)=\set{y+x:\norm{y}<r}$, say $y+x$, and consider its distance to $x$: $\norm{y+x-x}=\norm{y}<r$, so that $y\in B_r(x)$, and $x+B_r(0)\subseteq B_r(x)$.

Now take any element of $B_r(x)$, say $y$. Consider the distance of $y+x$ to $x$: $\norm{y+x-x}<r$, so that $y\in x+B_r(0)$, and $B_r(x)\subseteq x+B_r(0)$. Putting this and the above together, we have that $B_r(x)=x+B_r(0)$.

We have $B_1(0)=\set{x:\norm{x}<1}$, multiplying each element of $B_1(0)$ by $r$ gives us $rB_1(0)=\set{rx:\norm{x}<1}$
\end{proof} 

\begin{exercise}{33}
In a metric space $M$ with metric $d$ show that limits are unique. [Hint: $d(x,y)\leq d(x,x_n)+d(x_n,y)$].
\end{exercise}
\begin{proof}
Suppose $(x_n)\to x$ and $(x_n)\to y$. Then for a fixed $\epsilon>0$, there exist an $N\in\N$ so that for all $n>N$, it holds that $d(x_n,x)<\epsilon/2$. Likewise, there exists an $N'\in\N$ so that for all $n>N'$, it holds that $d(x_n,y)<\epsilon/2$. Let $n>\max\set{N,N'}$, we then have $d(x,y)\leq d(x,x_n)+d(x_n,y)\leq \epsilon$, so that $x=y$, as required.
\end{proof} 

\begin{exercise}{34}
If $x_n\to x$ in $(M,d)$, show that $d(x_n,y)\to d(x,y)$ for any $y\in M$. More generally, if $x_n\to x$ and $y_n\to y$, show that $d(x_n,y_n)\to d(x,y)$.
\end{exercise}
\begin{proof}
From the hypotheses, we have that for all $\epsilon>0$, there exists an $N\in\N$ so that $d(x_n,x)<\epsilon/2$ whenever $n>N$. Likewise, there exists an $N'\in\N$ so that $d(y_n,y)<\epsilon/2$ whenever $n>N'$. Let $n>\max\set{N,N'}$, we then have 
\[
d(x_n,y_n) 
\leq d(x_n,x)+d(x,y_n) 
\leq d(x_n,x)+d(x,y)+d(y,y_n)
<\epsilon+d(x,y).
\]
Subtracting $d(x,y)$ on both sides of the inequality gives us that $d(x_n,y_n)\to d(x,y)$, as required.
\end{proof} 

\begin{exercise}{35}
In a metric space $M$ with metric $d$ show that if $x_n\to x$, then $x_{n_k}\to x$ for any subsequence $(x_{n_k})$ of $(x_n)$.
\end{exercise}
\begin{proof}
Suppose $x_n\to x$. Then for all $\epsilon>0$, there exists an $N\in\N$ so that for all $n>N$, it holds that $d(x_n,x)<\epsilon$. 

Let $(x_{n_k})$ be an arbitrary subsequence of $(x_n)$. Since $n_k$ is increasing, we can choose a $K\in\N$ so that for all $k>K$ it holds that $n_k>N$. Thus, $d(x_{n_k},x)<\epsilon$ because, as argued above, the condition holds for all $n>N$.
\end{proof} 

\begin{exercise}{36}
In a metric space $M$ with metric $d$ show that a convergent sequence is Cauchy, and a Cauchy sequence is bounded (that is, the set $\set{x_n:n\geq 1}$ is bounded).
\end{exercise}
\begin{proof}
(Convergent implies Cauchy) Let $(x_n)$ be a convergent sequence. Then for all $\epsilon>0$ there exists an $N\in\N$ so that for all $n>N$ it holds that $d(x_n,x)<\epsilon$. Let $n,m>N$. We have that $d(x_n,x_m)\leq d(x_n,x)+d(x_m,x)<\epsilon$, so that $(x_n)$ is Cauchy, as desired.

(Cauchy is bounded) Let $(x_n)$ be a Cauchy sequence. Then for all $\epsilon>0$, there exists an $N\in\N$ so that for all $n,m>N$, it holds that $d(x_n,x_m)<\epsilon$. Let $\epsilon'=\max\set{d(x_1,x_N),d(x_2,x_N),\dots,d(x_{N-1},x_N)}$, and let $r=\epsilon+2\epsilon'$. We have that $(x_n)\subseteq B_r(x_n)$. To see that, let $x_i,x_j\in B_r(x_N)$. We have three cases: (i) if $i,j>N$, then $d(x_i,x_j)<\epsilon<r$ by the Cauchy criterion, (ii) if $i,j<N$, then $d(x_i,x_j)\leq d(x_i,x_N)+d(x_N,x_j)<2\epsilon' <r$ by the definition of $\epsilon'$, and (iii) if $j<N<i$, we have that $d(x_i,x_j)\leq d(x_i,x_N)+d(x_j,x_N)<\epsilon+\epsilon'<r$.
\end{proof} 

\begin{exercise}{37}
In a metric space $M$ with metric $d$ show that a Cauchy sequence with a convergent subsequence converges.
\end{exercise}
\begin{proof}
Let $(x_n)$ be Cauchy and $x_{n_k}\to x$. Thus, for all $\epsilon>0$, there exists a $K\in\N$ so that for all $k>K$, it holds that $d(x_{n_k},x)<\epsilon/2$. Furthermore, there exists an $M\in\N$ so that for all $n,m>M$, it holds that $d(x_n,x_m)<\epsilon/2$. Now let $N=\max\set{n_K,M}$. It is then the case that $d(x_n,x) \leq d(x_n, x_{n_K})+d(x_{n_K},x) <\epsilon$, so that $(x_n)\to x$, as desired.
\end{proof} 

\begin{exercise}{38}
A sequence $(x_n)$ has a Cauchy subsequence if and only if it has a subsequence $(x_{n_k})$ for which $d(x_{n_k},x_{n_{k+1}})<2^{-k}$ for all $k$.
\end{exercise}
\begin{proof}
($\Rightarrow$) 
Suppose $(x_n)$ has a Cauchy subsequence. 
Then for all $\epsilon=2^k>0$ there exists an $K\in\N$ so that whenever $k,m>K$, it holds that $d(x_{n_k},x_{n_{m}})<2^{-k}$.
We can build the subsequence with the desired property noticing the above property holds for $m=k+1$ and any choice of $k$ with $\epsilon=2^{-k}>0$.

($\Leftarrow$)
Suppose $(x_n)$ has a subsequence $(x_{n_k})$ for which $d(x_{n_k},x_{n_{k+1}})<2^{-k}$ for all $k$. 
We will prove that $(x_{n_k})$ is itself Cauchy. 
Fix $\epsilon>0$, and choose $K\in\N$ so that $1/2^{K-1}<\epsilon$. 
We then have, for all $k,l>K$
\begin{align*}
    d(x_{n_k},x_{n_l}) 
    \leq& d(x_{n_k},x_{n_{k+1}})+\dots+d(x_{n_{l-1}},x_{n_l})\\
    \leq& \sum_{i=K}^\infty d(x_{n_i},x_{n_{i+1}})\\
    <& \sum_{i=K}^\infty 2^{-i}\\
    =& \sum_{i=0}^\infty 2^{-i}-\sum_{i=0}^{K-1}2^{-i}\\
    =& 2-(2^K-1)/2^{K-1}\\
    =& 1/2^{K-1}<\epsilon.
\end{align*}
That is, $(x_{n_k})$ is Cauchy, as desired.
\end{proof} 

\begin{exercise}{39}
In a metric space $M$ with metric $d$ show that if every subsequence of $(x_n)$ has a further subsequence that converges to $x$, then $(x_n)$ converges to $x$.
\end{exercise}
\begin{proof}
Suppose, for the sake of contradiction, that $(x_n)$ does not converge to $x$, but all subsequences of $(x_n)$ have a further subsequence that converge to $x$. The fact that $(x_n)$ does not converge to $x$ means that there exists an $\epsilon>0$ so that for all $N\in\N$, there exists an $n>N$ so that $d(x_n,x)\geq\epsilon$. We now construct a subsequence of $(x_n)$ with no further subsequence that converges to $x$. Let $x_{n_j}$ be the $j$-th element of $(x_n)$ with $d(x_{n_j},x)\geq\epsilon$. There are infinitely many such elements, given that we have at least one for all $N\in\N$. We then have that for all $k\in\N$ it holds that $d(x_{n_k},x)\geq\epsilon$ so that no subsequence of $(x_{n_k})$ converges to $x$, giving us a contradiction.
\end{proof} 

\begin{exercise}{41}
Given $x,y\in l_2$, recall that $\brackets{x,y}=\sum_{i=1}^\infty x_iy_i$. Show that if $x^{(k)}\to x$ and $y^{(k)}\to y$ in $l_2$, then $\brackets{x^{(k)},y^{(k)}}\to\brackets{x,y}$.
\end{exercise}
\begin{proof}
Before we prove the main result we will prove a helpful Lemma: if $(x^{(k)})$ is a sequence in $l_2$, then $\sup\set{\norm{x^{(k)}}: k\in\N}<\infty$. Proof: From exercise 36, we know that a convergent sequence is Cauchy and a Cauchy sequence is bounded. Furthermore, we know that the sequence is bounded if and only if for all $a\in l_2$ it holds that $\sup_{x\in (x^{(k)})}d(x, a)<\infty$. Choosing $a=0$, we have that $\sup\set{\norm{x^{(k)}}:k\in\N}=C<\infty$.

Since $x^{(k)}\to x$ and $y^{(k)}\to y$, then for $\epsilon>0$, there exist a $N,M\in\N$ so that whenever $n>N$ and $m>M$, it holds that $\norm{x^{(n)}-x}<\epsilon/\norm{y}_2$ and $\norm{y^{(m)}-y}<\epsilon/C$. Let $k>\max\set{N,M}$. We have,
\begin{align*}
    \absoluteValue{\brackets{x^{(k)}, y^{(k)}}-\brackets{x,y}}
    =& \absoluteValue{\brackets{x^{(k)}, y^{(k)}}- \brackets{x^{(k)}, y} + \brackets{x^{(k)}, y} -\brackets{x,y}}\\
    \leq& \absoluteValue{\brackets{x^{(k)}, y^{(k)}}- \brackets{x^{(k)}, y}}
    + \absoluteValue{\brackets{x^{(k)}, y} -\brackets{x,y}}\\
    =& \absoluteValue{\sum_{i=1}^\infty x^{(k)}_iy^{(k)}_i -\sum_{i=1}^\infty x^{(k)}_iy_i}
    + \absoluteValue{\sum_{i=1}^\infty x^{(k)}_iy_i -\sum_{i=1}^\infty x_iy_i}\\
    =& \absoluteValue{\sum_{i=1}^\infty x^{(k)}_i(y^{(k)}_i -y_i)}
    + \absoluteValue{\sum_{i=1}^\infty (x^{(k)}_i-x_i)y_i}\\
    \leq& \sum_{i=1}^\infty \absoluteValue{x^{(k)}_i(y^{(k)}_i -y_i)}
    + \sum_{i=1}^\infty \absoluteValue{(x^{(k)}_i-x_i)y_i}\\
    \leq& \norm{x^{(k)}}_2\norm{y^{(k)}-y}_2
    + \norm{y}_2\norm{x^{(k)}-x}_2\\
    <& \frac{\norm{x^{(k)}}_2\epsilon}{C} + \frac{\norm{y}_2\epsilon}{\norm{y}_2} < \epsilon,
\end{align*}
giving us the desired result.
\end{proof} 

\begin{exercise}{42}
Two metric $d$ and $\rho$ on a set $M$ are said to be equivalent if they generate the same convergent sequences; that is, $d(x_n,x)\to 0$ if and only if $\rho(x_n,x)\to 0$. If $d$ is any metric on $M$, show that the metrics $\rho,\sigma$ and $\tau$, defined on exercise 6, are all equivalent to $d$.
\end{exercise}
\begin{proof}
(Equivalence between $d$ and $\tau$) Notice that if a sequence $x_n\to x$ under $d$ or $\tau$, then $d(x_n,x)<1$ for $n>N$ for some $N\in\N$, but then in that case $d=\tau$, so that they are equivalent.

(Equivalence between $d$ and $\rho$) Let $x_n\to x$ under $d$, then for all $\epsilon>0$, there exists an $N\in\N$ so that for all $n>N$, it holds that $\tau(x_n,x)\leq \sqrt{d(x_n,x)}\leq d(x_n,x)<\epsilon$, so that $x_n\to x$ under $\tau$. Now let $x_n\to x$ under $\tau$, then for all $\epsilon$ there exists an $N\in\N$ so that $\tau(x_n,x) =\sqrt{d(x_n,x)} <\sqrt{\epsilon}$ whenever $n>N$. Squaring both sides of the inequality, shows us that $d(x_n,x)<\epsilon$ so that $x_n\to x$ under $d$.

(Equivalence between $d$ and $\sigma$) Let $x_n\to x$ under $d$. Then for all $\epsilon>0$, there exists an $N\in\N$ so that for all $n>N$ it holds that $\sigma(x_n,x) =d(x_n,x)/(1+d(x_n,x)) \leq d(x_n,x) <\epsilon$, so that $x_n\to x$ under $\sigma$. Now suppose $x_n\to x$ under $\sigma$. Then for all $\epsilon>0$, there exists an $N\in\N$ so that whenever $n>N$ it holds that $\sigma(x_n,x) =d(x_n,x)/(1+d(x_n,x)) <\epsilon/(1+\epsilon)$. That is, $d(x_n,x)(1+\epsilon)<\epsilon(1+d(x_n,x))$ which is the same as $d(x_n,x)<\epsilon$, so that $x_n\to x$ under $d$. 

Since the definition of equivalence of metrics is an equivalence relation, then all the metrics are equivalent to each other, and to $d$.
\end{proof} 

\begin{exercise}{43}
Show that the usual metric on $\N$ is equivalent to the discrete metric. Show that any metric on a finite set is equivalent to the discrete metric.
\end{exercise}
\begin{proof}
($\N$ with the usual metric is equivalent to the discrete metric) To see this equivalence, notice that for a sequence to be convergent in $\N$ under the usual metric or the discrete metric, the sequence needs to eventually be constant. That is, there must exist $N\in\N$ so that for all $n>N$ it holds that $x_n=x$. If this was not the case, then $x_n\not\to x$. But this is the same as saying that $\N$ under the usual metric and the discrete metric are equivalent, since a sequence converges under these metrics if and only if it is eventually constant.

(Any metric on a finite set is equivalent to the discrete metric) This follows from essentially the same argument as above. For a sequence to be convergent in a finite set under any metric, it is necessary that the sequence is eventually constant. To see this, let $X=\set{x_1,\dots,x_n}$ and $d$ be a metric defined on $X$. Let $\epsilon=\min\set{d(x_i,x_j):x_i,x_j\in X}$. The only way in which a sequence $(x_n)$ can get arbitrarily close (less than $\epsilon$) to a potential limit $x$, is by having $x_n=x$ for all $n>N$ for some 
$N\in\N$. Since this is also the case for the discrete metric, then these metrics are equivalent.
\end{proof} 

\begin{exercise}{44}
Show that the metrics induced by $\norm{\cdot}_1,\norm{\cdot}_2$, and $\norm{\cdot}_\infty$ on $\R^n$ are all equivalent. [Hint: See exercise 18].
\end{exercise}
\begin{proof}
We will first prove a Lemma: Let $(X,\norm{\cdot}_0)$ and $(X,\norm{\cdot})$ be metric spaces. If there exist constants $a,b$ such that for all $x\in X$, $a\norm{x} \leq\norm{x}_0 \leq b\norm{x}$ holds (this is the definition of equivalent norms in Kreyszig), then a sequence $(x_n)$ in $X$ converges in $(X,\norm{\cdot})$, if and only if it converges in $(X,\norm{\cdot}_0)$.

\textit{Proof}: we will only prove one direction, the other direction can be proved in a similar fashion, mutatis mutandis. Let $x_n\to x$ under $\norm{\cdot}$. Then for all $\epsilon>0$, there exists an $N\in\N$ such that if $n>N$, then it holds that $\norm{x_n-x}_0/b\leq\norm{x_n-x}<\epsilon/b$. Multiplying both sides of the inequality by $b$ gives us that $\norm{x_n-x}_0<\epsilon$. That is, $x_n\to x$ under $\norm{\cdot}_0$, as required.

Having proved this, we can use exercise 18 where we proved such constants exists between $\norm{\cdot}_1, \norm{\cdot}_2$ and $\norm{\cdot}_\infty$ on $\R^n$, so that the metrics induced by these norms are equivalent in Carothers' sense.
\end{proof} 

\begin{exercise}{46}
Given two metric spaces $(M,d)$ and $(N,\rho)$, we can define a metric on the product $M\times N$ in a variety of ways. Our only requirement is that a sequence of pairs $(a_n,x_n)$ in $M\times N$ should converge precisely when both coordinate sequences $(a_n)$ and $(x_n)$ converge (in $(M,d)$ and $(N,\rho)$, respectively). Show that each of the following define metrics on $M\times N$ enjoy this property and that all three are equivalent:
\begin{align*}
    d_1((a,x),(b,y)) =& d(a,b)+\rho(x,y)\\
    d_2((a,x),(b,y)) =& (d(a,b)^2+\rho(x,y)^2)^{1/2}\\
    d_\infty((a,x),(b,y)) =& \max\set{d(a,b),\rho(x,y)}.
\end{align*}
Henceforth, any implicit reference to ``the'' metric on $M\times N$, sometimes called the product metric, will mean one of $d_1,d_2$, or $d_\infty$. Any one of them will serve equally well; use whichever looks most convenient for the argument at hand.
\end{exercise}
\begin{proof}
(For the property) Suppose $a_n\to a$ and $x_n\to x$. Then for all $\epsilon>0$, there exist $N,M\in\N$ so that for $n>N$ and $m>M$, it holds that $d(a_n,a)<\epsilon/2$ and $\rho(x_n,x)<\epsilon/2$. Let $n>\max\set{N,M}$. Then it is true that 
\begin{align*}
    d_1((a_n,a),(x_n,x)) =& d(a_n,a)+\rho(x_n,x) <\epsilon,\\
    d_2((a_n,a),(x_n,x)) =& (d(a_n,a)^2+\rho(x_n,x)^2)^{1/2}\\
    <& ((\epsilon/2)^2+(\epsilon/2)^2)^{1/2} <\epsilon/2^{1/2}<\epsilon,\\
    d_\infty((a_n,a),(x_n,x)) =&\max\set{d(a_n,a), \rho(x_n,x)} <\epsilon/2<\epsilon.
\end{align*}

On the other hand, suppose $(a_n,x_n)\to (a,x)$ under $d_1,d_2$ and $d_\infty$. Then for all $\epsilon>0$ there exists an $N\in\N$ so that for all $n>N$, it holds that
\begin{align*}
    d(a_n,a),\rho(x_n,x)
    \leq& d(a_n,a)+\rho(x_n,x) =d_1((a_n,a),(x_n,x)) <\epsilon\\
    d(a_n,a),\rho(x_n,x)
    \leq& (d(a_n,a)^2+\rho(x_n,x)^2)^{1/2} =d_2((a_n,a),(x_n,x)) <\epsilon\\
    d(a_n,a),\rho(x_n,x)
    \leq& \max\set{d(a_n,a), \rho(x_n,x)} =d_\infty((a_n,a),(x_n,x)) <\epsilon,
\end{align*}
so that in all cases, it also holds that $a_n\to a$ and $x_n\to x$.

(Equivalence between $d_1$ and $d_\infty$) Let $(a_n,x_n)\to (a,x)$ under $d_1$. Then for all $\epsilon>0$, there exists an $N\in\N$ so that for all $n>N$, it holds that 
\begin{align*}
    d_\infty((a_n,x_n),(a,x)) 
    =& \max\set{d(a_n,a),\rho(x_n,x)}\\
    <& d(a_n,a)+\rho(x_n,x)\\
    =& d_1((a_n,x_n),(a,x)) <\epsilon.
\end{align*}
On the other hand, if $(a_n,x_n)\to (a,x)$ under $d_\infty$, then there exists an $N\in\N$ so that whenever $n>N$, it holds that $d_\infty((a_n,x_n),(a,x)) <\epsilon/2$. Hence,
\begin{align*}
    d_1((a_n,x_n),(a,x)) 
    <& d(a_n,a)+\rho(x_n,x)\\
    \leq& 2\max\set{d(a_n,a),\rho(x_n,x)}\\
    =& 2d_\infty((a_n,x_n),(a,x)) <2\epsilon/2 = \epsilon.
\end{align*}
Putting these two results together gives us that $d_1$ and $d_\infty$ are equivalent.

(Equivalence between $d_2$ and $d_\infty$) Let $(a_n,x_n)\to (a,x)$ under $d_2$. Then for all $\epsilon>0$, there exists an $N\in\N$ so that for all $n>N$, it holds that 
\begin{align*}
    d_\infty((a_n,x_n),(a,x)) 
    =& \max\set{d(a_n,a),\rho(x_n,x)}\\
    <& (d(a_n,a)^2+\rho(x_n,x)^2)^{1/2}\\
    =& d_2((a_n,x_n),(a,x)) <\epsilon.
\end{align*}
On the other hand, if $(a_n,x_n)\to (a,x)$ under $d_\infty$, then there exists an $N\in\N$ so that whenever $n>N$, it holds that $d_\infty((a_n,x_n),(a,x)) <\epsilon/\sqrt{2}$. Hence,
\begin{align*}
    d_2((a_n,x_n),(a,x)) 
    <& (d(a_n,a)^2+\rho(x_n,x)^2)^{1/2}\\
    \leq& \sqrt{2}\max\set{d(a_n,a),\rho(x_n,x)}\\
    =& \sqrt{2}d_\infty((a_n,x_n),(a,x)) <\sqrt{2}\epsilon/\sqrt{2} = \epsilon.
\end{align*}
Putting these two results together gives us that $d_2$ and $d_\infty$ are equivalent.
\end{proof} 
