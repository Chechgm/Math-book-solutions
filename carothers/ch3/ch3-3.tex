\subsection{More inequalities}

More Inequalities
24*
25*
26 (for L: generalize to infinite sums)


\begin{exercise}{24}
The conclusion of Lemma 3.7 also holds in the case $p=1$ and $q=\infty$. [Lemma 3.7. (H\"olders inequality) let $1<p<\infty$ and let $q$ be defined by $1/p+1/q=1$. Given $x\in l_p$ and $y\in l_q$, we have $\sum_{i=1}^\infty \absoluteValue{x_iy_i}\leq\norm{x}_p\norm{y}_q$]. Why?
\end{exercise}
\begin{proof}
fill
\end{proof} 

\begin{exercise}{25}
The same techniques can be used to show that $\norm{f}_p=(\int_0^1\absoluteValue{f(t)}^pdt)^{1/p}$ defines a norm on $C[0,1]$ for any $1<p<\infty$. State and prove the analogues of Lemma 3.7 and Theorem 3.8 in this case. (Does Lemma 3.7 still hold in this setting for $p=1$ and $q=\infty$?).
\end{exercise}
\begin{proof}
fill
\end{proof} 

\begin{exercise}{x}
fill
\end{exercise}
\begin{proof}
fill
\end{proof} 
