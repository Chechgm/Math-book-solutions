\section{Metric spaces}


\begin{exercise}{2}
If $d$ is a metric on $M$, show that $\absoluteValue{d(x,z)-d(y,z)}\leq d(x,y)$ for any $x,y,z\in M$.
\end{exercise}
\begin{proof}
We have $d(x,z)\leq d(x,y)+d(y,z)$, so that $d(x,z)-d(y,z)\leq d(x,y)$. On the other hand, $d(y,z)\leq d(x,y)+d(x,z)$, so that $-d(x,y)\leq d(x,z)-d(x,y)$. Combining these results together, we obtain that $\absoluteValue{d(x,z)-d(y,z)}\leq d(x,y)$.
\end{proof} 

\begin{exercise}{5}
There are other, albeit less natural choices for a metric on $\R$. For instance, check that $\rho(a,b)=\sqrt{\absoluteValue{a-b}},\sigma(a,b)=\absoluteValue{a-b}/(1+\absoluteValue{a-b})$, and $\tau(a,b)=\min\set{\absoluteValue{a-b},1}$ each define metrics on $\R$. [Hint: To show that $\sigma$ is a metric, you might first show that the function $F(t)=t/(1+t)$ is increasing and satisfies $F(s+t)\leq F(s)+F(t)$ for $s,t\geq 0$. A similar approach will also work for $\rho$ and $\tau$].
\end{exercise}
\begin{proof}
This is a Corollary to exercise 6, since $\absoluteValue{a-b}$ is a metric.
\end{proof} 

\begin{exercise}{6}
If $d$ is any metric on $M$, show that $\rho(x,y)=\sqrt{d(x,y)}$, $\sigma(x,y)=d(x,y)/(1+d(x,y))$, and $\tau(x,y)=\min\set{d(x,y),1}$ are also metrics on $M$. [Hint: $\sigma(x,y)=F(d(x,y))$, where $F$ is as in exercise 5].
\end{exercise}
\begin{proof}
(i) (ii) and (iii) For all the potential metrics, it follows that $0\leq \theta(x,y)\leq\infty$, $\theta(x,y)=0$ if and only if $x=y$ and $\theta(x,y)=\theta(y,x)$, because all of them are metrics and the functions they are applied to don't change these properties.

(iv) For $\rho$ and $\sigma$ one can prove that $\sqrt{d}$ and $F$ are increasing functions on $d$ (through their derivative, for example), so that the triangle inequality follows from the triangle inequality of $d$. For $\tau(x,y)$, we have $d(x,y)\leq d(x,z)+d(z,y)$, then $\tau(x,y) =\min\set{d(x,y),1}\leq \min\set{d(x,z)+d(z,y),1}$. If $d(x,z)+d(z,y)\geq 1$, then $\tau(x,y)\leq \tau(x,z)+\tau(z,y)$, otherwise, both $d(x,z)<1$ and $d(z,y)<1$, in which case $\tau(x,y)\leq 2=\tau(x,z)+\tau(z,y)$, as required.
\end{proof} 

\begin{exercise}{9}
Recall that $2^\N$ denotes the set of all sequences (or ``strings'') of 0s and 1s. Show that $d(a,b)=\sum_{i=1}^\infty 2^{-n}\absoluteValue{a_n-b_n}$, where $a=(a_n)$ and $b=(b_n)$ are sequences of 0s and 1s, defines a metric on $2^\N$.
\end{exercise}
\begin{proof}
(i) $d$ as defined is real valued, nonnegative and for finiteness, we have $d(a,b) =\sum_{i=1}^\infty 2^{-n}\absoluteValue{a_n-b_n} \leq \sum_{i=1}^\infty 2^{-n}=1$, where the first inequality follows from the fact that $\absoluteValue{a_n-b_n}\leq 1$ for all $n$ by definition. 

(ii) If $a=b$, then $\absoluteValue{a_n-b_n}=0$ for all $n$, and $d(a,b)=0$. On the other hand, if $a\neq b$, then for some $n$, it holds that $a_n\neq b_n$ and $\absoluteValue{a_n-b_n}\neq 0$. Since all terms in the series are positive, no element of the sum can cancel $\absoluteValue{a_n-b_n}$ so that $d(a,b)\neq 0$.

(iii) The symmetry of $d$ follows from the symmetry of the absolute value.

(iv) Let $c=(c_n)\in 2^\N$. Then 
\begin{align*}
    d(a,b) =&
    \sum_{i=1}^\infty 2^{-n}\absoluteValue{a_n-b_n}\\
    \leq& \sum_{i=1}^\infty 2^{-n}(\absoluteValue{a_n-c_n}+\absoluteValue{c_n-b_n})\\
    =& \sum_{i=1}^\infty 2^{-n}\absoluteValue{a_n-c_n}
    + \sum_{i=1}^\infty2^{-n}\absoluteValue{c_n-b_n}) 
    = d(a,c)+d(c,b).
\end{align*}
The first inequality following from the triangle inequality of the absolute value.
\end{proof} 

\begin{exercise}{12}
Check that $d(f,g)=\max_{a\leq t\leq b}\absoluteValue{f(t)-g(t)}$ defines a metric on $C[a,b]$, the collection of all continuous, real-valued functions defined on the closed interval $[a,b]$.
\end{exercise}
\begin{proof}
Let $f,g,h\in C[a,b]$.

(i) Since $\absoluteValue{x}\geq 0$ for all $x$, and continuous functions are bounded on closed sets, we have $0\leq d(f,g)\leq \infty$ for all $f,g\in C[a,b]$.

(ii) If $f=g$, then $f(x)-g(x)=0$ for all $x\in[a,b]$, so that $d(f,g)=0$. On the other hand, if $f\neq g$, then there exists an $x\in[a,b]$ so that $f(x)\neq g(x)$ and hence $d(f,g)\neq 0$.

(iii) The symmetry of $d$ follows from the symmetry of the absolute value.

(iv) We have 
\begin{align*}
d(f,g) =& \max_{x\in[a,b]}\absoluteValue{f(x)-g(x)}\\
\leq& \max_{x\in[a,b]}\absoluteValue{f(x)-h(x)}
+\absoluteValue{h(x)-g(x)}\\
\leq& \max_{x\in[a,b]}\absoluteValue{f(x)-h(x)}
+\max_{x\in[a,b]}\absoluteValue{h(x)-g(x)}\\
=& d(f,h)+d(h,g).
\end{align*}
\end{proof} 

\begin{exercise}{14}
We say that a subset $A$ of a metric space $M$ is bounded if there is some $x_0\in M$ and some constant $C<\infty$ such that $d(a,x_0)\leq C$ for all $a,\in A$. Show that a finite union of bounded sets is again bounded.
\end{exercise}
\begin{proof}
We will prove this by induction. 

Base case: Let $A$ and $B$ be bounded sets, so that $x_0\in A$ is such that $d(x_0,a)<C<\infty$ for all $a\in A$ and $y_0\in B$ is such that $d(y_0,b)<C'<\infty$ for all $b\in B$. Let $x\in A\cup B$. Then $d(x_0,x)\leq d(x_0,y_0)+d(y_0,x)$. If $x\in A$ then $d(x_0,x)<C$ by hypothesis. If $x\in B$, then $d(x_0,x)<d(x_0,y_0)+C'$ (where $d(x_0,y_0)$ is fixed). Hence, $x_0\in A\cup B$ is such that $d(x_0,x)<\max[C,d(x_0,y_0)+C']$ for all $x\in A\cup B$.

Induction case: Suppose the statement holds for the union of $k$ sets. Then $A_1\cup\dots\dots\cup A_k$ is bounded. Defining $A'=A_1\cup\dots\dots A_k$ and using the argument of the base case mutatis mutandis on $A'\cup A_{k+1}$ we obtain the desired result.
\end{proof} 

\begin{exercise}{15}
We define the diameter of a nonempty subset $A\subseteq M$ by $\text{diam}(A)=\sup\set{d(a,b):a,b\in A}$. Show that $A$ is bounded if and only if $\text{diam}(A)$ is finite. 
\end{exercise}
\begin{proof}
($\Rightarrow$) Suppose $A$ is bounded. Then there exists $x_0$ and $C<\infty$ so that $d(x_0,x)<C$ for all $x\in A$. Let $x,y\in A$ be arbitrary. We have $d(x,y)\leq d(x,x_0)+d(x_0,y)<2C$, so that $\text{diam}(A)<\infty$.

($\Leftarrow$) Let $\text{diam}(A)=\sup\set{d(a,b):a,b\in A}=K$. Let $x_0,y\in A$. Then $d(x_0,y)<K$, so that $x_0$ and $K$ are the point and the bound that make $A$ bounded.
\end{proof} 
