\section{Monotone functions}


\begin{exercise}{32}
Deduce from Theorem 2.17 that a monotone function $f:\R\to\R$ has points of continuity in every open interval.
\end{exercise}
\begin{proof}
We know that every open interval in the real line is not countable (because we can find a bijection between $(a,b)$ and $\R$. Hence, if there existed an open interval without a point of continuity, then the monotone function would have uncountably many points of discontinuity, which contradicts Theorem 2.17.
\end{proof} 

\begin{exercise}{33}
Let $f:[a,b]\to\R$ be monotone. Given $n$ distinct points $a<x_1<x_2<\dots<x_n<b$, show that $\sum_{i=1}^n\absoluteValue{f(x_i+)-f(x_i-)}\leq \absoluteValue{f(b)-f(a)}$. Use this to give another proof that $f$ has at most countably many (jump) discontinuities.
\end{exercise}
\begin{proof}
Without loss of generality, suppose $f$ is monotonically increasing. We will prove $\sum_{i=1}^n\absoluteValue{f(x_i+)-f(x_i-)}\leq \absoluteValue{f(b)-f(a)}$ by induction.

Base case: let $y\leq b$ and $x_i\in[a,y]$. From monotonicity, we have that for any point $x_i$, $f(x_i+)-f(x_i-)\leq f(y)-f(a)$ (as otherwise $f$ wouldn't be monotonous or it wouldn't be the case that $a<x_i<y$).

Suppose the statement holds for $x_k$. Then $\sum_{i=1}^k\absoluteValue{f(x_i+)-f(x_i-)}\leq \absoluteValue{f(y)-f(a)}$. We will use the same argument as above, but this time with $b$, the upper limit of the interval. By monotonicity we have that $f(x_{k+1}+)-f(x_{k+1}-)\leq f(b)-f(y)$. Now, since $\sum_{i=1}^k f(x_i+)-f(x_i-)\leq f(y)-f(a)$ (by the induction hypothesis, and the fact that $x_k<x_{k+1}$), we can add those together, giving us the desired result.

% Base case: from monotonicity, we have that for any point $x_i$, $f(x_i+)-f(x_i-)\leq f(b)-f(a)$ (as otherwise $f$ wouldn't be monotonous or it wouldn't be the case that $a<x_i<b$).

% Suppose the statement holds for $x_k$. Then $\sum_{i=1}^k\absoluteValue{f(x_i+)-f(x_i-)}\leq \absoluteValue{f(b)-f(a)}$. We will use the same argument as above. Let $y=(x_{k+1}+x_k)/2$, so that $f(x_k+)<f(y)<f(x_{k+1}-)$. By monotonicity we have that $f(x_{k+1}+)-f(x_{k+1}-)\leq f(b)-f(y)$. Now, since $\sum_{i=1}^k f(x_i+)-f(x_i-)\leq f(y)-f(a)$ (by the induction hypothesis, and the fact that $x_k<x_{k+1}$), we can add those together, giving us the desired result.

To see that $f$ has at most countably many (jump) discontinuities, let $D$ the set of all discontinuities, and define
\[
    D_n=\left\{x_i\in D: f(x_i+)-f(x_i-)\geq\frac{f(b)-f(a)}{n}\right\}.
\]
Now $D_n$ has at most $n$ elements, as otherwise, it would be the case that 
\[
f(b)-f(a)\geq\sum_{i=1}^{n+1}f(x_i+)-f(x_i-) \geq\sum_{i=1}^{n+1}\frac{f(b)-f(a)}{n} > f(b)-f(a),
\]
which is a contradiction. Since $D=\bigcup_nD_n$ and all $D_n$ are finite, then $D$ is countable, as required.
\end{proof} 