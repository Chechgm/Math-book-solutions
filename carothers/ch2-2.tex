\subsection{The Cantor set}

21*
22*
23*
26*

29*

\begin{exercise}{21}
Show that any terniary decimal of the form $0.a_1a_2\dots a_n11$ (base 3), i.e., any finite-length decimal ending in two (or more) 1s, is not an element of $\Delta$.
\end{exercise}
\begin{proof}
fill
\end{proof} 

\begin{exercise}{22}
Show that $\Delta$ contains no (nonempty) open intervals. In particular, show that if $x,y\in\Delta$ with $x<y$, there there is one $z\in[0,1]\setminus\Delta$ with $x<z<y$. (It follows from this that $\Delta$ is nowhere dense, which is another way of saying that $\Delta$ is ``small'').
\end{exercise}
\begin{proof}
fill
\end{proof} 

\begin{exercise}{23}
The endpoints of $\Delta$ are points in $\Delta$ having a finite-length base 3 decimal expansion (not necessarily in the proper form), that is, all of the points in $\Delta$ of the form $a/3^n$ for some integers $n$ and $0\leq a\leq 3^n$. Show that the endpoints of $\Delta$ other than 0 and 1 can be written as $0.a_1a_2\dots a_{n+1}$ (base 3), where each $a_k$ is 0 or 2, except $a_{n+1}$, which is either 1 or 2. That is, the discarded ``middle third'' intervals are of the form ($0.a_1a_2\dots a_{n}1,\, 0.a_1a_2\dots a_n2$) where both entries are points of $\Delta$ written in base 3.
\end{exercise}
\begin{proof}
fill
\end{proof} 

\begin{exercise}{26}
Let $f:\Delta\to[0,1]$ be the Cantor function (defined above) and let $x,y\in\Delta$ with $x<y$. Show that $f(x)\leq f(y)$. If $f(x)=f(y)$, show that $x$ has two distinct binary decimal expansions. Finally, show that $f(x)=f(y)$ if and only if $x$ and $y$ are ``consecutive'' endpoints of the form $x=0a_1a_2\dots a_n1$ and $y=0.a_1a_2\dots a_n2$ (base 3).
\end{exercise}
\begin{proof}
fill
\end{proof} 

\begin{exercise}{29}
Prove that the extended Cantor function $f:[0,1]\to[0,1]$ (as defined above) is increasing. [Hint: Consider cases].
\end{exercise}
\begin{proof}
fill
\end{proof}
