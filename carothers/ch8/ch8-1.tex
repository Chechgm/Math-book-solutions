\subsection{Compact metric spaces}

1* Interpretation of compactness in R
2* Closed and bounded is not compact in general, but really only in $R^n$
3* This is a really typical application of compactness. Compactness leans itself very well it existence arguments
4* Easy but fundamental 
6* This is a special case of a truly fundamental and powerful result, Tychonoff's theorem.
7C Not that important, but good to reinforce concepts
8C Don't be too fancy, show it is closed and bouded in the Euclidean norm
10* I love this kind of stuff, showing things are equivalent with R being complete. Incidentally, it actually suffices to assume any closed interval [a,b] is compact to get completeness. Can you show this general version?
11* Very important. We shown in previous chapter that connectedness is not relative, and neither is compact. So you are not compact IN a space, the space is compact in itself.
15* This gives another interpretation of totally bounded and compact as the sets whose closure are compact.
16* Link between totally bounded, compact and completions, very cool
17* Both separable as compact can be seen as "smallness conditions". This shows that compact spaces are smaller than separable ones.

20* Saying that all continuous functions to R are bounded is called pseudocompact. In topology, pseudocompact does not necessarily imply compact
21* Filling in details from the theory 
22*
23* These two are probably among the most useful consequences of compactness. It shows that to show homeomorphism, we don't need to check the continuity of the inverse. Used all the time in practice.
25* Compare to a previous exercise where it was shown these are connectedness.
20L Cool generalization of the Banach fixed point theorem, but it does require compactness

30* Working out details of the theory
31* Far reaching generalization of the nested set theorem 
32* Working out details of the theory
36* Another typical existence proof which is so characteristic of compactness 

\begin{exercise}{1}
fill
\end{exercise}
\begin{proof}
fill
\end{proof} 

\begin{exercise}{2}
fill
\end{exercise}
\begin{proof}
fill
\end{proof} 

\begin{exercise}{3}
fill
\end{exercise}
\begin{proof}
fill
\end{proof} 

\begin{exercise}{4}
fill
\end{exercise}
\begin{proof}
fill
\end{proof} 

\begin{exercise}{6}
fill
\end{exercise}
\begin{proof}
fill
\end{proof} 

\begin{exercise}{7}
fill
\end{exercise}
\begin{proof}
fill
\end{proof} 

\begin{exercise}{8}
fill
\end{exercise}
\begin{proof}
fill
\end{proof} 

\begin{exercise}{10}
fill
\end{exercise}
\begin{proof}
fill
\end{proof} 

\begin{exercise}{11}
fill
\end{exercise}
\begin{proof}
fill
\end{proof} 

\begin{exercise}{15}
fill
\end{exercise}
\begin{proof}
fill
\end{proof} 

\begin{exercise}{16}
fill
\end{exercise}
\begin{proof}
fill
\end{proof} 

\begin{exercise}{17}
fill
\end{exercise}
\begin{proof}
fill
\end{proof} 

\begin{exercise}{20}
fill
\end{exercise}
\begin{proof}
fill
\end{proof} 

\begin{exercise}{21}
fill
\end{exercise}
\begin{proof}
fill
\end{proof} 

\begin{exercise}{22}
fill
\end{exercise}
\begin{proof}
fill
\end{proof} 

\begin{exercise}{23}
fill
\end{exercise}
\begin{proof}
fill
\end{proof} 

\begin{exercise}{25}
fill
\end{exercise}
\begin{proof}
fill
\end{proof} 

\begin{exercise}{30}
fill
\end{exercise}
\begin{proof}
fill
\end{proof} 

\begin{exercise}{31}
fill
\end{exercise}
\begin{proof}
fill
\end{proof} 

\begin{exercise}{32}
fill
\end{exercise}
\begin{proof}
fill
\end{proof} 

\begin{exercise}{36}
fill
\end{exercise}
\begin{proof}
fill
\end{proof} 
