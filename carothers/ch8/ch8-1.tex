\subsection{Compact metric spaces}


\begin{exercise}{1}
If $K$ is a nonempty compact subset of $\R$, show that $\sup K$ and $\inf K$ are elements of $K$.
\end{exercise}
\begin{proof}
By Corollary 8.3 we know that $K$ is closed in $\R$, since $\sup K,\inf K\in \bar{K}$, then $\sup K, \inf K\in K$.
\end{proof} 

\begin{exercise}{2}
Let $E=\set{x\in\Q: 2<x^2<3}$, considered as a subset of $\Q$ (with its usual metric).
Show that $E$ is closed and bounded but not compact.
\end{exercise}
\begin{proof}
$E$ is bounded because for every $x,y\in E$, $d(x,y)=\absoluteValue{x-y}<1$.
Moreover, $E$ is closed because for every convergent sequence in $E\in\Q$ converges to some $x$ in $E$.
However, $E$ is not complete because the sequence $x_n = (2+1/n)^2$ for $n=2,3,\dots$ does not converge to any point in $\Q$ (it would converge to $\sqrt{2}\notin \Q$).
\end{proof} 

\begin{exercise}{3}
If $A$ is compact in $M$, prove that $\text{diam}(A)$ is finite.
Moreover, if $A$ is nonempty, show that there exist points $x$ and $y$ in $A$ such that $\text{diam}(A)=d(x,y)$.
\end{exercise}
\begin{proof}
If $A$ is compact in $M$, then it is totally bounded and thus by example 7.2.(a) also bounded so that by exercise 3.15 its diameter is finite.

For the second part, $\text{diam}(A)=\sup{\set{d(x,y):x,y\in A}}$, so that there exists a sequence $d(x_n,y_n)\to \text{diam}(A)$, where $x_n,y_n\in A$.
But $A$ is compact, so that by Theorem 8.2, $(x_n)$ has a convergent subsequence $x_{n_k}\to x\in A$.
Likewise, since $A$ is compact, the sequence $(y_{n_k})$ has a convergent subsequence $y_{n_{k_l}}\to y \in A$.
We have that $d(x_{n_{k_l}}, y_{n_{k_l}})$ is a subsequence of the convergent sequence $d(x_n,y_n)$, so $d(x_{n_{k_l}}, y_{n_{k_l}})$ must converge to the same limit;
that is, $d(x_{n_{k_l}}, y_{n_{k_l}}) \to d(x,y) = \text{diam}(A)$, so that $x$ and $y$, both in $A$ satisfy the desired condition. 
\end{proof} 

\begin{exercise}{4}
If $A$ and $B$ are compact sets in $M$, show that $A\cup B$ is compact.
\end{exercise}
\begin{proof}
We know that the finite union of closed sets is closed.

Furthermore, by Lemma 7.1, a set is totally bounded if and only if, given $\epsilon>0$, there are finitely many sets $A_1,\dots,A_n\subseteq A$ with $\text{diam}(A_i)<\epsilon$ for all $i$, such that $A\subseteq\bigcup A_i$.
Thus, if both $A$ and $B$ are totally bounded, we will still have a finite number of sets $A_1,\dots,A_n,B_1,\dots, B_m$ satisfying $A\cup B\subseteq \bigcup A_i \cup \bigcup B_j$, so that $A\cup B$ is totally bounded too.

Putting these two results together gives us the desired result.
\end{proof} 

\begin{exercise}{6}
If $A$ is compact in $M$ and $B$ is compact in $N$, show that $A\times B$ is compact in $M\times N$ (see exercise 3.46).
\end{exercise}
\begin{proof}
From exercise 7.17, we know that $M\times N$ is complete if and only if both $M$ and $N$ are complete. 

We will now prove that $M\times N$ is totally bounded if and only if $M$ and $N$ are totally bounded:

($\Leftarrow$)
Suppose $N$ and $M$ are totally bounded.
Then for all $\epsilon>0$, there exists a finite number of points $N'=\set{x_1,\dots,x_n}\subseteq N$ and $M'=\set{y_1,\dots,y_n}\subseteq M$ such that $N\subseteq\bigcup B_\epsilon(x_i)$ and $M\subseteq\bigcup B_\epsilon(y_i)$.
Thus, the finite set of points $N'\times M'$ as the property that $N\times M\subseteq\bigcup B_\epsilon((x,y)_i)$, because for any $(x,y)\in N\times M$, there exists a $x_i\in N'$ and a $y_j\in M'$ with $d(x_i,x)<\epsilon$ and $d(y_j,y)<\epsilon$ so that $d_\infty((x_i,y_j), (x,y)) = \max\set{d(x_i,x), d(y_j,y)}<\epsilon$.

($\Rightarrow$)
Suppose $N\times M$ is totally bounded.
Then for all $\epsilon>0$, there exists a finite set of points $(x,y)_1,\dots, (x,y)_n$ such that 
\begin{align*}
    N\times M
    \subseteq\bigcup B_\epsilon((x,y)_i) 
    = \bigcup \set{(x,y) &\in N\times M:\\
    &d_\infty((x,y)_i, (x,y)) 
    = \max\set{d(x_i,x),d(y_i,y)} 
    <\epsilon},
\end{align*}
where $x_i$ and $y_i$ correspond to $x$ and $y$ in $(x,y)_i$.
Notice, however, that this implies $N$ and $M$ are themselves totally bounded, as the set of points $x_1,\dots,x_n$ have the property that 
\begin{align*}
    N\subseteq\bigcup B_\epsilon(x_i) = \bigcup \set{x\in N: d(x_i,x) < d_\infty((x,y)_i, (x,y)) < \epsilon}.    
\end{align*}
And likewise for $M$.

Thus, we get the $N\times M$ is complete and totally bounded (hence compact) if and only if $N$ and $M$ are each complete and totally bounded (so each is compact).
\end{proof} 

\begin{exercise}{7}
If $K$ is a compact subset of $\R^2$, show that $K\subseteq [a,b]\times[c,d]$ for some pair of compact intervals $[a,b]$ and $[c,d]$.
\end{exercise}
\begin{proof}
From example 8.1, a subset $K$ of $\R^n$ is compact if and only if $K$ is closed and bounded.
Thus, since $K$ must be bounded, then there exist intervals $(a',b')$ and $(c',d')$ such that $K\subseteq (a',b')\times (c',d')$.
Furthermore, $K$ must be closed, so that we can make the intervals smaller to include the limit points of $K$, that is, there exist closed intervals $[a,b]$ and $[c,d]$ such that $K\subseteq [a,b]\times[c,d]$.
Since $[a,b]$ and $[c,d]$ are closed and bounded intervals of $\R$, they are compact, as desired.
\end{proof} 

\begin{exercise}{8}
Prove that the set $X=\set{x\in\R^n: \norm{x}_1=1}$ is compact in $\R^n$ under the euclidean norm.
\end{exercise}
\begin{proof}
(Bounded)
By exercise 3.18, we know that for all $x\in X$, $\norm{x}_2 \leq \norm{x}_1 = 1$, thus $X$ is bounded under the euclidean norm.

(Closed)
Let $(x^m_n)_{m=1,\dots}$ be a sequence in $\R^n$.
Then for all $\epsilon>0$, there exists an $M\in\N$ such that whenever $m>M$, it holds that $\norm{x^m-x}_2<\epsilon/\sqrt{n}$.
We have
\begin{align*}
    \epsilon 
    >& \sqrt{n}\norm{x^m-x}_2\\
    =& \norm{x^m-x}_1\\
    =& \sum_{i=1}^n \absoluteValue{x^m_i-x_i}\\
    \geq& \sum_{i=1}^n \absoluteValue{\absoluteValue{x^m_i}-\absoluteValue{x_i}}\\
    \geq& \sum_{i=1}^n \absoluteValue{x^m_i}-\absoluteValue{x_i}\\
    =& \sum_{i=1}^n \absoluteValue{x^m_i}- \sum_{i=1}^n\absoluteValue{x_i}\\
    =& 1- \sum_{i=1}^n\absoluteValue{x_i}.
\end{align*}
Where the first lines follows again from exercise 3.18.
Thus $\absoluteValue{\norm{x}_1-1}<\epsilon$;
that is $x\in X$, as required.

Since $X$ is closed and bounded, $X$ is compact under the euclidean norm.
\end{proof} 

\begin{exercise}{10}
Show that the Heine-Borel Theorem (closed and bounded sets in $\R$ are compact) implies the Bolzano-Weierstrass Theorem.
Conclude that the Heine-Borel Theorem is equivalent to the completeness of $\R$.
\end{exercise}
\begin{proof}
Proving that the Heine-Borel Theorem implies the Bolzano-Weierstrass in $\R$ is equivalent (by Theorem 8.2 and Theorem 1.11) to prove that every sequence in a compact subset $X\subseteq\R$, that is a sequence in a closed and bounded subset of $\R$, has a subsequence that converges to a point in $X$ implies that every bounded sequence of real numbers has a convergent subsequence.

However, the Bolzano-Weierstrass Theorem does not require the limit point of the convergent subsequence to be an element of the sequence itself. 
Thus, we let $X=(x_n)$, the set composed of all the elements of a bounded sequence, and take its closure $\bar{X}$, obtaining a closed and bounded set, which by the Heine-Borel Theorem is compact.
By Theorem 8.2, every sequence on the compact set has a convergent subsequence, so we take the sequence $(x_n)$, giving us the desired result.

The fact that Heine-Borel is equivalent to the completeness of $\R$ follows from Theorem 7.11 which says that $(M,d)$ being complete is equivalent to the Bolzano-Weierstrass Theorem.
We have just proved that the Heine-Borel Theorem implies the Bolzano-Weierstrass Theorem.
In the other direction, the Bolzano-Weierstrass trivially implies the Heine-Botel Theorem. 
To see this, take $X$ to be closed and bounded in $\R$.
By exercise 7.2, $X\subseteq\R$ is bounded if and only if it is totally bounded.
Furthermore, since $X$ is bounded, then there is a compact interval $[a,b]$ such that $X\subseteq [a,b]$, by Corollary 8.2, $X$ is compact.
\end{proof} 

\begin{exercise}{11}
Prove that compactness is not a relative property.
That is, if $K$ is compact in $M$, show that $K$ is compact in any metric space that contains it (isometrically).
\end{exercise}
\begin{proof}
To see this, we analyse whether the two conditions for compactness change in an isometric space.
Total boundedness does not change, as it is a property defined by distances between a finite number of points and the rest of the set.
Thus in a different (but isometric) space from $M$, $K$ still will be totally bounded.
Likewise completeness, the fact that Cauchy sequences converge is preserved when distances are preserved, so that $K$ would be complete in a different but isometric space.
Thus, $K$ is compact regardless of what space it is contained in, as long as the new space is isometric to the first where $K$ was originally defined.
\end{proof} 

\begin{exercise}{15}
If $A$ is a totally bounded subset of a complete metric space $M$, show that $\bar{A}$ is compact in $M$.
For this reason, totally bounded sets are sometimes called precompact or conditionally compact.
In fact, any set with compact closure might be labeled precompact.
\end{exercise}
\begin{proof}
Recall that the closure of a set is a closed set.
By Theorem 7.9, $(A,d)$ is a complete if and only if $A$ is closed in $M$, and by exercise 7.5, $A$ is totally bounded if and only if $\bar{A}$ is totally bounded.
Thus, $A$ is totally bounded and complete, so it is compact.
\end{proof} 

\begin{exercise}{16}
Show that a metric space $M$ is totally bounded if and only if its completion $\hat{M}$ is compact.
\end{exercise}
\begin{proof}
($\Rightarrow$)
Suppose $M$ is totally bounded.
To see its completion $\hat{M}$ is also totally bounded, let $\epsilon>0$, then there exist a finite number of points $x_1,\dots,x_n\in M$ so that $M\subseteq\bigcup B_\epsilon(x_i)$;
that is, for all $x\in M$, it holds that $d(x_i,x)<\epsilon$.
Now let $y\in \hat{M}$, either $y$ is in $M$, in which case $y\in\bigcup B_\epsilon(x_i)$ for some $i$, or $y$ is the limit of some Cauchy sequence, $(y_n)$ in $M$.
In the second case, we know there exists an $N\in\N$ such that $d(y_n,y)<\epsilon$ for $n\geq N$.
We then have
\begin{align*}
    d(y,x_i) \leq d(y,y_n) + d(y_n,x_i) < 2\epsilon.
\end{align*}
So that $y\in B_{2\epsilon}(x_i)$, and thus, $\hat{M}$ is totally bounded.

By definition, $\hat{M}$ is complete, so that $\hat{M}$ is totally bounded and complete and thus compact.

($\Leftarrow$)
Suppose $\hat{M}$ is compact.
Then $\hat{M}$ is totally bounded.
In exercise 7.1 we proved that a subset of a totally bounded subset is totally bounded, so that $M\subseteq\hat{M}$ is totally bounded (isometrically, of course).
\end{proof} 

\begin{exercise}{17}
If $M$ is compact, show that $M$ is also separable.
\end{exercise}
\begin{proof}
We will construct a countable dense subset of $M$.
Since $M$ is totally bounded, then for all $n\in\N$, there exists a finite set of points in $M$, $M_n=\set{x_1,\dots,x_k}$ such that $M\subseteq\bigcup B_{1/n}(x_i)$.
Thus, let our countable set be $X=\bigcup M_n$ ($X$ is countable because it is the countable union of finite sets).
To see $X$ is dense in $M$, let $y\in M$ but $\notin X$.
We can construct a sequence $(y_n)$ in $X$ that converges to $y$ by letting $y_n=x_i$ such that $y\in B_{1/n}(x_i)$, whose existence is guaranteed because of the total boundedness condition.
Thus, $X$ is countable and dense in $M$ as desired.
\end{proof} 

\begin{exercise}{20}
Let $E$ be a noncompact subset of $\R$.
Find a continuous function $f:E\to\R$ that is 
\begin{enumerate}
    \item not bounded;
    \item bounded but has no maximum value.
\end{enumerate}
\end{exercise}
\begin{proof}
\begin{enumerate}
    \item Let $E$ be noncompact because it is not complete, say $y$ is a limit of $E$ not in $E$.
    Then the function $f(x)=1/d(x,y)$ is not bounded.

    If, on the other hand, $E$ is noncompact because it is not totally bounded, then choose any point in $y\in E$, and let $f(x)=d(x,y)$, which is also not bounded because in $\R$ being bounded and totally bounded are equivalent.
    \item If $E$ is noncompact because it is not complete, then the function $f(x)=1/(1+d(x,y))$, where $y$ is a limit of $E$ not in $E$.

    On the other hand, if $E$ is noncompact because $E$ is not totally bounded, then $f(x)=\arctan(d(x,y))$ for a fixed $y\in E$ is bounded but has no maximum value.
\end{enumerate}
\end{proof} 

\begin{exercise}{21 Corollary 8.6}
If $f:[a,b]\to\R$ is continuous, then the range of $f$ is a compact interval $[c,d]$ for some $c,d\in\R$.
\end{exercise}
\begin{proof}
In Corollary 6.7 we concluded that if $I\subseteq\R$ is an interval and $f$ is a nonconstant continuous function, then $f(I)$ is an interval. 
Moreover, Theorem 8.4 tells us that the continuous image of a compact set is itself compact, so that $I$ must be compact too.
Putting these two results together gives us that $f([a,b])=[c,d]$.
\end{proof} 

\begin{exercise}{22}
If $M$ is compact and $f:M\to N$ is continuous, prove that $f$ is a closed map.
\end{exercise}
\begin{proof}
Let $f(A)$ be a subset of the range of $f$ with $A\subseteq M$.
Let $(f(x_n))$ be a convergent sequence in $f(A)$, so that $f(x_n)\to y$, for $x_n\in A$.
We want to prove that $y\in f(A)$.
Notice, first, that since $M$ is compact, then $x_n$ has a convergent subsequence $(x_{n_k})$ with $x_{n_k}\to x$.
Hence, $\lim f(x_{n_k}) = y = f(x)$, since a subsequence of an already convergent sequence converges to the same limit.
Since $A$ is closed, then $x\in A$ and thus $f(x)=y\in A$, as required.
\end{proof} 

\begin{exercise}{23}
Suppose that $M$ is compact and that $f:M\to N$ is continuous, one-to-one and onto.
Prove that $f$ is a homeomorphism.
\end{exercise}
\begin{proof}
This is a Corollary to the previous exercise (8.22) and exercise 5.57 which says that if $f$ is continuous, one-to-one, and onto, then $f$ being closed and $f^{-1}$ being continuous are equivalent.
Since $f$ is bijective and both $f$ and $f^{-1}$ are continuous, then $f$ is a homeomorphism.
\end{proof} 

\begin{exercise}{25}
Let $V$ be a normed vector space, and let $x\neq y\in V$.
Show that the map $f(t)=x-t(y-x)$ is a homeomorphism from $[0,1]$ into $V$.
The range of $f$ is the line segment joining $x$ and $y$;
it is often written $[x,y]$.
\end{exercise}
\begin{proof}
We proved this in 6.27.
\end{proof} 

\begin{exercise}{30 Lemma 8.8}
In a metric space $M$, the following are equivalent:
\begin{enumerate}
    \item If $\GGG$ is any collection of open sets in $M$ with $M\subseteq \bigcup\set{G:G\in\GGG}$, then there are finitely many sets $G_1,\dots, G_n \in \GGG$ with $M\subseteq \bigcup G_i$.
    \item If $\FFF$ is any collection of closed sets in $M$ such that $\bigcap F_i \neq \emptyset$ for all choices of finitely many sets $F_1,\dots, F_n\in \FFF$, then $\bigcap \set{F:F\in\FFF} \neq \emptyset$.
\end{enumerate}
\end{exercise}
\begin{proof}
($\Rightarrow$)
Suppose, for the sake of contradiction, that $\bigcap \set{F:F\in\FFF} = \emptyset$, so that by the DeMorgan law, we have that $\bigcup \set{F^c:F\in\FFF} = M$, however for all choices of finite sets in $\FFF$, $f_1,\dots,F_n$ it holds that $\bigcap F_i \neq \emptyset$.
The contradiction statement tells us that the set $\set{F^c:\FFF}$ is a collection of open sets in $M$, so that by the first condition and the DeMorgan law, we get that there are finitely many sets $F_1^c,\dots,F_n^c$ such that $M\subseteq \bigcup F_i^c$.
Applying again the DeMorgan law, we get that $\bigcap F_i^c \subseteq M^c = \emptyset$, arising a contradiction.

($\Leftarrow$)
We will proceed as above.
Suppose, for the sake of contradiction, that there exists a family of open sets $\GGG$ such that $\bigcup \set{G:G \in \GGG}$ but for all finite sets $G_1,\dots,G_n \in \GGG$, it is the case that $M\not\subseteq\bigcup G_i$.
Then this implies, by the DeMorgan law, that $\bigcap G_i^c \not\subseteq\emptyset$;
that is $\bigcap G_i^c \neq \emptyset$, so that by the second condition (which we assume), the family of sets $\FFF = \set{G^c:G\in\GGG}$ has the property that $\bigcap\set{F:F\in\FFF}$.
By the DeMorgan law again, we get that $\bigcup \set{G: G\in\GGG}\neq M$, giving us a contradiction.
\end{proof} 

\begin{exercise}{31}
Given an arbitrary metric space $M$, show that a decreasing sequence of nonempty compact sets in $M$ has nonempty intersection.
\end{exercise}
\begin{proof}
Let $A_1\supseteq A_2\supseteq \dots$ be a decreasing sequence of nonempty compact sets in $M$.
Let $(x_n)$ be a sequence defined as $x_n\in A_n$ for all $n$.
Since $(x_n)$ is a sequence in a compact set, it contains a subsequence that converges in the set, and furthermore, the limit must be an element of all of the sets.
Thus if $x_{n_k}\to x$, then $x\in\bigcap A_i$, as required.
\end{proof} 

\begin{exercise}{32}
Prove Corollary 8.11 by showing that the following two statements are equivalent:
\begin{enumerate}
    \item Every decreasing sequence of nonempty closed sets in $M$ has a nonempty intersection.
    \item Every countable open cover of $M$ admits a finite subcover;
    that is, if $(G_n)$ is a sequence of open sets in $M$ satisfying $M \subseteq \bigcup G_i$, then $M\subseteq \bigcup_{i=1}^N G_i$ for some finite $N$.
\end{enumerate}
\end{exercise}
\begin{proof}
($\Rightarrow$)
Suppose, for the sake of contradiction, that there is an increasing sequence $G_1 \subseteq G_2 \subseteq \dots$ of open sets with $M\subseteq\bigcup G_i$, so that there is no finite subcover;
that is, for all $N$ it holds that $M\not\subseteq \bigcup_{i=1}^N G_i$.
By the DeMorgan law, we have that for all $N$, $\bigcap_{i=1}^N G_i^c \not\subseteq \emptyset$, that is, it is not empty.
Let $F_n = \bigcap_{i=1}^n G_i^c$.
We have that $F_1 \supseteq F_2 \supseteq \dots$, and as we saw before, $F_i$ is nonempty for all $i$.
Thus by assumption, $\bigcap F_i \neq \emptyset$ so that by the DeMorgan law, $\bigcup G_i \neq M$, which contradicts the assumption that $(G_n)$ is a cover of $M$.

($\Leftarrow$)
Suppose, for the sake of contradiction, that there exists a decreasing sequence of nonempty closed sets $F_1\subseteq F_2\subseteq \dots \in M$ such that $\bigcap F_i = \emptyset$.
By the DeMorgan law, we have that $\bigcup F_i^c = M$.
Since $(F_n^c)$ is a sequence of open sets, the second condition tells us that there exists a finite $N$ so that $M \subseteq \bigcup_{i=1}^N F^c_i$.
Using again the DeMorgan law, we get that $\bigcap_{i=1}^N F_i \subseteq \emptyset$.
But if we sort $F_1,\dots,F_N$ by decreasing order (which we can because they were already decreasing), we get a contradiction: either they are not decreasing, because then there would be a point in common, are at least one of them is empty, giving us the desired result.
\end{proof} 

\begin{exercise}{36}
Let $F$ and $K$ be disjoint, nonempty sets of a metric space $M$ with $F$ closed and $K$ compact.
Show that $d(F,K) = \inf\set{d(x,y):x\in F, y\in K}>0$.
Show that this may fail if we assume only that $F$ and $K$ are disjoint closed sets.
\end{exercise}
\begin{proof}
Suppose $F$ and $K$ are disjoint and suppose $K$ is compact.
Suppose, for the sake of contradiction, that $d(F,K)=0$, which implies there exists sequences $(x_n)$ in $F$ and $(y_n)$ in $K$, such that $d(x_n,y_n) \to 0$.
Since $K$ is compact, then by Theorem 8.2, $(y_n)$ contains a subsequence $(y_{n_k})$ that converges, say to $y\in K$ (because $K$ is compact, so that the limit of convergent sequences are in $K$).
We have that the subsequence given by $d(x_{n_k}, y_{n_k}) \to 0$ as it is the subsequence of a convergent sequence.
But this also implies that $x_{n_k}\to y$.
To see this, notice that $d(x_{n_k}, y) \leq d(x_{n_k},y_{n_k}) + d(y_{n_k}, y)$, and since both sequences on the right hand side converge to 0, then $d(x_{n_k},y)$ converges to 0;
that is $x_{n_k}\to y$.
Because $F$ is closed, this means that $y\in F$, and so $F$ and $K$ are not disjoint, giving us a contradiction.

The second part is the content of exercise 4.32.
\end{proof} 
