\subsection{Uniform continuity}

65C
67C
69* Providing details to proof
70C Cool counterexample to a possible generalization of the fixed point theorems

\begin{exercise}{44}
Show that any Lipschitz map $f:(M,d)\to (N,\rho)$ is uniformly continuous. 
In particular, any isometry is uniformly continuous.
\end{exercise}
\begin{proof}
Let $f$ be Lipschitz.
Then there exists a constant $K>0$ such that $\norm{f(x)-f(y)} < K\norm{x-y}$ for all $x$ and $y$.
Let $\epsilon >0$, and choose $\delta = \epsilon/K$, we then have that for all $x$ and $y$ it holds that $\norm{f(x)-f(y)}<K\norm{x-y}=\epsilon$, as required.
\end{proof} 

\begin{exercise}{45}
Prove that every map $f:\N\to\R$ is uniformly continuous.
\end{exercise}
\begin{proof}
We have that for every $\epsilon>0$, $0<\delta<1$ is enough to have $f(B_\delta(x))\subseteq B_\epsilon(f(x))$, since $B_\delta(x)=\set{x}$.
Given that this is true for all $x$, then $f$ is uniformly continuous.
\end{proof} 

\begin{exercise}{46}
Show that $\absoluteValue{d(x,z)-d(y,z)} \leq d(x,y)$ and conclude that the map $x\mapsto d(x,z)$ is uniformly continuous on $M$ for each fixed $z\in M$.
\end{exercise}
\begin{proof}
The first part is the content of exercise 5.20.
Notice that when we proved continuity, the choice of $\delta$ did not depend on $x$ but only on $\epsilon$.
Thus, $d(x,z)$ is uniformly continuous for any fixed $z$.
\end{proof} 

\begin{exercise}{47}
Given a nonempty subset $A$ of $M$, show that $\absoluteValue{d(x,A) - d(y,A)} \leq d(x,y)$ and conclude that the map $x \mapsto d(x,A)$ is uniformly continuous on $M$.
\end{exercise}
\begin{proof}
The first part of the exercise is the content of Proposition 5.4.
To see that the map is uniformly continuous, notice that for any $\epsilon>0$, choosing $\delta=\epsilon$ suffices to satisfy $\absoluteValue{d(x,A) - d(y,A)}< \epsilon$ for all $x$ and $y$.
\end{proof} 

\begin{exercise}{48}
Prove that a uniformly continuous map sends Cauchy sequences to Cauchy sequences.
\end{exercise}
\begin{proof}
Let $f:M\to N$ be uniformly continuous and $(x_n)$ a Cauchy sequence in $M$.
Thus for $\epsilon>0$, there exists a $\delta>0$ so that for all $x,y\in M$ we have $\norm{f(x)-f(y)}<\epsilon$ whenever $\norm{x-y}<\delta$.
Let $N\in\N$ be such that whenever $n,m>N$, it holds that $\norm{x_n-x_m}<\delta$.
Then we have that $\norm{f(x_n)-f(x_m)}<\epsilon$, giving us the Cauchy sequence $(f(x_n))$, as required.
\end{proof} 

\begin{exercise}{49}
Show that the sum of two uniformly continuous maps is uniformly continuous.
Is the product of uniformly continuous maps always uniformly continuous?
Explain.
\end{exercise}
\begin{proof}
The sum of two continuous functions is continuous:
\begin{align*}
    \norm{f(x)+g(x) - f(y)-g(y)}
    \leq& \norm{f(x)-f(y)} +\norm{g(x)-g(y)}.
\end{align*}
Because both $f$ and $g$ are uniformly continuous, it suffices to choose the minimum of the $\delta_f$ and $\delta_g$ that makes the norm of the difference to be $\epsilon/2$ close.

The product of to uniformly continuous functions might not be uniformly continuous.
We have 
\begin{align*}
    \absoluteValue{f(x)g(x) - f(y)g(y)} 
    =& \absoluteValue{f(x)g(x) + f(x)g(y) - f(x)g(y) - f(y)g(y)}\\
    =& \absoluteValue{f(x)[g(x)-g(y)] - g(y)[f(y)-f(x)]}\\
    \leq& \absoluteValue{f(x)[g(x)-g(y)]} + \absoluteValue{g(y)[f(y)-f(x)]}\\
    \leq& \absoluteValue{f(x)}\absoluteValue{g(x)-g(y)} + \absoluteValue{g(y)}\absoluteValue{f(y)-f(x)}.
\end{align*}
Although the function is continuous, the function might not be uniformly continuous, given that the choice of the proximity between $x$ and $y$ depends on the values of the functions evaluated at $x$ and $y$.
A concrete example is $f(x)=g(x)=x$ in $\R$.
\end{proof} 

\begin{exercise}{54}
Let $E$ be a bounded, noncompact subset of $\R$.
Show that there exists a continuous function $f:E\to\R$ that is not uniformly continuous.
\end{exercise}
\begin{proof}
Let $E$ be a bounded noncompact subset of $\R$, so that $E$ is not closed.
Because $E$ is not closed, then there exists a limit point of $E$, say $a$ not in $E$.
The function $f(x)=1/d(x,a)$ is not uniformly continuous (for a general metric space $E$).
\end{proof} 

\begin{exercise}{55}
Give an example of a bounded continuous map $f:\R\to\R$ that is not uniformly continuous.
Can an unbounded continuous function $f:\R\to\R$ be uniformly continuous?
Explain.
\end{exercise}
\begin{proof}
The function $f(x)=\sin(x^2)$ is bounded but not uniformly continuous.

A continuous unbounded function from $\R$ to $\R$ can be uniformly continuous, an example is the identity.
\end{proof} 

\begin{exercise}{56}
Prove that $f:(M,d)\to(N,\rho)$ is uniformly continuous if and only if $\rho(f(x_n),f(y_n))\to 0$ for any pair of sequences $(x_n)$ and $(y_n)$ in $M$ satisfying $d(x_n,y_n)\to 0$.
[Hint: For the backward implication, assume that $f$ is not uniformly continuous and work towards a contradiction].
\end{exercise}
\begin{proof}
($\Rightarrow$)
Suppose $f$ is uniformly continuous.
Then for all $\epsilon>0$, there exists a $\delta>0$ so that for all $x$ and $y$ it holds that $\rho(f(x),f(y))<\epsilon$, whenever $d(x,y)<\delta$.
Since $d(x_n,y_n) \to 0$, then there exists an $N\in\N$, such that for $n>N$, it holds that $d(x_n,y_n)<\delta$.
But putting these two conclusions together, we have that for $n>N$, the condition $\rho(f(x_n),f(y_n))<\epsilon$ holds, given that the condition $d(x_n,y_n)<\delta$ holds too.

($\Leftarrow$)
Following the hint, suppose for the sake of contradiction that $f$ is not uniformly continuous.
Then there exists an $\epsilon>0$, such that for all $\delta>0$ we can find a pair $x,y$ such that even though $d(x,y)<\delta$, we still have that $\rho(f(x),f(y))\geq\epsilon$.
Let $\epsilon>0$ and $\delta=1/n$, moreover, let $(x_n)$ and $(y_n)$ be such that $x_n\to x$ and $y_n\to y$, where $x$ and $y$ are given by the non-uniform continuity of $f$.
Thus $d(x_n,y_n)<1/n$ and $\rho(f(x_n), f(y_n))\geq \epsilon$.
But this contradicts that $d(x_n,y_n)\to 0$ implies $\rho(f(x_n),f(y_n))\to 0$;
indeed $\lim_{n\to\infty}\rho(f(x_n),f(y_n)) \geq \epsilon$, a contradiction.
\end{proof} 

\begin{exercise}{57}
A function $f:\R \to \R$ is said to satisfy a Lipschitz condition of order $\alpha$, where $\alpha>0$, if there is a constant $K<\infty$ such that $\absoluteValue{f(x)-f(y)}\leq K\absoluteValue{x-y}^\alpha$ for all $x,y$.
Prove that such a function is uniformly continuous.
\end{exercise}
\begin{proof}
In exercise 59 we prove that a function with Lipschitz condition with $\alpha\geq 1$ is constant and hence uniformly continuous.
Suppose $f$ has the Lipschitz condition for $\alpha>0$.
Then $\absoluteValue{f(x)-f(y)} \leq K\absoluteValue{x-y}^\alpha$, then choosing $\delta<\sqrt[\alpha]{\epsilon}/K$ suffices, given that then $\absoluteValue{f(x)-f(y)} \leq K\absoluteValue{x-y}^\alpha < K(\sqrt[\alpha]{\epsilon})^\alpha/K = \epsilon$ for all $x$ and $y$, as required.
\end{proof} 

\begin{exercise}{58}
Show that any function $f:\R\to\R$ having a bounded derivative is Lipschitz of order 1.
[Hint: Use the mean value Theorem].
\end{exercise}
\begin{proof}
Suppose $f$ has a bounded derivative, so that $\absoluteValue{f'(x)}<K$ for a real $K$.
Let $[x,y]$ be an arbitrary closed interval.
By the mean value Theorem, we have that there exists a $c\in[x,y]$, such that 
\begin{align*}
    \absoluteValue{f'(c)} 
    = \frac{\absoluteValue{f(y)-f(x)}}
    {\absoluteValue{y-x}} < K,
\end{align*}
so that $\absoluteValue{f(y)-f(x)}<K\absoluteValue{y-x}$; 
that is, $f$ is Lipschitz of order 1.
\end{proof} 

\begin{exercise}{59}
The Lipschitz condition is interesting only for $\alpha \leq 1$;
show that a function satisfying a Lipschitz condition of order $\alpha>1$ is constant.
\end{exercise}
\begin{proof}
By the Lipschitz condition, we have that
\begin{align*}
    K\absoluteValue{x-y}^{\alpha-1} 
    = \frac{f(x)-f(y)}{x-y}.
\end{align*}
Taking the limit as $y\to x$ on both sides of the equation gives us that $f'(x)=\lim_{y\to x}K\absoluteValue{x-y}^{\alpha-1}=0$ for any $\alpha\geq 1$;
that is, $f$ is constant for all $x$.
\end{proof} 

\begin{exercise}{61}
Two metric spaces $(M,d)$ and $(N,\rho)$ are said to be uniformly homeomorphic if there is a one-to-one and onto map $f:M\to N$ such that both $f$ and $f^{-1}$ are uniformly continuous.
In this case we say that $f$ is a uniform homeomorphism.
Prove that completeness is preserved by uniform homeomorphisms.
\end{exercise}
\begin{proof}
In 48 we proved that uniformly continuous functions preserve Cauchy sequences, so that if a Cauchy sequence $(x_n)$ in $M$ converges to a point in $M$, then the sequence $(f(x_n))$ in $N$ converges to a point in $N$.
That is, $N$ is complete.
\end{proof} 

\begin{exercise}{62}
Two metrics $d$ and $\rho$ on a set $M$ are said to be uniformly equivalent if the identity map between $(M,d)$ and $(M,\rho)$ is uniformly continuous in both directions (i.e., if the identity map is a uniform homeomorphism).
If there are constants $0<c<C<\infty$ such that $c\rho(x,y)<d(x,y)<C\rho(x,y)$ for every points $x, y\in M$, prove that $d$ and $\rho$ are uniformly equivalent.
\end{exercise}
\begin{proof}
Let $f$ be the identity in $M$.
$f$ is trivially a bijection. 
We will prove that $f$ is uniformly continuous, the proof that $f^{-1}$ is also uniformly continuous (and thus $d$ and $\rho$ are uniformly equivalent) follows mutatis mutandis using the same strategy.

Let $\epsilon>0$ and $\delta<c\epsilon$.
We have $\rho(f(x),f(y)) = \rho(x,y) \leq d(x,y)/c < \epsilon$, for all $x,y\in M$, thus $f$ is uniformly continuous.
\end{proof} 

\begin{exercise}{65}
If $f:(0,1)\to\R$ is continuous, and if both $f(0+)=\lim_{x^+\to 0}f(x)$ and $f(-1)$ exist, show that the function $F$ defined by $F(0)=f(0+), F(1)=f(1-)$, and $F(x)=f(x)$ of $0<x<1$ is uniformly continuous on $[0,1]$.
\end{exercise}
\begin{proof}
Theorem 8.15
\end{proof} 

\begin{exercise}{67}
Define $f:l_2\to l_1$ by $f(x)=(x_n/n)_{n=1}^\infty$.
Show that $f$ is uniformly continuous.
\end{exercise}
\begin{proof}
fill
\end{proof} 

\begin{exercise}{69}
fill
\end{exercise}
\begin{proof}
fill
\end{proof} 

\begin{exercise}{70}
fill
\end{exercise}
\begin{proof}
fill
\end{proof} 
