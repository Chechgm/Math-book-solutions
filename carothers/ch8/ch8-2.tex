\subsection{Uniform continuity}

44* Lipschitz is stronger than uniform continuity and is the strongest version of continuity we have
45* Easy but cool
46*
47* Two important examples of uniform continuity 
48* This is not equivalent to uniform continuity (counterexample?)
49* Some elementary structure 
54* In fact, for a metric space, compact is equivalent to saying all continuous real-valued maps on it are uniformly continuous!
55* Bittersweet memories since this was once my exam question and I couldn't solve it in time 
56* This is the sequential characterization of uniformly continuous which I personally find extremely interesting 
57* This class of maps is very important for the future and try to interpolate between Lipschitz (of order 1) and uniformly continuous
58* So bounded differentiable => Lipschitz of order 1 => Lipschitz of order a for decreasing a => uniformly continuous => continuous 
59* I got a cool story related to this 
61* There are three ways for two metric spaces to be "isomorphic". They can be homeomorphic, uniformly homeomorphic or isometric. These three give rise to three different categories of metric spaces. The fact that metric spaces don't have a unique category (as opposed to groups which do) makes category theory not as useful in analysis as in topology or abstract algebra!
62* There are also three ways metrics can be equivalent. This is covered later.

65C
67C
69* Providing details to proof
70C Cool counterexample to a possible generalization of the fixed point theorems

\begin{exercise}{44}
fill
\end{exercise}
\begin{proof}
fill
\end{proof} 

\begin{exercise}{45}
fill
\end{exercise}
\begin{proof}
fill
\end{proof} 

\begin{exercise}{46}
fill
\end{exercise}
\begin{proof}
fill
\end{proof} 

\begin{exercise}{47}
fill
\end{exercise}
\begin{proof}
fill
\end{proof} 

\begin{exercise}{48}
fill
\end{exercise}
\begin{proof}
fill
\end{proof} 

\begin{exercise}{49}
fill
\end{exercise}
\begin{proof}
fill
\end{proof} 

\begin{exercise}{55}
fill
\end{exercise}
\begin{proof}
fill
\end{proof} 

\begin{exercise}{56}
fill
\end{exercise}
\begin{proof}
fill
\end{proof} 

\begin{exercise}{57}
fill
\end{exercise}
\begin{proof}
fill
\end{proof} 

\begin{exercise}{58}
fill
\end{exercise}
\begin{proof}
fill
\end{proof} 

\begin{exercise}{59}
fill
\end{exercise}
\begin{proof}
fill
\end{proof} 

\begin{exercise}{61}
fill
\end{exercise}
\begin{proof}
fill
\end{proof} 

\begin{exercise}{62}
fill
\end{exercise}
\begin{proof}
fill
\end{proof} 

\begin{exercise}{65}
fill
\end{exercise}
\begin{proof}
fill
\end{proof} 

\begin{exercise}{67}
fill
\end{exercise}
\begin{proof}
fill
\end{proof} 

\begin{exercise}{69}
fill
\end{exercise}
\begin{proof}
fill
\end{proof} 

\begin{exercise}{70}
fill
\end{exercise}
\begin{proof}
fill
\end{proof} 
