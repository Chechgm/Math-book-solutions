\section{Discontinuous functions}


\begin{exercise}{1}
If $f$ is increasing, show that $\omega_f(a)=f(a+)-f(a-)$.
\end{exercise}
\begin{proof}
We have that 
\begin{align*}
    \omega_f(a) = \lim_{h\to0+}\omega(f;(a-h,a+h)) = \lim_{h \to 0+} \sup\set{\absoluteValue{f(x)-f(y)}: x,y\in (a-h,a+h)}.
\end{align*}
Since $f$ is increasing, we have that for all $h>0$, $f(a-h)<f(a)$ and $f(a)<f(a+h)$, so that $\absoluteValue{f(a+h)-f(a-h)} = f(a+h)-f(a-h)$.
Taking the limit as $h$ goes to 0 gives us the desired result.
\end{proof} 

\begin{exercise}{2}
Prove that $f$ is continuous at $a$ if and only if $\omega_f(a)=0$.
\end{exercise}
\begin{proof}
($\Rightarrow$)
Suppose $f$ is continuous.
Then for all $\epsilon>0$, we can find a $\delta>0$ such that whenever $\norm{a-x}<\delta$, it holds that $\norm{f(a)-f(x)}<\epsilon$. 
Since this holds for all $\epsilon$, then $\omega_f(a) = \inf_{a\in I, I\text{ open}}\omega(f; I) = 0$.

($\Leftarrow$)
Suppose $\omega_f(a)= \lim_{h\to 0+}\text{ diam}f(B_h(a)) = 0$, then for $\epsilon>0$, we can find an $h>0$, so that whenever $x\in B_h(a)$, it holds that $f(x) \in B_\epsilon(f(B_h(a))$;
that is, $f$ is continuous.
\end{proof} 

\begin{exercise}{5}
If $A$ is a subset of $\R$, and if $x$ is in the interior of $A$, show that $x$ is a point of continuity for $\chi_A$ (the characteristic function of $A$).
Are there any other points of continuity?
\end{exercise}
\begin{proof}
Recall that $\text{int}(A)=\set{x\in A:B_\epsilon(x)\subseteq A\text{ for some }\epsilon>0}$.
Thus, $\omega_{\chi_A}(x) = \lim_{h\to 0+}\text{diam}\chi_A(B_h(x)) = 0$, and by exercise 2, $\chi_A$ is continuous in $x$.
The other points of continuity of $\chi_A$ are the interior of the complement of $A$.
\end{proof} 

\begin{exercise}{9}
If $E$ is a closed set in $\R$, show that $E=D(f)$ for some bounded function $f$.
[Hint: A sum of two characteristic functions will do the trick].
\end{exercise}
\begin{proof}
Consider the function given by $f(x) = 0$ if $x \in \R \setminus E$, $f(x) = 1$ if $x \in E\cap\Q$, and $f(x) = 2$ if $x \in E\cap(\R \setminus \Q)$.
This set is exactly discontinuous at $E$.
\end{proof} 

\begin{exercise}{12}
More generally, in any metric space, show that every open set is an $F_\sigma$ and that every closed set is a $G_\delta$.
\end{exercise}
\begin{proof}
To see the closed set $A\subseteq X$ is a $G_\delta$, let $A_n=\set{x\in X: d(x,A)<1/n}$, each of which are open.
We have that $\bigcap A_n = A$, since $a\in A$ if and only if $d(a,A)=0$ which is the case given the previous countable intersection.

Now to see an open set $B\subseteq X$ is $F_\sigma$, consider $B^C$ which is open.
We know we can write $B^C$ as the intersection of open sets $B_n^C$, so that $B^C = \bigcap B_n^C$ and thus $B = \parens{\bigcap B_n^C}^C = \bigcup B_n$, where $B_n$ is closed for all $n$. 
Thus $B$ is $F_\sigma$, as required.
\end{proof} 
