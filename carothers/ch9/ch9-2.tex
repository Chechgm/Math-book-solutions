\section{The Baire category Theorem}

16* Link between nowhere dense and isolated point, good for intuition 
19* Interesting characterizations of nowhere dense 
21C Some interesting properties for your intuition 
24*
25*
26* Filling in details from the theory 
28* This makes sense since first category sets are intuitive "small sets". This property makes sense in light of that intuition
29* Second category sets are intuitively big sets 
30* first category is a relative property!
31* Showing completeness is crucial in Baire 
32* 
33* Test your intuition! 
36* Relate this again to the notion of big/small 
37* Again related to big/small intuition. First category is so small that its complement is big in the sense of cardinality 
38* Nice example 
40C Seeing how these notions behave with functions 
42C Cool way to extend Baire to other spaces 
43* This suggest a topological property. Baire spaces are the ones that satisfy the conclusion of Baire's theorem.
46* This property would not at all be easy to prove without Baire!
47* This shows how Baire's theorem can be used to show spaces not being complete, see exercise 49
48* I think this should be intuitive enough, but still not easy to prove 
49* What is more, is that V does not admit ANY norm that makes it a complete normed space! This is an example of a vector space which can never be made into a Banach space, no matter what norm you pick. Very cool. 


\begin{exercise}{14}
Prove that $A$ has an empty interior in $M$ if and only if $A^C$ is dense in $M$.
\end{exercise}
\begin{proof}
This is the content of exercise 4.46.
\end{proof} 

\begin{exercise}{15}
If $G$ is open and dense in $\R$, show that the same is true of $G\setminus\set{x}$ for any $x\in\R$. 
Is this true in any metric space?
Explain.
\end{exercise}
\begin{proof}
Let $G$ be open and dense in $\R$, so that $\bar{G}=\R$.
Thus, for any $\epsilon>0$, and any $x\in\R$, $B_\epsilon(x)\cap G\neq \emptyset$.
In fact, $B_\epsilon(x)\cap G$ is infinite.
If this were not the case, say if $B_\epsilon(x)\cap G = A$ was finite, then we could take $\epsilon' < \min\set{d(x,a): a\in A}$, and certainly $B_{\epsilon'}(x)\cap G$ would be empty. 
Thus, $G\setminus\set{x}$ must be dense in $\R$ as well.

This is not true in every metric space.
Consider the space $M=[0,1]\cup \set{10}$ with the usual metric of $\R$.
The set $G = (\Q\cap[0,1]) \cup \set{10}$ is dense in $M$ but certainly $G\setminus\set{10}$ is not.
The important characteristic of this metric space is that $10$ is an isolated point.
The statement holds for any infinite metric space without isolated points.
\end{proof} 

\begin{exercise}{16}
Show that $\set{x}$ is nowhere dense in $M$ if and only if $x$ is not an isolated point of $M$.
\end{exercise}
\begin{proof}
($\Rightarrow$)
Suppose $x$ is an isolated point of $M$.
Then there exists an $\epsilon > 0$, such that $B_\epsilon(x)\cap M = \set{x}$.

A set is nowhere dense if its closure contains no nonempty open set.
Alternatively, a set is nowhere dense if the complement of its closure is open and dense in the space.

($\Leftarrow$)
Suppose $\set{x}$ is not nowhere dense in $M$.


\end{proof} 

\begin{exercise}{19}
fill
\end{exercise}
\begin{proof}
fill
\end{proof} 

\begin{exercise}{21}
fill
\end{exercise}
\begin{proof}
fill
\end{proof} 

\begin{exercise}{24}
fill
\end{exercise}
\begin{proof}
fill
\end{proof} 

\begin{exercise}{25}
fill
\end{exercise}
\begin{proof}
fill
\end{proof} 

\begin{exercise}{26}
fill
\end{exercise}
\begin{proof}
fill
\end{proof} 

\begin{exercise}{28}
fill
\end{exercise}
\begin{proof}
fill
\end{proof} 

\begin{exercise}{29}
fill
\end{exercise}
\begin{proof}
fill
\end{proof} 

\begin{exercise}{30}
fill
\end{exercise}
\begin{proof}
fill
\end{proof} 

\begin{exercise}{31}
fill
\end{exercise}
\begin{proof}
fill
\end{proof} 

\begin{exercise}{32}
fill
\end{exercise}
\begin{proof}
fill
\end{proof} 

\begin{exercise}{33}
fill
\end{exercise}
\begin{proof}
fill
\end{proof} 

\begin{exercise}{36}
fill
\end{exercise}
\begin{proof}
fill
\end{proof} 

\begin{exercise}{37}
fill
\end{exercise}
\begin{proof}
fill
\end{proof} 

\begin{exercise}{38}
fill
\end{exercise}
\begin{proof}
fill
\end{proof} 

\begin{exercise}{40}
fill
\end{exercise}
\begin{proof}
fill
\end{proof} 

\begin{exercise}{42}
fill
\end{exercise}
\begin{proof}
fill
\end{proof} 

\begin{exercise}{43}
fill
\end{exercise}
\begin{proof}
fill
\end{proof} 

\begin{exercise}{46}
fill
\end{exercise}
\begin{proof}
fill
\end{proof} 

\begin{exercise}{47}
fill
\end{exercise}
\begin{proof}
fill
\end{proof} 

\begin{exercise}{48}
fill
\end{exercise}
\begin{proof}
fill
\end{proof} 

\begin{exercise}{49}
fill
\end{exercise}
\begin{proof}
fill
\end{proof} 