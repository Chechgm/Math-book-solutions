\subsection{fill}

The Baire Category Theorem
14* Useful for intuition, it means properties of nowhere dense and everywhere dense are very similar 
15* Good for intuition 
16* Link between nowhere dense and isolated point, good for intuition 
17L Very cool result and yet a different way to prove uncountability of R 
18L I like this result, since it seems something intuitively plausible, and it is satisfying to see that a weird counterexample doesn't screw it up
19* Interesting characterizations of nowhere dense 
21C Some interesting properties for your intuition 
24*
25*
26* Filling in details from the theory 
28* This makes sense since first category sets are intuitive "small sets". This property makes sense in light of that intuition
29* Second category sets are intuitively big sets 
30* first category is a relative property!
31* Showing completeness is crucial in Baire 
32* 
33* Test your intuition! 
36* Relate this again to the notion of big/small 
37* Again related to big/small intuition. First category is so small that its complement is big in the sense of cardinality 
38* Nice example 
40C Seeing how these notions behave with functions 
42C Cool way to extend Baire to other spaces 
43* This suggest a topological property. Baire spaces are the ones that satisfy the conclusion of Baire's theorem.
44L Very cool extension of Baire's theorem
45L Compare this to space-filling functions which are as far as possible removed from these functions. They cannot be injective then.
46* This property would not at all be easy to prove without Baire!
47* This shows how Baire's theorem can be used to show spaces not being complete, see exercise 49
48* I think this should be intuitive enough, but still not easy to prove 
49* What is more, is that V does not admit ANY norm that makes it a complete normed space! This is an example of a vector space which can never be made into a Banach space, no matter what norm you pick. Very cool. 


\begin{exercise}{x}
fill
\end{exercise}
\begin{proof}
fill
\end{proof} 

\begin{exercise}{x}
fill
\end{exercise}
\begin{proof}
fill
\end{proof} 

\begin{exercise}{x}
fill
\end{exercise}
\begin{proof}
fill
\end{proof} 

\begin{exercise}{x}
fill
\end{exercise}
\begin{proof}
fill
\end{proof} 

\begin{exercise}{x}
fill
\end{exercise}
\begin{proof}
fill
\end{proof} 

\begin{exercise}{x}
fill
\end{exercise}
\begin{proof}
fill
\end{proof} 

\begin{exercise}{x}
fill
\end{exercise}
\begin{proof}
fill
\end{proof} 

\begin{exercise}{x}
fill
\end{exercise}
\begin{proof}
fill
\end{proof} 

\begin{exercise}{x}
fill
\end{exercise}
\begin{proof}
fill
\end{proof} 

\begin{exercise}{x}
fill
\end{exercise}
\begin{proof}
fill
\end{proof} 

\begin{exercise}{x}
fill
\end{exercise}
\begin{proof}
fill
\end{proof} 

\begin{exercise}{x}
fill
\end{exercise}
\begin{proof}
fill
\end{proof} 

\begin{exercise}{x}
fill
\end{exercise}
\begin{proof}
fill
\end{proof} 

\begin{exercise}{x}
fill
\end{exercise}
\begin{proof}
fill
\end{proof} 

\begin{exercise}{x}
fill
\end{exercise}
\begin{proof}
fill
\end{proof} 

\begin{exercise}{x}
fill
\end{exercise}
\begin{proof}
fill
\end{proof} 

\begin{exercise}{x}
fill
\end{exercise}
\begin{proof}
fill
\end{proof} 

\begin{exercise}{x}
fill
\end{exercise}
\begin{proof}
fill
\end{proof} 

\begin{exercise}{x}
fill
\end{exercise}
\begin{proof}
fill
\end{proof} 