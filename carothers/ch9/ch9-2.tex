\section{The Baire category Theorem}

16* Link between nowhere dense and isolated point, good for intuition. Solve exercise 4.54...
19* Interesting characterizations of nowhere dense. Solve exercise 4.60

26* Filling in details from the theory 

28* This makes sense since first category sets are intuitive "small sets". This property makes sense in light of that intuition
29* Second category sets are intuitively big sets 
30* first category is a relative property!
31* Showing completeness is crucial in Baire 
32* 

33* Test your intuition! 
36* Relate this again to the notion of big/small 
37* Again related to big/small intuition. First category is so small that its complement is big in the sense of cardinality 
38* Nice example 
40C Seeing how these notions behave with functions 
42C Cool way to extend Baire to other spaces 
43* This suggest a topological property. Baire spaces are the ones that satisfy the conclusion of Baire's theorem.
46* This property would not at all be easy to prove without Baire!
47* This shows how Baire's theorem can be used to show spaces not being complete, see exercise 49
48* I think this should be intuitive enough, but still not easy to prove 
49* What is more, is that V does not admit ANY norm that makes it a complete normed space! This is an example of a vector space which can never be made into a Banach space, no matter what norm you pick. Very cool. 


\begin{exercise}{14}
Prove that $A$ has an empty interior in $M$ if and only if $A^C$ is dense in $M$.
\end{exercise}
\begin{proof}
This is the content of exercise 4.46.
\end{proof} 

\begin{exercise}{15}
If $G$ is open and dense in $\R$, show that the same is true of $G\setminus\set{x}$ for any $x\in\R$. 
Is this true in any metric space?
Explain.
\end{exercise}
\begin{proof}
Let $G$ be open and dense in $\R$, so that $\bar{G}=\R$.
Thus, for any $\epsilon>0$, and any $x\in\R$, $B_\epsilon(x)\cap G\neq \emptyset$.
In fact, $B_\epsilon(x)\cap G$ is infinite.
If this were not the case, say if $B_\epsilon(x)\cap G = A$ was finite, then we could take $\epsilon' < \min\set{d(x,a): a\in A}$, and certainly $B_{\epsilon'}(x)\cap G$ would be empty. 
Thus, $G\setminus\set{x}$ must be dense in $\R$ as well.

This is not true in every metric space.
Consider the space $M=[0,1]\cup \set{10}$ with the usual metric of $\R$.
The set $G = (\Q\cap[0,1]) \cup \set{10}$ is dense in $M$ but certainly $G\setminus\set{10}$ is not.
The important characteristic of this metric space is that $10$ is an isolated point.
The statement holds for any infinite metric space without isolated points.
\end{proof} 

\begin{exercise}{16}
Show that $\set{x}$ is nowhere dense in $M$ if and only if $x$ is not an isolated point of $M$.
\end{exercise}
\begin{proof}
This is the content of exercise 4.54.
\end{proof} 

\begin{exercise}{19}
Show that each of the following is equivalent to the statement ``$A$ is nowhere dense'':
\begin{enumerate}
    \item $\bar{A}$ contains no nonempty open set.
    \item Each nonempty open set in $M$ contains a nonempty open subset that is disjoint from $A$.
    \item Each nonempty open set in $M$ contains an open ball that is disjoint from $A$.
\end{enumerate}
\end{exercise}
\begin{proof}
This is the content of exercise 4.60.
\end{proof} 

\begin{exercise}{21}
If $x_n \to x$ in $\R$, show that the set $E = \set{x} \cup \set{x_n: n \geq 1}$ is nowhere dense in $\R$.
Is the same true if $\R$ is replaced by an arbitrary metric space $M$?
Is every countable set nowhere dense?
Explain
\end{exercise}
\begin{proof}
Let $\epsilon >0$ and $x \in E$.
We have that $B_\epsilon(x) \not\subseteq E$, since $B_\epsilon(x)$ is an uncountable set (Theorem 2.9), while $E$ is not.
Thus, $E = \bar{E}$ is nowhere dense, as required.

This is not true for all metric spaces, a discrete metric space being a simple example.
Likewise, it is not true that every countable set is nowhere dense.
In example 9.9.b we discussed how $\Q$, is not nowhere dense in $\R$.
\end{proof} 

\begin{exercise}{24}
If $\R = \bigcup E_n$, then the closure of some $E_n$ contains an interval;
that is, $\operatorname{int}(\bar{E}) \neq \emptyset$ for some $n$.
\end{exercise}
\begin{proof}
We will use a proof technique similar to that of Corollary 9.6.
Because $\R = \bigcup E_n$, we have that $\R = \bigcup \bar{E}_n$, so that for the open sets $G_n = \R \setminus \bar{E}_n$, it holds that $\bigcap G_n = \emptyset$.
By the Baire's Category Theorem, it must be the case that some $G_n$ is not dense, so that it misses an entire open interval. 
Thus $\bar{E}_n$ contains an entire open interval.
\end{proof} 

\begin{exercise}{25}
If $\R \setminus \Q = \bigcup E_n$, then the closure of some $E_n$ contains an interval.
Deduce that the conclusion of Baire's Theorem holds for $\R \setminus \Q$.
\end{exercise}
\begin{proof}
We have that $\R \setminus \Q \subseteq \bigcup \bar{E}_n$, so that by adding the (at most countably many) points in $\R \setminus \bigcup \bar{E}_n$ for $\R$, we get that $\R = \bigcup \bar{E}_n \cup \bigcup \set{x_n}$.
Thus, for a sequence of open sets $G_n$, we have that $\emptyset = \bigcap G_n$, so that for some $n$, $G_n$ is not dense, implying that it misses a whole open interval.
We have that $\R \setminus \set{x_n}$ is dense in $\R$, so it must be the case that the not dense $G_n$ corresponds to the complement of $\bar{E}_n$, giving us that $\bar{E}_n$ contains an open interval.

This proof tells us precisely that Baire's Category Theorem also holds for $\R \setminus \Q$.
\end{proof} 

\begin{exercise}{26 [The Baire Category Theorem]}
A complete metric space is of the second category in itself.
That is, if $M$ is a complete metric space, and if we write $M = \bigcup E_n$, then the closure of some $E_n$ contains an open ball.
Equivalently, if $(G_n)$ is a sequence of dense open sets in $M$, then $\bigcap G_n \neq \emptyset$;
in fact $\bigcap G_n$ is dense in $M$.
\end{exercise}
\begin{proof}
fill
\end{proof} 

\begin{exercise}{28}
fill
\end{exercise}
\begin{proof}
fill
\end{proof} 

\begin{exercise}{29}
fill
\end{exercise}
\begin{proof}
fill
\end{proof} 

\begin{exercise}{30}
fill
\end{exercise}
\begin{proof}
fill
\end{proof} 

\begin{exercise}{31}
fill
\end{exercise}
\begin{proof}
fill
\end{proof} 

\begin{exercise}{32}
fill
\end{exercise}
\begin{proof}
fill
\end{proof} 

\begin{exercise}{33}
fill
\end{exercise}
\begin{proof}
fill
\end{proof} 

\begin{exercise}{36}
fill
\end{exercise}
\begin{proof}
fill
\end{proof} 

\begin{exercise}{37}
fill
\end{exercise}
\begin{proof}
fill
\end{proof} 

\begin{exercise}{38}
fill
\end{exercise}
\begin{proof}
fill
\end{proof} 

\begin{exercise}{40}
fill
\end{exercise}
\begin{proof}
fill
\end{proof} 

\begin{exercise}{42}
fill
\end{exercise}
\begin{proof}
fill
\end{proof} 

\begin{exercise}{43}
fill
\end{exercise}
\begin{proof}
fill
\end{proof} 

\begin{exercise}{46}
fill
\end{exercise}
\begin{proof}
fill
\end{proof} 

\begin{exercise}{47}
fill
\end{exercise}
\begin{proof}
fill
\end{proof} 

\begin{exercise}{48}
fill
\end{exercise}
\begin{proof}
fill
\end{proof} 

\begin{exercise}{49}
fill
\end{exercise}
\begin{proof}
fill
\end{proof} 