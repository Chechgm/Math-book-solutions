\section{The Baire category Theorem}


40C Seeing how these notions behave with functions 
42C Cool way to extend Baire to other spaces 

\begin{exercise}{14}
Prove that $A$ has an empty interior in $M$ if and only if $A^C$ is dense in $M$.
\end{exercise}
\begin{proof}
This is the content of exercise 4.46.
\end{proof} 

\begin{exercise}{15}
If $G$ is open and dense in $\R$, show that the same is true of $G\setminus\set{x}$ for any $x\in\R$. 
Is this true in any metric space?
Explain.
\end{exercise}
\begin{proof}
Let $G$ be open and dense in $\R$, so that $\bar{G}=\R$.
Thus, for any $\epsilon>0$, and any $x\in\R$, $B_\epsilon(x)\cap G\neq \emptyset$.
In fact, $B_\epsilon(x)\cap G$ is infinite.
If this were not the case, say if $B_\epsilon(x)\cap G = A$ was finite, then we could take $\epsilon' < \min\set{d(x,a): a\in A}$, and certainly $B_{\epsilon'}(x)\cap G$ would be empty. 
Thus, $G\setminus\set{x}$ must be dense in $\R$ as well.

This is not true in every metric space.
Consider the space $M=[0,1]\cup \set{10}$ with the usual metric of $\R$.
The set $G = (\Q\cap[0,1]) \cup \set{10}$ is dense in $M$ but certainly $G\setminus\set{10}$ is not.
The important characteristic of this metric space is that $10$ is an isolated point.
The statement holds for any infinite metric space without isolated points.
\end{proof} 

\begin{exercise}{16}
Show that $\set{x}$ is nowhere dense in $M$ if and only if $x$ is not an isolated point of $M$.
\end{exercise}
\begin{proof}
This is the content of exercise 4.54.
\end{proof} 

\begin{exercise}{19}
Show that each of the following is equivalent to the statement ``$A$ is nowhere dense'':
\begin{enumerate}
    \item $\bar{A}$ contains no nonempty open set.
    \item Each nonempty open set in $M$ contains a nonempty open subset that is disjoint from $A$.
    \item Each nonempty open set in $M$ contains an open ball that is disjoint from $A$.
\end{enumerate}
\end{exercise}
\begin{proof}
This is the content of exercise 4.60.
\end{proof} 

\begin{exercise}{21}
If $x_n \to x$ in $\R$, show that the set $E = \set{x} \cup \set{x_n: n \geq 1}$ is nowhere dense in $\R$.
Is the same true if $\R$ is replaced by an arbitrary metric space $M$?
Is every countable set nowhere dense?
Explain
\end{exercise}
\begin{proof}
Let $\epsilon >0$ and $x \in E$.
We have that $B_\epsilon(x) \not\subseteq E$, since $B_\epsilon(x)$ is an uncountable set (Theorem 2.9), while $E$ is not.
Thus, $E = \bar{E}$ is nowhere dense, as required.

This is not true for all metric spaces, a discrete metric space being a simple example.
Likewise, it is not true that every countable set is nowhere dense.
In example 9.9.b we discussed how $\Q$, is not nowhere dense in $\R$.
\end{proof} 

\begin{exercise}{24}
If $\R = \bigcup E_n$, then the closure of some $E_n$ contains an interval;
that is, $\operatorname{int}(\bar{E}) \neq \emptyset$ for some $n$.
\end{exercise}
\begin{proof}
We will use a proof technique similar to that of Corollary 9.6.
Because $\R = \bigcup E_n$, we have that $\R = \bigcup \bar{E}_n$, so that for the open sets $G_n = \R \setminus \bar{E}_n$, it holds that $\bigcap G_n = \emptyset$.
By the Baire's Category Theorem, it must be the case that some $G_n$ is not dense, so that it misses an entire open interval. 
Thus $\bar{E}_n$ contains an entire open interval.
\end{proof} 

\begin{exercise}{25}
If $\R \setminus \Q = \bigcup E_n$, then the closure of some $E_n$ contains an interval.
Deduce that the conclusion of Baire's Theorem holds for $\R \setminus \Q$.
\end{exercise}
\begin{proof}
We have that $\R \setminus \Q \subseteq \bigcup \bar{E}_n$, so that by adding the (at most countably many) points in $\R \setminus \bigcup \bar{E}_n$ for $\R$, we get that $\R = \bigcup \bar{E}_n \cup \bigcup \set{x_n}$.
Thus, for a sequence of open sets $G_n$, we have that $\emptyset = \bigcap G_n$, so that for some $n$, $G_n$ is not dense, implying that it misses a whole open interval.
We have that $\R \setminus \set{x_n}$ is dense in $\R$, so it must be the case that the not dense $G_n$ corresponds to the complement of $\bar{E}_n$, giving us that $\bar{E}_n$ contains an open interval.

This proof tells us precisely that Baire's Category Theorem also holds for $\R \setminus \Q$.
\end{proof} 

\begin{exercise}{26 [The Baire Category Theorem]}
A complete metric space is of the second category in itself.
That is, if $M$ is a complete metric space, and if we write $M = \bigcup E_n$, then the closure of some $E_n$ contains an open ball.
Equivalently, if $(G_n)$ is a sequence of dense open sets in $M$, then $\bigcap G_n \neq \emptyset$;
in fact $\bigcap G_n$ is dense in $M$.
\end{exercise}
\begin{proof}
Let $x_0 \in M$ and let $I_0$ be an open set containing $x_0$.
We will prove both conclusions at the same time by showing that $I_0 \cap (\bigcap G_n) \neq \emptyset$.

Since $G_1$ is dense, we know that $I_0 \cap G_1 \neq \emptyset$ (because $I_0$ is open and it contains $x_0$, which must be the limit of a sequence contained in $G_1$).
But since $G_1$ is also open, this means that we can find some open interval $I_1 \subseteq I_0 \cap G_1$.
By shrinking $I_1$ (if necessary), we may suppose that $\operatorname{diam}(I_1) \leq 1$ and $\bar{I}_1 \subseteq I_0 \cap G_1$.

Now use $I_1$ in place of $I_0$ and $G_2$ in place of $G_1$.
Since $G_2$ is dense we have that $I_1 \cap G_2 \neq \emptyset$.
But $G_2$ is open, so there is some open interval $I_2$ with $\operatorname{diam}(I_2) \leq 1/2$ such that $\bar{I}_2 \subseteq I_1 \cap G_2 \subseteq I_0 \cap G_1 \cap G_2$.

Repeat this using $I_2$ and $G_3$ in place of $I_1$ and $G_2$, and so on.
What we get is a sequence of nested closed intervals, $\bar{I}_1 \supseteq \bar{I}_2 \supseteq \dots$ with $\operatorname{diam}(I_n) \leq 1/n$ and $\bar{I}_n \subseteq I_0 \cap (\bigcap G_k)$.
Thus by the nested set Theorem (7.11.ii), $I_0 \cap (\bigcap G_k) \supseteq \bigcap I_n \neq \emptyset$. 
Consequently, $\bigcap G_k$ is nonempty and dense.
\end{proof} 

\begin{exercise}{28}
In a metric space $M$, show that any subset of a first category set is still first category, and that a countable union of first category sets is again first category.
\end{exercise}
\begin{proof}
A subset $A$ of $M$ is said to be of the first category in $M$ if $A$ can be written as a countable union of sets, each of which is nowhere dense in $M$.

Taking a subset of $A$ does not affect this property, as we will only have to take elements of the sets of such union, which will still render them nowhere dense in $M$.

Likewise, since countable unions of countable sets are still countable, a countable union of first category sets can still be written as the countable union (the countable union of countable unions) of nowhere dense sets.
\end{proof} 

\begin{exercise}{29}
In a metric space $M$, prove that any superset of a second category set is itself a second category set.
\end{exercise}
\begin{proof}
A subset $B$ is a second category set in $M$ if, whenever we write $B = \bigcup E_n$ some $E_n$ fails to be nowhere dense in $M$;
that is, $\operatorname{int}(\bar{E}_n) \neq \emptyset$ for some $n$.

Taking a supserset of a second category set will not affect this property, as we will only be adding to the sets $E_n$ which compose the union.
If we either add elements to a $E_n$, or we add more sets to the countable union, some $E_n$ will still fail to be nowhere dense in $M$.
\end{proof} 

\begin{exercise}{30}
Show that $\N$ is first category in $\R$ but second category in itself.
\end{exercise}
\begin{proof}
We can write $\N = \bigcup \set{n}$.
We have that $\set{n}$ is a nowhere dense set in $\R$ for all $n$ (this is because $(\overline{\set{n}})^c$ is open and dense in $\R$).
Thus $\N$ is of first category in $\R$.
On the other hand, in $\N$, $\operatorname{int}(\overline{\set{n}})$ is nonempty for all $n$, as it contains the open set $B_\epsilon(n)$, where $\epsilon < 1$.
Hence, $\N$ is of second category in itself.
\end{proof} 

\begin{exercise}{31}
Show that $\Q$ is first category in itself (thus completeness is essential in Baire's Theorem).
\end{exercise}
\begin{proof}
Let $(E_n)$ be a sequence of sets enumerating $\Q$, so that $\Q = \bigcup E_n$.
Take any $E_n = \set{q}$ for some rational $q$.
Now $\overline{\set{q}} = \set{q}$, but we have that for any $\epsilon > 0$, $B_\epsilon(q) \not\subseteq \set{q}$, as there exists infinitely many rationals in $(q-\epsilon, q+\epsilon)$.
Thus $\operatorname{int}(\overline{\set{q}}) = \emptyset$;
that is, $\Q$ is of first category in itself.
\end{proof} 

\begin{exercise}{32}
In $\R$, show that any open interval (and hence nonempty, open set) is a second category set.
\end{exercise}
\begin{proof}
This is a Corollary to exercise 33.
\end{proof} 

\begin{exercise}{33}
If $M$ is complete, is every nonempty, open set a second category set?
\end{exercise}
\begin{proof}
Yes.
By exercise 7.30, we know that every nonempty open set is homeomorphic to a complete metric space.
Furthermore, in exercise 43 we prove that if $N$ is homeomorphic to a complete metric space $M$, the conclusion of Baire's Theorem holds in $N$.
Both of these results give us in conjuction the desired result.
\end{proof} 

\begin{exercise}{36}
If $M$ is complete, show that the complement of a first category set in $M$ is a dense set of the second category in $M$.
In particular, a first category set in a complete metric space must have an empty interior.
\end{exercise}
\begin{proof}
Let $A$ be a set of the first category in $M$.
Thus, $A = \bigcup \bar{E}_n$, where $E_n$ is nowhere dense.
By the DeMorgan law, we have that $A^C = \bigcap (\bar{E}_n)^C$, where now $(\bar{E}_n)^C$ is each an open dense set in $M$.
By the Baire's category Theorem, we have that $A^C$ is a dense set in $M$.
\end{proof} 

\begin{exercise}{37}
Show that the complement of a first category set in $\R$ is uncountable.
\end{exercise}
\begin{proof}
We will prove this by contrapositive.
To do so, we have to prove that if for a set $A$, $A^C$ is countable, then $A$ is not of the first category.
If $A^C$ is countable, then it is a set of the first category, since it can be written as a countable union of singletons.
This is because $\R$ has no isolated points so that applying exercise 16, every singleton is nowhere dense.
Now applying exercise 36, and using the fact that $\R$ is complete, then $A$ is of the second category.
\end{proof} 

\begin{exercise}{38}
Is the complement of a first category set necessarily a second category set?
Likewise, is the complement of a second category set necessarily a first category set?
Explain.
\end{exercise}
\begin{proof}
Neither of those is true.
For the first example consider $\N \subseteq \Q$.
$\N$ is a set of the first category in $\Q$, as it can be written as a countable union of singletons, which are nowhere dense sets in $\Q$ given that each natural is not an isolated point in $\Q$.
However, $\N^C \subseteq \Q$ is not a set of the second category in $\Q$.
We proved in exercise 31 that $\Q$ is first category in itself, and on exercise 28 that any subset of a set of the first category is in itself of the first category.

For the second example, in exercise 33 we proved that in a complete metric space, every nonempty open set is a second category set.
Take any discrete metric, which we know is complete.
In this case any subset of the underlying set is open, so that it is of second category, importantly, its complement is also open, so that its complement is also a second category set.
\end{proof} 

\begin{exercise}{40}
Let $f: \R \to \R$ be a continuous function that is nonconstant on any interval.
If $A$ is a second category set in $\R$, show that $f(A)$ is also second category.
[Hint: if B is closed and nowhere dense, show that $f^{-1}(B)$ is closed and nowhere dense].
\end{exercise}
\begin{proof}
Following the hint, we will prove this by contrapositive
That is, we will prove that if $f(A)$ is a set of the first category then $A$ is a set of the first category too.
In other words, $f(A)$ can be written as $f(A) = \bigcup E_n$ where all $E_n$ are nowhere dense implies that $A = \bigcup F_n$ where all $F_n$ are nowhere dense.
By the properties of inverse functions, we have that 
\begin{align*}
    f^{-1}(f(A)) = A = f^{-1}(\bigcup E_n) = \bigcup f^{-1}(E_n).
\end{align*}
We want to prove that each $f^{-1}(E_n)$ is a closed nowhere dense set.

The fact that $f^{-1}(E_n)$ is closed follows by continuity.
Furthermore, we can prove that $f^{-1}(E_n)$ is nowhere dense.
If $f^{-1}(E_n)$ were not nowhere dense, then it would contain a nonempty open set, say $B \subseteq f^{-1}(E_n)$.
We then have $f(B) \subseteq f(f^{-1}(E_n)) \subseteq E_n$, where the second containment follows by the properties of inverses (see 5.1).
Since $B$ is an open set, it contains some interval, so that by the corollary to the generalised intermediate value Theorem (Corollary 6.7), and given that $f$ is nonconstant, $f$ maps such interval to an interval.
This would imply that some interval is contained in $f(B) \subseteq E_n$ and $E_n$ wouldn't be nowhere dense.

Thus, we have that $A = \bigcup f^{-1}(E_n)$ where each $f^{-1}(E_n)$ is nowhere dense;
that is, $A$ is of the first category.
\end{proof} 

\begin{exercise}{42}
While completeness is essential in the proof of Baire's Theorem, the conclusion may still hold for some incomplete spaces.
Show that it holds in $\N$ if we us the metric $d(m,n) = \absoluteValue{m-n}/mn$, but that $(\N,d)$ is not complete.
[Hint: $d$ is equivalent to the usual metric.
See exercise 7.14].
\end{exercise}
\begin{proof}
As written in the hint, we proved in exercise 7.14 that $d$ is equivalent to the usual metric and that $(\N,d)$ is not complete.
We will now prove that the conclusion of Baire's Theorem still holds in such space.

Thus, we need to prove that if $\N = \bigcup E_n$, then the closure of some $E_n$ contains an open ball.
To see why this is the case, notice (trivially) that any nonempty $E_n$ must contain an element of $x$ of $\N$ and that every element of $\N$ is an isolated point of $\N$ under the usual metric.
Hence, $\set{x}$ is not nowhere dense, as we proved in exercise 16, giving us that Baire's category Theorem holds in $(\N,d)$.
\end{proof} 

\begin{exercise}{43}
If $N$ is homeomorphic to a complete metric space $M$, show that the conclusion of Baire's Theorem holds in $N$.
[Hint: homeomorphisms preserve open dense sets
Why?].
\end{exercise}
\begin{proof}
As the hint suggests, we just need to prove that homeomorphisms preserve dense open sets.
We want to prove that if $(G_n)$ is a sequence of dense open sets in $M$, then $\bigcap G_n \neq \emptyset$ and indeed, $\bigcap G_n$ is dense in $M$ implies that given a sequence of dense open sets $(O_n)$ in $N$, then $\bigcap O_n \neq \emptyset$ and $\bigcap O_n$ is dense in $N$.

Let $f: N\to M$ be a homeomorphism and let $O_n$ be a sequence of open dense sets in $N$.
By Theorem 5.5, we have that $f(O_n)$ is open for all $n$.
We will now prove that $f(O_n)$ is dense for all $n$.
Let $y \in M$.
Since $f$ is a homeomorphism, then there exists an $x \in N$ (but not necessarily in $f(O_n)$) such that $f(x) = y$.
However, we know that $O_n$ is dense in $N$, so that there exists a sequence $(x_n)$ in $O_n$ with $x_n \to x$.
Finally, since $f$ is continuous, we have that $f(x_n) \to f(x) = y$, so that $O_n$ is dense.

We now have a sequence of open dense sets in $M$, namely $(f(O_n))$, by the Baire's category Theorem, we have that $\bigcap f(O_n)$ is dense in $M$, so that $f^{-1}(\bigcap f(O_n)) = \bigcap f^{-1}(f(O_n)) = \bigcap O_n$ is dense in $N$ (by the same argument as above, just now applied to $f^{-1})$, giving us the desired result.
\end{proof} 

\begin{exercise}{46}
Show that $\R^2$ cannot be written as a countable union of lines.
\end{exercise}
\begin{proof}
We can write a line as the set $l = \set{(x,f(x) = ax+b): a,b,x \in \R} \subseteq \R^2$.
For all $a,b \in \R$, we have that $l$ nowhere dense in $\R^2$.
To see this, simply consider $\epsilon>0$ and notice that any point within $\epsilon$ above or below of $x$ is not contained in $l$ (of course with the exception of any vertical line, in which case any point left or right of $f(x)$ will not be contained in $l$).
Thus, $\bar{l}$ contains no nonempty open set.

If we could write $\R^2$ as the countable union of lines, then we would contradict the Baire's category Theorem.
\end{proof} 

\begin{exercise}{47}
Let $\cP$ be the vector space of all polynomials supplied with the norm $\norm{p} = \max\set{\absoluteValue{a_i}: i=0,\dots,n}$, where $p(x) = a_0 + a_1x + \dots + a_nx^n \in \cP$.
Show that $\cP$ is not complete.
\end{exercise}
\begin{proof}
This is a Corollary to exercise 48, as the set of monomials is a proper, closed, linear subspace of $\cP$.
That is, for every $k\in\N$, the set consisting of $\vecspan\set{x^k}$ is a closed, linear subspace of $\cP$.
Furthermore, we have that each of these spaces is finite dimensional and $\cP$ is the countable union of such sets.
Thus by the Baire's category Theorem, $\cP$ cannot be complete. 
\end{proof} 

\begin{exercise}{48}
If $W$ is a proper, closed, linear subspace of a normed vector space $V$, show that $W$ is nowhere dense in $V$.
[Hint: if $B_r(x) \subseteq W$, then $nB_1(0) \subseteq W$ for every $n$.
Why?]
\end{exercise}
\begin{proof}
We will prove this by contrapositive.
Suppose $W$ is not nowhere dense in $V$, so that there exists an $r>0$ and $x\in W$, such that $B_r(x) \subseteq W$.
Because $W$ is a vector space, then $-x \in W$ for all $x$, but also $(1/r)x \in W$ for all $x$.
Thus we can transform all the elements of $B_r(x)$ using the linear map $f_n(w) = (n/r)(w-x)$ to obtain the open set $B_n(0) \subseteq W$, for all $n\in\N$.

However, this implies that all $V$ is in $W$.
To see this, take any $v\in V$ and let $\epsilon = \norm{v}$.
We just need to choose $n>\epsilon$, so that $v \in B_n(0) \subseteq W$.
Thus, $W$ is not a proper subspace of $V$, as required.
\end{proof} 

\begin{exercise}{49}
Let $V$ be an infinite-dimensional normed vector space, and suppose that $V = \bigcup W_n$, where each $W_n$ is a finite-dimensional subspace of $V$.
Prove that $V$ is not complete.
\end{exercise}
\begin{proof}
This is a Corollary to the previous exercise, as in this case, each $W_n$ is a proper closed, linear subspace of $V$, implying that $V$ can be written as the countable union of nowhere dense sets.
Thus, by the Baire's category Theorem, $V$ cannot be complete.
\end{proof} 