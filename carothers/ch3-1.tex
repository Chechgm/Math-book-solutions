\subsection{Metric spaces}

Carothers 3 Metrics and Norms
Metric Spaces
2*
3
5* (suffices to do 6 or 7)
6*
9
12*
14*
15*


\begin{exercise}{2}
If $d$ is a metric on $M$, show that $\absoluteValue{d(x,z)-d(y,z)}\leq d(x,y)$ for any $x,y,z\in M$.
\end{exercise}
\begin{proof}

\end{proof} 

\begin{exercise}{3}
Shared
\end{exercise}
\begin{proof}
fill
\end{proof} 

\begin{exercise}{5}
There are other, albeit less natural choices for a metric on $\R$. For instance, check that $\rho(a,b)=\sqrt{\absoluteValue{a-b}},\sigma(a,b)=\absoluteValue{a-b}/(1+\absoluteValue{a-b})$, and $\tau(a,b)=\min\set{\absoluteValue{a-b},1}$ each define metrics on $\R$. [Hint: To show that $\sigma$ is a metric, you might first show that the function $F(t)=t/(1+t)$ is increasing and satisfies $F(s+t)\leq F(s)+F(t)$ for $s,t\geq 0$. A similar approach will also work for $\rho$ and $\tau$].
\end{exercise}
\begin{proof}
fill
\end{proof} 

\begin{exercise}{6}
If $d$ is any metric on $M$, show that $\rho(x,y)=\sqrt{d(x,y)}$, $\sigma(x,y)=d(x,y)/(1+d(x,y))$, and $\tau(x,y)=\min\set{d(x,y),1}$ are also metrics on $M$. [Hint: $\sigma(x,y)=F(d(x,y))$, where $F$ is as in exercise 5].
\end{exercise}
\begin{proof}
fill
\end{proof} 

\begin{exercise}{9}
Shared
\end{exercise}
\begin{proof}
fill
\end{proof} 

\begin{exercise}{12}
Check that $d(f,g)=\max_{a\leq t\leq b}\absoluteValue{f(t)-g(t)}$ defines a metric on $C[a,b]$, the collection of all continuous, real-valued functions defined on the closed interval $[a,b]$.
\end{exercise}
\begin{proof}
fill
\end{proof} 

\begin{exercise}{14}
We say that a subset $A$ of a metric space $M$ is bounded if there is some $x_0\in M$ and some constant $C<\infty$ such that $d(a,x_0)\leq C$ for all $a,\in A$. Show that a finite union of bounded sets is again bounded.
\end{exercise}
\begin{proof}
fill
\end{proof} 

\begin{exercise}{15}
We define the diameter of a nonempty subset $A\subseteq M$ by $\text{diam}(A)=\sup\set{d(a,b):a,b\in A}$. Show that $A$ is bounded if and only if $\text{diam}(A)$ is finite. 
\end{exercise}
\begin{proof}
fill
\end{proof} 
