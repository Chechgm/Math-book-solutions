\subsection{Complete metric spaces}


\begin{exercise}{12}
Let $A$ be a subset of an arbitrary metric space $(M,d)$.
If $(A,d)$ is complete, show that $A$ is closed in $M$.
\end{exercise}
\begin{proof}
Let $(x_n)$ be a convergent sequence in $A$.
Then, $(x_n)$ is Cauchy.
Since $A$ is complete, then $(x_n)$ converges to a point in $A$.
Thus $A$ is closed.
\end{proof} 

\begin{exercise}{13}
Show that $\R$ endowed with the metric $\rho(x,y)=\absoluteValue{\arctan x-\arctan y}$ is not complete.
How about if we try $\tau(x,y)=\absoluteValue{x^3-y^3}$?
\end{exercise}
\begin{proof}
($\arctan$)
Consider the sequence given by $x_n = n$.
We have that $x_n$ is Cauchy under the $\arctan$ metric, but $\arctan n\to \pi/2$, which is not in the image of $\arctan$, and thus it is not complete.

(Cubic)
Let $(x_n)$ be a Cauchy sequence under the $\tau$ metric.
By Exercise 3.36, $(x_n)$ is a bounded subset in $(\R,\tau)$, let the bound be $B$.
For any $x\in (x_n)$, we have $\absoluteValue{x^3}\leq B$, so that $\absoluteValue{x}\leq B^{1/3}$.
Thus, $(x_n)$ is bounded in $\R$ with its usual metric and by Theorem 7.6, the Bolzano-Weierstrass Theorem, it has a convergent subsequence.
This convergent subsequence converges also in $(\R,\tau)$, and since a Cauchy sequence with a convergent subsequence converges, then $(x_n)$ is convergent in $(\R,\tau)$ giving us that $\R$ is complete under $\tau$, as required.
\end{proof} 

\begin{exercise}{15}
Prove or disprove:
If $M$ is complete and $f:(M,d)\to (N,\rho)$ is continuous, then $f(M)$ is complete.
\end{exercise}
\begin{proof}
This is not true.
We know that $\R$ is complete, and homeomorphic to $(0,1)$ which is not.
Thus for the homeomorphism, $f$ between $\R$ and the open unit interval, it doesn't hold that $f(\R)$ is complete.
\end{proof} 

\begin{exercise}{16}
Prove that $\R^n$ is complete under any of the norms $\norm{\cdot}_1, \norm{\cdot}_2,$ or $\norm{\cdot}_\infty$.
[This is interesting because completeness is not usually preserved by the mere equivalence of metrics.
Here we used the fact that all of the metrics involved are generated by norms.
Specifically, we need the norms in question to be equivalent as functions:
$\norm{\cdot}_\infty \leq \norm{\cdot}_2 \leq \norm{\cdot}_1 \leq n\norm{\cdot}_\infty$.
As we will see later, any two norms on $\R^n$ are comparable this way].
\end{exercise}
\begin{proof}
Let $\epsilon>0$ and $(x_n)$ be a Cauchy sequence in $\R^n$, so that there exists $K\in\N$ such that whenever $k,l>K$, it holds that $\norm{x_k-x_l}_\infty < \epsilon$.
Thus for all $i\in \set{1,\dots,n}$, we have that $\absoluteValue{x_n^i-x_m^i}<\epsilon$, meaning that for all $i$, $(x^i_n)$ is Cauchy in $\R$ and hence it converges to $x^i$ by the completeness of $\R$.

Given this convergence, choose an $K\in\N$ (this is different from the $K$ above), such that whenever $k>K$, it holds that $\absoluteValue{x^i_k-x^i}<\epsilon$ for all $i$.
Let $x=(x^i)$, to see $(x_n)\to x$, notice that $\norm{x_k-x}_\infty = \max_i\absoluteValue{x_k^i-x^i} < \epsilon$, as required.
\end{proof} 

\begin{exercise}{17}
Given metric spaces $N$ and $M$, show that $M\times N$ is complete if and only if both $N$ and $M$ are complete.
\end{exercise}
\begin{proof}
($\Rightarrow$)
Suppose $M\times N$ is complete, so that any Cauchy sequence in $M\times N$ converges to a point in $M\times N$ under $\norm{\cdot}_\infty$.
Let $\epsilon>0$, the sequence $(x^N_k)$ be Cauchy in $N$ and $(x^M_k)$ Cauchy in $M$.
Hence, there exists $K\in\N$ with $d_N(x^N_k,x^N_l)<\epsilon$, and $d_M(x^M_k,x^M_l)<\epsilon$ whenever $k,l>K$.
We have that the sequence $(x_k) = (x^M_k,x^N_k)$ is Cauchy in $M\times N$ and by assumption it converges to $x=(x^N,x^M)$ under $\norm{\cdot}_\infty$.
We have 
\begin{align*}
    d_N(x^N_k,x^N) \leq \max_{i\in\set{N,M}}d_i(x^i_k,x^i) = \norm{x_k,x}_\infty < \epsilon,    
\end{align*}
and the same holds replacing $d_N(x^N_k,x^N)$ for $d_M(x^M_k,x^M)$, so that both $(x^N_k)$ and $(x^M_k)$ converge, as required.

($\Leftarrow$)
Suppose $N$ and $M$ are complete.
Let $\epsilon>0$ and $(x_k)$ be Cauchy in $M\times N$, so that we can find a $K\in\N$ with $\norm{x_k-x_l}_\infty<\epsilon$, whenever $k,l>K$.
As in the previous exercise, we have that both $d_N(x^N_k, x^N_l)<\epsilon$ and $d_M(x^M_k, x^M_l)<\epsilon$, and given that $N$ and $M$ are complete, the sequences $(x^N_k)$ and $(x^M_k)$ converge to, say $x^N$ and $x^M$. 

Let $x=(x^N, x^M)$, and choose $K\in\N$ (different from the one above) so that whenever $k>K$ it holds that $d_N(x^N_k, x^N)<\epsilon$ and $d_M(x^M_k, x^M)<\epsilon$.
We have that $\norm{x_k-x}_\infty = \max_{i\in \set{N,M}}d_i(x^i_k-x^i) < \epsilon$, so that $(x_k)$ converges, as required.
\end{proof} 

\begin{exercise}{18}
Fill in the details of the proofs that $l_1$ and $l_\infty$ are complete.
\end{exercise}
\begin{proof}
($l_1$)
Suppose $(f_n)$ is a Cauchy sequence in $l_1$.
Then for $\epsilon>0$, there exists an $N\in\N$, so that whenever $n,m>N$, it holds that $\norm{f_n-f_m}_1 = \sum_k \absoluteValue{f_n(k)-f_m(k)} < \epsilon$.
Thus, for all $k$, $f_n(k)$ is a Cauchy sequence and because $f_n(k)$ is in $\R$ (or $\C$) it converges, say $f_n(k)\to f(k)$. 
Our candidate limit of $l_1$ will be $f$, with $f(k)$ defined as above.

To see $f\in l_1$, notice that as $(f_n)$ is Cauchy, then by exercise 3.36, it is bounded by a constant $B$, and so $\sum_k^K \absoluteValue{f(k)} = \lim_{n\to\infty} \sum_k^K \absoluteValue{f_n(k)} \leq B$.
Since this holds for all $K$, we get that $\norm{f}_1 = \sum_k^\infty \absoluteValue{f(k)} < B$;
that is $f\in l_1$.

Finally, $f_n\to f$.
As above, we have that for all $K$, $\sum_k^K \absoluteValue{f_n(k)-f(k)} = \lim_{m\to \infty} \sum_k^K \absoluteValue{f_n(k)-f_m(k)} < \epsilon$, so that $\norm{f_n(k)-f(k)}_1 = \sum_k^\infty \absoluteValue{f_n(k)-f(k)} < \epsilon$ by the order limit Theorem.

($l_\infty$)
Let $(f_n)$ be a Cauchy sequence in $l_\infty$.
Thus for all $\epsilon>0$, there exists an $N\in\N$, so that whenever $n,m>N$, it holds that $\norm{f_n-f_m}_\infty = \sup_k\absoluteValue{f_n(k)-f_m(k)} < \epsilon$.
That is, for all $k$, $(f_n(k))$ is Cauchy  in $\R$ or $\C$ and thus converges, say $f_n(k)\to f(k)$.
Our candidate limit will be $f$, where $f(k)$ is as defined above.

To see $f\in l_\infty$, notice that for the $n$ defined above, we have 
\begin{align*}
    \norm{f}_\infty 
    =& \sup_k \absoluteValue{f(k)}\\ 
    =& \sup_k \absoluteValue{f(k) - f_n(k) + f_n(k)}\\
    \leq& \sup_k\absoluteValue{f(k) - f_n(k)} + \sup_k\absoluteValue{f_n(k)}\\
    <& \epsilon + B.
\end{align*}
Where $\sup_k\absoluteValue{f_n(k)}<B$, because $(f_n)\in l_\infty$.

Finally, we prove that $f_n\to f$.
To see this, we have $\norm{f_n-f}_\infty = \sup_k \absoluteValue{f_n(k) - f(k)} < \epsilon$, because for each $k$, $f_n(k)\to f(k)$, as we have argued above.
Thus $l_\infty$ is complete. 
\end{proof} 

\begin{exercise}{19}
Prove that $c_0$ is complete by showing that $c_0$ is closed in $l_\infty$.
[Hint: if $(f_n)$ is a sequence in $c_0$ converging to $f\in l_\infty$, note that $\absoluteValue{f(k)} \leq \absoluteValue{f(k)-f_n(k)} + \absoluteValue{f_n(k)}$.
Now choose $n$ so that the $\absoluteValue{f(k)-f_n(k)}$ is small independent of $k$]. 
\end{exercise}
\begin{proof}
Following the hint, suppose $f_n\to f$ under $\norm{\cdot}_\infty$.
That is, for all $\epsilon>0$, there exists an $N\in\N$, such that whenever $n>N$, it holds that $\norm{f_n-f} = \sup_k\absoluteValue{f_n(k)-f(k)} < \epsilon/2$.
Likewise, since $f_n\in c_0$, then $f(k) \to 0$, so that we can find an $K\in\N$ such that whenever $k>K$, it holds that $\absoluteValue{f_n(k)} <\epsilon/2$ (this holds for all $n$, so it holds for the $n$ chosen above).
We have
\begin{align*}
    \absoluteValue{f(k)} 
    \leq& \absoluteValue{f(k) + f_n(k) - f_n(k)}\\
    \leq& \absoluteValue{f(k) - f_n(k)} + \absoluteValue{f_n(k)}\\
    \leq& 2\epsilon/2 = \epsilon,
\end{align*}
so that $f(k)\to 0$, and thus $f\in c_0$, as required.

Now since $c_0$ is closed in $l_\infty$ which is a complete space, then by Theorem 7.9, $c_0$ is complete.
\end{proof} 

\begin{exercise}{20}
If $(x_n)$ and $(y_n)$ are Cauchy in $(M,d)$, show that $(d(x_n,y_n))^\infty_{n=1}$ is Cauchy in $\R$.
\end{exercise}
\begin{proof}
Let $\epsilon>0$ and let $N$ be such that whenever $n,m>N$, it holds $d(x_n,x_m)<\epsilon/2$ and $d(y_n,y_m)<\epsilon/2$.
We have
\begin{align*}
    d(x_n,y_n)-d(x_m,y_m)
    \leq& d(x_n,x_m) + d(x_m, y_n) - d(x_m,y_m)\\
    \leq& d(x_n,x_m) + d(y_m, y_n) + d(x_m,y_m) - d(x_m,y_m)\\
    <& 2\epsilon/2 = \epsilon.
\end{align*}
Likewise, we can obtain that $d(x_m,y_m)-d(x_n,y_n)<\epsilon$, which is the same as $d(x_n,y_n)-d(x_m,y_m)>-\epsilon$.
Putting those together, we get that $\absoluteValue{d(x_n,y_n)-d(x_m,y_m)}<\epsilon$, so that $(d(x_n,y_n))^\infty_{n=1}$ is Cauchy.
\end{proof} 

\begin{exercise}{21}
If $(M,d)$ is complete prove that two Cauchy sequences $(x_n)$ and $(y_n)$ have the same limit if and only if $d(x_n,y_n)\to 0$.
\end{exercise}
\begin{proof}
($\Rightarrow$)
Suppose $(x_n)$ and $(y_n)$ have the same limit, say $a$.
We have
\begin{align*}
    \absoluteValue{d(x_n,y_n)}
    =& d(x_n,y_n)\\
    \leq& d(x_n, a) + d(a, y_n) \to 0,
\end{align*}
since both $d(x_n,a)\to 0$ and $d(y_n,a)\to 0$.

($\Leftarrow$)
We will prove this by contrapositive.
Suppose $x_n\to a$ and $y_n\to b$, for $a\neq b$ and consider the candidate convergence $d(x_n,y_n)\to d(a,b)\neq 0$.
Fix $\epsilon>0$.
Since $x_n\to a$ and $y_n\to b$, we can find $N\in\N$, such that whenever $n>N$, it holds that both $d(x_n,a)<\epsilon/2$ and $d(y_n,b)<\epsilon/2$.
Thus, let $n>N$.
We have
\begin{align*}
    d(x_n,y_n) - d(a,b)
    \leq& d(x_n,a) + d(a,y_n) - d(a,b)\\
    \leq& d(x_n,a) + d(a,b) + d(b,y_n) - d(a,b)\\
    \leq& d(x_n, a) + d(b,y_n) < 2\epsilon/2 = \epsilon.
\end{align*}

Likewise
\begin{align*}
    d(a,b) - d(x_n,y_n)
    \leq& d(a,x_n) + d(b,x_n) - d(x_n,y_n)\\
    \leq& d(a,x_n) + d(x_n,y_n) + d(b,y_n) - d(x_n,y_n)\\
    \leq& d(a,x_n) + d(b,y_n) < 2\epsilon/2 = \epsilon,
\end{align*}
which is the same as $d(x_n,y_n)-d(a,b)>-\epsilon$.

Putting these inequalities together we get that $\absoluteValue{d(x_n,y_n)-d(a,b)}<\epsilon$, as desired.
\end{proof} 

\begin{exercise}{25}
True or False?
If $f:\R\to\R$ is continuous and $(x_n)$ is Cauchy, then $(f(x_n))$ is Cauchy.
Examples?
How about if we insist that $f$ be strictly increasing?
Show that the answer is ``True'' if $f$ is Lipschitz.
\end{exercise}
\begin{proof}
The answer is true.
$(x_n)$ Cauchy in $\R$ is convergent, so that $(f(x_n))$ is convergent in $\R$ too and thus Cauchy.
An obvious example is $f(x)=x$ and $(x_n)=1/n$, $f$ is continuous and $(x_n)=(f(x_n))$ is Cauchy.
The answer does not change if $f$ is strictly increasing.

For the Lipschitz result, let $(x_n)$ be Cauchy, let $M$ be the Lipschitz constant of $f$.
Let $N$ be such that whenever $n,m>N$, it is the case that $M\absoluteValue{x_n-x_m}<\epsilon$.
We have $\absoluteValue{f(x_n)-f(x_m)} \leq M\absoluteValue{x_n-x_m} < \epsilon$, so that $(f(x_n))$ is Cauchy, as required.
\end{proof} 

\begin{exercise}{26}
Just as with the nested interval Theorem, it is essential that the sets $F_n$ used in the nested set Theorem be both closed and bounded.
Why?
Is the condition $\text{diam}(F_n)\to 0$ really necessary?
Explain.
\end{exercise}
\begin{proof}
We will look at a couple of examples in $\R$ with its usual metric.
Take $F_n=(0,1/n)$, none of which is closed but all are bounded, then $\bigcap_n F_n =\emptyset$.
Likewise, if $F_n=[n,\infty)$, so that no $F_n$ is bounded, even though they are all closed, then $\bigcap_nF_n=\emptyset$.
In  both cases, Theorem 7.11 does not hold.

The condition that the diameter converges to 0 is not necessary for the intersection of $F_n$ to be nonempty, since the sets are nested and thus there will be more than one element in the intersection. 
If, however, we want the intersection to have only one element, then the diameter must converge to 0, as if the diameter is not 0 then it must be the case that the intersection contains more than one element.
\end{proof} 

\begin{exercise}{27}
Note that the version of the Bolzano-Weierstrass Theorem given in Theorem 7.11 replaces boundedness with total boundedness.
Is this really necessary?
Explain.
\end{exercise}
\begin{proof}
The condition is necessary.
Consider the subset of $l_2$ consisting of sequences where all but one element of the sequence are 0 and the other element is 1.
The distance between any of these sequences and $0\in l_2$ is 1, so that the subset is bounded.
However,  there is no limit point in $l_2$ of such subset because the distance between any two sequences is 2, so that we cannot get arbitrarily close to a point in $l_2$.
\end{proof} 

\begin{exercise}{31}
If $\sum^\infty_{n=1}x_n$ is a convergent series in a normed vector space $X$, show  that $\norm{\sum^\infty_{n=1}x_n}\leq \sum^\infty_n{n=1}\norm{x_n}$.
\end{exercise}
\begin{proof}
We have that for all $N$, $\norm{\sum^N_{n=1}x_n} \leq \sum^N_{n=1} \norm{x_n} \leq \sum^\infty_{n=1} \norm{x_n}$.
Taking limits at the left hand side of the inequality and using the order limit Theorem gives us the desired result.
\end{proof} 

\begin{exercise}{33}
Let $s$ denote the vector space of all finitely nonzero real sequences;
that is, $x=(x_n)\in s$ if $x_n=0$ for all but finitely many $n$.
Show that $s$ is not complete under the $\sup$ norm $\norm{x}_\infty=\sup_n\absoluteValue{x_n}$.
\end{exercise}
\begin{proof}
Consider the sequence $(x^k_n)$ of sequences, where for the $k$-th sequence (of sequences), the first $k$ elements are $x_n^k=1/n$, and $x_n=0$ for $n>k$.
This sequence converges under the $\sup$ norm to the sequence $x_n=1/n$ for all $n$.
However, this sequence is not in $s$, as all of its elements are different from 0.
\end{proof} 
