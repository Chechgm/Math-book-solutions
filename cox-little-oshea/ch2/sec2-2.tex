\section{Orderings on the monomials in $k[x_1,\dots, x_n]$}


\begin{exercise}{4}
Show that grlex is a monomial order according to Definition 1.
\end{exercise}
\begin{proof}
Recall the definition of grlex:
Let $\alpha, \beta \in \N^n$.
We say $\alpha >_{\text{grlex}} \beta$ if $\absoluteValue{\alpha} = \sum \alpha_i > \sum \beta_i = \absoluteValue{\beta}$, or $\absoluteValue{\alpha}=\absoluteValue{\beta}$ and $\alpha >_{\text{lex}} \beta$.

First, since $>$ is an ordering on the naturals, and $>_{\text{lex}}$ is an ordering on $\N^n$, $>_{\text{grlex}}$ is an ordering on $\N^n$.

Second, let $\gamma \in \N^n$ and $\alpha > \beta$.
We need to prove that $\alpha + \gamma > \beta + \gamma$ for the first condition of grlex, because we know the condition holds for lex, thus we assume $\absoluteValue{\alpha} > \absoluteValue{\beta}$.
We have $\sum \alpha_i > \sum \beta_i$ and adding each $\gamma_i$ on both sides of the inequality, we obtain $\sum \alpha_i + \gamma_i > \sum \beta_i \gamma_i$ which is the same as $\absoluteValue{\alpha + \gamma} > \absoluteValue{\beta + \gamma}$, as required.

Third, $>_{\text{grlex}}$ is a well-ordering because $>$ is a well ordering on the naturals (the first condition) and $>_{\text{lex}}$ is a well-ordering on $\N^n$.
\end{proof} 

\begin{exercise}{7}
Let $>$ be any monomial order.
\begin{enumerate}
    \item Show that $\alpha \geq 0$ for all $\alpha \in \N^n$.
    [Hint: Proof by contradiction].
    \item If $x^\alpha$ divides $x^\beta$, then $\alpha \leq \beta$.
    Is the converse true?
    \item If $\alpha \in \N^n$, then $\alpha$ is the smalles element of $\alpha + \N^n = \set{\alpha + \beta: \beta \N^n}$.
\end{enumerate}
\end{exercise}
\begin{proof}
\begin{enumerate}
    \item Suppose, for the sake of contradiction, that there exists an $\alpha \in \N^n$ such that $\alpha < 0$.
    By the second property of monomial orderings, we have $\alpha + \alpha = 2\alpha < \alpha$ and, in general, $n\alpha < (n-1)\alpha$.
    But this implies that there is a decreasing sequence that never terminates, in particular, $\alpha > 2\alpha > 3\alpha > \dots$ which contradicts Lemma 2.
    \item If $x^\alpha$ divides $x^\beta$, then there exists a $\gamma \in \N^n$ such that $x^\beta = x^\alpha x^\gamma + r$, thus $\beta = \alpha +\gamma$ (where equality here is in the monomial order). 
    Hence either $\beta = \alpha$, if $\gamma = 0$ or $\beta > \alpha$, as required.
    
    The converse is not true.
    Consider the lex order in $k[x,y]$.
    We have that $x>y$ but $x$ does not divide $y$.
    \item We have that $\beta \geq 0$ for all $\beta \in \N^n$.
    By the second property of nomonomial orders, $\beta + \alpha \geq \alpha$, so that $\alpha$ is the smallest element of $\alpha + \N^n$.
\end{enumerate}
\end{proof} 

\begin{exercise}{11}
Let $>$ be a monomial order on $k[x_1, \dots, x_n]$.
\begin{enumerate}
    \item Let $f \in k[x_1, \dots, x_n]$ and let $m$ be a monomial.
    Show that $\operatorname{LT}(m\cdot f) = m \cdot \operatorname{LT}(f)$.
    \item Let $f,g \in k[x_1, \dots, x_n]$.
    Is $\operatorname{LT}(f \cdot g)$ necessarily the same as $\operatorname{LT}(f) \cdot \operatorname{LT}(g)$?
    \item If $f_i, g_i \in k[x_1, \dots, x_n], 1 \leq i \leq s$, is $\operatorname{LM}(\sum_i^s f_i \cdot g_i)$ necessarily equal to $\operatorname{LM}(f_i) \cdot \operatorname{LT}(g_i)$ for some $i$?
\end{enumerate}
\end{exercise}
\begin{proof}
\begin{enumerate}
    \item Let $m = x^\alpha$, $f = a_1x^{\alpha_1} + \dots + a_nx^{\alpha_n}$ and $\operatorname{LT}(f) = a_i x^{\alpha_i}$ for some $i$.
    We have that $\alpha_i > \alpha_j$ for $j \neq i$.
    By the second property of monomial orderings, we have $\alpha_i + \alpha > \alpha_j + \alpha$ so that $mf = a_1x^{\alpha_1 + \alpha} + \dots + a_nx^{\alpha_n + \alpha}$ with $\operatorname{LT}(mf) = a_ix^{\alpha_i + \alpha}$.
    Thus $m\operatorname{LT}(f) = a_ix^{\alpha_i + \alpha} = \operatorname{LT}(mf)$, as required.
    \item Yes, the argument is similar as above.
    Let $\alpha_f = \operatorname{multideg}(f)$ so that for any other degree of $f$. say $\beta$ we have $\alpha_f > \beta$.
    Using the second property of the monomial orderings, we have that $\alpha_j + \gamma > \beta + \gamma$, in particular, if $\gamma = \operatorname{multideg}(g)$, we have that the multi degree remains unchanged under polynomial product (the same argument follows for the multidegree of $g$).
    But this is precisely saying that $\operatorname{LM}(fg) = \operatorname{LM}(f)\operatorname{LM}(g)$.
    Since the leading coefficient has no effect in this calculation, the desired result follows.
    \item No, let $f_1 = f_2 \neq 0$ be arbitrary, let $g_1 = 1$ and $g_2 = -1$, then $f_1 \cdot g_1 + f_2 \cdot g_2 = 0$.
    The sum has a different leading monomial than any of the individual products.
\end{enumerate}
\end{proof} 

\begin{exercise}{12}
Lemma 8 gives two properties of the multidegree.
\begin{enumerate}
    \item Prove Lemma 8. 
    Hint: the arguments used in exercise 11 may be relevant.
    \item Suppose that $\operatorname{multideg}(f) = \operatorname{multideg}(g)$ and $f + g \neq 0$.
    Give examples to show that $\operatorname{multideg}(f+g)$ may or may not be equal $\max(\operatorname{multideg}(f), \operatorname{multideg}(g))$.
\end{enumerate}
\end{exercise}
\begin{proof}
\begin{enumerate}
    \item Lemma 8:
    Let $f, g \in k[x_1, \dots, x_n]$ be nonzero polynomials.
    Then:
    \begin{enumerate}
        \item $\operatorname{multideg}(fg) = \operatorname{multideg}(f) + \operatorname{multideg}(g)$.
        \item If $f + g \neq 0$, then $\operatorname{multideg}(f + g) \leq \max(\operatorname{multideg}(f), \operatorname{multideg}(g))$.
        If, in addition, $\operatorname{multideg}(f) \neq \operatorname{multideg}(g)$, then equality occurs.
    \end{enumerate}
    The argument for the first part is the same argument as in exercise 11.2, where we don't focus anymore on the leading coefficient but only on the multidegrees.

    For the second part, if $f$ and $g$ have the same multidegree, then either it gets canceled in the sum or not. 
    If it gets canceled, then the multidegree of the sum is certainly less than the multidegree of either of them.
    If it doesn't get canceled, then the multidegree of the sum is essentially the multidegree of either of them.
    Furthermore, if $\operatorname{multideg}(f) \neq \operatorname{multideg}(g)$, then adding the polynomials together doesn't cancel the highest multidegree between them, so that the equality holds.
    \item Let $f \neq 0$ and $g = 1$, then $\operatorname{multideg}(f + g) = \max(\operatorname{multideg}(f), \operatorname{multideg}(g)) = \operatorname{multideg}(f)$, so equality holds.
    For an example where the inequality is strict and $f + g \neq 0$, consider $f = x$ and $g = -x + 1$, then $\operatorname{multideg}(f + g) = \operatorname{multideg}(1) \leq \max(\operatorname{multideg}(f), \operatorname{multideg}(g))$.
\end{enumerate}
\end{proof} 
