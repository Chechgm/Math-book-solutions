\subsection{Polynomials and Affine Space}


\begin{exercise}{1}
Let $\F_2 = \{0,1\}$, and define addition and multiplication by $0+0 = 1+1 = 0$, $0+1 = 1+0 = 1$, $0\cdot 0 = 0\cdot 1 = 1\cdot 0 = 0$ and $1\cdot 1 = 1$. 
Explain why $\F_2$ is a field. 
(You need not check the associative and distributive properties, but you should verify the existence of identities and inverses, both additive and multiplicative).
\end{exercise}
\begin{proof}
    We know that $0$ is the additive identity since
    \begin{align*}
        0 + 0 = 0 = 0+0,~\text{and} 1+0 = 1 = 0+1.
    \end{align*}
    Likewise, $1$ is the multiplicative identity since
    \begin{align*}
        1\cdot 0 = 0 = 0\cdot 1,~\text{and} 1\cdot 1 = 1 = 1\cdot 1.
    \end{align*}
   The additive inverse of $0$ is $0$ since $0+0=0$, and the additive inverse of $1$ is $1$ since $1+1=1$. 
   The multiplicative inverse of $1$ is $1$ since $1\cdot 1 = 1$.
\end{proof}

\begin{exercise}{2}
Let $\F_2$ be the field from Exercise $1$.
\begin{enumerate}
    \item Consider the polynomial $g(x,y) = x^2 y + y^2 x\in \F_2[x,y]$. 
    Show that $g(x,y) = 0$ for every $(x,y)\in \F_2^2$, and explain why this does not contradiction Proposition $5$.
    \item Find a nonzero polynomial in $\F_2[x,y,z]$ which vanishes at each point of $\F_2^3$. 
    Try to find one involving all three variables.
    \item Find a nonzero polynomial in $\F_2[x_1,...,x_n]$ which vanishes at every point in $\F_2^n$. Can you find one in which all of $x_1$,...,$x_n$ appear.
\end{enumerate}
\end{exercise}
\begin{proof}
\begin{enumerate}
    \item We have the following situations:
    \begin{itemize}
        \item For $(x,y) = (0,0)$,
        \begin{align*}
            x^2 y + y^2 x = 0^2\cdot 0 + 0^2 \cdot 0 = 0\cdot 0 + 0\cdot 0 = 0 + 0 = 0.
        \end{align*}
        \item For $(x,y) = (1,0)$,
        \begin{align*}
            x^2 y + y^2 x = 1^2\cdot 0 + 0^2 \cdot 1 = 1\cdot 0 + 0\cdot 1 = 0 + 0 = 0.
        \end{align*}
        \item For $(x,y) = (0,1)$,
        \begin{align*}
            x^2 y + y^2 x = 0^2\cdot 1 + 1^2 \cdot 0 = 0\cdot 1 + 1\cdot 0 = 0 + 0 = 0.
        \end{align*}
        \item For $(x,y) = (1,1)$,
        \begin{align*}
            x^2 y + y^2 x = 1^2\cdot 1 + 1^2 \cdot 1 = 1\cdot 1 + 1\cdot 1 = 1 + 1 = 0.
        \end{align*}
    \end{itemize}
    This does cont contradict Proposition $5$, since that assumes an infinite field which $\F_2$ isn't.
    \item Try $g(x,y,z) = (x-y)(x-z)(y-z)$. 
    Take $(x,y,z)\in \F_2^3$ and since $x,y,z\in \{0,1\}$, it must be the case that $x=y$ or $y=z$ or $x=z$ by the pigeonhole principle. 
    In each of these cases, it clearly holds that $g(x,y,z) = 0$.
    \item Analogously with (b), we can define
    \begin{align*}
        g(x_1,...,x_n) = \prod_{1\leq i<j\leq n} (x_i - x_j).
    \end{align*}
\end{enumerate}    
\end{proof}

\begin{exercise}{3}
Let $p$ be a prime number. 
The ring of integers modulo $p$ is a field with $p$ elements, which we will denote $\F_p$.
\begin{enumerate}
    \item Explain why $\F_p \setminus \set{0}$ is a group under multiplication.
    \item Use Lagrange's Theorem to show that $a^{p-1}=1$ for all $a\in \F_p \setminus \set{0}$.
    \item Prove that $a^p=a$ for all $a\in\F_p$. Hint: treat the cases $a=0$ and $a\neq 0$ separately.
    \item Find a nonzero polynomial in $\F_p[x]$ that vanishes at all points of $\F_p$. 
    Hint: use part 3.
\end{enumerate}
\end{exercise}
\begin{proof}
 \begin{enumerate}
     \item Since $\F_p$ is a field, it is an integral domain, so that it does not have divisors of zero. 
     That is, if $a,b\in\F_p$ and $ab=0$ then either $a=0$ or $b=0$. 
     In other words, if $a\neq 0$ and $b\neq 0$ then $ab \in \F_p \setminus\set{0}$.
     Furthermore, since $\F_p$  is a field, it is closed under inverses, so that every element in $\F_p\setminus\set{0}$ is invertible.
     \item By Lagrange's Theorem, we know that the order of any element of a finite group divides the order of the group. 
     Since $\F_p\setminus\set{0}$ has order $p-1$, then for any $a\in\F_p\setminus\set{0}$, the order of $a$ (say $n$) divides $p-1$ so that $p-1=mn$. 
     We then have $(a^n)^m=a^{p-1}=1^m=1$.
     \item If $a=0$, then $0^p=0$. 
     If $a\neq 0$, we multiply both sides of the equation in exercise 2 by $a$ we obtain $a^p=a$.
     \item Since $a^p=a$ for all $a\in\F_p$, then the polynomial $p(x)=x^p-x$ has the property that $p(a)=0$ for all $a\in\F_p$.
 \end{enumerate}
\end{proof}

\begin{exercise}{4}
Let $F$ be a finite field with $q$ elements. 
Adapt the argument of exercise 3 to prove that $x^q-x$ is a nonzero polynomial in $F[x]$ which vanishes at every point of $F$. 
This shows that Proposition 5 fails for all finite fields.
\end{exercise}
\begin{proof}
 Since in exercise 3 we did not use the primality of the order of $\F_p$, then the result of this questions follows directly from it.
\end{proof}

\begin{exercise}{5}
In the proof of Proposition $5$, we took $f\in k[x_1,...,x_n]$ and wrote it as a polynomial in $x_n$ with coefficients in $k[x_1,...,x_{n-1}]$. 
To see what this looks like in a specific case, consider the polynomial
\begin{align*}
    f(x,y,z) = x^5 y^2 z - x^4 y^3 + y^5 + x^2 z - y^3 z + xy + 2x - 5z + 3.
\end{align*}
\begin{enumerate}
    \item Write $f$ as a polynomial in $x$ with coefficients in $k[y,z]$.
    \item Write $f$ as a polynomial in $y$ with coefficients in $k[x,z]$.
    \item Write $f$ as a polynomial in $z$ with coefficients in $k[x,y]$.
\end{enumerate}
\end{exercise}
\begin{proof}
    \begin{enumerate}
        \item We have
        \begin{align*}
            f(x,y,z) = (y^2 z)x^5 + (-y^3)x^4 + z x^2 + (y + 2)x + (y^5 - y^3 z - 5z + 3).
        \end{align*}
        \item We have
        \begin{align*}
            f(x,y,z) = y^5 + (-x^4 - z)y^3 + (x^5 z)y^2 + xy + (x^2 z + 2x - 5z + 3).
        \end{align*}
        \item We have
        \begin{align*}
            f(x,y,z) = (x^5 y^2 + x^2 - y^3 - 5)z + (-x^4y^3 + y^5 +xy + 2x + 3).  
        \end{align*}
    \end{enumerate}
\end{proof}

\begin{exercise}{6}
    Inside of $\C^n$, we have the subset $\Z^n$, which consists of all points with integer coordinates.
    \begin{enumerate}
        \item Prove that if $f\in \C[x_1,...,x_n]$ vanishes at every point of $\Z^n$, then $f$ is the zero polynomial.
        \item Let $f\in \C[x_1,...,x_n]$, and let $M$ be the largest power of any variable that appears in $f$. 
        Let $\Z_{M+1}^n$ be the set of points of $\Z^n$, all coordinates of which lie between $1$ and $M+1$, inclusive. 
        Prove that if $f$ vanishes at all points of $\Z_{M+1}^n$, then $f$ is the zero polynomial.
    \end{enumerate}
\end{exercise}
\begin{proof}
    \begin{enumerate}
        \item When $n=1$, it is well known that a nonzero polynomial in $\C[x]$ of degree $m$ has at most $m$ distinct roots. 
        For our particular $f\in \C[x]$, we are assuming $f(a)=0$ for all $a\in \Z$. 
        Since $\Z$ is infinite, this means that $f$ has infinitely many roots, and, hence, $f$ must be the zero polynomial.
        
        Now assume that the statement is true for $n-1$ and let $f\in \C[x_1,...,x_n]$ be a polynomial that vanishes at all points of $\Z^n$. 
        By collecting the various powers of $x_n$, we can write $f$ in the form
        \begin{align*}
            f = \sum_{i=0}^N g_i(x_1,...,x_{n-1})x_n^i,
        \end{align*}
        where $g_i\in \C[x_1,...,x_{n-1}]$. 
        We will show that each $g_i$ is the zero polynomial in $n-1$ variables, which will force $f$ to be the zero polynomial in $\C[x_1,...,x_n]$.
        
        If we fix $(a_1,...,a_{n-1})\in \Z^{n-1}$, we get the polynomial $f(a_1,...,a_{n-1}, x_n)\in \C[x_n]$. 
        By our hypothesis on $f$, this vanishes for each $a_n\in \Z$. 
        It follows from the case $n=1$ that $f(a_1,...,a_{n-1},x_n)$ is the zero polynomial in $\C[x_n]$. 
        Using the above formula for $f$, we see that the coefficients of $f(a_1,...,a_{n-1},x_n)$ are $g_i(a_1,...,a_{n-1})$, and thus $g_i(a_1,...,a_{n-1})=0$ for all $i$. 
        Since $(a_1,...,a_{n-1})$ was arbitrarily chosen in $\Z^{n-1}$, it follows that each $g_i\in \C[x_1,...,x_{n-1}]$ gives the zero function on $\Z^{n-1}$. 
        Our inductive assumption then implies that each $g_i$ is the zero polynomial in $\C[x_1,...,x_{n-1}]$. 
        This forces $f$ to be the zero polynomial in $\C[x_1,...,x_n]$ and completes the proof.
        \item When $n=1$, it is well known that a nonzero polynomial in $\C[x]$ of degree $<N$ has at most $N$ distinct roots. 
        For our particular $f\in \C[x]$, we are assuming $f(a)=0$ for all $a\in \Z_{M+1}$. 
        Since this contains $M+1$ points, this means that $f$ must be the zero polynomial.
        
        Now assume that the statement is true for $n-1$ and let $f\in \C[x_1,...,x_n]$ be a polynomial that vanishes at all points of $\Z^n_{M+1}$. 
        By collecting the various powers of $x_n$, we can write $f$ in the form
        \begin{align*}
            f = \sum_{i=0}^N g_i(x_1,...,x_{n-1})x_n^i,
        \end{align*}
        where $g_i\in \C[x_1,...,x_{n-1}]$. 
        We will show that each $g_i$ is the zero polynomial in $n-1$ variables, which will force $f$ to be the zero polynomial in $\C[x_1,...,x_n]$.
        
        If we fix $(a_1,...,a_{n-1})\in \Z^{n-1}_{M+1}$, we get the polynomial $f(a_1,...,a_{n-1}, x_n)\in \C[x_n]$. 
        By our hypothesis on $f$, this vanishes for each $a_n\in \Z_{M+1}$. 
        It follows from the case $n=1$ that $f(a_1,...,a_{n-1},x_n)$ is the zero polynomial in $\C[x_n]$. 
        Using the above formula for $f$, we see that the coefficients of $f(a_1,...,a_{n-1},x_n)$ are $g_i(a_1,...,a_{n-1})$, and thus $g_i(a_1,...,a_{n-1})=0$ for all $i$. 
        Since $(a_1,...,a_{n-1})$ was arbitrarily chosen in $\Z^{n-1}_{M+1}$, it follows that each $g_i\in \C[x_1,...,x_{n-1}]$ gives the zero function on $\Z^{n-1}$. 
        Our inductive assumption then implies that each $g_i$ is the zero polynomial in $\C[x_1,...,x_{n-1}]$. 
        This forces $f$ to be the zero polynomial in $\C[x_1,...,x_n]$ and completes the proof.
    \end{enumerate}
\end{proof}
















