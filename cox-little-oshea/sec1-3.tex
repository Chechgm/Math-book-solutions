\subsection{Parametrizations of Affine Varieties}


\begin{exercise}{4}
Consider the parametric representation
\begin{align*}
    x=\frac{t}{1+t},\,\,y=1-\frac{1}{t^2}.
\end{align*}
\begin{enumerate}
    \item Find the equation of the affine variety determined by the above parametric equations.
    \item Show that the above equations parametrize all points of the variety found in part (1) except for the point $(1,1)$.
\end{enumerate}
\end{exercise}
\begin{proof}
\begin{enumerate}
    \item We have 
    \begin{align*}
        &x = \frac{t}{1+t} &&\iff\\
        &x(1+t) = t &&\iff\\
        &x = t(1-x) &&\iff\\
        &t = \frac{x}{1-x},
    \end{align*}
    so that replacing in the equation of $y$ we obtain the equation of the affine variety
    \begin{align*}
        y =& 1 -\frac{1}{\left(\frac{x}{1-x}\right)^2}\\
        =& 1-\frac{(1-x)^2}{x^2}\\
        =& \frac{x^2-1+2x-x^2}{x^2}\iff\\
        &yx^2-2x+1=0.
    \end{align*}
    \item To show this, we simply replace the parametrised equation for $x$ and $y$ in the equation of the affine variety
    \begin{align*}
        &\parens{1-\frac{1}{t^2}}\parens{\frac{t}{1+t}}^2-2\parens{\frac{t}{1+t}}+1=0\\
        &\parens{\frac{t^2(t^2-1)}{t^2(1+t)^2}}-2\parens{\frac{t^3(1+t)}{t^2(1+t)^2}}+1=0\\
        &\parens{\frac{t^4-t^2-2t^3-2t^4}{t^2(1+t)^2}}+1=0\\
        &\frac{t^4-t^2-2t^3-2t^4+t^2(1+t)^2}{t^2(1+t)^2}=0\\
        &\frac{t^4-t^2-2t^3-2t^4+t^2(1+2t+t^2)}{t^2(1+t)^2}=0\\
        &\frac{t^4-t^2-2t^3-2t^4+t^2+2t^3+t^4)}{t^2(1+t)^2}=0\\
        &0=0.
    \end{align*}
    Notice, however, that if $x=1$, then $1=t/(1+t)$ so that $1+t=t$ which is a contradiction.
\end{enumerate}
\end{proof}

\begin{exercise}{6}
The goal of this problem is to show that the sphere $x^2+y^2+z^2=1$ in 3-dimensional space can be parametrised by
\begin{align*}
    x=\frac{2u}{u^2+v^2+1},\,\, y=\frac{2v}{u^2+v^2+1},\,\, z=\frac{u^2+v^2-1}{u^2+v^2+1}.
\end{align*}
The idea is to adapt the argument used for the circle $x^2+y^2=1$ to 3-dimensional space.
\begin{enumerate}
    \item Given a point $(u,v,0)$ in the $(x,y)$-plane, draw the line form this point to the ``north pole'' $(0,0,1)$ of the sphere, and let $(x,y,z)$ be the other point where the line meets the sphere. Draw a picture to illustrate this, and argue geometrically that mapping $(u,v)$ to $(x,y,z)$ gives a parametrisation of the sphere minus the north pole.
    \item Show that the line connecting $(0,0,1)$ to $(u,v,0)$ is parametrised by $(tu,tv,1-t)$, where $t$ is a parameter that moves along the line.
    \item Substitute $x=tu, y=tv$ and $z=1-t$ into the equation for the sphere $x^2+y^2+z^2=1$. Use this to derive the formulas given at the beginning of the problem.
\end{enumerate}
\end{exercise}
\begin{proof}
\begin{enumerate}
    \item Since the described line intersects the unit sphere in only one point, and for all points of the sphere (besides $(0,0,1)$ we can find $u$ and $v$ so that the point of the sphere is the intersection with the line, then the whole sphere (with the exception of the north pole) is identified with $(u,v)$. 
    \begin{figure}[H]
     \centering
     \includegraphics[width=.5\textwidth]{cox-little-oshea/assets/sec1-3-ex6a.png}
     \caption{Line between $(u,v,0)$ and $(0,0,1)$ for $u=v=2$ and the unit sphere.}
     \label{fig:sec1-3-ex8a}
    \end{figure}
    Remark: This appears on Brannan, Espen and Gray as the Stereographic projection.
    \item We know that the equation of a parametrised line in 3D is given by $r(t)=(x_0,y_0,z_0)+t(a,b,c)$, where $(a,b,c)$ are the slopes of the line. We can find the slopes, given two points, by simply subtracting one point from the other. We then have, $a=u, b=v$ and $c=-1$ then $r(t)=(0,0,1)+t(u,v,-1)=(tu,tv,1-t)$, as desired.
    \item We have 
    \begin{align*}
        &u^2t^2 + v^2t^2 + (1-t)^2 = 1 &&\iff\\
        &u^2t^2 + v^2t^2 + 1 - 2t + t^2 = 1 &&\iff\\
        &t^2(u^2 + v^2 + 1) - 2t = 0 &&\iff\\
        &t(u^2 + v^2 + 1) - 2 = 0 &&\iff\\
        &t = \frac{2}{u^2 + v^2 + 1}.
    \end{align*}
    Replacing $t$ in the above parametric equations of the line we obtain the desired result.
\end{enumerate}
\end{proof}

\begin{exercise}{8}
Consider the curve defined by $y^2=cx^2-x^3$, where $c$ is some constant. Here is a picture of the curve when $c>0$:
\begin{figure}[H]
     \centering
     \includegraphics[width=.5\textwidth]{cox-little-oshea/assets/sec1-3-ex8.png}
     \label{fig:sec1-2-ex8}
\end{figure}
Our goal is to parametrise this curve.
\begin{enumerate}
    \item Show that a line will meet this curve at either 0, 1, 2, or 3 points. Illustrate your answer with a picture. Hint: let the equation of the line be either $x=a$ or $y=mx+b$.
    \item Show that a nonvertical line through the origin meets the curve at exactly one other point when $m^2\neq c$. Draw a picture to illustrate this, and see if you can come up with an intuitive explanation as to why this happens.
    \item Now draw the vertical line $x=1$. Given a point $(1,t)$ on this line, draw the line connecting $(1,t)$ to the origin. This will intersect the curve in a point $(x,y)$. Draw a picture to illustrate this, and argue geometrically that this gives a parametrisation of the entire curve.
    \item Show that the geometric description from part (3) leads to the parametrisation $x=c-t^2, y=t(c-t^2)$. 
\end{enumerate}
\end{exercise}
\begin{proof}
\begin{enumerate}
    \item Using the hint, consider the following two cases:

    Case 1. The line is vertical, with equation $x=a$. Then we have
    \begin{align*}
        y^2 =& ca^2-a^3\\
        =& a^2(c-a)\implies\\
        y =& a\sqrt{c-a}.
    \end{align*}
    We then have two extra cases. Case 1a: If $c-a\geq0$, then the line intersects the curve in the following two points $y=\pm a\sqrt{c-a}$, covering the 2 intersection case. Case 1b: If $c-a<0$, then the line does not intersect the curve, covering the 0 case.

    Case 2. The line is given by the following function $y=mx+b$. We have
    \begin{align*}
        &(mx+b)^2 = cx^2 -x^3 &&\implies\\
        &m^2x^2+2mbx+b^2 = cx^2-x^3 &&\implies\\
        &x^3+(m^2-c)x^2+2mbx+b^2 = 0 &&\text{(*)},
    \end{align*}
    which can have either 1 or 3 real roots (because complex roots come in pairs, covering the 1 or 3 intersection case.

    Cases 1a, 1b and 2 tell us that the lines can meet in 0, 1, 2 or 3 points.
    \item If $m^2\neq c$, and $b=0$ (because the line goes through the origin, we have that (*) is: $x^3+(m^2-c)x^2 = x^2(x+m^2-c)=0$, so that the roots of such polynomial are 0 and $x=c-m^2$. 
    
    My intuition of why this happens is that as we saw on (*), and replacing $b=0$, we must have 3 real roots, 0 being repeated. Hence, we are only left with $x+m^2-c=0$ which is a single intersection between the curves.
    \begin{figure}[H]
         \centering
         \includegraphics[width=.5\textwidth]{cox-little-oshea/assets/sec1-3-ex8-sol2.png}
         \label{fig:sec1-2-ex8-sol2}
    \end{figure}

    \item We just argued in the previous exercise that a line that goes through the origin meets the curve at a single point. Geometrically, this is a parametrisation of the whole curve because the line through $(1,t)$ and the origin will have any slope we want, with functional form $y=mx$. Since the intersection between this line and $y$ is given by (*) (with $b=0$), we cover all values of $y$. To see this, just choose $(1,t)$ so that $m$ gives the desired value of $y$ whenever we replace $x=c-m^2$ in the original function of the equation.
    \begin{figure}[H]
         \centering
         \includegraphics[width=.5\textwidth]{cox-little-oshea/assets/sec1-3-ex8-sol3.png}
         \label{fig:sec1-2-ex8-sol3}
    \end{figure}
    \item We just argued before about $x=c-m^2$, which translates to $x=c-t^2$ given that we are using the vertical curve $x=1$. To see the parametrisation for $y$, we substitute as we intuitively argued above: $y=\sqrt{c(c-t^2)^2-(c-t^2)^3}=\sqrt{t^2(c-t^2)^2}=t(c-t^2)$, as required.
\end{enumerate}
\end{proof}

\begin{exercise}{11}
In this problem, we will derive the parametrisation
\begin{align*}
    x = t(u^2-t^2),\,\, y = u,\,\, z = u^2-t^2
\end{align*}
of the surface $x^2-y^2z^2+z^3=0$ considered in the text.
\begin{enumerate}
    \item Adapt the formulas in part (4) of exercise 8 to show that the curve $x^2 = cz^2-z^3$ is parametrized by 
    \begin{align*}
        z=c-t^2,\,\, x=t(c-t^2).
    \end{align*}
    \item Now replace the c in part (1) by $y^2$, and explain how this leads to the above parametrisation of $x^2-y^2z^2+z^3=0$.
    \item Explain why this parametrisation covers the entire surface $\bV(x^2-y^2z^2+z^3$. Hint: See part (3) of exercise 8.
\end{enumerate}
\end{exercise}
\begin{proof}
\begin{enumerate}
    \item These are exactly the same formulas as in part (4) of 8, where $x=y$ and $z=x$. 
    \item After replacing $c=y^2$, we have $x^2=y^2z^2-z^3$, which is exactly the parametrisation we want.
    \item As argued on the two previous solutions, the formulas here presented are the same as in Exercise 8 where $x=y$ and $z=x$. Furthermore, instead of having $c$ as a parameter as in Exercise 8, we introduce a new variable to take on that value, namely $y^2=c$. Since the reasons of why the parametrisation in Exercise 8 did not depend on the value of $c$, then we simply introduce a new variable, $y$, and parameter, $u$, without affecting the fact that the entired surface is covered by the parametrisation.
\end{enumerate}
\end{proof}
