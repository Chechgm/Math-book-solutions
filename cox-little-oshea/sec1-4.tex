\subsection{Ideals}


\begin{exercise}{1}
Consider the equations
\begin{align*}
    x^2+y^2-1 =& 0,\\
    xy-1 =& 0
\end{align*}
which describe the intersection of a circle with a hyperbola.
\begin{enumerate}
    \item Use algebra to eliminate $y$ from the above equations.
    \item Show how the polynomial found in part 1 lies in $\brackets{x^2+y^2-1, xy-1}$. Your answer should be similar to what we did in equation (1), page 30. Hint: multiply the second equation by $xy-1$.
\end{enumerate}
\end{exercise}
\begin{proof}
\begin{enumerate}
    \item To eliminate $y$ from the above equations, multiply the first one by $x^2$ to obtain $x^4+x^2y^2-x^2=0$ and notice that from the second we have $xy=1$, so that $x^2y^2=1$, giving us $x^4+1-x^2=0$.
    \item Following the hint, we have $x^2(x^2+y^2-1)-(xy-1)(xy+1) =(x^4+x^2y^2-x^2)-(x^2y^2-1) =x^4+1-x^2$, as required.
\end{enumerate}
\end{proof}

\begin{exercise}{2}
Let $I\subseteq k[x_1\dots,x_n]$ be an ideal, and let $f_1,\dots,f_s\in k[x_1,\dots,x_n]$. Prove that the following statements are equivalent:
\begin{enumerate}
    \item $f_1,\dots,f_s\in I$.
    \item $\brackets{f_1,\dots,f_s}\subseteq I$.
\end{enumerate}
\end{exercise}
\begin{proof}
Suppose $f_1,\dots,f_s\in I$. Then for $f_i$ there exists polynomials $g_{i,1},\dots,g_{i,t_i}$ so that $f_i=\sum_r^{t_i}h_rg_{i,r}$. Now take any linear combination of $f_1,\dots,f_s$, say $r=\sum_i^sl_if_i$. Replacing each $f_i$ as a linear combination of $g_{i,k}$ gives us that $r\in I$ hence $\brackets{f_1,\dots,f_s}\subseteq I$. 

If we assume 2, clearly 1 holds.
\end{proof}

\begin{exercise}{4}
Prove proposition 4: If $f_1,\dots,f_s$ and $g_1,\dots,g_t$ are bases of the same ideal in $k[x_1,\dots,x_n]$, so that $\brackets{f_1,\dots,f_s}=\brackets{g_1,\dots,g_t}$, then we have $\bV(f_1,\dots,f_s)=\bV(g_1,\dots,g_t)$.
\end{exercise}
\begin{proof}
Let $x\in \bV(f_1,\dots,f_s)$. Then $f_1(x)=\dots=f_s(x)=0$. Since $\brackets{f_1,\dots,f_s}=\brackets{g_1,\dots,g_t}$, then we can represent $g_i$ as a linear combination of $f_1,\dots,f_s$, say $g_i=\sum_i^sh_if_i$, but this implies $g_i(x)=0$ and hence $x\in\bV(g_1,\dots,g_t)$, so that $\bV(f_1,\dots,f_s)\subseteq\bV(g_1,\dots,g_t)$. The converse of the proof is symmetric to this one. Hence $\bV(f_1,\dots,f_s)=\bV(g_1,\dots,g_t)$, as required.
\end{proof}

\begin{exercise}{7}
Show that $\bI(\bV(x^n,y^m))=\brackets{x,y}$ for any positive integers $n$ and $m$.
\end{exercise}
\begin{proof}
We have that $\bV(x^n,y^m)$ implies $x^n=y^m=0$, so that $\bV(x^n,y^m)=\set{(0,0)}$. However, any polynomial in $\brackets{x,y}$ equals 0 when evaluated at $(0,0)$. Now consider any polynomial $p(x,y)\in\brackets{x,y}$ and $r\in k$ so that $q(x,y)=p(x,y)+r\notin\brackets{x,y}$. We have $q(0,0)=p(0,0)+r=r\neq 0$, hence $q(x,y)\notin\bI(\bV(x^n,y^m))$. That is, $\bI(\bV(x^n,y^m))=\brackets{x,y}$, as required.
\end{proof}

\begin{exercise}{8}
The ideal $\bI(V)$ of a variety has a special property not shared by all ideals. Specifically, we define an ideal $I$ to be the radical if whenever a power of $f^m$ of a polynomial $f$ is in $I$, then $f$ itself is in $I$. More succinctly, $I$ is a radical when $f\in I$ if and only if $f^m\in I$ for some positive integer $m$.
\begin{enumerate}
    \item Prove that $\bI(V)$ is always a radical ideal.
    \item Prove that $\brackets{x^2,y^2}$ is not a radical ideal. This implies that $\brackets{x^2,y^2}\neq \bI(V)$ for any variety $V\subseteq k^2$.
\end{enumerate}
Radical ideals will play an important role in Chapter 4. In particular, the Nullstellensatz will imply that there is a one-to-one correspondence between varieties in $\C^n$ and radical ideals in $\C[x_1,\dots,x_n]$.
\end{exercise}
\begin{proof}
\begin{enumerate}
    \item ($\Rightarrow$) Suppose $f\in\bI(V))$, because $\bI(V))$ is an ideal, then all powers of $f$ are in $\bI(V))$. Hence, $f^m\in\bI(V))$.

    ($\Leftarrow$) Suppose $f^m\in\bI(V))$. Then $f^m(x)=0$ for some $\bx\in k^n$. However, we have that $f^m(x) =f^{m-1}(x)f(x)=0$. If $f(x)=0$, we are done, otherwise, we can argue similarly with $f^{m-1}$ until $f^2(x)$ which gives us $f(x)=0$, so that $f\in\bI(V))$ as well.
    \item Well, we have that all polynomials in $\brackets{x^2,y^2}$ are of the form $h_1x^2+h_2y^2$ for $h_1,h_2\in k^n[x,y]$. By choosing $h_1=1$ and $h_2=0$, we have that $x^2\in\brackets{x^2,y^2}$. However, $x\notin\brackets{x^2,y^2}$, as all the polynomials in $\brackets{x^2,y^2}$ are of degree at least 2. Hence, $\brackets{x^2,y^2}$ is not a radical ideal.
\end{enumerate}
\end{proof}

\begin{exercise}{14}
This exercise is concerned with Proposition 8.
\begin{enumerate}
    \item Prove that part (ii) of the proposition follows from part (i).
    \item Prove the following corollary of the proposition: if $V$ and $W$ are affine varieties in $k^n$, then $V\nsubseteq W$ if and only if $\bI(V)\nsupseteq\bI(W)$.
\end{enumerate}
\end{exercise}
\begin{proof}
\begin{enumerate}
    \item Since $V=W$ is the same as $V\subseteq W$ and $W\subseteq V$, (ii) follows from using part (i) twice.
    \item This result is simply the negation of (i).
\end{enumerate}
\end{proof}

\begin{exercise}{15}
In the text, we defined $\bI(V)$ for a variety $V\subseteq k^n$. We can generalize this as follows: if $S\subseteq k^n$ is any subset, then we set
\[
\bI(S)=\set{f\in k[x_1,\dots,x_n]\mid f(a_1,\dots,a_n)=0\text{ for all }(a_1,\dots,a_n)\in S}.
\]
\begin{enumerate}
    \item Prove that $\bI(S)$ is an ideal.
    \item Let $X=\set{(a,a)\in\R^2\mid a\neq 1}$. By exercise 1.2.8, we know that $X$ is not an affine variety. Determine $\bI(X)$. Hint: What you proved in Exercise 1.2.8 will be useful. See also Exercise 10 of this section.
    \item Let $\Z^n$ be the points of $\C^n$ with integer coordinates. Determine $\bI(\Z^n)$. Hint: See Exercise 1.1.6.
\end{enumerate}
\end{exercise}
\begin{proof}
\begin{enumerate}
    \item Let $f,g\in\bI(S)$, $x\in S$ and $r\in k[x_1,\dots,x_n]$. 

    Addition: Consider $f+g$. We have $(f+g)(x) =f(x)+g(x) =0$, so that $f+g\in\bI(S)$.

    Absorption: Consider $rf$. We have $(rf)(x) =r(x)f(x) =r(x)0 =0$, so that $rf\in\bI(S)$.
    \item In the proof of exercise 1.2.8 we didn't use any defining characteristic of varieties, so the proof is the same here. That is, if $f$ vanishes in $S$, then $f$ has infinitely many roots, if that is the case, then it must be the case that $f=0$, hence $\bI(X)=\set{0}$.
    \item As above, so that we get $\bI(\Z^n)=\set{0}$.
\end{enumerate}
\end{proof}
