\section*{Chapter 29. Degrees of Field Extensions}
\addcontentsline{toc}{section}{Chapter 29. Degrees of Field Extensions}


\begin{exercise}{A Examples of finite extensions}
\begin{enumerate}
    \item Find a basis for $\Q(i\sqrt{2})$ over $\Q$ and describe the elements of $\Q(i\sqrt{2})$. (See the two examples immediately following Theorem 1).
    \item Show that every element of $\R(2+3i)$ can be written as $a+bi$, where $a,b\in R$. Conclude that $\R(2+3i)=\C$
\end{enumerate}
\end{exercise}
\begin{proof}
 \begin{enumerate}
     \item We have $x=i\sqrt{2}$, then $x^2=i^2\sqrt{2}^2=-2$, so that $x^2+2=0$ is the minimal polynomial of $i\sqrt{2}$ over $\Q$. Then $\Q(i\sqrt{2})$ has as basis $\{1,\sqrt{2}\}$, so that $\Q(i\sqrt{2})=\{a+bi\sqrt{2}: a,b\in\Q\}$.
     \item We know that $\R(2+3i)$ is the smallest field that contains $\R$ and $2+3i$. Then if $a,b/3\in\R$, we have that $(2+3i)(b/3)= 2b+bi\in\R(2+3i)$ and $(2/3)+a-(2/3)+bi=a+bi\in\R(2+3i)$. So that $\R(2+3i)=\{a+bi: a,b\in\R\}= \C$, as required.
 \end{enumerate}
\end{proof}

\begin{exercise}{C Finite extensions of finite fields}
By the proof of the basic theorem of field extensions, if $p(x)$ is an irreducible polynomial of degree $n$ in $F[x]$, then $\quot{F[x]}{\langle p(x)\rangle}\cong F(c)$ where $c$ is a root of $p(x)$. By Theorem 1 in this chapter, $F(c)$ is of degree $n$ over $F$. Using the paragraph preceding Theorem 1:
\begin{enumerate}
    \item Prove that every element of $F(c)$ can be written uniquely as $a_0+\dots+a_{n-1}c^{n-1}$, for some $a_0,\dots,a_{n-1}\in F$.
    \item Construct a field of four elements. (It is to be an extension of $\Z_2$). Describe its elements, and supply its addition and multiplication tables.
    \item Construct a field of eight elements. (It is to be an extension of $\Z_2$).
    \item Prove that if $F$ has $q$ elements, and $a$ is algebraic over $F$ of degree $n$, then $F(a)$ has $q^n$ elements.
    \item Prove that for every prime number $p$, there is an irreducible quadratic $\Z_p[x]$. Conclude that for every prime $p$, there is a field with $p^2$ elements. 
\end{enumerate}
\end{exercise}
\begin{proof}
 \begin{enumerate}
     \item Suppose for the sake of contradiction, that an element in $F(c)$ can be written as $a_0+\dots+a_{n-1}c^{n-1}$ or $b_0+\dots+b_{n-1}c^{n-1}$ for $a_i,b_i\in F$. We then have $a_0+\dots+a_{n-1}c^{n-1}=b_0+\dots+b_{n-1}c^{n-1}$ which implies $(a_0-b_0)+\dots+(a_{n-1}-b_{n-1})c^{n-1}=0$. Since $1,c,\dots,c^{n-1}$ is a basis of $F(c)$, then it must be the case that $(a_0-b_0)=\dots=(a_{n-1}-b_{n-1})=0$, so that $a_i=b_i$ implying that the element can only be written as a unique linear combination.
    \item We know that $x^2+x+1$ has no roots in $\Z_2$, call the root of this polynomial $\alpha$. Then $\Z_2(\alpha)$ has 4 elements because there are 4 polynomials $\Z_2[x]$ of degree less than $n$. The elements of $Z_2(\alpha)$ can be identified with $0,1,x,x+1$ from which we can follow the usual addition and multiplication rules in $\Z_2[x]$ in addition to $x^2+x=-1=1$ to find the tables.
    \item From Theorem 2, we just have to find an extension of degree 2 over $\Z_2(\alpha)$ for an extension with 8 elements in $\Z_2$. Consider, as above, the polynomial $\alpha^2+\alpha+1$ which has no roots in $\Z_2(\alpha)$ giving us the desired result. 
    \item If $a$ is algebraic over $F$ of degree $n$, it means that there exists a polynomial of degree $n$ such that $F(a)\cong\quot{F[x]}{\langle p(x)\rangle}$. But this quotient contains all the polynomials of degree less than $n$. Using the multiplication principle of counting, we can conclude that there are $q^n$ different polynomials of degree less than $n$ in a field with $q$ elements, as desired.
    \item We will try to prove this using a counting argument. There are $p^2$ monic quadratic polynomials in $Z_p[x]$ and there are $p(p-1)/2+p$ linear polynomials in $Z_p[x]$. 
    
    We will now prove by induction that $p(p-1)/2+p<p^2$, to conclude the desired result. for $p=2$, we have $2(2-1)/2+2=2<4$. Now suppose that the inequality holds for all $n$ up to $p-1$, so that
    \begin{align}
        &\frac{(p-1)((p-1)-1)}{2}+(p-1) < (p-1)^2\\
        &\frac{(p-1)(p-2)+2(p-1)}{2} < p^2-2p+1\\
        &\frac{(p-1)(p-2+2)}{2} < p^2\\
        &\frac{p(p-1)}{2}+2p-1 < p^2\\
        &\frac{p(p-1)}{2}+p < p^2.
    \end{align}
    The last inequality follows from the fact that $p$ is a prime greater than 1, so that $2p-1>p$. 
    
    Since there are more quadratic polynomials than linear ones, then there must be some irreducible quadratic polynomial. Since the multiplicative order of any element in $\Z_p$ is $p$, then those irreducible quadratic polynomials have no solution in $\Z_p$, which by exercise 4, gives us the desired result.
 \end{enumerate}
\end{proof}

\begin{exercise}{D Degrees of extensions (applications of Theorem 2)}
Let $F$ be a field, and $K$ a field extension of $F$. Prove the following:
\begin{enumerate}
    \item $[K:F]=1$ if and only if $K=F$.
    \item If $[K:F]$ is a prime number, there is no field properly between $F$ and $K$ (that is, there is no field $L$ such that $F\not\subseteq L\not\subseteq K$).
    \item If $[K:F]$ is a prime, then $K=F(a)$ for every $a\in K-F$.
    \item Suppose $a,b\in K$ are algebraic over $F$ with degrees $m$ and $n$, where $m$ and $n$ are relatively prime. Then:
    \begin{enumerate}
        \item $F(a,b)$ is of degree $mn$ over $F$.
        \item $F(a)\cap F(b)=F$.
    \end{enumerate}
    \item If the degree of $F(a)$ over $F$ is a prime, then $F(a)=F(a^n)$ for any $n$ (on the condition that $a^n\notin F$).
    \item If an irreducible polynomial $p(x)\in F[x]$ has a root in $K$, then $\deg p(x)\vert[K:F]$.
\end{enumerate}
\end{exercise}
\begin{proof}
 \begin{enumerate}
     \item ($\Rightarrow$) We know that $K$ contains $F$ so that it must be the case that a basis of $K$ contains a scalar $a\in F$. Since $[K:F]=1$, then the basis of $K$ is simply $\{a\}$, which is $F$, so that $K=F$.

     ($\Leftarrow$) If $K=F$, then a basis of $K$ is as above, so that $[K:F]=1$, as required.
     \item Suppose $F\subseteq L\subseteq K$, then we would have, for a prime $p$, that $[K:F]=p=[K:L][L:F]$, but since we cannot factorise a prime, it must be the case that either $[K:L]=p$ and $[L:F]=1$, or $[K:L]=1$ and $[L:F]=p$, so that $L$ is no field properly between $F$ and $K$.
     \item We have that $F\subseteq F(a)\subseteq K$, then by exercise 2, necessarily $[F(a):F]=[K:F]$ (given that $a\in K$.
     \item 
         \begin{enumerate}
             \item From 2, we know that neither $F(a)\subset F(b)$ nor $F(b)\subset F(a)$. Since $F(a)$ is the smallest field containing $a$, this means $a\notin F(b)$ and likewise $b\notin F(a)$. Hence, $[F(a,b):F]=[F(a):F][F(b):F]=mn$, as required.
             \item By the same argument as above, the only common elements of $F(a)$ and $F(b)$ are the elements of $F$ so that $F(a)\cap F(b)=F$, as required.
         \end{enumerate}
    \item This is a special case of exercise 3, where $F(a)=K$ and $F(a^n)=F(b)$, because $F(a)$ is a field, and $a^n\in F(a)$. 
    \item Let $a\in K$ be the root of $p(x)$. Then $F(a)\subseteq K$, by Theorem 2, we have that $[K:F]=[K:F(a)][F(a):F]=[K:F(a)](\deg p(x))$, as required.
 \end{enumerate}
\end{proof}

\begin{exercise}{E Short questions relating to degrees of extensions}
Let $F$ be a field.

Prove parts 1-3:
\begin{enumerate}
    \item The degree of $a$ over $F$ is the same as the degree of $1/a$ over $F$. It is also the same as the degrees of $a+c$ and $ac$ over $F$, for any $c\in F$.
    \item $a$ is of degree 1 over $F$ if and only if $a\in F$.
    \item If a real number $c$ is a root of an irreducible polynomial of degree greater than 1 in $\Q[x]$, then $c$ is irrational.
    \item Use part 3 and Eisenstein's irreducibility criterion to prove that $\sqrt{m/n}$ (where $m,n\in\Z$) is irrational if there is a prime number which divides $m$ but not $n$, and whose square does not divide $m$.
    \item Show that part 4 remains true for $\sqrt[q]{m/n}$, where $q>1$.
    \item If $a$ and $b$ are algebraic over $F$, prove that $F(a,b)$ is a finite extension of $F$.
\end{enumerate}
\end{exercise}
\begin{proof}
 \begin{enumerate}
     \item In Ch.27.D.1 and 2, and Ch.27.E.1 we proved that $F(a)=F(a+c)=F(ca)$ so that the degrees are the same. Furthermore, the technique we used to prove that if $1/a$ is algebraic over $F$ didn't modify the degree of the irreducible polynomial for which $a$ is a root, hence, $F(a)$ has the same degree as $F(1/a)$ over $F$.
     \item We have that $a$ is of degree 1 over $F$, if and only if there is a polynomial of degree 1 in $F[x]$ where $a$ is a root, in particular $x-a=0$, if and only if $x=a\in F$, as required.
     \item This is a Corollary to exercise 2, where $F=\Q$.
     \item Consider the polynomial given by $p(x)=nx^2-m$. Suppose there is a prime number with the desired characteristics. Then if we apply Eisenstein's criterion, we have that $p(x)$ is irreducible over $\Q$. We have that if $nx^2-m=0$, then $\sqrt{m/n}$ is a root of $p(x)$. By exercise 3, this root is irrational.
    \item Since Eisenstein's criterion does not depend on the degree of the polynomial being quadratic, we can apply the proof as above verbatim and only changing the exponent of $x$ in $p(x)$ to be $q$.
    \item We have that if $a$ is algebraic over $F$, then it is also algebraic over $F(b)$, so we can consider $F(b)$ an intermediate extension of $F(a,b)$. By Theorem 2, $[F(a,b):F]=[F(a,b):F(b)][F(b):F]$. We know that both factors on the left hand side are finite, so that $F(a,b)$ is a finite extension, as desired.
 \end{enumerate}
\end{proof}

\begin{exercise}{G Fields of algebraic elements: algebraic numbers}
Let $F\subseteq K$ and $a,b\in K$. We have seen on page 295 that if $a$ and $b$ are algebraic over $F$, then $F(a,b)$ is a finite extension of $F$.

Use the above to prove parts 1 and 2.
\begin{enumerate}
    \item If $a$ and $b$ are algebraic over $F$, then $a+b,\, a-b,\, ab$ and $a/b$ are algebraic over $F$. (In the last case assume $b\neq 0$).
    \item The set $\{x\in K:x\text{ is algebraic over }F\}$ is a subfield of $K$, containing $F$.

    Any complex number which is algebraic over $\Q$ is called an algebraic number. By part 2, the set of all the algebraic numbers is a field, which we shall designated by $\A$.

    Let $a(x)=a_0+\dots+a_nx^n$ be in $\A[x]$, and let $c$ be any root of $a(x)$. We will prove that $c\in\A$. To begin with, all the coefficients of $a(x)$ are in $\Q(a_0,\dots,a_n)$.
    \item Prove: $\Q(a_0,\dots,a_n)$ is a finite extension of $\Q$.

    Let $\Q(a_0,\dots,a_n)=\Q_1$. Since $a(x)\in\Q_1[x]$, $c$ is algebraic over $\Q_1$.

    Prove parts 4 and 5:
    \item $\Q_1(c)$ is a finite extension of $\Q_1$, hence a finite extension of $\Q$. (Why?).
    \item $c\in\A$.

    Conclusion: the roots of any polynomial whose coefficients are algebraic numbers are themselves algebraic numbers.

    A field $F$ is called algebraically closed if the roots of every polynomial in $F[x]$ are in $F$. We have thus proved that $\A$ is algebraically closed.
\end{enumerate}
\end{exercise}
\begin{proof}
 \begin{enumerate}
     \item If $a$ and $b$ are algebraic over $F$ that means that $F(a,b)$ has finite degree over $F$. Furthermore, all of $a+b,a-b,ab$ and $a/b$ are in $F(a,b)$, from Theorem 3, we know that all elements of the extension are algebraic over the original field, as desired.
     \item From 1, we know that the algebraic elements of $K$ satisfy the field axioms. Furthermore, as we proved in exercise E.2., all elements of $F$ are algebraic over $F$.
     \item Since $a(x)\in\A[x]$, by definition the coefficients of $a(x)$ are algebraic. That means that for each one of those coefficients, there is a polynomial of finite degree in $\Q[x]$ such that the coefficient is the root of such polynomial. Then we can construct $\Q_1$ with an iterated extension and conclude by Theorem 2 that $\Q_1$ has a finite degree over $\Q$.
    \item $\Q_1(c)$ is a finite extension of $\Q_1$ because $c$ is the root of a polynomial in $\Q_1[x]$. By Theorem 1, we know that the degree of $\Q_1(c)$ is the degree of the minimum polynomial of $c$ over $\Q_1$.

    Since $\Q_1$ is itself a finite extension of $\Q$, we have, by Theorem 2, $[\Q_1(c):\Q]=[\Q_1(c):\Q_1][\Q_1:\Q]$, we concluded that both terms in the right hand side are finite, hence the left hand side is finite too.
    \item From 4 we concluded that $c$ is the root of a polynomial in $\Q$, because there is a polynomial, but this is precisely the definition of an algebraic number, so that $c\in\A$.
 \end{enumerate}
\end{proof}