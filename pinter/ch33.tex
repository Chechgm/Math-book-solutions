\section*{Chapter 33. Solving Equations by Radicals}
\addcontentsline{toc}{section}{Chapter 33. Solving Equations by Radicals}


\begin{exercise}{A Finding radical extensions}
\begin{enumerate}
    \item Find radical extensions of $\Q$ containing the following complex numbers:
    \begin{enumerate}
        \item $(\sqrt{5}-\sqrt[5]{2})/(\sqrt[4]{3}+\sqrt[3]{4})$.
        \item $\sqrt{(1-\sqrt[9]{2})/\sqrt[3]{1-\sqrt{5}}}$.
        \item $\sqrt[5]{(\sqrt{3}-2i)^3/(i-\sqrt{11}}$.
    \end{enumerate}
    \item Show that the following polynomials in $\Q[x]$ are not solvable by radicals:
    \begin{enumerate}
        \item $2x^5-5x^4+5$.
        \item $x^5-4x^2+2$.
        \item $x^5-4x^4+2x+2$.
    \end{enumerate}
    \item Show that $a(x)=x^5-10x^4+40x^3-80x^2+79x-30$ is solvable by radicals over $\Q$, and give its root field. [Hint: Compute $(x-2)^5-(x-2)$].
    \item Show that $ax^8+bx^6+cx^4+dx^2+e$ is solvable by radicals over any field. [Hint: Let $y=x^2$; use the fact that every fourth degree polynomial is solvable by radicals].
    \item Explain why parts 3 and 4 do not contradict the principal finding of this chapter: that polynomial equations of degree greater than or equal to 5 do not have general solution by radicals.
\end{enumerate}
\end{exercise}
\begin{proof}
 \begin{enumerate}
     \item We are unsure of what is being asked.
     \item
     \begin{enumerate}
         \item By Eisenstein's criterion, the polynomial is not reducible. Furthermore, the polynomial has two local optima at $x=0$ and $x=2$, so that it has three real roots and two complex roots, all denoted as $r_i$. Hence, we have that, if we denote $K$ as the root field of the polynomial, $[K:\Q]=[K:\Q(r_1)][\Q(r_1):\Q]=[K:\Q(r_1)]\cdot 5$. Hence, 5 is a factor of $[K:\Q]$, so that there is a 5-cycle in $[K:\Q]$. Finally, since the polynomial has two complex roots, we have the transposition given by $a+bi\to a-bi$ in $Gal(K:\Q)$ which implies that $Gal(K:\Q)\cong S_5$ so that the polynomial is unsolvable.
         \item As above, by Eisenstein's criterion, the polynomial is irreducible. Furthermore, it has two local optima, giving us 3 real roots and 2 complex roots. The rest of the proof follows as above.
         \item Same technique as above, though this time the optimum have no closed form. Nevertheless, this has no impact on the proof technique. 
     \end{enumerate}
     \item We have that $a(x)=x^5-10x^4+40x^3-80x^2+79x-30= (x-2)(x-3)(x-1)(x^2-4x+5)$, so that it has 2, 3 and 1 as real roots, and $2\pm i$ as complex roots and hence is solvable.
     \item If we do the substitution $y=x^2$, we obtain the polynomial given by $p(y)=ay^4+by^3+cy^2+dy+e$. Since every polynomial of fourth degree is solvable, then we can obtain the roots of the above polynomial by radicals. For any root $r_i$ of $p(y)$, we can substitute $\sqrt{r_i}$ which is a root of the original polynomial. Hence, the original polynomial is solvable by radicals.
    \item Neither 3 or 4 contradict the unsolvability of the quintic because the unsolvability of the quintic talks about the solution by radicals of a general quintic. Both 3 and 4 are specific cases of polynomials.
 \end{enumerate}
\end{proof}

\begin{exercise}{B Solvable groups}
Let $G$ be a group. The symbol $H\triangleleft G$ is commonly used as an abbreviation of ``$H$ is a normal subgroup of $G$''. A normal series of $G$ is a finite sequence $H_0,H_1,\dots,H_n$ of subgroups of $G$ such that $\{e\}=H_0\triangleleft H_1\triangleleft\dots\triangleleft H_n=G$. Such a series is called a solvable series if each quotient group $\quot{H_{i+1}}{H_i}$ is abelian. $G$ is called a solvable group if it has a solvable series.
\begin{enumerate}
    \item Explain why every abelian subgroup is, trivially, a solvable group.
    \item Let $G$ be a solvable group. with a solvable series $H_0,\dots,H_n$. Let $K$ be a subgroup of $G$. Show that $J_0=K\cap H_0,\dots,J_n=K\cap H_n$ is a normal series of $K$.
    \item Use the remark immediately preceding Theorem 2 to prove that $J_0,\dots,J_n$ is a solvable series of $K$.
     \item Use parts 2 and 3 to prove: every subgroup of a solvable group is solvable.
     \item Verify that $\{e\}\subseteq\{e,\beta,\delta\}\subseteq S_3$ is a solvable series for $S_3$. Conclude that $S_3$, and all of its subgroups, are solvable.
     \item In $S_4$, let $A_4$ be the group of all the even permutations, and let $B=\{e,(12)(34),(13)(24),(14)(23)\}$. Show that $\{e\}\subseteq B\subseteq A_4\subseteq S_4$ is a solvable series for $S_4$. Conclude that $S_4$ and all its subgroups are solvable.
\end{enumerate}
\end{exercise}
\begin{proof}
 \begin{enumerate}
     \item We have the chain given by $\quot{G}{\{e\}}\cong G$ which is abelian, giving us our desired chain.
     \item This follows directly from the fact that the intersection between a subgroup and a normal subgroup is normal. So that $J_0,\dots,J_{n}$ are all normal given that all $H_0,\dots,H_{n}$ are normal too.
     \item Since $G$ is solvable, then we know that $\quot{H_{i+1}}{H_i}$ is abelian. We know that a quotient group is abelian if and only if  the divisor, (in this case $H_i$) contains all the commutators of $H_{i+1}$. But this implies that $J_i=K\cap H_i$ contains all the commutators of $J_{i+1}$ (as $K\cap H_i$ contains all the commutators of $K$ contained in $H_i$). Since $J_{i}$ contains all the commutators of $J_{i+1}$, then the quotient is abelian, using the same result as above.
     \item This is essentially the conclusion of exercises 2 and 3, by considering $J_0,\dots,J_n$ the required series of subgroups that make $K$ solvable.
     \item From the table on page 72, we can conclude that $\{e,\beta,\delta\}$ is abelian, hence $\quot{S_3}{\{e,\beta,\delta\}}$ is abelian too and clearly $\quot{\{e,\beta,\delta\}}{\{e\}}$ is abelian too. Hence, we have our desired solvable series. By the result in exercise 4, every subgroup of $S_3$ is also solvable.
     \item We have that $\quot{S_4}{A_4}\cong \Z_2$, which is abelian. By computation, we can also conclude that $B$ is abelian so that $\quot{A_4}{B}$ is abelian too, giving us our desired solvable series. By exercise 4, all subgroups of $S_4$ are solvable.

     Remark: The greater conclusion of exercises 5 and 6 is that all polynomials of degree 3 and 4 are solvable by radicals.
 \end{enumerate}
\end{proof}