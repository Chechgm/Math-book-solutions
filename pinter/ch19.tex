\subsection*{Chapter 19. Quotient rings}
\addcontentsline{toc}{subsection}{Chapter 19. Quotient rings}


\begin{exercise}{B Examples of the use of the FHT}
In each of the following, use the FHT (Fundamental Homomorphism Theorem) to prove that the two given groups are isomorphic. Then display their tables
  \begin{enumerate}
      \item $\mathbb{Z}_{5}$ and $\quot{\mathbb{Z}_{20}}{\langle 5\rangle}$
      \item $\mathbb{Z}_{3}$ and $\quot{\mathbb{Z}_{6}}{\langle 3\rangle}$
  \end{enumerate}
\end{exercise}
\begin{proof}
 \begin{enumerate}
     \item Let $f:\Z_{20}\rightarrow\Z_{5}$ be defined as $f(a)=[a]$. $\ker f$ is $\langle 5\rangle$. Hence, $\quot{\Z_{20}}{\langle 5\rangle}\cong\Z_{5}$ by the FHT.
     \item Let $f:\Z_{6}\rightarrow\Z_{3}$ be defined as $f=
     \begin{pmatrix}
     0 & 1 & 2 & 3 & 4 & 5\\
     0 & 1 & 2 & 0 & 1 & 2
     \end{pmatrix}$, we have $\ker f=\langle 3 \rangle$ so that $\quot{\Z_{6}}{\langle 3\rangle}\cong\Z_{3}$ by the FHT.
 \end{enumerate}
\end{proof}


\begin{exercise}{E Properties of quotient rings $\quot{A}{J}$ in relation to properties of $J$}
Let $A$ be a ring and $J$ an ideal of $A$. Use conditions (1), (2) and (3) of this chapter. Prove each of the following
  \begin{enumerate}
      \item Every element of $\quot{A}{J}$ has a square root iff for every $x\in A$, there is some $y\in A$ such that $x-y^{2}\in J$.
      \item Every element of $\quot{A}{J}$ is its own negative iff $x+x\in J$ for every $x\in A$.
      \item $\quot{A}{J}$ is a boolean ring iff $x^{2}-x\in J$ for every $x\in A$. (A ring $S$ is called a boolean ring iff $s^{2}=s$ for every $s\in S$).
      \item If J is an ideal of all the nilpotent elements of a commutative ring $A$, then $\quot{A}{J}$ has no nilpotent elements (except zero). (Nilpotent elements from Chapter 17 M.: ``An element $a$ of a ring is nilpotent if $a^{n}=0$ for some positive integer $n$'').
      \item Every element of $\quot{A}{J}$ is nilpotent iff $J$ has the following property: for every $x\in A$, there is a positive integer $n$ such that $x^{n}\in J$.
      \item $\quot{A}{J}$ has unity element iff there exists an element $a\in A$ such that $ax-x\in J$ and $xa-x\in J$ for every $x\in A$.
 \end{enumerate}
\end{exercise}
\begin{proof}
 \begin{enumerate}
     \item ($\Rightarrow$) Suppose every element of $\quot{A}{J}$ has a square root. Then for all $x\in A$, there exists a $y\in A$ with $(J+y)^{2}=J+y^{2}=J+x$. By property (2), $x-y^{2}\in J$, as required.

     ($\Leftarrow$) Suppose that for every $x\in A$, there exists a $y\in A$ such that $x-y^{2}\in J$. Then by (2), we have $J+x=J+y^{2}=(J+y)^{2}$ so that $y$ is the square root of $x$.
     \item ($\Rightarrow$) Suppose every element of $\quot{A}{J}$ is its own inverse, then for every $x\in A$, $(J+x)+(J+x)=J+x+x=J$, by property (3), $x+x\in J$.

     ($\Leftarrow$) Suppose $x+x\in J$ for all $x\in A$. By property (3), we have $J+x+x=(J+x)+(J+x)=J$ so that $(J+x)$ is its own negative.
     \item ($\Rightarrow$) Suppose $\quot{A}{J}$ is a boolean ring, so that $(J+x)^{2}=(J+x)$ for all $x\in A$. Then $J+x^{2}=J+x$ and by property (2), $x^{2}-x\in J$.

     ($\Leftarrow$) Suppose that for all $x\in A$, $x^{2}-x\in A$ holds. Then by property (2) $J+x^{2}=(J+x)^{2}=J+x$, hence $\quot{A}{J}$ is a boolean ring.
     \item Suppose, $(J+x)$ is nilpotent. Then there exists a positive integer $n$ such that $(J+x)^{n}=J+x^{n}=J$. By property (3), $x^{n}\in J$, so that there is a $y\in J$ such that $x^{n}=y$, but $y$ is nilpotent, so that there exists a positive integer, $m$, such that $y^{m}=0$, hence $y^{m}=(x^{n})^{m}=x^{nm}=0$ and $x$ is nilpotent and $x\in J$. But this implies that the only nilpotent element in $\quot{A}{J}$ is $J$, as required.
     \item ($\Rightarrow$) Suppose every element of $\quot{A}{J}$ nilpotent. Then there exists a positive integer $n$, such that $(J+x)^{n}=J+x^{n}=J$. By property (3), $x^{n}\in J$.

     ($\Leftarrow$) Suppose for every $x\in A$, there is a positive integer $n$ such that $x^{n}\in J$. By property (3), $J+x^{n}=(J+x)^{n}=J$, so that $(J+x)$ is nilpotent.
     \item ($\Rightarrow$) Suppose $\quot{A}{J}$ has a unity, $J+a$. Then for every $x\in A$, $(J+a)(J+x)=J+ax=J+x$, so by property (2), $ax-x\in J$. Likewise, $(J+x)(J+a)=J+xa=J+x$, and, $xa-x\in J$, as required.

     ($\Leftarrow$) Suppose there exists an $a\in A$ such that $ax-x, xa-x\in J$ for all $x\in A$. By property (2), we have $J+ax=(J+a)(J+x)=J+x$ and $J+xa=(J+x)(J+a)=J+x$, but this is the same as saying $J+a$ is the unity of $\quot{A}{J}$.
 \end{enumerate}
\end{proof}


\begin{exercise}{F Prime and maximal ideals}
Let $A$ be a commutative ring with unity, and $J$ an ideal of $A$. Prove each of the following:
  \begin{enumerate}
      \item $\quot{A}{J}$ is a commutative ring with unity.
      \item $J$ is a prime ideal iff $\quot{A}{J}$ is an integral domain.
      \item Every maximal ideal of $A$ is a prime ideal. (Hint: Use the fact, proved in this chapter, that if $J$ is a maximal ideal then $\quot{A}{J}$ is a field).
      \item If $\quot{A}{J}$ is a field, then $J$ is a maximal ideal. (Hint: See exercise I2 of Chapter 18: ``If $f:A\rightarrow B$ is a homomorphism from $A$ onto $B$, and $B$ is a field, then the kernel of $f$ is a maximal ideal'').
  \end{enumerate}
\end{exercise}
\begin{proof}
 \begin{enumerate}
     \item We know that $\quot{A}{J}$ is a ring. It is left to prove i) that it is commutative and ii) it has unity

     i) Commutativity: Let $J+a, J+b\in\quot{A}{J}$. We have $(J+a)(J+b)=J+ab=J+ba=(J+b)(J+a)$, as required.

     ii) Unity: Let $a$ be the unity of $A$, so that for all $x\in A$, $xa=ax=x$. Let $J+x\in\quot{A}{J}$. We have $(J+a)(J+x)=J+ax=J+x$ and $(J+x)(J+a)=J+xa=J+x$, so that $J+a$ is the unity of $\quot{A}{J}$.
     \item ($\Rightarrow$) Suppose $J$ is a prime ideal. Since $A$ is a commutative ring $\quot{A}{J}$ is a commutative ring with unity. It is only left to prove that $\quot{A}{J}$ has the cancellation property. Let $a,b,c\in A$. Suppose $(J+a)(J+b)=J+ab=J+ac=(J+a)(J+c)$. By property (2), $ab-ac=a(b-c)\in J$. 

     Since $J$ is a prime ideal, either $a\in J$ or $b-c\in J$. If $a\in J$, then $(J+a)(J+b)=J+b=J+c=(J+a)(J+c)$. If $b-c\in J$, then by property (2), $J+b=J+c$. In both cases, $\quot{A}{J}$ has the cancellation property and it is an integral domain.

     ($\Leftarrow$) Suppose $\quot{A}{J}$ is an integral domain. Let $ab\in J$, then by property (3), $J+ab=(J+a)(J+b)=J$. Integral domains have no divisors of zero. Because the zero element in $\quot{A}{J}$ is $J$, then either $J+a=J$ or $J+b=J$ which implies that either $a$ or $b$ are in $J$, and $J$ is prime.
     \item Let $J$ be a maximal ideal of $A$. Then $\quot{A}{J}$ is a field, and because all fields are integral domains, $\quot{A}{J}$ is an integral domain. By 3), $J$ is a prime ideal.
     \item Consider the natural homomorphism $f:A\rightarrow\quot{A}{J}$ given by $f(a)=J+a$, we know $\ker f=J$. Exercise Ch.18.I.2 tells us that if $\quot{A}{J}$ is a field, then $\ker f$ is a maximal ideal, so $J$ is a maximal ideal, as required.
 \end{enumerate}
\end{proof}


\begin{exercise}{G Further properties of quotient rings in relation to their ideals}
Let $A$ be a ring and $J$ an ideal of $A$. (In parts 1-3 and 5 assume that $A$ is a commutative ring with unity).
  \begin{enumerate}
      \item Prove that $\quot{A}{J}$ is a field iff for every element $a\in A$, where $a\notin J$, there is some $b\in A$ such that $ab-1\in J$.
      \item Prove that every nonzero element of $\quot{A}{J}$ is either invertible or a divisor of zero iff the following property holds, where $a,x\in A$: For every $a\notin J$, there is some $x\notin J$ such that either $ax\in J$ or $ax-1\in J$.
      \item An ideal $J$ of a ring is called primary iff for all $a,b\in A$, if $ab\in J$, then either $a\in J$ or $b^{n}\in J$ for some positive integer $n$. Prove that every zero divisor in $\quot{A}{J}$ is nilpotent iff $J$ is primary.
      \item An ideal $J$ of a ring $A$ is called semiprime iff it has the following property: For every $a\in A$, if $a^{n}\in J$ for some positive integer $n$, then necessarily $a\in J$. Prove that $J$ is semiprime iff $\quot{A}{J}$ has no nilpotent elements (except zero).
      \item Prove that an integral domain can have no nonzero nilpotent elements. Then use part 4, together with Exercise F2 to prove that every prime ideal in a commutative ring is semiprime.
  \end{enumerate}
\end{exercise}
\begin{proof}
\begin{enumerate}
 \item ($\Leftarrow$) From F.1, we know $\quot{A}{J}$ is a commutative ring with unity. It remains to be proven that every element of $\quot{A}{J}$ is invertible. Let $a\in A$, with $a\notin J$. By hypothesis, there exists $b\in A$ such that $ab-1\in J$. By property (2), we have $J+ab=(J+a)(J+b)=J+1$. Because $\quot{A}{J}$ is commutative, then $(J+a)(J+b)=(J+b)(J+a)$ and $J+a$ and $J+b$ are inverses of each other.

 ($\Rightarrow$) Let $\quot{A}{J}$ be a field, that is, a commutative ring with unity and all its elements are invertible. Then for any $J+a\in\quot{A}{J}$, there exists a $J+b$ such that $(J+a)(J+b)=J+ab=J+1$. By property (2), this implies $ab-1\in J$ Since this is true for all $J+a$, then it is true for all $a\in A$.
 \item ($\Leftarrow$) Let $a\in A$ with $a\notin J$. Then there exists $x\in A$ with $x\notin J$ such that either: 
 
 i) $ax\in J$, which by property (3) implies $J+ax=(J+a)(J+b)=J$, that is $J+a$ and $J+b$ are divisors of zero. 
 
 Or ii) $ax-1\in J$, which by property (2) implies $J+ax=(J+a)(J+x)=J+1$, and because $\quot{A}{J}$ is a commutative ring, $J+a$ and $J+x$ are inverses. 

 ($\Rightarrow$) Suppose every nonzero element of $\quot{A}{J}$ is either i) invertible or ii) a divisor of zero. Let $J+a\in\quot{A}{J}$. If $J+a$ is:

 i) Invertible, then there exists $J+x$ such that $(J+a)(J+x)=J+ax=J+1$. By property (2) this implies $ax-1\in J$.

 ii) Divisor of zero, then there exists $J+x$ such that $(J+a)(J+x)=J+ax=J$, which by property (3) implies $ax\in J$.
 \item ($\Rightarrow$) Suppose every zero divisor in $\quot{A}{J}$ is nilpotent. Then for any divisor of zero, say $J+a$, in $\quot{A}{J}$, there exists a positive integer $n$ such that $(J+a)^{n}=J+a^{n}=J$. Suppose $ab\in J$, then by property (3), $J+ab=(J+a)(J+b)=J$ so that $J+a$ and $J+b$ are divisors of zero. This means either $a$ or $b$ are in $J$ or, by the nilpotent property, $a^{n}$ or $b^{m}$ belong to $J$ for some positive integers $n$ and $m$.

 ($\Rightarrow$) Suppose $J$ is primary. Let $J+a, J+b$ be divisors of zero in $\quot{A}{J}$ so that $(J+a)(J+b)=J+ab=J$. By property (3) this implies $ab\in J$. Because $J$ is primary, then $a^{n},b^{m}$ for some positive integers $n$ and $m$. By property (3) we have $J+a^{n}=(J+a)^{n}=J$, and $J+b^{m}=(J+b)^{m}=J$ so that $J+a$ and $J+b$ are nilpotent.
 \item ($\Rightarrow$) Suppose $J$ is semiprime. Then for all $a\in A$, if $a^{n}\in J$ for some positive integer $n$, then necessarily $a\in J$. But $a^{n}\in J$ is the same as $J+a^{n}=(J+a)^{n}=J$ because of property (3), thus $J+a=J$ and $\quot{A}{J}$ has no nilpotent elements.

 ($\Leftarrow$) Suppose $\quot{A}{J}$ has no nilpotent elements. So that if $(J+a)^{n}=J$ for any $a\in A$ and some positive integer $n$, then $a\in J$. Notice that $(J+a)^{n}=J+a^{n}=J$ implies $a^{n}\in J$ by property (3) but this means $J$ is a semiprime, as required.
 \item i) Let $A$ be an integral domain. Suppose $a\in A$ is a nonzero nilpotent element of $A$. Then there exists a positive integer $n$ such that $a^{n}=0$. But this implies $aa^{n-1}=0$ so that $a$ and $a^{n-1}$ are divisors of zero, which is a contradiction.

 ii) Let $J$ be a prime ieal in a commutative ring $A$. Then by F.2, $\quot{A}{J}$ is an integral domain. We just proved that integral domains have no nonzero nilpotent elements, hence, by 4), $J$ is semiprime, as required.
\end{enumerate}
\end{proof}


\begin{exercise}{H $\Z_{n}$ as a homomorphic image of $\mathbb{Z}$}
Recall that the function $f(a)=\bar{a}$ is the natural homomorphism from $\Z$ onto $\Z_{n}$. If a polynomiaal equation $p=0$ is satisfied in $\Z$, necessarily $f(p)=f(0)$ is true in $\Z_{n}$. Let us take a specific example; there are integers $x$ and $y$ satisfying $11x^{2}-8y^{2}+29=0$ (we may take $x=3$ and $y=4$). It follows that there must be elements $\bar{x}$ and $\bar{y}$ in $\Z_{6}$ which satisfy $\bar{11}\bar{x}^{2}-\bar{8}\bar{y}^{2}+\bar{29}=\bar{0}$ in $\Z_{6}$, that is $\bar{5}\bar{x}^{2}-\bar{2}\bar{y}^{2}+\bar{5}=0$. (We take $\bar{x}=\bar{3}$ and $\bar{y}=\bar{4}$). The problems which follow are based on this observation.
  \begin{enumerate}
      \item Prove that the equation $x^{2}-7y^{2}-24=0$ has no integer solutions. (Hint: if there are integers $x$ and $y$ satisfying this equation, what equation will $\bar{x}$ and $\bar{y}$ satisfy in $\Z_{7}$?)
      \item Prove that $x^{2}+(x+1)^{2}+(x+2)^{2}=y^{2}$ has no integer solutions.
      \item Prove that $x^{2}+10y^{2}=n$ (where $n$ is an integer) has no integer solutions if the last digit of $n$ is 2, 3, 7, or 8.
      \item Prove that the sequence 3, 8, 13, 18, 23,\dots. does not include the square of any integer. (Hint: the image of each number on this list, under the natural homomorphism from $\Z$ to $\Z_{5}$, is $\bar{3}$).
      \item Prove that the sequence 2, 10, 18, 26,\dots does not include the cube of any integer.
      \item Prove that the sequence 3, 11, 19, 27,\dots does not include the sum of two squares of integers.
      \item Prove that if $n$ is a product of two consecutive integers, its units digit must be 0, 2, or 6.
      \item Prove that if $n$ is the product of three consecutive integers, its units digit must be 0, 4, or 6.
  \end{enumerate}
\end{exercise}
\begin{proof}
 \begin{enumerate}
     \item In $\Z_{7}$, $x^{2}-7y^{2}-24=0$ becomes $\bar{x}^{2}=\bar{24}=\bar{3}$. Since this equation cannot be satisfied in $\Z_{7}$, then the equation has no integer solutions.
     \item First, we simplify the expression: $x^{2}+(x+1)^{2}+(x+2)^{2}=x^{2}+x^{2}+2x+1+x^{2}+4x+4=3x^{2}+6x+5=y^{2}$. In $Z_{3}$ the expression becomes $\bar{y}^{2}=\bar{5}=\bar{2}$. Since this equation cannot be satisfied in $Z_{3}$, then it has no integer solutions.
     \item For this exercise we consider $\Z_{10}$. Any integer $n$ modulo 10 will belong to the congruence class of the value of its last digit. For the equation, we have that $x^{2}+10y^{2}=n$ becomes $\bar{x}^{2}=\bar{n}$. We then check all possibilities: $\bar{0}^{2}=\bar{0}, \bar{1}^{2}=\bar{1}, \bar{2}^{2}=\bar{2}, \bar{3}^{2}=\bar{3}, \bar{4}^{2}=\bar{6}, \bar{5}^{2}=\bar{5}, \bar{6}^{2}=\bar{6}, \bar{7}^{2}=\bar{9}, \bar{8}^{2}=\bar{4}, \bar{9}^{2}=\bar{1}$. Since 2,3,7 or 8 do not appear on the right hand side of these equalities, then the equation cannot be satisfied when $n$ has any of these numbers as last digit.
     \item Using the natural homomorphism, we have that in $\Z_{5}$ we want to satisfy the equation $\bar{x}^{2}=\bar{3}$. However, this is not possible, so the equation cannot hold in $\Z$.
     \item As above, we want to satisfy $\bar{x}^{3}=\bar{2}$ in $\Z_{8}$. If we try this equation for all congruence classes in $\Z_{8}$ we find that the equation cannot be satisfied. Thus, it has no integer solutions.
     \item Consider $\Z_{8}$, then the sequence should fulfill the following equation, if any of those numbers were a sum of two square integers: $\bar{x}^{2}+\bar{y}^{2}=\bar{3}$, using the following algorithm, I verified on Python that the equation cannot be satisfied:
     \begin{algorithmic}
        \FOR{$x=0,\dots,7$}
        \STATE{
            \FOR{$y=0,\dots,7$}
            \STATE{
                    \IF{$\bar{x}^{2}+\bar{y}^{2}=\bar{3}$}
                        \STATE{Break}
                    \ELSE 
                        \STATE{Continue}
                    \ENDIF
            }
            \ENDFOR
        }
        \ENDFOR
    \end{algorithmic}
    \item Using a similar algorithm as above, I verified that in $\Z_{10}$, the product of two consecutive integers must have unit digit 0, 2 or 6. Because $\Z_{10}$ is the quotient ring of $\Z$ that preserves the information of the last digit of every integer, then if the statement is true in $\Z_{10}$, it is also true for all integers.
    \item As above.
 \end{enumerate}
\end{proof}