\subsection*{Chapter 32. Galois Theory: the Heart of the Matter}
\addcontentsline{toc}{subsection}{Chapter 32. Galois Theory: the Heart of the Matter}


\begin{exercise}{A Computing a Galois group}
\begin{enumerate}
    \item Show that $\Q(i,\sqrt{2})$ is the root field of $(x^2+1)(x^2-2)$ over $\Q$.
    \item Find the degree of $\Q(i,\sqrt{2})$ over $\Q$.
    \item List the elements of $Gal(\Q(i,\sqrt{2}):\Q)$ and exhibit its table.
    \item Write the inclusion diagram for the subgroups of $Gal(\Q(i,\sqrt{2}):\Q)$, and the inclusion diagram for the fields intermediate between $\Q$ and $\Q(i,\sqrt{2})$. Indicate the Galois correspondence.
\end{enumerate}
\end{exercise}
\begin{proof}
 \begin{enumerate}
     \item The roots of $x^2+1$ are $\pm i$ and the roots of $x^2-2$ are $\pm\sqrt{2}$, so that $\Q(i,\sqrt{2})$ is the root field of $(x^2+1)(x^2-2)$.
     \item We know that $[Q(i,\sqrt{2}:\Q]=[Q(i,\sqrt{2}):\Q(i)][\Q(i):\Q]=2\cdot 2=4$.
     \item Following a similar approach to the example in the chapter, we have that $e$ is the transformation that leaves the roots invariant, $\alpha$ transforms $\sqrt{2}$ into $-\sqrt{2}$, $\beta$ transforms $i$ into $-i$ and $\gamma$ transforms both $\sqrt{2}$ and $i$ into their negative counterparts. The multiplication table is the same as $Gal(\Q(\sqrt{2},\sqrt{3}):\Q)$.
     \item The Galois correspondence is the same as the correspondence between $Gal(\Q(\sqrt{2},\sqrt{3}):\Q)$ and the intermediate fields of $\Q(\sqrt{2},\sqrt{3})$. The main difference is that $\Q(\sqrt{6})$ is replaced by $\Q(i\sqrt{2})$, which is the fixfield of the subgroup $\{e,\gamma\}\in Gal(\Q(\sqrt{2},\sqrt{3}):\Q)$.
 \end{enumerate}
\end{proof}

\begin{exercise}{B Computing a Galois group of eight elements}
\begin{enumerate}
    \item Show that $\Q(\sqrt{2},\sqrt{3},\sqrt{5})$ is the root field of $(x^2-2)(x^2-3)(x^2-5)$ over $\Q$.
    \item Show that the degree of $\Q(\sqrt{2},\sqrt{3},\sqrt{5})$ over $\Q$ is 8.
    \item List the 8 elements of $\mathbf{G}=Gal(\Q(\sqrt{2},\sqrt{3},\sqrt{5}):Q)$ and write its table.
    \item List the subgroups of $\mathbf{G}$. (By Lagrange's Theorem, any proper subgroup of $\mathbf{G}$ has either two or four elements).
     \item For each subgroup of $\mathbf{G}$ find its fixfield.
     \item Indicate the Galois correspondence by means of a diagram like the one on page 329.
\end{enumerate}
\end{exercise}
\begin{proof}
 \begin{enumerate}
     \item We have that $\pm\sqrt{2}$ are the roots of $x^2-2$, $\pm\sqrt{3}$ the roots of $x^2-3$, and $\pm\sqrt{5}$ the roots of $x^2-5$, so that $\Q(\sqrt{2},\sqrt{3},\sqrt{5})$ is the root field of $(x^2-2)(x^2-3)(x^2-5)$ over $\Q$.
     \item We know that 
     \begin{align*}
         [\Q(\sqrt{2}&,\sqrt{3},\sqrt{5}):\Q]\\
         =& [\Q(\sqrt{2},\sqrt{3},\sqrt{5}):\Q(\sqrt{2},\sqrt{3})][\Q(\sqrt{2},\sqrt{3}):\Q]\\
         =& [\Q(\sqrt{2},\sqrt{3},\sqrt{5}):\Q(\sqrt{2},\sqrt{3})][\Q(\sqrt{2},\sqrt{3}):\Q(\sqrt{2})][\Q(\sqrt{2}):\Q]\\
         =& 2\cdot 2\cdot 2=8,
     \end{align*}
     as required.
     \item The 8 elements of $\bG$ are the automorphisms that change the roots $\sqrt{2},\sqrt{3}$ and $\sqrt{5}$ for its negative counterparts. Since there are 3 square roots, there are $2^3$ automorphisms including the automorphism that changes no square roots (the identity) and the one that changes all. 
     \item Taking the example on the book as reference, we can see that each element of $\bG$ with the identity element forms a subgroup of $\bG$, furthermore, that gives us 6 subgroups (and 8 if we count the trivial subgroups). Furthermore, for each pair in $\{2,3,5\}$, we can take the subgroup that includes the automorphism that send each square root to its negative and that takes their product to their negative. For example, the automorphisms that: take $\sqrt{2}$ to its negative, take $\sqrt{3}$ to its negative, and take $\sqrt{6}$ to its negative plus the identity form a subgroup of $\bG$. Since there are 3 elements in the set above, there are $C(3,2)=3$ such subgroups. Giving us a total of 11 subgroups of $\bG$.
     \item For each subgroup of order 2, their fixfield is simply the field extension of the element which is taken to its negative: $\sqrt{2}, \sqrt{3}, \sqrt{5},\sqrt{6}, \sqrt{10},\dots$. For the subgroups of order 4, the  fixfields are the combinations of those roots that are changed: $\Q(\sqrt{2},\sqrt{3}), \Q(\sqrt{2},\sqrt{5}),\dots$. Notice that, for example $\Q(\sqrt{6})\subseteq\Q(\sqrt{2},\sqrt{3})$.
     \item The correspondence follows directly from the previous exercise. The figure would have 4 hierarchies instead of 3 as in the book example.
 \end{enumerate}
\end{proof}

\begin{exercise}{H The group of automorphisms of $\C$}
\begin{enumerate}
    \item Prove: the only automorphism of $\Q$ is the identity function. [Hint: if $h$ is an automorphism, $h(1)=1$; hence $h(2)=2$, and so on].
    \item Prove: any automorphism of $\R$ sends squares of numbers to squares of numbers, hence positive numbers to positive numbers.
    \item Using part 2, prove that if $h$ is any automorphism of $\R$, $a<b$ implies $h(a)<h(b)$.
    \item Use parts 1 and 3 to prove that the only automorphism of $\R$ is the identity function.
     \item List the elements of $Gal(\C:\R)$.
     \item Prove that the identity function and the function $a+bi\rightarrow a-bi$ are the only autmorphisms of $\C$ which fix $\R$.
\end{enumerate}
\end{exercise}
\begin{proof}
 \begin{enumerate}
     \item Let $h$ be an automorphism of $\Q$. We will first prove by induction that for all $n\in\N$, $h(n)=n$. For the base case, notice that $h(1)=1$ by the properties of homomorphisms. Now suppose the statement holds for $n$. We then have $h(n+1)=h(n)+h(1)=n+1$, as required. By the properties of ring homomorphisms, we have that $h(-n)=-h(n)=-n$, so that $h$ maps every integer to itself. Finally, by the properties of ring homomorphisms, $h(n^{-1})=h(n)^{-1}=n^{-1}$. Notice that any rational can be represented as $m/n$ for integers $m$ and $n$. So that $h(m/n)=h(m)h(n)^{-1}=m/n$. That is, $h$ is the identity.
     \item Let $h$ be an automorphism of $\R$, and let $a\in\R$. We have $h(a^2)=h(aa)=h(a)h(a)=h(a)^2$, as required.
     \item Because $b>a$, there exists an element $x\in\R$ so that $b-a=x^2$. From 2, we know that $h(x^2)=h(x)^2$ so that $h(b-a)=h(b)-h(a)$ is positive and hence $h(b)>h(a)$.
     \item Suppose, for the sake of contradiction that there exists a real $x$ with $h(x)\neq x$. Suppose without loss of generality that $x>h(x)$. Then there exists a rational, $a$ so that $h(x)<a<x$. But notice that by 3, if $x>a$, then $h(x)>h(a)$, and by 1, $h(a)=a$, so that $h(x)<a<x$ is a contradiction.
     \item By Theorem 1, the degree of $Gal(\C:\R)=2$. Then, we know that the automorphisms of $\C$ which fix $\R$ are simply those automorphisms that send $a+bi$ to itself or that send it to $a-bi$.
     \item This is the same as the question above.
 \end{enumerate}
\end{proof}

\begin{exercise}{I Further questions relating to Galois groups}
Throughout this set of questions, let $K$ be a root field over $F$, let $\mathbf{G}=Gal(K:F)$, and let $I$ be any intermediate field. Prove the following:
\begin{enumerate}
    \item $I^\ast=Gal(K:I)$ is a subgroup of $\mathbf{G}$.
    \item If $H$ is a subgroup of $\mathbf{G}$ and $H^\circ=\{a\in K: \pi(a)=a\text{ for every }\pi\in H\}$, then $H^\circ$ is a subfield of $K$, and $F\subseteq H^\circ$.
    \item Let $H$ be the fixer of $I$, and $I'$ the fixfield of $H$. Then $I\subseteq I'$. Let $I$ be the fixfield of $H$, and $I^\ast$ the fixer of $I$. Then $H\subseteq I^\ast$.
    \item Let $I$ be a normal extension of $F$ (that is, a root field of some polynomial over $F$). If $\mathbf{G}$ is abelian, then $Gal(K:I)$ and $Gal(I:F)$ are abelian. (Hint: Use Theorem 4).
     \item Let $I$ be a normal extension of $F$. If $\mathbf{G}$ is a cyclic group, then $Gal(K:I)$ and $Gal(I:F)$ are cyclic groups.
     \item If $\mathbf{G}$ is a cyclic group, there exists exactly one intermediate field $I$ of degree $k$, for each integer $k$ dividing $[K:F]$.
\end{enumerate}
\end{exercise}
\begin{proof}
 \begin{enumerate}
     \item Let $g,h\in I^\ast$. We know that $I^\ast$ is closed under composition because if $g$ and $h$ fix $I$, then certainly $g\circ h$ and $h\circ g$ too. Furthermore, since $g$ moves only elements of $K$ that are not in $I$, then $g^{-1}$ moves only those elements that are originally moved, in other words, $g^{-1}$ fixes $I$, so that it is in $I^\ast$. Hence, $I^\ast$ is a subgroup of $\bG$.
     \item Let $a,b\in H^\circ$. For all $\pi\in H$, we have that $\pi(a-b)=\pi(a)-\pi(b)=a-b$ so that $a-b$ is also fixed by the elements of $H$. Furthermore, $\pi(ab)=\pi(a)\pi(b)=ab$, so that $ab$ is also fixed by the elements of $H$. As a result, $H^\circ$ is a subfield of $K$.

     To prove that $F\subseteq H^\circ$, notice that because $\bG=Gal(K:F)$, then $F$ is fixed by any element of $\bG$ and hence by any subgroup of $\bG$, so that $F$ is contained in $H^\circ$.
     \item By definition, if $H$ is the fixer of $I$, all elements of $H$ fix $I$. On the other hand, $I'$ the fixfield of $H$, contains all the elements in $K$ that are not moved by all elements of $H$. Hence for $a\in I$, $a\in I'$ and $I\subseteq I'$.

     If $I$ is the fixfield of $H$, then all the elements in $I$ are not moved by all the elements of $H$. Likewise, if $I^\ast$ is the fixer of $I$, then all the elements of $I^\ast$ fix $I$, as a result, $H\subseteq I^\ast$.
     \item We know that $Gal(K:I)$ is abelian because from exercise 1, $Gal(K:I)\subseteq Gal(K:F)$ an any subgroup of an abelian group is abelian.

     To conclude that $Gal(I:F)$ is abelian, notice that from Theorem 4, $Gal(I:F)\cong\quot{Gal(K:F)}{Gal(K:I)}$. But from group theory, we know that the quotient of an abelian group is abelian, then $Gal(I:F)$ is abelian as desired.
     \item Since $Gal(K:I)$ is a subgroup of $Gal(K:F)$, and we know that the subgroup of a cyclic group is cyclic, then $Gal(K:I)$ is cyclic.

     Again, by Theorem 4, we have that $Gal(I:F)\cong\quot{Gal(K:F)}{Gal(K:I)}$. But the quotient of a cyclic group is cyclic, so that $Gal(I:F)$ is cyclic, as desired.
     \item First, we know that the intermediate fields between $K$ and $I$ are identified by the subgroups of $\bG$ by the Galois correspondence. Hence, we can write this question in its group-only form as: ``If $\bG$ is cyclic, there exists exactly one subgroup of order $k$ for each integer dividing the order of $\bG$''. But this is a result from group theory, so that the desired result is fulfilled.
 \end{enumerate}
\end{proof}

\begin{exercise}{J Normal extensions and normal subgroups}
Suppose $F\subseteq K$, where $K$ is a normal extension of $F$. (This means simply that $K$ is the root field of some polynomial in $F[x]$: see Chapter 31 exercise K). Let $I_1\subseteq I_2$ be intermediate fields.
\begin{enumerate}
    \item Deduce from Theorem 4 that, if $I_2$ is a normal extension of $I_1$, then $I^\ast_2$ is a normal subgroup of $I^\ast_1$.
    \item Prove the following for any intermediate field $I$: Let ${h\in Gal(K:F)},\, g\in I^\ast,\, a\in I$, and $b=h(a)$. Then $[h\circ g\circ h^{-1}](b)=b$. Conclude that $hI^\ast h^{-1}\subseteq h(I)^\ast$.
    \item Use part 2 to prove that $hI^\ast h^{-1}=h(I)^\ast$.

    Two intermediate fields $I_1$ and $I_2$ are called conjugate if and only if there is an automorphism [i.e., an element $i\in Gal(K:F)$] such that $i(I_1)=I_2$.
    \item Use part 3 to prove that $I_1$ and $I_2$ are conjugate if and only if $I^\ast_1$ and $I^\ast_2$ are conjugate subroups in the Galois group.
    \item Use part 4 to prove that for any intermediate fields $I_1$ and $I_2$: if and only if $I^\ast_2$ is a normal subgroup of $I^\ast_1$, then $I_2$ is a normal extension of $I_1$.

    Combining parts 1 and 5 we have: $I_2$ is a normal extension of $I_1$ if and only if $I^\ast_2$ is a normal subgroup of $I^\ast_1$. (Historically, this result is the origin of the word ``normal'' in the term ``normal subrgroup'').
\end{enumerate}
\end{exercise}
\begin{proof}
 \begin{enumerate}
     \item We have that $I_1^\ast=Gal(K:I_1)$ and $I_2^\ast=Gal(K:I_2)$. Furthermore, if $I_1\subseteq I_2\subseteq K$, and $I_2$ is a normal extension of $I_1$ and $K$ is a normal extension of $I_1$ because $F\subseteq I_1\subseteq K$ and $K$ is a normal extension of $F$. Then by Theorem 4, we would have that $Gal(I_2:I_1)=\quot{Gal(K:I_1)}{Gal(K:I_2)}=\quot{I_1^\ast}{I_2^\ast}$ and $I_2^\ast$ is a normal subgroup of $I_1^\ast$.
     \item By definition, $h^{-1}(b)=a$, but $a\in I$, and $g\in I^\ast$, so that $g(a)=a$. Finally, by definition $h(a)=b$. Putting all of these together, we have $[h\circ g\circ h^{-1}](b)=b$. 

     Since the chosen $a, b$ and $g$ were arbitrary, that means that any element of $hI^\ast h^{-1}$ fixes any element of $I$ mapped by $h$, mathematically, $hI^\ast h^{-1}\subseteq h(I)^\ast$.
     \item We already proved that $hI^\ast h^{-1}\subseteq h(I)^\ast$. We will now prove that $hI^\ast h^{-1}$ has the same number of elements as $h(I)^\ast$. Since $h$ is an automorphism, $h(I)^\ast$ must have the same number of elements as $I^\ast$. Likewise, $hI^\ast h^{-1}$ must have the same elements as $I^\ast$. Hence, $h(I)^\ast$ and $hI^\ast h^{-1}$ have the same number of elements and using part 2, they are equal.
     \item ($\Rightarrow$) Suppose $I_1$ and $I_2$ are conjugate. Then there exists $h\in Gal(K:F)$ such that $h(I_1)=I_2$, which implies $h(I_1)^\ast=I_2^\ast$. By 3, we have $I_2^\ast= h(I_1)^\ast= hI_1^\ast h^{-1}$, so that $I_1^\ast$ and $I_2^\ast$ are conjugate, as desired.

     ($\Leftarrow$) Suppose $I_1^\ast$ and $I_2^\ast$ are conjugate. Then there exists $h\in Gal(K:F)$ such that $hI_1^\ast h^{-1}=I_2^\ast$. By 3, $I_2^\ast= hI_1^\ast h^{-1}= h(I_1)^\ast$. But then $I_2=h(I_1)$. which means that $I_1$ and $I_2$ are conjugate fields, as required.
    \item ($\Rightarrow$) We will use the following Lemma: a field extension is normal if and only if it equals all its conjugates. We will only prove the converse, as this is the only result we need for the exercise. For the proof, suppose a field $F$ equals all its conjugates, and let $K$ be a root field containing $F$. Furthermore, assume $F\neq K$. Take an irreducible polynomial so that some but not all its roots are in $F$. However, by Theorem 7 of Chapter 31, $K$ must contain all the roots of $p(x)$, and there must be an automorphism that permutes the roots of said polynomial. But because $F$ is its only conjugate then, by definition, all automorphisms send $F$ to itself. Hence, $F$ must contain all the roots of that polynomial, as all the automorphisms that permute the roots always send $F$ to itself. 

    For the proof of the exercise, let $I_2^\ast$ be a normal subgroup of $I_1^\ast$. By the definition of normal subgroup, $I_2^\ast$ is conjugate to itself. By exercise 4, $I_2$ is conjugate to itself, and by the lemma above, $I_2$ is a normal extension. Finally, because $I_2^\ast\subseteq I_1^\ast$, then $I_2\subseteq I_2$ which implies $I_2$ is a normal extension of $I_1$.
    
    ($\Leftarrow$) This is the content of exercise 1.
 \end{enumerate}
\end{proof}