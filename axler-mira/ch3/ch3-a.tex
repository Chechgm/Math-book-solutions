\section{Integration with respect to a measure}
\addcontentsline{toc}{section}{fill}

Axler 3A Integration with Respect to a Measure
2 In physics, there is a thing called the Dirac delta function which is zero everywhere except at x=0, and which has a total integral of 1. This function is not a usual function in math. This exercise gives one way in which math is able to rigorize the Dirac delta function.
5 This is very cool since it means summation is a particular integral. So all theorem we prove about integrals (like swapping limit and integral, swapping derivative and integral, etc.) are all valid for summations too
7
9 in probability, f would be the pdf of the probability measure nu. 
10 This allows us to swap infinite sum and integral of positive functions. in particular, we can swap any two infinite sums of positive numbers!
13
14 two examples showing the limitations of generalizing the monotone convergence theorem 
15 Two results which allow for substitutions in the Lebesgue integral. It is interesting how the Lebesgue integral is a very strong integral, but doesn't really allow substitution theorems easily. The Riemann integral is much more superior when it comes to those. 
17 Fatou's lemma is another landmark result in measure theory
19 This should intuitively be true by comparing the area under a curve with two rectangles below and above the curve. 
20 A fun generalization of the monotone convergence theorem
21 This is the main intuition behind the Lebesgue integral, but it takes a while to see why the Lebesgue integral works like this. It is also not at all clear why it should lead to a stronger integral to begin with.

2.D.24

\begin{exercise}{2}
Suppose $X$ is a set, $\cS$ is a $\sigma$-algebra, and $c \in X$.
Define the Dirac measure $\delta_c$ on $(X,\cS)$ by $\delta_c(E) = 1$ if $c \in E$ and $\delta_c(E) = 0$ otherwise.
Prove that if $f: X \to [0, \infty]$ is $\cS$-measurable then $\int f d\delta_c = f(c)$.
[Careful: $\set{c}$ may not be in $\cS$].
\end{exercise}
\begin{proof}
Let $\cP$ be any partition of $\cS$.
We have that
\begin{align}
    \cL(f, \cP) = \sum_{A_j \in \cP} \delta_c(A_j) \inf_{A_j}f = \inf_{A_i} f \leq f(c),
\end{align}
for the unique $A_i$ such that $c \in A_i$.
Thus, $\int f = \sup \cL(f,\cP) \leq f(c)$.
\end{proof} 

\begin{exercise}{5}
fill
\end{exercise}
\begin{proof}
fill
\end{proof} 

\begin{exercise}{7}
fill
\end{exercise}
\begin{proof}
fill
\end{proof} 

\begin{exercise}{9}
fill
\end{exercise}
\begin{proof}
fill
\end{proof} 

\begin{exercise}{10}
fill
\end{exercise}
\begin{proof}
fill
\end{proof} 

\begin{exercise}{13}
fill
\end{exercise}
\begin{proof}
fill
\end{proof} 

\begin{exercise}{14}
fill
\end{exercise}
\begin{proof}
fill
\end{proof} 

\begin{exercise}{15}
fill
\end{exercise}
\begin{proof}
fill
\end{proof} 

\begin{exercise}{17}
fill
\end{exercise}
\begin{proof}
fill
\end{proof} 

\begin{exercise}{19}
fill
\end{exercise}
\begin{proof}
fill
\end{proof} 

\begin{exercise}{20}
fill
\end{exercise}
\begin{proof}
fill
\end{proof} 

\begin{exercise}{21}
fill
\end{exercise}
\begin{proof}
fill
\end{proof} 
