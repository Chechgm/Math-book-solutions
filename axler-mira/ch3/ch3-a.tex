\section{3A Integration with respect to a measure}
\addcontentsline{toc}{section}{3A Integration with respect to a measure}


\begin{exercise}{2}
Suppose $X$ is a set, $\cS$ is a $\sigma$-algebra, and $c \in X$.
Define the Dirac measure $\delta_c$ on $(X,\cS)$ by $\delta_c(E) = 1$ if $c \in E$ and $\delta_c(E) = 0$ otherwise.
Prove that if $f: X \to [0, \infty]$ is $\cS$-measurable then $\int f d\delta_c = f(c)$.
[Careful: $\set{c}$ may not be in $\cS$].
\end{exercise}
\begin{proof}
To compute this integral we will follow the so-called bootstrapping method.
We will compute the integral for 
i) the characteristic function,
ii) for linear combinations of characteristic functions, that is, simple functions,
iii) use the monotone convergence Theorem to prove it for arbitrary positive functions,
iv) prove it for general functions by using $f = f^+ - f^-$.

We will then begin by computing the integral for the characteristic function,
that is, we want to prove that $\int \chi_E d\delta_c = \chi_E(c)$.
To see this is the case, notice that 3.4 gives us the first equality in
\begin{align*}
    \int \chi_E d\delta_c 
    = \delta_c(E)
    = \begin{cases}
        1 \text{ if } c \in E\\
        0 \text{ otherwise}
    \end{cases}
    = \chi_E(c),
\end{align*}
as required.

Now let $f$ be a simple function, so that $f = \sum_{k=1}^n a_k \chi_{E_k}$, where the $E_k$ are disjoint.
We want to prove $\int (\sum_{k=1}^n a_k \chi_{E_k}) d\delta_c = \sum_{k=1}^n a_k \chi_{E_k} (c)$.
To see this is the case, notice that 3.7 tells us that
\begin{align*}
    \int (\sum_{k=1}^n a_k \chi_{E_k}) d\delta_c = \sum_{k=1}^n a_k \delta_c(E).
\end{align*}
Notice that $c$ can only belong to one of $E_k$, say $E_j$, so that 
\begin{align*}
    \sum_{k=1}^n a_k \delta_c(E)
    = a_j \delta_c(E_j)
    = a_j
    = \sum_{k=1}^n a_k \chi_{E_k}(c) 
    = f(c).
\end{align*}

For the third step, let $f$ be an arbitrary positive function.
By 2.89, we can approximate $f$ with a sequence of simple functions, such that $0 \leq f_1 \leq f_2 \leq \dots$ and $f_n \to f$.
Then by the monotone convergence Theorem, we have $\int f d\delta_c = \lim_{n \to \infty} \int f_n d\delta_c$.
We have already seen that $\int f_n d\delta_c = f_n(c)$ for a simple function, so that 
\begin{align*}
    \int f d\delta_c 
    = \lim_{n \to \infty} \int f_n d\delta_c 
    = \lim_{n \to \infty} f_n(c) 
    = f(c)
\end{align*}

To complete the proof, let $f$ be arbitrary.
By the definition of the integral of a real valued function, we have that $\int f d\delta_c = \int f^+ d\delta_c - \int f^- d\delta_c = \int f^+ d\delta_c$, if $f(c) \geq 0$ and $\int f d\delta_c = \int f^- d\delta_c$ if $f(c) \leq 0$.
In both cases, we get that $\int f d\delta_c = f(c)$, as required.
\end{proof} 

\begin{exercise}{5}
Prove that integration with respect to counting measure is summation.
Suppose $\mu$ is counting measure on $\N$ and $b_1, b_2, \dots$ is a sequence of nonnegative numbers.
Think of $b$ as the function from $\N$ to $[0, \infty)$ defined by $b(k) = b_k$.
Then $\int b d\mu = \sum_{k=1}^\infty b_k$.
\end{exercise}
\begin{proof}
We will proceed using the bootstrap procedure as in exercise 2.
We begin proving the statement for $b= \chi_E$;
that is, we prove that $\int \chi_E d\mu = \sum b_k$, and $\chi_E(k) = b_k$ if $k \in E$, as usual.
For a finite $E$, we have $\int \chi_E d\mu = \mu(E) = n = \sum_{k \in E} \chi_E(k)$.
The statement holds for infinite $E$ mutatis mutandis.

We will now prove the statement for a simple function $b = \sum_{m = 1}^M b_m \chi_{E_m}$, with $E_m$ disjoint, so that $b(k) = b_m$ if $k \in E_m$ for some $m$ and 0 otherwise.
Let $n_m$ be the cardinality of $E_m$, we have $\int b d\mu = \sum_{m=1}^M b_m \mu(E_m) = \sum_{m=1}^M b_m n_m$, where the first equality follows by 3.7.
Likewise, $\sum_{k \in \N}^\infty b(k) = \sum_{k \in \N}^\infty (\sum_{m = 1}^M b_m \chi_{E_m}(k)) = \sum_{m=1}^M b_m n_m$, so that $\int b d\mu = \sum_{k \in \N}^\infty b(k) = \sum_{k \in \N}^\infty b_k$.

Let $b$ be an arbitrary positive function, that is, $b(k) = b_k \geq 0$.
We can approximate $b$ by the sequence $b^n = \sum_{k = 1}^n b_k\chi_{\set{k}}$, we have that $b^n \to b$. 
Thus, we can apply the monotone convergence Theorem to conclude that $\int b d\mu = \lim_{n \to \infty} \int b^n d\mu$.
We have seen that $\int b^n d\mu = \sum_{k=1}^n b_k$, so that taking limit as $n \to \infty$ gives us the desired result.

Since $b$ is defined to be positive from the start, we don't need to do the last step of the bootstrap for an arbitrary function.
\end{proof} 

\begin{exercise}{7}
Suppose $X$ is a set, $\cS$ is the $\sigma$-algebra of all subsets of $X$, and $w: X \to [0, \infty]$ is a function.
Define a measure $\mu$ on $(X, \cS)$ by $\mu(E) = \sum_{x \in E} w(x)$ for $E \subseteq X$.
Prove that if $f: X \to [0,\infty]$ is a function, then $\int f d\mu = \sum_{x \in X} w(x)f(x)$, where the infinite sums above are defined as the supremum of all sums over finite subsets of $E$ (first sum) or $X$ (second sum).
\end{exercise}
\begin{proof}
We will prove this through a double inequality, that is, we prove that 
\begin{align*}
    \int f d\mu \geq \sum_{x \in X} w(x)f(x) = \sup_{D \subseteq X; D \text{ finite}} \sum_{x \in D} w(x) f(x),
\end{align*}
and
\begin{align*}
    \int f d\mu \leq \sup_{D \subseteq X; D \text{ finite}} \sum_{x \in D} w(x) f(x).
\end{align*}

To see the first inequality holds, let $D \in X$ be finite.
Consider the function $g(x) = f(x)\chi_{D}(x)$, so that $g$ is a simple function as we can write $f(x)\chi_{D}(x) = \sum_{x \in D} f(x) \chi_{\set{x}} = \sum_{x \in D} a_x \chi_{\set{x}}$.
We know that the integral of $g$ is as follows (3.7):
\begin{align*}
    \int g d\mu
    = \sum_{x \in D} f(x)\mu(\set{x})
    = \sum_{x \in D} f(x)w(x).
\end{align*}
Now, notice that $g \leq f$ so that $\int g d\mu \leq \int f d\mu$.
Since this is true for any finite set $D$, we have $\int f d\mu \geq \sup_{D \subseteq X; D \text{ finite}} \sum_{x \in D} f(x)w(x)$, giving us the first result.

To see the second inequality holds, notice that $\int f d\mu$ is the supremum of the integrals over all simple functions, say $g =\sum_{k = 1
}^n a_k \chi_{E_k}$ with $E_k$ disjoint, so that $g \leq f$.
Using 3.7, the integral of a simple function, we have that 
\begin{align*}
    \int g d\mu 
    = \sum_{k = 1}^n a_k \mu(E_k) 
    = \sum_{k = 1}^n a_k \parens{\sup_{D_k \subseteq E_k; D_k \text{ finite}} \sum_{x \in D_k} w(x)}.
\end{align*}
By the homogeneity of the supremum, for each $k$ we can push the constant inside the supremum, giving us 
\begin{align*}
    \int g d\mu 
    =  \sup_{D_1, \dots, D_k} \sum_{k = 1}^n \sum_{x \in D_k} a_k w(x),
\end{align*}
where, of course, we still have  $D_k \subseteq E_k$  and $D_k$ finite.
Since $h(x) = a_k$ if $x \in E_k$ and $h(x) \leq f(x)$, we have $a_k w(x) \leq f(x) w(x)$, so that 
\begin{align*}
    \int h d\mu 
    \leq \sup_{D \subseteq X; D \text{ finite}} \sum_{x \in D} f(x) w(x).
\end{align*}
Taking the supremum on both sides of this inequality gives us the other required inequality, completing the proof.
\end{proof}

\begin{exercise}{9}
Suppose $\mu$ is a measure on a measurable space $(X, \cS)$ and $f: X \to [0,\infty]$ is and $\cS$-measurable function.
Define $v: \cS \to [0, \infty]$ by $v(A) = \int \chi_A f d\mu$ for $A \in \cS$.
Prove that $v$ is a measure on $(X, \cS)$.
\end{exercise}
\begin{proof}
Since $f > 0$, we have that $\int \chi_A f d\mu \geq 0$ for all $A$, thus, the critical property of measures we want to prove is the countable additivity;
that is, for a set of disjoint sets $E_1, E_2, \dots \in \cS$, we want that 
\begin{align*}
    v(\bigcup E_k) 
    = \int \chi_{\bigcup E_k} f d\mu 
    = \sum \int \chi_{E_k} f d\mu 
    = \sum v(E_k).
\end{align*}

We know that for disjoint sets, $\chi_{\bigcup E_k} = \sum \chi_{E_k}$, so that we can write $\int \chi_{\bigcup E_k} f d\mu = \int \sum \chi_{E_k} f d\mu$.
Additionally, we have that $\lim_{n\to \infty} \sum_{k=1}^n \chi{E_k}f = \sum_{k = 1}^\infty f$, and since $f$ is positive, $\sum_{k=1}^n \chi{E_k}f \leq \sum_{k=1}^{n+1} \chi{E_k}f$, for all $n$.
Thus we can apply the monotone convergence Theorem (in the second equality), obtaining
\begin{align*}
    v(\bigcup E_k)
    =& \int \sum_{k=1}^\infty \chi_{E_k}f d\mu\\
    =& \lim_{n \to \infty} \int \sum_{k = 1}^n \chi_{E_k} f d\mu\\
    =& \lim_{n \to \infty} \sum_{k = 1}^n \int \chi_{E_k}f d\mu\\
    =& \sum_{k = 1}^\infty \int \chi_{E_k}f d\mu
    = \sum v(E_k),
\end{align*}
completing the proof.
\end{proof} 

\begin{exercise}{10}
Suppose $(X, \cS, \mu)$ is a measure space and $f_1, f_2, \dots$ is a sequence of nonnegative $\cS$-measurable functions.
Define $f: X \to [0, \infty]$ by $f(x) = \sum_{k=1}^\infty f_k(x)$.
Prove that $\int f d\mu = \sum_{k=1}^\infty \int f_k d\mu$.
\end{exercise}
\begin{proof}
Notice that the sequence of functions $(g_n)$ defined by $g_n(x) = \sum_{k=1}^n f_k(x)$ has the property that $g_1 \leq g_2 \leq \dots$ and furthermore $g_n \to f$.
We can now use the monotone convergence Theorem to conclude that $\lim_{n \to \infty} \int g_n d\mu = \int f d\mu$.
But then we have that
\begin{align*}
    \int f d\mu
    = \lim_{n \to \infty} \int g_n d\mu
    = \lim_{n \to \infty} \int \sum_{k=0}^n f_k d\mu
    = \lim_{n \to \infty} \sum_{k=0}^n \int f_k d\mu
    = \sum_{k=0}^\infty \int f_k d\mu, 
\end{align*}
where the third equality follows from the linearity of the integral.
\end{proof} 

\begin{exercise}{13}
Give an example to show that the monotone convergence Theorem can fail if the hypothesis that $f_1, f_2, \dots$ are nonnegative functions is dropped.
\end{exercise}
\begin{proof}
Consider the sequence
\begin{align*}
    f_n(x) = 
    \begin{cases}
        1 \text{ if } x \in [0,n]\\
        0 \text{ if } x > n\\
        -1 \text{ if } x < 0.
    \end{cases}
\end{align*}
We have that $\int f_n d\lambda = \int f^+ d\lambda - \int f^- d\lambda = n - \infty = -\infty$.
Furthermore, $\lim f_n = f$, where $f$ is as in example 3.19.
We have that $\lim_{n \to \infty} \int f_n d\lambda = -\infty$, however, we concluded in example 3.19, that $\int f d\lambda$ is not defined.
\end{proof} 

\begin{exercise}{14}
Give an example to show that the monotone convergence Theorem can fail if the hypothesis of an increasing sequence of functions is replaced by a hypothesis of decreasing sequence of functions.
[This exercise shows that the monotone convergence Theorem should be called the increasing convergence Theorem.
However, see exercise 20].
\end{exercise}
\begin{proof}
Consider the sequence of functions $f_n(x) = \chi_{\R \setminus [-n, n]}(x)$.
We have that $f_n \to 0$ and $\int f_n d\lambda = \lambda(\R \setminus [-n,n]) = -\infty - 2n = -\infty$, so that 
\begin{align*}
    \lim \int f_n d\lambda = -\infty \neq \int f d\lambda = 0,
\end{align*}
completing the proof.
\end{proof} 

\begin{exercise}{15}
Suppose $\lambda$ is Lebesgue measure on $\R$ and $f: \R \to [-\infty, \infty]$ is a Borel measurable functions such that $\int f d\lambda$ is defined.
\begin{enumerate}
    \item For $t \in \R$, define $f_t: \R \to [-\infty, \infty]$ by $f_t(x) = f(x-t)$.
    Prove that $\int f_t d\lambda = \int f d\lambda$ for all $t \in \R$.
    \item For $t \in \R$, define $f_t: \R \to [-\infty, \infty]$ by $f_t(x) = f(tx)$.
    Prove that $\int f_t d\lambda = (1/\absoluteValue{t}) \int f d\lambda$ for all $t \in \R \setminus \set{0}$.
\end{enumerate}
\end{exercise}
\begin{proof}
We will proceed as in exercise 2 and 5.
Thus, we begin by proving the statement holds when $f = \chi_E$;
that is, we prove that $\int \chi_{E,t} d\lambda = \int \chi_E d\lambda$, where $\chi_{E,t}(x) = \chi_E(x-t)$.
Notice that $\chi_{E,t}(x) = \chi_E(x-t) = 1$ when $x \in E+t$.
Furthermore, we have that $\int \chi_E d\lambda = \lambda(E) = \lambda(E+t)$, where the last equality follows from 2.7.
Thus, $\int \chi_E d\lambda = \int \chi_{E,t} d\lambda$.

For the second step, suppose $f = \sum_{k = 1}^n a_k \chi_{E_k}$, for disjoint $E_k$.
We have $\int f d\lambda = \sum_{k = 1}^n a_k \lambda(E_k) = \sum_{k = 1}^n a_k\lambda(E_k + t)$, so that using a similar strategy as above for each $E_k$ we can conclude that $\int f d\lambda = \int f_t d\lambda$.

For the third step, let $f$ be positive, and approximate $f$ with an increasing sequence of simple functions $(f^n)$ (we use an upper script instead of a lower one given that $f_t$ is already in use in the problem statement).
Using the monotone convergence Theorem, we have $\lim_{n \to \infty} \int f^n d\lambda = \int f d\lambda$, so that 
\begin{align*}
    \lim_{n \to \infty} \int \sum_{k = 1}^n a_k\chi_{E_k} d\lambda
    = \lim_{n \to \infty} \sum_{k = 1}^n a_k \lambda(E_k) 
    = \lim_{n \to \infty} \sum_{k = 1}^n a_k \lambda(E_k + t).
\end{align*}
Using a similar argument as in the previous two cases, and the limit on the series, we get that $\int f d\lambda = \int f_t d\lambda$.

The result follows for an arbitrary function by using $\int f d\lambda = \int f^+ d\lambda - \int f^- d\lambda$.

The second result follows using a similar strategy, where instead we use the result in exercise 2A.2, where we concluded that the outer measure and thus Lebesgue measure) has the following property: for $t \in \R$, $\lambda(tA) = \absoluteValue{t} \lambda(A)$.
\end{proof} 

\begin{exercise}{17}
For $x_1, x_2, \dots$ a sequence in $[-\infty, \infty]$, define $\liminf_{k \to \infty} x_k$ by $\liminf_{k \to \infty} x_k = \lim_{k \to \infty} \inf\set{x_k, x_{k+1}, \dots}$.
Note that $\inf \set{x_k, x_{k+1}, \dots}$ is an increasing function of $k$;
thus the limit above on the right exists in $[-\infty, \infty]$.

Suppose that $(X, \cS, \mu)$ is a measure space and $f_1, f_2, \dots$ is a sequence of non-negative $\cS$-measurable functions on $X$.
Define a function $f: X \to [0, \infty]$ by $f(x) = \liminf_{k \to \infty} f_k(x)$.
\begin{enumerate}
    \item Show that $f$ is an $\cS$-measurable function.
    \item Prove that $\int f  d\mu \leq \liminf_{k \to \infty} \int f_k d\mu$.
    \item Give an example showing that this inequality can be a strict inequality even when $\mu(X) < \infty$ and the family of functions $\set{f_k}_{k \in \N}$ is uniformly bounded.
\end{enumerate}
[The second result is called Fatou's Lemma.
Some textbooks prove Fatou's Lemma and then use it to prove the monotone convergence Theorem.
Here we are taking the reverse approach, you should be able to use the monotone convergence Theorem to give a clean proof of Fatou's Lemma].
\end{exercise}
\begin{proof}
\begin{enumerate}
    \item In 2.53 we proved that the function $g(x) = \inf\set{f_n(x): n \in \N}$ is a $\cS$-measurable function.
    This property is not changed if instead of taking $n \in \N$ inside the set we take $n > k$, so that the function $g_k(x) = \inf\set{f_n(x): n > k}$ is a $\cS$-measurable function.
    Now we have that $f(x) = \liminf_{k \to \infty} f_k(x) = \lim_{k \to \infty} \inf \set{f_n(x): n > k} = \lim_{k\to \infty} g_k(x)$, which we know is a $\cS$-measurable function by 2.48.
    \item As stated in the exercise, $g_k(x) = \inf{f_n(x): n > k}$ is an increasing function of $k$, so that $g_k$ are nonnegative and increasing, giving us the possibility to use the monotone convergence Theorem;
    that is, $\int f d\mu = \lim \int g_k d\mu$.
    
    Notice that for each $k$, it holds that $g_k = \inf \set{f_n(x): n > k} \leq f_k(x)$ which implies that $\int g_k d\mu \leq \int f d\mu$, by the order Theorem of the Lebesgue integrals (3.22).
    We can now take liminf in both sides of the inequality to obtain $\liminf_{k \to \infty} \int g_k d\mu \leq \liminf_{k \to \infty} \int f_k d\mu$.
    Finally, we know that $\liminf_{k \to \infty} \int g_k d\mu = \lim_{k \to \infty}\int g_k d\mu$, since when a sequence of real numbers converges, the the limit and the liminf coincide.
    Putting all of these together we have $\int f d\mu = \lim \int g_k d\mu \leq \liminf \int f_k d\mu$, completing the proof.
    \item Consider the sequence of functions $f_n: [-1,1] \to \R$ given by
    \begin{align*}
        &f_n
        = \begin{cases}
            1 \text{ if } x \in [-1, 0),\\
            0 \text{ if } x \in [0, 1];
        \end{cases}
        &&f_{2n}
        = \begin{cases}
            0 \text{ if } x \in [-1, 0),\\
            1 \text{ if } x \in [0, 1],
        \end{cases}
    \end{align*}
    for $n \in \N$.
    We have that $f_n \to 0$ (0 as a function), so that $\int f d\lambda = 0$.
    However, $\liminf \int f_k d\lambda = 1$, since $\int f_k d\lambda = 1$ for all $k$, so that the inequality is strict.
\end{enumerate}
\end{proof} 

\begin{exercise}{19}
Show that if $(X, \cS, \mu)$ is a measure space and $f: X \to [0,\infty)$ is $\cS$-measurable, then $\mu(X) \inf_X f \leq \int f d\mu \leq \mu(X) \sup_X f$.
\end{exercise}
\begin{proof}
Consider the partition $P$, give by $X, \emptyset$.
We then have $\cL (f, P) = \mu(\emptyset) \inf_\emptyset f + \mu(X) \inf_X f$, and since $\int f d\mu = \sup_P \cL(f,P)$, we get that $\mu(X)\inf_X f \leq \int f d\mu$.

Moreover, for any partition $P$, we have 
\begin{align*}
    \cL(f, P) 
    = \sum_{j=0}^m \mu(A_j) \inf_{A_j} f 
    \leq \sum_{j=0}^m \mu(A_j) \sup_{A_j} f 
    \leq \sup_X f \sum_{j=0}^m \mu(A_j)
    = \sup_X f \mu(X),
\end{align*}
so that $\sup_P \cL(f, P) = \int f d\mu \leq \mu(X) \sup_X f$, completing the proof.
\end{proof} 

\begin{exercise}{20}
Suppose $(X, \cS, \mu)$ is a measure space and $f_1, f_2, \dots$ is a monotone (meaning either increasing or decreasing) sequence of $\cS$-measurable functions.
Define $f: X \to [-\infty, \infty]$ by $f(x) = \lim_{k \to \infty} f_k(x)$.
Prove that if $\int \absoluteValue{f_1} d\mu < \infty$, then $\lim_{k \to \infty} \int f_k d\mu = \int f d\mu$.
\end{exercise}
\begin{proof}
We will prove this by cases.

\paragraph{Case 1} If $(f_n)$ is increasing and positive, the result follows by the monotone convergence Theorem.

\paragraph{Case 2} Suppose $(f_n)$ is increasing, but $f_1$ is negative in at least one $x$.
Consider then the sequence $g_n = f_n + f_1^-$, so that $0 \leq g_1 \leq g_2 \leq \dots$.
We have that $\lim_{n \to \infty} g_n = f + f_1^-$.
We can now apply the monotone convergence Theorem to conclude that 
\begin{align*}
    \lim \int f_n + f_1^- d\mu
    = \lim \int g_n d\mu 
    = \int \lim g_n d\mu 
    = \int f + f_1^- d\mu.
\end{align*}
Since $\int \absoluteValue{f} d\mu = \int f^+ + f^- d\mu < \infty$, then $\int f^- d\mu < \infty$, so that from the equality above and the additivity of integration (3.21), we get 
\begin{align*}
    \lim \int f_n d\mu + \lim \int f_1^- d\mu
    = \int f d\mu + \int f_1^- d\mu,
\end{align*}
so that $\lim \int f_n d\mu = \int f d\mu$, as desired.

\paragraph{Case 3} Suppose $(f_n)$ is negative and decreasing $0 \geq f_1 \geq f_2 \geq \dots$.
Then the sequence $g_n = -f_n$ is increasing and positive so that the result follows from the monotone convergence Theorem and the homogeneity of integration (3.20).

\paragraph{Case 4} Suppose $(f_n)$ is decreasing but for some $x$, $f_1$ is positive.
This case follows the same strategy of case 2 mutatis mutandis.
Specifically, we use $-f_1^+$ instead of $f_1^-$.

The 4 cases exahust all the possibilities, completing the proof.
\end{proof} 

\begin{exercise}{21}
Henri Lebesgue wrote the following about his method of integration:
\begin{quotation}
I have to pay a certain sum, which I have collected in my pocket.
I take the bills and coins out of my pocket and give them to the creditor in the order I find them  until I have reached the total sum.
This is the Riemann integral.
But I can proceed differently.
After I have taken all the money out of my pocket I order the bills and coins according to identical values and then I pay the several heaps one after the other to the creditor.
This is my integral.
\end{quotation}
Use 3.15 to explain what Lebesgue meant and to explain why integration of a function with respect to a measure can be thought of as partitioning the range of the function, in contrast to Riemann integration, which depends on partitioning the domain of the function.
[The quote above is taken from page 796 of the Princeton companion to mathematics, edited by Timothy Gowers].
\end{exercise}
\begin{proof}
The way I interpret this quote is as follows:
Every bill or coin in Lebesgue's pocket is an `$x$' in the domain of our function and the denomination of the bill is the function evaluation $f(x)$.
The Riemann integral consists of adding up the function evaluations of each of our elements in the domain of our function (we partition it), we can think of this as a Riemann sum.
On the other hand, the Lebesgue integral takes all the elements in our domain that have a specific function evaluation and then the integral is simply the product of the `quantity' (measure) of that set times the particular function evaluation, so that we are partitioning the range.
\end{proof} 
