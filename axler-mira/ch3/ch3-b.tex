\section{3B Limits of integrals and integrals of limits}
\addcontentsline{toc}{section}{3B Limits of integrals and integrals of limits}

6 In fact, in this example it can be shown that the extended Riemann integral even exists, but still the Lebesgue integral does not. This shows that the Lebesgue integral is very much not perfect. The Henstock integral is the perfect integral on R. But it only really exists on $R^n$. The benefit of Lebesgue is that it works on very general measure spaces, like in probability, not that it can integrate many functions on R.
8 Verifying theory 
9 Verifying theory 
10 How does this relate to $L^p$ spaces? What inclusion is shown here?

\begin{exercise}{1}
Give an example of a sequence $f_1, f_2, \dots$ of functions from $\N$ to $[0, \infty)$ such that $\lim_{k \to \infty} f_k(m) = 0$ for every $m \in \N$ but $\lim_{k \to \infty} \int f_k d\mu = 1$ where $\mu$ is counting measure on $\N$.
\end{exercise}
\begin{proof}
Consider the sequence given by $f_k(m) = 1$ if $m = k$ and 0 otherwise.
We see that for any $m$, $\lim_{k \to \infty} f_k(m) = 0$ but also $\int f_k d\mu = 1$ under the counting measure, so that $\lim f_k d\mu = 1$, completing the proof.
Notice that this even though $g(m) = 1$ dominates every $f_k$, the dominated convergence Theorem does not apply, as $g$ has no finite integral.
\end{proof} 

\begin{exercise}{2}
Give an example of a sequence $f_1, f_2, \dots$ of continuous functions from $\R$ to $[0,1]$ such that $\lim_{k \to \infty} f_k(x) = 0$ for every $x \in \R$ but $\lim_{k \to \infty} f_k d\lambda = \infty$, where $\lambda$ is Lebesgue measure on $\R$
\end{exercise}
\begin{proof}
Let 
\begin{align*}
    f_k(x) = 
    \begin{cases}
        0   &\text{ if }    x < k\\
        x-k &\text{ if }  x \in [k, k+1]\\
        1   &\text{ if }    x > k+1.
    \end{cases}
\end{align*}
Each function in this sequence is continuous.
Moreover, we have that for all $x$, $\lim_{k \to \infty} f_k(x) = 0$, but also for all $k$, $\int f_k d\lambda = \infty$, so that $\lim_{k \to \infty} f_k d\lambda = \infty$, completing the proof.
\end{proof} 

\begin{exercise}{3}
Suppose $\lambda$ is Lebesgue measure on $\R$ and $f: \R \to \R$ is a Borel measurable function such that $\int \absoluteValue{f} d\lambda < \infty$.
Define $g: \R \to \R$ by $g(x) = \int_(-\infty, x) f d\lambda$.
Prove that $g$ is uniformly continuous on $\R$.
\end{exercise}
\begin{proof}
We have 
\begin{align*}
    \absoluteValue{g(x) - g(y)}
    =& \absoluteValue{\int_{(-\infty, x)}f d\lambda - \int_{(-\infty, y)}f d\lambda}\\
    =& \absoluteValue{\int f\chi_{(-\infty, x)} d\lambda - \int f\chi_{(-\infty, y)} d\lambda}\\
    =& \absoluteValue{\int f\chi_{(-\infty, x)} - f\chi_{(-\infty, y)} d\lambda}\\
    =& \absoluteValue{\int f\chi_{[x, y)} d\lambda}\\
    =& \absoluteValue{\int_{[x, y)} f d\lambda}\\
    \leq& \int_{[x, y)}\absoluteValue{f} d\lambda,
\end{align*}
where the last inequality follows from 2.45, and we have assumed, without loss of generality, that $x < y$.

Let $\epsilon > 0$.
Since we assume $\int \absoluteValue{f} d\lambda < \infty$, we can use 3.29 to conclude there is a $\delta > 0$ such that $\int_B \absoluteValue{f} d\lambda < \epsilon$ for all Borel $B$ with $\lambda(B) < \delta$.
translating this to our result above, we have that for any $x,y \in \R$ with $\absoluteValue{x - y} < \delta$, we get that $\absoluteValue{g(x) - g(y)} \leq \int_{[x, y)} \absoluteValue{f} d\lambda < \epsilon$, giving us the desired uniform continuity.
\end{proof} 

\begin{exercise}{6}
Let $\lambda$ denote Lebesgue measure on $\R$.
Give an example of a continuous function $f: [0, \infty) \to \R$ such that $\lim_{t \to \infty} \int_{[0,t]} f d\lambda$ exists (in $\R$) but $\int_{[0, \infty)} f d\lambda$ is not defined.
\end{exercise}
\begin{proof}
$\sin(x)/x$


\end{proof} 

\begin{exercise}{8}
Let $f: [a,b] \to \R$ be a bounded function and $P_n$ be a partition that divides $[a,b]$ into $2^n$ subintervals of equal size with $I_1, I_2, \dots, I_{2^n}$ the corresponding closed subintervals.
Define $g_n = \sum_{j=1}^{2^n} (\inf_{I_j}) \chi_{I_j}$, $h_n = \sum_{j=1}^{2^n} (\sup_{I_j}) \chi_{I_j}$, and the limits of these functions as $f^L(x) = \lim g_n(x)$, and $f^U(x) = \lim h_n(x)$.
Prove that $\set{x \in [a,b]: f^U(x) \neq f^L(x)} = \set{x \in [a,b]: f \text{ is not continuous at } x}$.
\end{exercise}
\begin{proof}
fill
\end{proof} 

\begin{exercise}{9}
Suppose $(X, \cS, \mu)$ is a measure space and $E_1, E_2, \dots, E_n$ are disjoint subsets of $X$.
Suppose $a_1, \dots, a_n$ are distinct nonzero real numbers.
Prove that $a_1 \chi_{E_1} + \dots a_n \chi_{E_n} \in \cL^1(\mu)$ if and only if $E_k \in \cS$ and $\mu(E_k) < \infty$ for all $k \in \set{1, \dots, n}$.
Furthermore, $\norm{a_1 \chi_{E_1} + \dots + a_n \chi_{E_n}}_1 = \absoluteValue{a_1} \mu(E_1) + \dots + \absoluteValue{a_n} \mu(E_n).$
\end{exercise}
\begin{proof}
($\Rightarrow$)
Notice that if $E_k \in \cS$ then $\chi_{E_k}$ is measurable (2.38), so that given that the sum of measurable functions is measurable (2.46), then $\sum_{k=1}^n a_n \chi_{E_k}$ is itself measurable.
Furthermore, since 
\begin{align*}
    \norm{\sum a_k \chi_{E_k}}_1
    =& \int \absoluteValue{\sum a_k \chi_{E_k}} d\mu\\
    \leq& \sum \int \absoluteValue{ a_k \chi_{E_k}} d\mu\\
    =& \sum \int \absoluteValue{a_k} \chi_{E_k} d\mu\\
    =& \sum\absoluteValue{a_k} \mu(E_k),
\end{align*}
which is finite only if $\mu(E_k) < \infty$, for all $k$.

($\Leftarrow$)
Prove exercise 2B.13 using the same technique as 2.38, so that this follows as a Corollary.


Suppose $\sum a_k \chi_{E_k} \in \cL^1(\mu)$.
Then $\sum a_k \chi_{E_k}$ is measurable so that each $E_k$ must be in $\cS$.
Suppose this were not the case, so that $E_i$ is not in $\cS$ for some $E_i$.
There are two cases, either $\sum_{k\neq i} a_k \chi_{E_k}$ is measurable, in which case we get a contradiction, because the subtraction of two measurable functions is measurable (2.46).
The second case is when $\sum_{k\neq i} a_k \chi_{E_k}$ is not measurable, in which case we can repeat the process, until we arrive at a contradiction of the same nature, otherwise, there will be $E_i$ and $E_j$ such that $a_i \chi_{E_i} + a_j \chi_{E_j}$ is not measurable, 



This would imply that $\sum_{k\neq i} a_k \chi_{E_k}$
\end{proof} 

\begin{exercise}{10}
\begin{enumerate}
    \item Suppose $(X, \cS, \mu)$ is a measure space such that $\mu(X) < \infty$.
    Suppose $p, r$ are positive numbers with $p < r$.
    Prove that if $f: X \to [0, \infty)$ is an $\cS$-measurable function such that $\int f^r d\mu < \infty$, then $\int f^p d\mu < \infty$.
    \item Give an example to show that the previous result can be false without the hypothesis that $\mu(X) < \infty$.
\end{enumerate}
\end{exercise}
\begin{proof}
\begin{itemize}
    \item Notice that for any partition $P$ of $X$, we have 
    \begin{align*}
        \cL(f^r, P) 
        = \sum_{j=1}^m \mu(A_j) \inf_{A_j}f^r 
        \geq \sum_{j=1}^m \mu(A_j) \inf_{A_j}f^p 
        = \cL(f^p, P),
    \end{align*}
    the inequality holding true since for all $x \in X$ and all positive functions 
\end{itemize}
\end{proof} 
