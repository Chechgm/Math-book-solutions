\section{2E Convergence of measurable functions}
%\addcontentsline{toc}{section}{2E Convergence of measurable functions}


\begin{exercise}{1}
Suppose $X$ is a finite set.
Explain why a sequence of functions from $X$ to $\R$ that converges pointwise on $X$ also converges uniformly on $X$.
\end{exercise}
\begin{proof}
If a sequence of functions $(f_n)$ converges pointwise on $X$, it means that for all $x\in X$ and all $\epsilon>0$, there exists an $N_x \in \N$, such that whenever $n > N_x$, it holds that $\absoluteValue{f_n(x) -f(x)} < \epsilon$.
If we take $N = \max \set{N_x: x \in X}$, then we have that whenever $n>N$, it holds that $\absoluteValue{f_n(x) - f(x)} < \epsilon$, for all $x \in X$.
This is precisely the definition of uniform convergence.
\end{proof} 

\begin{exercise}{2}
Give an example of a sequence of functions from $\N$ to $\R$ that converges pointwise on $\N$ but does not converge uniformly on $\N$.
\end{exercise}
\begin{proof}
Consider the sequence of functions given by $f_n(x) = \chi_{\set{n}}(x)$ for all $n$.
This sequence converges to $f(x) = 0$ for all $n$ pointwise, but there is no $N \in \N$, such that whenever $n>N$ it holds that $\absoluteValue{f_n(x)} < \epsilon$.
\end{proof} 

\begin{exercise}{3}
Give an example of a sequence of functions $f_1, f_2, \dots$ from $[0,1]$ to $\R$ that converges pointwise to a function $f:[0,1] \to \R$ that is not a bounded function.
\end{exercise}
\begin{proof}
Consider the sequence of functions given by $f_n(x) = n^2 x$, whenever $0 \leq x \leq 1/n$ and $f_n(x) = 1/x$ whenever $1/n < x \leq 0$.
Both of the pieces that define the function are continuous in themselves and furthermore $n^2 x= 1/x$ at $x=1/n$ for all $n$, so that the function is continuous there. 
$(f_n)$ converges pointwise to $f(x) = 1/x$ in $(0,1]$ and $f(0) = 0$, but $f$ is not bounded.
\end{proof} 

\begin{exercise}{4}
Prove or give a counterexample:
If $A \subseteq \R$ and $f_1,f_2,\dots$ is a sequence of uniformly continuous functions from $A$ to $\R$ that converges uniformly to a function $f: A \to \R$, then $f$ is uniformly continuous on $A$.
\end{exercise}
\begin{proof}
Let $\epsilon > 0$.
Since $f_n \to f$ uniformly, then there exists an $N \in \N$, such that whenever $n \geq N$, it holds that $\absoluteValue{f_n(x) - f(x)} < \epsilon/3$, for all $x \in A$.
Furthermore, since $f_N$ (notice that $N$ is fixed here) is uniformly continuous, then there exists a $\delta > 0$ such that whenever $\absoluteValue{x - y} < \delta$, it holds that $\absoluteValue{f_N(x) - f_N(y)} < \epsilon/3$.

Putting these together, we have
\begin{align*}
    \absoluteValue{f(x) - f(y)}
    =& \absoluteValue{f(x) - f_n(x) + f_N(x) - f_N(y) + f_N(y) -f(y)}\\
    \leq& \absoluteValue{f(x) - f_N(x)} 
    + \absoluteValue{f_N(x) - f_N(y)}
    + \absoluteValue{f_N(y) -f(y)} < \epsilon,
\end{align*}
for all $x,y \in A$.
\end{proof} 

\begin{exercise}{5}
Give an example to show that Egorov's Theorem can fail without the hypothesis that $\mu(X) < \infty$.
\end{exercise}
\begin{proof}
We can modify the function in exercise 2 as follows.
Let $f_n : \R_{\geq 0} \to \R$ be defined as $f_n(x) = \chi_{[n-1,n)}(x)$.
We have that this sequence of functions converges pointwise to $f(x) = 0$ but there is no subset, $E$ with $\mu(E) = \infty$ such that $f_n$ converges uniformly to the zero function.
\end{proof} 

\begin{exercise}{7}
Suppose $F$ is a closed bounded subset of $\R$ and $g_1, g_2, \dots$ is an increasing of continuous real valued functions on $F$ (thus $g_1(x) \leq g_2(x) \leq \dots$ for all $x \in F$) such that $\sup\set{g_1(x), g_2(x),\dots} < \infty$ for each $x \in F$.
Define a real valued function $g$ on $F$ by $g(x) = \lim g_k(x)$.
Prove that $g$ is continuous on $F$ if and only if $g_1,g_2,\dots$ converges uniformly on $F$ to $g$.
[The result above is called Dini's Theorem].
\end{exercise}
\begin{proof}
($\Rightarrow$)
Suppose $g$ is continuous and let $f_n(x) - g(x) - g_n(x)$.
Because $g_n \to g$, then $f_n(x) \to 0$.
Additionally $f_n$ is decreasing, because $g_n$ is increasing (for all $x \in F)$.
Let $\epsilon > 0$ and define $A_n = \set{x \in F: f_n(x) < \epsilon}$.
The sequence $(A_n)$ is an open cover of $F$, this is true because as we argued, $f_n$ converges pointwise to 0, so that for all $\epsilon > 0$, there exists an $N\in\N$, such that whenever $n>N$, it holds that $\absoluteValue{f_n(x)} < \epsilon$.
We have that $F$ is a closed and bounded set, so that by the Heine-Borel Theorem, $(A_n)$ must contain a finite subcover of $F$.
Thus, if we define $N$ to be the minimum of the indices of the finite subcover, we have that for $n>N$, it holds for all $x\in F$ that $\absoluteValue{f_n(x)} < \epsilon$.
Adding $g$ back to $f_n$ does not change this property and thus $g_n \to g$ uniformly.

($\Leftarrow$)
Let $\epsilon > 0$ and suppose $g_n \to g$ uniformly.
Then there exists an $N \in \N$, such that whenever $n \geq N$, it holds that $\absoluteValue{g_n(x) - g(x)} < \epsilon/3$ for all $x \in F$.
Likewise, since each $g_n$ is continuous, we have that there exists a $\delta > 0$ such that whenever $\absoluteValue{x - y} < \delta$, we have that $\absoluteValue{g_N(x) - g_N(y)} < \epsilon/3$ for all $x,y \in F$.
We have
\begin{align*}
    \absoluteValue{g(x) -g(y)}
    =& \absoluteValue{g(x) - g_N(x) + g_N(x) - g_N(y) + g_N(y) - g(y)}\\
    =& \absoluteValue{g(x) - g_N(x)} 
    + \absoluteValue{g_N(x) - g_N(y)} 
    + \absoluteValue{g_N(y) - g(y)}\\
    <& 3\epsilon/3 = \epsilon,
\end{align*}
whenever $\absoluteValue{x - y} < \delta$.
Thus $g$ is continuous.
\end{proof} 

\begin{exercise}{9}
Suppose $F_1, \dots, F_n$ are disjoint closed subsets of $\R$.
Prove that if $g: F_1 \cup \dots \cup F_n \to \R$ is a function such that $g|_{F_k}$ is a continuous function for each $k \in \set{1,\dots, n}$, then $g$ is a continuous function.
\end{exercise}
\begin{proof}
Since each $g|_{F_k}$ is continuous, then for all $\epsilon > 0$, there exists a $\delta_k > 0$, such that whenever $\absoluteValue{x-y} < \delta_k$, it holds that $\absoluteValue{g|_{F_k}(x) - g|_{F_k}(y)} < \epsilon$.
Take $\delta = \min\set{\delta_1,\dots, \delta_k}$ and let $\absoluteValue{x - y} < \delta$, we then have that $\absoluteValue{g(x) - g(y)} < \epsilon$, as required.
\end{proof} 

\begin{exercise}{13}
Prove or give a counterexample:
If $f_t: \R \to \R$ is a Borel measurable function for each $t \in \R$ and $f: \R \to (-\infty, \infty]$ is defined by $f(x) = \sup\set{f_t(x): t\in\R}$, then $f$ is a Borel measurable function.
\end{exercise}
\begin{proof}
Let $V$ be the Vitali set from 2.18. 
Let $t \in V$ and consider the Borel measurable function $f_t(x) = 1$ if $x=t$ and 0 otherwise.
We have that $f(x) = \sup\set{f_t(x): t\in\R} = \chi_V(x)$, which is not measurable because $V$ is not measurable (see 2.38), giving us the desired counterexample.
\end{proof} 

\begin{exercise}{15}
Suppose $B$ is a Borel set and $f: B \to \R$ is a Lebesgue measurable function.
Show that there exists a Borel measurable function $g: B \to \R$ such that $\absoluteValue{\set{x \in B: g(x) \neq f(x)}} = 0$.
\end{exercise}
\begin{proof}
To prove this we will follow closely the approach in 2.95.
Let $(\R, \cB)$ be the measurable space given by the reals and the borel sets.
Now define $(B, \cS)$, where $\cS = \set{F \cap B, F \in \cB}$, we know $\cS$ is a $\sigma$-algebra because of exercise 2.B.11.

Using 2.89 we can approximate $f$ by $f_1, f_2, \dots$, where $f_k$ is simple and Lebesgue measurable (because $f$ is itself Lebesgue measurable).
For a fixed $k \in \N$, $f_k = \chi_{A_1}c_1 + \dots + \chi_{A_n}c_n$, where each $A_i$ is Lebesgue measurable.

From 2.71, we know there exists $B_i \subseteq A_i$ where $B_i$ is Borel measurable and $\absoluteValue{A_i \setminus B_i} = 0$, for all $i$.
Define $g_k = \chi_{B_1}c_1 + \dots + \chi_{B_n}c_n$.
Then $g_k$ is Borel measurable and $\absoluteValue{\set{x \in B: g_k(x) \neq f_k(x)}} = 0$.

If $x \notin \bigcup \set{x \in B: g_k(x) \neq f_k(x)}$, then $g_k(x) = f_k(x)$ for all $k \in \N$ and hence $\lim g_k(x) = f(x)$.
Let $E = \set{x \in B: \lim g_k(x) \text{ exists in }B}$.
Then by exercise 2.B.14, $E$ is a Borel subset of $B$.
Furthermore, since $\lim g_k(x)$ exists, then $f_k(x) = g_k(x)$ for all $k \in \N$ and thus $B\setminus E = \bigcup \set{x \in B: g_k(x) \neq f_k(x)}$, so that $\absoluteValue{B \setminus E} = 0$.
To see this, notice that 
\begin{align*}
    \absoluteValue{B \setminus E} \leq \absoluteValue{\bigcup \set{x \in B: g_k(x) \neq f_k(x)}} \leq \sum \absoluteValue{\set{x \in B: g_k(x) \neq f_k(x)}} = 0.
\end{align*}

For $x \in B$, let $g(x) = \lim (\chi_E g_k)(x)$.
If $x \in E$, then the limit exists by definition, if $x \in B\setminus E$, the limit also exists because $(\chi_E g_k)(x) = 0$ for all $k \in \N$.
For all $k \in \N$, we have that $\chi_E g_k$ is Borel measurable.
This is true because of 2.44 and the fact that both $\chi_E$ and $g_k$ are Borel measurable, so that their composition is measurable too.
Furthermore, since since $g$ is the limit of Borel measurable functions, then 2.48 tells us $g$ is Borel measurable too.

Finally, we have that $\set{x \in B: g(x) \neq f(x)} \subseteq \bigcup \set{x \in B: g_k(x) \neq f_k(x)}$ so that $\absoluteValue{\set{x \in B: g(x) \neq f(x)}} = 0$, completing the proof. 

\end{proof} 
