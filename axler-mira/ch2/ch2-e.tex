\section{2E Convergence of measurable functions}
%\addcontentsline{toc}{section}{2E Convergence of measurable functions}

7 This is a surprisingly useful property. It shows that in many cases, pointwise convergence is enough to conclude uniform convergence. And uniform convergence is useful to swap integral and limit and other things.
13 This goes with the usual idea that it is very hard to construct non-Borel measurable functions and that they definitely don't show up "in nature". The sup is an easy well-behaved operations that does not yield non-measurable monsters
15 This shows that Borel measurable functions really are all we need. Any Lebesgue measurable functions differs from one except for measure 0 sets. And a measure 0 set will be shown to be totally irrelevant for integrals and other applications of measure theory.

2.D.24

\begin{exercise}{1}
Suppose $X$ is a finite set.
Explain why a sequence of functions from $X$ to $\R$ that converges pointwise on $X$ also converges uniformly on $X$.
\end{exercise}
\begin{proof}
If a sequence of functions $(f_n)$ converges pointwise on $X$, it means that for all $x\in X$ and all $\epsilon>0$, there exists an $N_x \in \N$, such that whenever $n > N_x$, it holds that $\absoluteValue{f_n(x) -f(x)} < \epsilon$.
If we take $N = \max \set{N_x: x \in X}$, then we have that whenever $n>N$, it holds that $\absoluteValue{f_n(x) - f(x)} < \epsilon$, for all $x \in X$.
This is precisely the definition of uniform convergence.
\end{proof} 

\begin{exercise}{2}
Give an example of a sequence of functions from $\N$ to $\R$ that converges pointwise on $\N$ but does not converge uniformly on $\N$.
\end{exercise}
\begin{proof}
Consider the sequence of functions given by $f_n(x) = \chi_{\set{n}}(x)$ for all $n$.
This sequence converges to $f(x) = 0$ for all $n$ pointwise, but there is no $N \in \N$, such that whenever $n>N$ it holds that $\absoluteValue{f_n(x)} < \epsilon$.
\end{proof} 

\begin{exercise}{3}
Give an example of a sequence of functions $f_1, f_2, \dots$ from $[0,1]$ to $\R$ that converges pointwise to a function $f:[0,1] \to \R$ that is not a bounded function.
\end{exercise}
\begin{proof}
Consider the sequence of functions given by $f_n(x) = n^2 x$, whenever $0 \leq x \leq 1/n$ and $f_n(x) = 1/x$ whenever $1/n < x \leq 0$.
Both of the pieces that define the function are continuous in themselves and furthermore $n^2 x= 1/x$ at $x=1/n$ for all $n$, so that the function is continuous there. 
$(f_n)$ converges pointwise to $f(x) = 1/x$ in $(0,1]$ and $f(0) = 0$ but $f$ is not bounded.
\end{proof} 

\begin{exercise}{4}
Prove or give a counterexample:
If $A \subseteq \R$ and $f_1,f_2,\dots$ is a sequence of uniformly continuous functions from $A$ to $\R$ that converges uniformly to a function $f: A \to \R$, then $f$ is uniformly continuous on $A$.
\end{exercise}
\begin{proof}
Let $\epsilon > 0$.
Since $f_n \to f$ uniformly, then there exists an $N \in \N$, such that whenever $n \geq N$, it holds that $\absoluteValue{f_n(x) - f(x)} < \epsilon/3$, for all $x \in A$.
Furthermore, since $f_N$ (notice that $N$ is fixed here) is uniformly continuous, then there exists a $\delta > 0$ such that whenever $\absoluteValue{x - y} < \delta$, it holds that $\absoluteValue{f_N(x) - f_N(y)} < \epsilon/3$.

Putting these together, we have
\begin{align*}
    \absoluteValue{f(x) - f(y)}
    =& \absoluteValue{f(x) - f_n(x) + f_N(x) - f_N(y) + f_N(y) -f(y)}\\
    \leq& \absoluteValue{f(x) - f_N(x)} 
    + \absoluteValue{f_N(x) - f_N(y)}
    + \absoluteValue{f_N(y) -f(y)} < \epsilon,
\end{align*}
for all $x,y \in A$.
\end{proof} 

\begin{exercise}{5}
Give an example to show that Egorov's Theorem can fail without the hypothesis that $\mu(X) < \infty$.
\end{exercise}
\begin{proof}
We can modify the function in exercise 2 as follows.
Let $f_n : \R_{\geq 0} \to \R$ be defined as $f_n(x) = \chi_{[n-1,n)}(x)$.
We have that this sequence of functions converges pointwise to $f(x) = 0$ but there is no subset, $E$ with $\mu(E) = \infty$ such that $f_n$ converges uniformly to the zero function.
\end{proof} 

\begin{exercise}{7}
fill
\end{exercise}
\begin{proof}
fill
\end{proof} 

\begin{exercise}{13}
fill
\end{exercise}
\begin{proof}
fill
\end{proof} 

\begin{exercise}{15}
fill
\end{exercise}
\begin{proof}
fill
\end{proof} 
