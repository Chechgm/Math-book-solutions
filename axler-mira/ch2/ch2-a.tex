\section{2A Outer measure on R}
\addcontentsline{toc}{section}{2A Outer measure on R}


\begin{exercise}{1}
Prove that if $A$ and $B$ are subsets of $\R$ and $\absoluteValue{B}=0$, then $\absoluteValue{A\cup B} = \absoluteValue{A}$.
\end{exercise}
\begin{proof}
Notice that $A\subseteq A\cup B$, so that $\absoluteValue{A}\leq \absoluteValue{A\cup B}$. 
On the other hand, from 2.8, we have that $\absoluteValue{A\cup B}\leq \absoluteValue{A} + \absoluteValue{B} = \absoluteValue{A}$.
The two inequalities together give us the desired result.
\end{proof} 

\begin{exercise}{2}
Suppose $A\subseteq\R$ and $t\in\R$.
Let $tA=\set{ta:a\in A}$.
Prove that $\absoluteValue{t}\absoluteValue{A}$.
[Assume that $0\cdot\infty$ is defined to be to be 0].
\end{exercise}
\begin{proof}
Notice that $t(a,b) = (ta,tb)$, so that $\ell((ta,tb))= tb-ta = \absoluteValue{t}(b-a) = \absoluteValue{t}\ell((a,b))$ (where the first equality follows with a suitable renaming of $a$ and $b$, if $t<0$).
Thus, any cover $I_1,I_2,\dots$ of $A$, can be converted into an open cover $tI_1,tI_2,\dots$ of $tA$, and any open cover of $tA$, $J_1,J_2,\dots$ of $tA$ can be converted into an open cover $J_1/t, J_2/t,\dots$ of $A$.
We would then obtain $\absoluteValue{tA} = \inf\set{\sum_k^\infty \ell(tI_k) = \sum_k^\infty \absoluteValue{t}\ell(I_k) = \absoluteValue{t}\sum_k^\infty \ell(I_k): A\subseteq \bigcup_k^\infty I_k}$.
Finally, by the properties of the infimum, we get that $\absoluteValue{tA} = \absoluteValue{t}\absoluteValue{A}$, as required.
\end{proof} 

\begin{exercise}{6}
Prove that if $a,b\in\R$ and $a<b$, then $\absoluteValue{(a,b)} = \absoluteValue{[a,b)} = \absoluteValue{(a,b]} = b-a$.
\end{exercise}
\begin{proof}
Notice that $[a,b), (b,a] \subseteq [a,b]$, so that by 2.5 and 2.14, we have that $\absoluteValue{[a,b)}, \absoluteValue{(a,b]} \leq b-a$.
Likewise, notice that $(a,b)\subseteq [a,b)$ and $(a,b)\subseteq (a,b]$, so that by 2.1 and 2.5, we have that $b-a = \absoluteValue{(a,b)} \leq \absoluteValue{[a,b)}$ and $b-a = \absoluteValue{(a,b)} \leq \absoluteValue{(a,b]}$, giving us the desired result.
\end{proof} 

\begin{exercise}{11}
Prove that if $I_1,I_2,\dots$ is a disjoint sequence of open intervals, then
\begin{align*}
    \absoluteValue{\bigcup_{k=1}^\infty I_k} = \sum_{k=1}^\infty \ell(I_k).
\end{align*}
\end{exercise}
\begin{proof}
An obvious remark is that $\bigcup_k^\infty I_k$ is covered exactly by $I_1,I_2,\dots$.
So that the infimum of sum of lengths over covers with open intervals of $\bigcup_k^\infty I_k$ is going to be precisely the sum of lengths of each of $I_k$.
Suppose this were not the case, that is, there is another covering of $\bigcup_k^\infty I_k$, such that the sum of lengths of the open intervals in the covering is less than the sum of lengths of $I_k$.
Then that would imply that the outer measure of at least one of the intervals of this potential new covering is less than the outer measure of one $I_k$, which is not possible.
As a result,
\begin{align*}
    \absoluteValue{\bigcup_{k=1}^\infty I_k} = \sum_{k=1}^\infty \ell(I_k),
\end{align*}
as required.
\end{proof} 
