\section{1A Review: Riemann integral}
\addcontentsline{toc}{section}{1A Review: Riemann integral}

\begin{exercise}{2}
Suppose $a \leq s < t \leq b$.
Define $f:[a,b]\to\R$ by $f(x)=1$ if $s<x<t$, and $f(x)=0$ otherwise.
Prove that $f$ is Riemann integrable on $[a,b]$ and that $\int_a^b f = t-s$.
\end{exercise}
\begin{proof}
We will follow a similar strategy as in example 1.10.
Let $P$ be the following partition: $a,s,t,b$.
By 1.5, we have
\begin{align*}
    \cU(f,[a,b])
    \leq (s-a)\sup_{x\in[a,s]} f(x) 
    + (t-s)\sup_{x\in[s,t]} f(x) 
    + (b-t)\sup_{x\in[t,b]} f(x)
    = t-s,
\end{align*}
and
\begin{align*}
    t-s
    =(s-a)\inf_{x\in[a,s]} f(x) 
    + (t-s)\inf_{x\in[s,t]} f(x) 
    + (b-t)\inf_{x\in[t,b]} f(x)
    \leq L(f,[a,b]);
\end{align*}
that is $\cU(f,[a,b]) \leq L(f,[a,b])$.
From 1.8 we know the reverse inequality holds, so that $\int_a^b f = \cU(f,[a,b]) = L(f,[a,b]) = t-s$.
\end{proof} 

\begin{exercise}{3}
Suppose $f:[a,b]\to\R$ is a bounded function. 
Prove that $f$ is Riemann integrable if and only if for each $\epsilon>0$, there exists a partition $P$ of $[a,b]$ such that $\cU(f,P,[a,b]) - L(f,P,[a,b]) < \epsilon$.
\end{exercise}
\begin{proof}
($\Rightarrow$)
Suppose $f$ is Riemann integrable.
Then 
\begin{align*}
    \inf_P \cU(f,P,[a,b]) = \cU(f,[a,b]) = L(f,[a,b]) = \sup_P L(f,P,[a,b]).
\end{align*}
Thus, for $\epsilon>0$, there exists a partition $P'$ such that $\cU(f,[a,b]) + \epsilon/2 > \cU(f,P',[a,b])$, and a partition $P''$ such that $L(f,[a,b]) - \epsilon/2 < L(f,P'',[a,b])$.
Let $P = P' \cup P''$.
\begin{align*}
    \cU(f,P,[a,b]) - L(f,P,[a,b]) 
    \leq& \cU(f,P',[a,b]) - L(f,P'',[a,b])\\
    <& (\cU(f,[a,b]) + \epsilon/2) 
    - (L(f,[a,b]) - \epsilon/2) = \epsilon.
\end{align*}

($\Leftarrow$)
If for every $\epsilon>0$ there exists a partition $P$ such that  $\cU(f,P,[a,b]) - L(f,P,[a,b]) < \epsilon$, then it must be the case that $\cU(f,[a,b]) = L(f,[a,b])$, that is, $f$ is integrable.
\end{proof} 

\begin{exercise}{8}
Suppose $f:[a,b]\to\R$ is Riemann integrable.
Prove that 
\begin{align*}
    \int_a^b f = \lim_{n\to\infty} \frac{b-a}{n}\sum_{j=1}^n f\parens{a+\frac{j(b-a)}{n}}.
\end{align*}
\end{exercise}
\begin{proof}
Notice that for any $n$, the partition $P_n$ of $[a,b]$ given by $a=x_1<x_2<\dots<x_{n-1}<x_n=b$, where $x_j-x_{j-1} = (b-a)/n$, gives us that
\begin{align*}
    L(f, P_n, [a,b]) 
    =& \sum_{j=2}^n (x_j-x_{j-1}) \inf_{x\in[x_j,x_{j-1}]}f(x)\\
    \leq& \sum_{j=2}^n (x_j-x_{j-1}) f\parens{a+\frac{j(b-a)}{n}}\\
    =& \frac{b-a}{n}\sum_{j=1}^n f\parens{a+\frac{j(b-a)}{n}} 
\end{align*}
and
\begin{align*}
    \sum_{j=2}^n (x_j-x_{j-1}) f\parens{a+\frac{j(b-a)}{n}}
    =& \frac{b-a}{n}\sum_{j=1}^n f\parens{a+\frac{j(b-a)}{n}}\\
    \leq& \sum_{j=2}^n (x_j-x_{j-1}) \sup_{x\in[x_j,x_{j-1}]}f(x)\\
    =& \cU(f,P_n,[a,b]).
\end{align*}
That is, for all $P_n$, it is the case that
\begin{align*}
    \cU(f,P_n,[a,b]) 
    \geq \sum_{j=2}^n (x_j-x_{j-1}) f\parens{a+\frac{j(b-a)}{n}}
    \geq L(f,P_n,[a,b]).
\end{align*}
Notice that as $n\to\infty$, $[a,b]$ is divided into $2^m$ intervals of equal size, where $m = \log_2 n$, so that we can use exercise 7, and the fact that $f$ is integrable, to conclude
\begin{align*}
    L(f, [a,b]) = \lim_{n\to\infty} \frac{b-a}{n}\sum_{j=1}^n f\parens{a+\frac{j(b-a)}{n}} = \cU(f,[a,b]).
\end{align*}
\end{proof} 

\begin{exercise}{13}
Suppose $f:[a,b]\to\R$ is an increasing function, meaning that $c,d\in [a,b]$ with $c<d$ implies $f(c)\leq f(d)$.
Prove that $f$ is Riemann integrable on $[a,b]$.
\end{exercise}
\begin{proof}
Let $\epsilon>0$.
Let $P$ be a partition of $[a,b]$ with $n$ elements, such that all the subintervals have the same length, and such that $(f(b)-f(a))/n(b-a)< \epsilon$.
We have
\begin{align*}
    \cU(f,P,[a,b]) - L(f,P,[a,b])
    =& \sum (x_j - x_{j-1})[\sup_{[x_j, x_{j-1}]} f - \inf_{[x_j, x_{j-1}]} f]\\
    =& \frac{b-a}{n} \sum [f(x_j) - f(x_{j-1})]\\
    =& \frac{b-a}{n} [f(a) - f(b)] < \epsilon.
\end{align*}
Where the second equality follows from the increasing assumption, and the last equality follows because the sum telescopes.

Thus, by exercise 3, $f$ is Riemann integrable.
\end{proof} 

\begin{exercise}{14}
Suppose $f_1, f_2,\dots$ is a sequence of Riemann integrable functions on $[a,b]$ such that $f_1,f_2,\dots$ converges uniformly on $[a,b]$ to a function $f:[a,b]\to\R$.
Prove that $f$ is Riemann integrable and $\lim_{n\to\infty}\int_a^b f_n = \int_a^b f$.
\end{exercise}
\begin{proof}
First we will prove that $f$ is Riemann integrable.
Let $\epsilon>0$, from the uniform convergence of $(f_n)$, we know there exists an $N\in\N$, such that whenever $n>N$, it holds that $\absoluteValue{f_n-f}<m\epsilon/3(b-a)$ for all $x\in [a,b]$.
Now, using exercise 3, we know there exists a partition $P$, $a=x_1<\dots<x_m=b$ such that $\cU(f_N,P,[a,b])-L(f_N,P,[a,b])<m\epsilon/3(b-a)$.
We have
\begin{align*}
    \cU(f,P,[a,b]) - L(f,P,[a,b])
    =&  \sum_j (x_j-x_{j-1})[\sup_{[x_j-x_{j-1}]} f - \inf_{[x_j-x_{j-1}]}f]\\
    =&  \sum_j (x_j-x_{j-1})
    [\sup_{[x_j-x_{j-1}]} f - \inf_{[x_j-x_{j-1}]}f\\
    &+ \sup_{[x_j-x_{j-1}]} f_N - \sup_{[x_j-x_{j-1}]}f_N\\
    &+ \inf_{[x_j-x_{j-1}]} f_N - \inf_{[x_j-x_{j-1}]}f_N]\\
    =&  \sum_j (x_j-x_{j-1})
    [\sup_{[x_j-x_{j-1}]} f - \sup_{[x_j-x_{j-1}]}f_N] \\
    &+ \sum_j (x_j-x_{j-1}) 
    [\inf_{[x_j-x_{j-1}]} f_N - \inf_{[x_j-x_{j-1}]}f] \\
    &+ \sum_j (x_j-x_{j-1})
    [\sup_{[x_j-x_{j-1}]} f_N - \inf_{[x_j-x_{j-1}]}f_N]\\
    <& \sum_j (x_j-x_{j-1}) m\epsilon/3(b-a)\\
    &+ \sum_j (x_j-x_{j-1}) m\epsilon/3(b-a)\\
    &+ \sum_j (x_j-x_{j-1}) m\epsilon/3(b-a)\\
    =& \epsilon.
\end{align*}
The first sums being less than $m\epsilon/3(b-a)$ because of uniform convergence and the last sum follows from the Riemann integrability of $f_N$.

Now we will see that the limit of Riemann integrals converges to the integral of the limit.
Let $\epsilon>0$, and choose $N\in\N$ be such that whenever $n>N$, it holds that $\absoluteValue{f_n - f} < \epsilon/(b-a)$ (for all $x\in[a,b]$, given that $(f_n)$ converges uniformly).
We have
\begin{align*}
    \absoluteValue{\int_a^b f_n - \int_a^b f}
    =& \absoluteValue{\int_a^b f_n - \int_a^b f}\\
    \leq& \int_a^b \absoluteValue{f_n - f}\\
    \leq& (b-a)\sup_{x\in[a,b]}\absoluteValue{f_n - f} 
    < \epsilon,
\end{align*}
where the first equality follows from the algebraic rules of the Riemann integral, the first inequality follows from exercise 12, the second to last inequality follows from 1.13, and the last inequality follows from the uniform convergence of $(f_n)$.
\end{proof} 
