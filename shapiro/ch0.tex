\subsection{Preface}

In the preface, Shapiro establishes the relations between the general questions of philosophy (Metaphysics, Semantics and Epistemology) and mathematics. Some examples of this questions are: (Metaphysics) What is the subject matter of mathematics? What is mathematics about?, (Semantics) What do mathematical statements mean? What is the nature of mathematical truth?, (Epistemology) How is mathematics known? Are proofs absolutely certain, immune from rational doubt? Are there unknowable mathematical truths?

Let me briefly discuss the question ``Are proofs absolutely certain, immune from rational doubt?''. From my limited perspective I would say the answer is yes, as long as we take the deductivist logic as given. Otherwise certain proofs might not be accepted, which I believe is why other views on mathematics (such as the constructivists) exist.

After asking these questions, Shapiro talks about the role of mathematics in the understanding of the world. He asks an interesting question: Why is it that we cannot get very far in understanding the world (in scientific terms) if we do not understand a lot of mathematics? I found this question very interesting and I am looking forward to its answer.

Before giving a summary of the book, Shapiro sets the philosophy of mathematics to be part of the philosophy of academic disciplines. 