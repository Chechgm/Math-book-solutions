\subsection{Convergence, Cauchy sequence, completeness}

\begin{exercise}{1 (Subsequence)}
If a sequence $(x_n)$ in a metric space $X$ is convergent and has limit $x$, show that every subsequence $(x_{n_k})$ of $(x_n)$ is convergent and has the same limit $x$.
\end{exercise}
\begin{proof}
Let $(x_{n_k})$ be a subsequence of $(x_n)$. Since $(x_n)$ converges to $x$, then for all $\epsilon>0$, there exists an $N\in\N$ so that for all $n>N$, it holds that $d(x_n,x)<\epsilon$. Fix $\epsilon>0$, and find $N\in\N$ as above. Now $N<n_K$ for some $K\in\N$, so that for all $k>K$ (and thus $n_k>N$, it holds that $d(x_{n_k},x)<\epsilon$ and $(x_{n_k})\to x$, as desired.
\end{proof}

\begin{exercise}{2}
If $(x_n)$ is Cauchy and has a convergent subsequence, say, $x_{n_k}\to x$, show that $(x_n)$ is convergent with the limit $x$.
\end{exercise}
\begin{proof}
Because $(x_n)$ is Cauchy, then for all $\epsilon>0$, there exists an $N\in\N$, so that for all $n,m>N$, it holds that $d(x_n,x_m)<\epsilon/2$. Likewise, there exists an $K\in\N$, so that for all $k>K$, it holds that $d(x_{n_K},x)<\epsilon/2$. Let $N'=\max[N,n_K]$, then for $n>N'$ and a fixed $n_k>N'$, we have $d(x_n,x)\leq d(x_n,x_{n_k})+d(x_{n_k},x)<\epsilon$
\end{proof}

\begin{exercise}{3}
Show that $x_n\to x$ if and only if for every neighborhood $V$ of $x$ there is an integer $n_0$ such that $x_n\in V$ for all $n>n_0$.
\end{exercise}
\begin{proof}
($\Rightarrow$) Suppose $x_n\to x$. Then for every $\epsilon>0$ there exists an $N$ so that for all $n>N$, it holds that $d(x_n,x)<\epsilon$. That is, no matter the size of the $\epsilon$-neighborhood inside $V$, we can find such $n_0$.

($\Leftarrow$) Suppose that for every neighborhood of $x$ we can find an $n_0$ so that for $n>n_0$ it holds that $x_n\in V$. Fix $\epsilon>0$ and let $V=B(x;\epsilon)$, then for $n>n_0$, it holds that $d(x_n,x)<\epsilon$. That is $x_n\to x$.
\end{proof}

\begin{exercise}{4 (Boundedness)}
Show that a Cauchy sequence is bounded.
\end{exercise}
\begin{proof}
A sequence is Cauchy if for all $\epsilon>0$, there exists an $N\in\N$ so that for all $n,m\geq N$, it holds that $d(x_n,x_m)<\epsilon$. Let $\delta=\max\set{d(x_N,x_k):k<N}$. We have that $\delta+\epsilon$ is a bound for the sequence because for arbitrary $x_n$ and $x_m$ in the sequence, $d(x_n,x_m)<\delta$ if $n,m<N$, $d(x_n,x_m)<\epsilon$ if $n,m\geq N$, and $d(x_n,x_m)\leq d(x_n,x_N)+d(x_N,x_m)=\delta+\epsilon$, if $n<N$ and $m\geq N$. 
\end{proof}

\begin{exercise}{5}
Is boundedness of a sequence in a metric space sufficient for the sequence to be Cauchy? Convergent?
\end{exercise}
\begin{proof}
Unfortunately no. Consider the real sequence (under the usual metric) given by $x_n=1$ if $n$ is even and $x_n=0$ if $n$ is odd. This sequence is pointwise bounded but certainly not Cauchy, nor convergent.
\end{proof}

\begin{exercise}{6}
If $(x_n)$ and $(y_n)$ are Cauchy sequences in a metric space $(X,d)$, show that $(a_n)$, where $a_n=d(x_n,y_n)$, converges. Give illustrative examples.
\end{exercise}
\begin{proof}
From exercise 1.1.12, we know that $\absoluteValue{d(x,y)-d(z,w)}\leq d(x,z)+d(y,w)$, so that $\absoluteValue{d(x_n,y_n)-d(x_m,y_m)}\leq d(x_n,x_m)+d(y_n,y_m)$. Fix $\epsilon>0$, since $(x_n)$ and $(y_n)$ are Cauchy, there exist $N,N'\in\N$, so that for all $n,m>N$ and $n',m'>N'$, it holds that $d(x_n,x_m)<\epsilon/2$ and $d(y_{n'},y_{m'})<\epsilon/2$. Let $M=\max[N,N']$, and $n,m>M$. We have $\absoluteValue{d(x_n,y_n)-d(x_m,y_m)}\leq d(x_n,x_m)+d(y_n,y_m)<\epsilon$, so that $(a_n)$ is Cauchy. Since $(a_n)$ is a Cauchy sequence in the reals, by the completeness of the reals (1.4-4), $(a_n)$ converges, as required.
\end{proof}

\begin{exercise}{8}
If $d_1$ and $d_2$ are metrics on the same set $X$ and there are positive numbers $a$ and $b$ such that for all $x,y\in X$, $ad_1(x,y)\leq d_2(x,y)\leq bd_1(x,y)$, show that the Cauchy sequences in $(X,d_1)$ and $(X,d_2)$ are the same.
\end{exercise}
\begin{proof}
Let $(x_n)$ be Cauchy in $(X,d_2)$. For $\epsilon>0$ there exists an $N\in\N$ such that for all $n,m>N$, it holds that $d_2(x_n,x_m)<\epsilon/a$, hence $d_1(x_n,x_m)<ad_2(x_n,x_m)<a\epsilon/a=\epsilon$ and $(x_n)$ is also Cauchy in $(X,d_1)$. We can prove that a Cauchy sequence in $(X,d_1)$ is Cauchy in $(X,d_2)$ mutatis mutandis.
\end{proof}

\begin{exercise}{9}
Using exercise 8, show that the metric space in exercises 13 to 15 of Section 1.2 have the same Cauchy sequences.
\end{exercise}
\begin{proof}
We have that 
\[
\max[d_1(x_1,y_1),d_2(x_2,y_2)]
\leq d_1(x_1,y_1)+d_2(x_2,y_2)
\leq 2\max[d_1(x_1,y_1),d_2(x_2,y_2)],
\]
so that $\bar{d}(x,y)$ and $d(x,y)$ are equivalent.

Furthermore,
\[
\max[d_1(x_1,y_1),d_2(x_2,y_2)]
\leq \sqrt{d_1(x_1,y_1)^2+d_2(x_2,y_2)^2}
\leq \sqrt{2}\max[d_1(x_1,y_1),d_2(x_2,y_2)],
\]
so that $\bar{d}(x,y)$ and $\hat{d}(x,y)$ are equivalent.

Since being equivalent in this sense is an equivalence relation, then $d(x,y)$ and $\hat{d}(x,y)$ are equivalent, as required.
\end{proof}

\begin{exercise}{10}
Using the completeness of $\R$, prove the completeness of $\C$.
\end{exercise}
\begin{proof}
Let $(x_n)$ be a Cauchy sequence in $\C$. That is, $x_n$ can be written as $a_n+ib_n$ for $a_n,b_n\in\R$. Let $\epsilon>0$, then we can find $N\in\N$ so that for any $n,m>N$, we have that $d(x_n,x_m)=\sqrt{(a_n-a_m)^2+(b_n-b_m)^2}<\epsilon$. 

From the previous statement, we can see that for the same $n$ and $m$, it holds that $d(a_n,a_m)\leq \epsilon$ and $d(b_n,b_m)\leq \epsilon$ under the euclidean metric. That is, $(a_n)$ and $(b_n)$ are Cauchy. Since they are Cauchy, they converge to say $a$ and $b$ so that $x_n$ converges to $a+ib$, as required.
\end{proof}