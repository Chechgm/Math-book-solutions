\subsection{Examples. Completeness proofs}

1 Can you find an easy criterion which subsets of R are complete?
2 $R^n$ is complete with different metrics, including this one, which is occasionally useful
3 Example of a noncomplete infinite dimensional space 
5 Example of a completeness proof
6 This is a pretty cool example since with this metric we can treat limits to infinity or being infinity on the same level as usual limits
8 Infinite dimensional completeness proof 
10 Example of a completeness proof
11
12 We see here that the space s has convergence the same as pointwise convergence, in the same way convergence in $l^infinity$ and C[a,b] is uniform convergence. The space s won't really occur too much in Kreyszig however, but it would in Carothers and general topo

\begin{exercise}{1}
Let $a,b\in\R$ and $a<b$. Show that the open interval $(a,b)$ is an incomplete subspace of $\R$, whereas the closed interval $[a,b]$ is complete.
\end{exercise}
\begin{proof}
fill
\end{proof}

\begin{exercise}{2}
Let $X$ be the space of all ordered $n$-tuples $x=(x_1,x_2,\dots,x_n)$ of real numbers and $d(x,y)=\max_j\absoluteValue{x_j,y_j}$ where $y=(y_n)$. Show that $(X,d)$ is complete.
\end{exercise}
\begin{proof}
fill
\end{proof}

\begin{exercise}{3}
Let $M\subseteq l^\infty$ be the subspace consisting of all sequences $x=(x_n)$ with at most finitely many nonzero terms. Find a Cauchy sequence in $M$ which does not converge in $M$, so that $M$ is not complete.
\end{exercise}
\begin{proof}
fill
\end{proof}

\begin{exercise}{5}
Show that the set $X$ of all integers with metric $d$ defined by $d(m,n)=\absoluteValue{m-n}$ is a complete metric space.
\end{exercise}
\begin{proof}
fill
\end{proof}

\begin{exercise}{6}
Show that the set of all real numbers constitutes an incomplete metric space if we choose $d(x,y)=\absoluteValue{\arctan x-\arctan y}$.
\end{exercise}
\begin{proof}
fill
\end{proof}

\begin{exercise}{8 (Space $C[a,b]$)}
Show that the subspace $Y\subseteq C[a,b]$ consisting of all $x\in C[a,b]$ such that $x(a)=x(b)$ is complete
\end{exercise}
\begin{proof}
fill
\end{proof}

\begin{exercise}{10 (Discrete metric)}
Show that a discrete metric space (cf. 1.1-8) is complete.
\end{exercise}
\begin{proof}
fill
\end{proof}

\begin{exercise}{11 (Space $s$)}
Show that in the space $s$ (cf. 1.2-1) we have $x_n\to x$ if and only if $x^n_j\to x_j$ for all $j=1,2,\dots$, where $x_n=(x^n_j)$ and $x=(x_j)$.
\end{exercise}
\begin{proof}
fill
\end{proof}

\begin{exercise}{12}
Using exercise 11, show that the sequence space $s$ in 1.2-1 is complete.
\end{exercise}
\begin{proof}
fill
\end{proof}