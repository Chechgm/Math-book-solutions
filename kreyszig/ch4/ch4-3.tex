\section{Hahn-Banach Theorem for complex vector spaces and normed spaces}

1
2 Some easy properties of seminorms
4 A very typical application of the Hahn-Banach theorem
5 This implies that we can always choose the seminorm in order to imply that our extension functional has a certain norm 
8 It is surprising that an easier proof is not possible. In finite dimensions this is pretty easy to prove. But no, complicated machinery Hahn-Banach is actually necessary here in general.
9 It is well known among functional analysts that separability can often be used to avoid the axiom of choice. You have already seen an example of this with the existence of an orthonormal basis in separable Hilbert space. Here is yet another version. Hahn-Banach for separable spaces is thus not dependent on axiom of choice, and it is actually the separable version that is used most of the time.
11 This is yet another useful result which is typically seen to rely on Hahn-Banach. It is actually a very useful technique in functional analysis that instead of proving a certain "fact" in Banach spaces, it is often useful to prove f("fact") for every bounded linear functional, and then to use Hahn-Banach to deduce "fact". This is easier since f("fact") is a statement of real/complex numbers and thus presumably easier to prove!
13 Another example of Hahn-Banach 
15 Similar comment to that of exercise 11.

\begin{exercise}{x}
fill
\end{exercise}
\begin{proof}
fill
\end{proof} 

\begin{exercise}{x}
fill
\end{exercise}
\begin{proof}
fill
\end{proof} 

\begin{exercise}{x}
fill
\end{exercise}
\begin{proof}
fill
\end{proof} 

\begin{exercise}{x}
fill
\end{exercise}
\begin{proof}
fill
\end{proof} 

\begin{exercise}{x}
fill
\end{exercise}
\begin{proof}
fill
\end{proof} 

\begin{exercise}{x}
fill
\end{exercise}
\begin{proof}
fill
\end{proof} 

\begin{exercise}{x}
fill
\end{exercise}
\begin{proof}
fill
\end{proof} 

\begin{exercise}{x}
fill
\end{exercise}
\begin{proof}
fill
\end{proof} 

\begin{exercise}{x}
fill
\end{exercise}
\begin{proof}
fill
\end{proof} 

\begin{exercise}{x}
fill
\end{exercise}
\begin{proof}
fill
\end{proof} 

\begin{exercise}{x}
fill
\end{exercise}
\begin{proof}
fill
\end{proof} 

\begin{exercise}{x}
fill
\end{exercise}
\begin{proof}
fill
\end{proof} 

\begin{exercise}{x}
fill
\end{exercise}
\begin{proof}
fill
\end{proof} 

\begin{exercise}{x}
fill
\end{exercise}
\begin{proof}
fill
\end{proof} 

\begin{exercise}{x}
fill
\end{exercise}
\begin{proof}
fill
\end{proof} 

\begin{exercise}{x}
fill
\end{exercise}
\begin{proof}
fill
\end{proof} 

\begin{exercise}{x}
fill
\end{exercise}
\begin{proof}
fill
\end{proof} 

\begin{exercise}{x}
fill
\end{exercise}
\begin{proof}
fill
\end{proof} 

\begin{exercise}{x}
fill
\end{exercise}
\begin{proof}
fill
\end{proof} 