\section{Hahn-Banach Theorem for complex vector spaces and normed spaces}


\begin{exercise}{1 (Seminorm)}
Show that $p(x+y) \leq p(x) + p(y)$ and $p(\alpha x) = \absoluteValue{\alpha} p(x)$ for every scalar $\alpha$ imply $p(0) = 0$ and $p(x) \geq 0$ so that $p$ is a seminorm.
\end{exercise}
\begin{proof}
First, take $\alpha = 0$, so that $p(0) = p(0 x) = \absoluteValue{0} p(x) = 0$.
Furthermore, $0 = p(0) = p(x + (-x)) \leq p(x) + p(-x) = 2p(x)$, thus $p(x) \geq 0$.
\end{proof} 

\begin{exercise}{2}
Show that $p(x+y) \leq p(x) + p(y)$ and $p(\alpha x) = \absoluteValue{\alpha} p(x)$ for every scalar $\alpha$ imply $\absoluteValue{p(x) - p(y)} \leq p(x-y)$.
\end{exercise}
\begin{proof}
First, we have that $p(x) = p(x-y + y) \leq p(x-y) + p(y)$, so that $p(x) - p(y) \leq p(x-y)$.
Furthermore $p(y) = \absoluteValue{-1}p(y) = p(-y) = p(x-y + x) \leq p(x-y) + p(x)$;
that is $-p(x-y) \leq p(x) - p(y)$.
Putting these together, we get the desired result.
\end{proof} 

\begin{exercise}{4}
Let $p$ be defined on a vector space $X$ and satisfy $p(x+y) \leq p(x) + p(y)$ and $p(\alpha x) = \absoluteValue{\alpha} p(x)$ for every scalar $\alpha$.
Show that for any given $x_0 \in X$ there is a linear functional $\tilde{f}$ on $X$ such that $\tilde{f}(x_0) = p(x_0)$ and $\absoluteValue{\tilde{f}(x)} \leq p(x)$ for all $x \in X$.
\end{exercise}
\begin{proof}
Let $Z \subseteq X$, where $Z = \set{x \in X: x = \alpha x_0, \alpha \in \R}$, with $x_0$ fixed.
Now we define $f$ to be a linear functional on $Z$ such that $ f(x) = \alpha p(x_0)$.
We know from exercise 4.2.9 that $f$ is linear on $Z$ and $f(x) \leq p(x)$.
Using the Hahn-Banach Theorem, we extend $f$ to $\tilde{f}$ so that $\tilde{f}(x) \leq p(x)$.
Finally, $\tilde{f}(-x) \leq p(-x)$ which implies $-\tilde{f}(x) \leq p(x)$; that is, $\tilde{f}(x) \geq -p(x)$, so that putting there inequalities together we obtain $\absoluteValue{\tilde{f}(x)} \leq p(x)$.
\end{proof} 

\begin{exercise}{5}
If $X$ in Theorem 4.3-1 is a normed space and $p(x) \leq k\norm{x}$ for some $k > 0$, show that $\norm{\tilde{f}} \leq k$.
\end{exercise}
\begin{proof}
We have that $\absoluteValue{\tilde{f}(x)} \leq p(x) \leq k\norm{x}$, so that $\absoluteValue{\tilde{f}(x)}/\norm{x} \leq k$.
Taking the supremum with respect to all $x$ such that $\norm{x}=1$ we get the desired result.
\end{proof} 

\begin{exercise}{8}
Let $X$ be a normed space and $X'$ its dual space.
If $X \neq \set{0}$, show that $X'$ cannot be $\set{0}$.
\end{exercise}
\begin{proof}
We will prove this by contrapositive.
Let $x_0 \in X$.
If $X = \set{0}$, then $f(x_0) = 0$ for all $f \in X'$, thus $x_0 = 0$ by 4.3-4.
\end{proof} 

\begin{exercise}{9}
Show that for a separable normed space $X$, Theorem 4.3-2 can be proved directly, without the use of Zorn's Lemma (which was used indirectly, namely in the proof of Theorem 4.2-1).
\end{exercise}
\begin{proof}
Since $X$ is a separable normed space, there exists a countable subset, $Y \subseteq X$, such that $\bar{Y} = X$.
Let $f$ be the bounded linear functional defined on $Z \subseteq X$, as in Theorem 4.3-2.
Let $y_1 \in Y \setminus Z$, and define $Z_1$ to be the subspace spanned by $Z$ and $y_1$.
Any $x \in Z_1$ can be written uniquely as $x = z + ay_1$.
We now define a functional, $g_1$ on $Z_1$ by $g_1(z + ax_1) = f(z) + ac$, where $c$ is any real constant.
For $a=0$, we have $g_1(z) = f(z)$.
Hence $g_1$ is a proper extension of $f$.

We will now prove that $g_1(x) \leq p(x)$ for all $x \in Z_1$.
Let $z, z' \in Z$ and $y_1$ be fixed.
We have 
\begin{align*}
    f(z) - f(z') = f(z - z') 
    \leq& p(z - z')\\
    =& p(z + y_1 - y_1 - z')\\
    \leq& p(z + y_1) + p(- y_1 - z'),
\end{align*}
so that $-p(- y_1 - z') - f(z') \leq p(z + y_1) - f(z)$.
Since $z$ does not appear on the left and $z'$ does not appear on the right, the inequality continues to hold if we take the supremum over $z \in Z$ on the left (say $m_0$) and the infimum over $z' \in Z$ on the right (say $m_1$).
Hence $m_0 \leq m_1$ and for a $c$ with $m_0 \leq c \leq m_1$. 
We then have 
\begin{align*}
    -p(-y_1 - z') - f(z') \leq c \text{ for all } z' \in Z
\end{align*}
and 
\begin{align*}
    c \leq p(z +y_1) - f(z) \text{ for all } z \in Z.
\end{align*}

Using the first inequality we will show that $g_1(x) \leq p(x)$ holds for negative $a$. 
For $a < 0$, replace $ z' = a^{-1}z$;
that is $-p(-y_1 - a^{-1}z) - f(a^{-1}z) \leq c$, so that multiplying by $-a$ we obtain $ap(-y_1 - a^{-1}z) + f(z) \leq -ac$ giving us
\begin{align*}
    g_1(x) 
    = f(z) + ac 
    \leq -ap(-y_1 - a^{-1}z)
    = p(ay_1 + z)
    = p(x).
\end{align*}

Now using the second inequality above, and $a > 0$.
Let $z = a^{-1}z$, we have that $c \leq p(a^{-1}z + y_1) - f(a^{-1}z$ so that multiplying by $a$, we get 
\begin{align*}
    ac 
    \leq ap(a^{-1}z + y_1) - f(z)
    = p(x) - f(z).
\end{align*}
That is, $g_1(x) = f(z) + ac \leq p(x)$.
\end{proof} 

We have proved that the inequality holds for the extension of $f$ in $Z_1$.
By induction, this holds for any $n \in \N$, so that $Y = \bigcup Z_i$;
that is, we have an extension of $f$, say $g$, in a dense set in $X$. 
Finally, we use Theorem 2.7-11 to conclude there is an extension of $g$, $\bar{g}$ in $\bar{Y} = X$, and furthermore $\norm{g} = \norm{\bar{g}}$.
Thus, 4.3-2 holds in separable spaces without using Zorn's Lemma.
\begin{exercise}{11}
If $f(x) = f(y)$ for every bounded linear functional $f$ on a normed space $X$ show that $x = y$.
\end{exercise}
\begin{proof}
We have $f(x-y) = 0$ for every functional.
By 4.3-4 we have that $x-y = 0$, giving us the desired result.
\end{proof} 

\begin{exercise}{13}
Let $X$ be a normed space and let $x_0 \neq 0$ be any element of $X$.
Then there is a bounded linear functional $\hat{f}$ on $X$ such that $\norm{\hat{f}} = \norm{x_0}^{-1}$ and $\hat{f}(x_0) = 1$.
\end{exercise}
\begin{proof}
We are going to follow a similar strategy as Theorem 4.3-3.
Let $Z$ be the subspace of $X$ defined as $x = ax$, where $a$ is a scalar.
Let $f(x) = f(a x_0) = a\norm{x_0}^{-1}\norm{x_0}$.
$f$ has norm $\norm{f} = \norm{x_0}^{-1}$ because $\absoluteValue{f(x)} = \absoluteValue{f(a x_0)} = \absoluteValue{a}\norm{x_0}^{-1}\norm{x_0} = \norm{x_0}^{-1}\norm{a x_0} = \norm{x_0}^{-1}\norm{x}$ implying that $\absoluteValue{f(x)}/\norm{x} = \norm{x_0}^{-1}$ and taking the supremum with respect to $x$ such that $\norm{x} = 1$ gives us the desired norm.
Finally, we have that $f(x_0) = \norm{x_0}^{-1}\norm{x_0} = 1$.
By the Hahn-Banach Theorem we can extend this functional to a functional, $\hat{f}$ in $X$, so that $\norm{\hat{f}} = \norm{f} = \norm{x_0}^{-1}$, as required.
\end{proof} 

\begin{exercise}{14 (Hyperplane)}
Show that for any sphere $S(0;r)$ in a normed space $X$ and any point $x_0 \in S(0;r)$ there is a hyperplane $H_0$ with $x_0 \in H_0$ such that the Ball $\tilde{B}(0;r)$ lies entirely in one of the two half spaces determined by $H_0$.
(See exercises 12 and 15, Section 2.8).
A simple illustration is shown in the following figure:
\begin{figure}[H]
    \centering
    \begin{tikzpicture}[
    scale=1.5,
    % Style for the hollow points (center and x0)
    point/.style={draw, circle, inner sep=0pt, minimum size=4pt, fill=white, thick}
]

    % --- Parameters ---
    \def\R{2.0}      % Radius of the circle
    \def\ang{-40}    % Angle of point x0 on the circle (approx based on image)
    \def\lenTop{2.5} % Length of tangent line extending upwards
    \def\lenBot{2.5} % Length of tangent line extending downwards

    % --- Coordinates ---
    \coordinate (O) at (0,0);
    \coordinate (X0) at (\ang:\R);

    % --- Drawing ---

    % 1. The Circle
    \draw[thick] (O) circle (\R);

    % 2. The Tangent Line (H0)
    % Calculated perpendicular to the radius at x0
    \draw[thick] 
        ($(X0) + (\ang+90:\lenTop)$) -- 
        ($(X0) + (\ang-90:\lenBot)$) 
        node[pos=1.0, below left] {$H_0$};

    % 3. The Center Point (0)
    \node[point] at (O) {}; 
    \node[below=5pt] at (O) {$0$};

    % 4. The Tangent Point (x0)
    % Drawn after the line so the hollow fill covers the line segment
    \node[point] at (X0) {};
    \node[above left=3pt] at (X0) {$x_0$};

    % 5. The Circle Label S(0; r)
    \node at (40:{\R+0.4}) {$S(0; r)$};

\end{tikzpicture}
    \caption{Illustration of exercise 14 in the case of Euclidean plane $\R^2$.}
    \label{fig:sec4-3-ex14}
\end{figure}
\end{exercise}
\begin{proof}
Another way of thinking about this exercise, is that we want a functional $\tilde{f} \in X'$, such that $\tilde{f}(x_0) = \norm{x_0} = r$, then the fact that $\tilde{B}(0;r) \subseteq H_0 = \set{x \in X: \tilde{f}(x) \leq \norm{x_0} = r}$ is guaranteed by definition.
The existence of such functional is follows by Theorem 4.3-3, as required.
\end{proof} 

\begin{exercise}{15}
If $x_0$ in a normed space $X$ is such that $\absoluteValue{f(x_0)} \leq c$ for all $f \in X'$ of norm 1, show that $\norm{x_0} \leq c$.
\end{exercise}
\begin{proof}
From 4.3-4, we have 
\begin{align*}
    \norm{x_0} 
    =& \sup_{f \in X', f \neq 0} \frac{\absoluteValue{f(x_0)}}{\norm{f}^{-1}}\\
    =& \sup_{f \in X', \norm{f}=1} \absoluteValue{f(x_0)}.
\end{align*}
By assumption, $\sup_{f \in X', \norm{f}=1} \absoluteValue{f(x_0)} \leq c$, thus $\norm{x_0} \leq c$, as required.
\end{proof} 
