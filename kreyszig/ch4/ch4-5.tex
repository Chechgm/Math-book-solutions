\section{Adjoint operator}


\begin{exercise}{1}
Let $X$ and $Y$ be normed spaces.
Let $T: X \to Y$ be a bounded linear operator and $g$ a bounded linear functional on $Y$.
Show that $f(x) = g(Tx)$ is linear.
\end{exercise}
\begin{proof}
Let $x, z \in Z$ and $a, b \in K$.
We have 
\begin{align*}
    f(ax + bz) 
    =& g(T(ax + bz))\\ 
    =& g(aTx + bTz)\\ 
    =& ag(Tx) + bg(Tz)\\
    =& af(x) + bf(z).
\end{align*}
\end{proof} 

\begin{exercise}{2}
What are the adjoints of a zero operator 0 and an identity operator $I$?
\end{exercise}
\begin{proof}
We have $g(0x) = 0$ for all $g$ and $x$, so that $(T^\times g)(x) = 0$ and $T^\times = 0$.
Likewise $g(Ix) = g(x)$ for all $g$ and $x$, so that $(T^\times g)(x) = g(x) = g(x)$ and $T^\times = I$.
\end{proof} 

\begin{exercise}{3}
Prove $(S+T)^\times = S^\times + T^\times$.
\end{exercise}
\begin{proof}
We have 
\begin{align*}
    (S+T)^\times(g)(x)
    =& g((S+T)x)\\
    =& g(Sx + Tx)\\
    =& g(Sx) + g(Tx)\\
    =& (S^\times g)(x) + (T^\times g)(x).
\end{align*}

\end{proof} 

\begin{exercise}{4}
Prove $(aT)^\times = aT^\times$
\end{exercise}
\begin{proof}
We have
\begin{align*}
    (aT^\times g)(x)
    = g(aTx)
    = ag(Tx)
    = a(T^\times g)(x)
\end{align*}
\end{proof} 

\begin{exercise}{5}
Prove $(ST)^\times = T^\times S^\times$.
\end{exercise}
\begin{proof}
We have
\begin{align*}
    ((ST)^\times g)(x)
    =& g((ST)x)\\
    =& g(S(Tx))\\
    =& (T^\times (gS))(x)\\
    =& T^\times (g(Sx))\\
    =& (S^\times(T^\times g))(x)\\
    =& ((S^\times T^\times) g)(x).
\end{align*}
\end{proof} 

\begin{exercise}{6}
Show that $(T^n)^\times = (T^\times)^n$.
\end{exercise}
\begin{proof}
We will prove this by induction.
The base case is exercise 5 where $S=T$.
As an inductive hypothesis, suppose $(T^n)^\times = (T^\times)^n$ holds for $n$.
Now using exercise 5, where $S = T^n$ gives us the desired result.
\end{proof} 

\begin{exercise}{7}
What formula for matrices do we obtain by combining exercise 5 and example 4.5-3?
\end{exercise}
\begin{proof}
We get the usual $(AB)^\top = B^\top A^\top$ for matrices $A$ and $B$.
\end{proof} 

\begin{exercise}{8}
Prove that if $T^{-1}$ exists, then $(T^\times)^{-1}$ also exists and $(T^\times)^{-1} = (T^{-1})^\times$.
\end{exercise}
\begin{proof}
We have
\begin{align*}
    T^\times (T^{-1})^\times
    = (T^{-1} T)^\times
    = I^\times
    = I
    = I^\times
    = (T T^{-1})^\times
    = (T^{-1})^\times T^\times,
\end{align*}
proving the desired result.
\end{proof} 

\begin{exercise}{9 (Annihilator)}
Let $X$ and $Y$ be normed spaces, $T: X \to Y$ a bounded linear operator and $M = \overline{\cR(T)}$, the closure of the range of $T$.
Show that $M^a = \cN(T^\times)$.
(See exercise 2.10.13).
\end{exercise}
\begin{proof}
Let $g \in \cN(T^\times)$.
Then $(T^\times g)(x) = 0x = 0$ so that $g \in M^a$, which implies $\cN(T^\times) \subseteq M^a$.
Conversely, Suppose $g \in M^a$, so that $g(Tx) = 0$ for all $x \in \overline{\cR(T)}$.
Thus $(T^\times g)(x) = 0$ and $g \in \cN(T^\times)$, which implies $M^a \subseteq \cN(T^\times)$.
Putting these two containments together gives us the desired result.
\end{proof} 

\begin{exercise}{10 (Anihilator)}
Let $B$ be a subset of the dual space $X'$ of a normed space $X$.
The anihilator $\prescript{a}{}{B}$ of $B$ is defined to be $\prescript{a}{}{B} = \set{x \in X: f(x)=0 \text{ for all }f \in B}$.
Show that in exercise 9, $\cR(T) \subseteq \prescript{a}{}{\cN(T^\times)}$.
What does this mean with respect to the task of solving an equation $Tx = y$.?
\end{exercise}
\begin{proof}
Let $y \in \cR(T)$, thus $y = Tx$ for some $x \in X$.
For any $g \in \cN(T^\times)$, we have $(T^\times g)(x) = g(Tx) = 0$, so that $y \in \prescript{a}{}{\cN(T^\times)}$.

Micro's comment on the task of solving an equation:
The annihilator may seem abstract, but it is something we are actually intrinsically interested in. 
Take a system of linear equations, like $x+y = 0$, $2x+3y = 0$. 
This induces functions $f(x,y) = x+y$ and $g(x,y) = 2x+3y$. 
The solution set of the system is now simply the annihilator of the set $\set{f,g}$. 
So annihilators are nothing more than solution sets of systems of equations.
\end{proof} 
