\section{Strong and weak convergence}


\begin{exercise}{1 (Pointwise convergence)}
If $x_n \in C[a,b]$ and $x_n \wto x \in C[a,b]$, show that $(x_n)$ is pointwise convergent on $[a,b]$, that is, $(x_n(t))$ converges for every $t \in [a,b]$.
\end{exercise}
\begin{proof}
Let $f_t \in (C[a,b])'$ be the evaluation functional, defined by $f_t(x) = x(t)$.
Since $(x_n)$ converges weakly, we have that $f_t(x) \to f_t(x)$, so that $x_n(t) \to x(t)$, since this holds for every $t \in [a,b]$, we get the desired result.
\end{proof} 

\begin{exercise}{2}
Let $X$ and $Y$ be normed spaces $T \in B(X,Y)$ and $(x_n)$ a sequence in $X$.
If $x_n \wto x$, show that $T x_n \wto Tx$.
\end{exercise}
\begin{proof}
Consider $g \in Y'$, notice that $g \circ T \in X'$, so that $(g \circ T)(x_n) \to (g \circ T)(x)$, since $f(x_n) \to f(x)$ due to the weak convergence of $(x_n)$.
But this is the same as saying that $g(T x_n) \to g(T x)$, so that $T x_n \wto T x$, as required.
\end{proof} 

\begin{exercise}{3}
IF $(x_n)$ and $(y_n)$ are sequences in the same normed space $X$, show that $x_n \wto x$ and $y_n \wto y$ implies $x_n + y_n \wto x+y$ as well as $ax_n \wto ax$ where $a$ is any scalar.
\end{exercise}
\begin{proof}
We have that for all $f \in X'$, $f(x_n) \to f(x)$ and $f(y_n) \to f(y)$.
This gives us that $f(x_n + y_n) = f(x_n) + f(y_n) \to f(x) + f(y) = f(x+y)$, so that $x_n + y_n \wto x+y$.
Likewise, for any scalar $a$, $f(ax_n) = af(x_n) \to af(x) = f(ax)$, so that $ax_n \wto ax$.
In both cases we have used the algebraic properties of limits and properties of linear operators.
\end{proof} 

\begin{exercise}{4}
Show that $x_n \wto x$ implies $\liminf \norm{x_n} \geq \norm{x}$.
(Use Theorem 4.3-3).
\end{exercise}
\begin{proof}
From Theorem 4.3-3 we know there exists a functional $f \in X'$ such that $\norm{f} = 1$ and $f(x) = \norm{x}$.
By the weak convergence of $x_n$, we have that $f(x_n) \to f(x) = \norm{x}$ so that $\absoluteValue{f(x)} \to \absoluteValue{f(x)} = \norm{x}$.
We have
\begin{align*}
    \norm{x}
    =& \lim \absoluteValue{f(x_n)}\\
    =& \liminf \absoluteValue{f(x_n)}\\
    \leq& \liminf \parens{\norm{f}\norm{x_n}}\\
    =& \liminf \norm{x_n},
\end{align*}
where the second equality follows from the fact that $\lim f(x_n) = \liminf f(x_n) = \limsup f(x_n)$ when $(f(x_n))$ is a convergent sequence in the reals.
\end{proof} 

\begin{exercise}{5}
If $x_n \wto x$ in a normed space $X$, show that $x \in \bar{Y}$, where $Y = \vecspan(x_n)$.
(Use Lemma 4.6-7).
\end{exercise}
\begin{proof}
Suppose, for the sake of contradiction, that $x \notin \overline{\vecspan(x_n)}$.
Then, from Lemma 4.6-7, there exists a functional with $f(x_n) = 0$ for all $n$, and $f(x) = \delta$, where $\delta = \inf_{y \in Y} \norm{y -x}$, but then $f(x_n) \not\to f(x)$, which contradicts $x_n \wto x$.
\end{proof} 

\begin{exercise}{6}
If $(x_n)$ is a weakly convergent sequence in a normed space $X$, say $x_n \wto x$, show that there is a sequence $(y_n)$ of linear combinations of elements of $(x_n)$ which converges strongly to $x$.
\end{exercise}
\begin{proof}
This is a Corollary to exercise 5, given that $x \in \overline{\vecspan{x_n}}$, which means exactly the desired result.
\end{proof} 

\begin{exercise}{7}
Show that any closed subspace $Y$ of a normed space $X$ contains the limits of all weakly convergent sequences of its elements.
\end{exercise}
\begin{proof}
This is a Corollary to exercise 5.
For any weakly convergent sequence $(x_n)$ in a closed subspace $Y$, $\overline{\vecspan(x_n)} \subseteq Y$, furthermore from exercise 5, we know $x \in \overline{\vecspan(x_n)}$ so that $x \in Y$, as required.
\end{proof} 

\begin{exercise}{8 (Weak Cauchy sequence)}
A weak Cauchy sequence in a real or complex normed space $X$ is a sequence $(x_n)$ in $X$ such that for every $f \in X'$ the sequence $(f(x_n))$ is Cauchy in $\R$ or $\C$, respectively.
[Note that then $\lim f(x_n)$ exists].
Show that a Cauchy sequence is bounded.
\end{exercise}
\begin{proof}
Notice that in the proof of Lemma 4.8-3.c, the property of the weak convergence of $(x_n)$ we used is that $(f(x_n))$ is bounded.
This property also holds for weak Cauchy sequences (since $(f(x_n))$ is Cauchy).
Thus, we can use the same proof, mutatis mutandis, to conclude that $\norm{x_n}$ is bounded, as required.
\end{proof} 

\begin{exercise}{9}
Let $A$ be a normed space $X$ such that every nonempty subset of $A$ contains a weak Cauchy sequence. 
Show that $A$ is bounded.
\end{exercise}
\begin{proof}
Suppose for the sake of contradiction, that $A$ is not bounded. 
Then there exists a sequence in $A$, say $(x_n)$ such that $\lim \norm{x_n} = \infty$.
Furthermore, we can choose a subsequence $(x_{n_k})$ so that $\norm{x_{n_k}}$ is monotonically increasing.
Thus, assuming that the sequence must be injective (so non-trivial same element sequences), then the set consisting of the elements of $(x_{n_k})$ has no (injective) Cauchy sequence.
\end{proof} 

\begin{exercise}{10 (Weak completeness)}
A normed space $X$ is said to be weakly complete if each weak Cauchy sequence in $X$ converges weakly in $X$.
If $X$ is reflexive, show that $X$ is weakly complete.
\end{exercise}
\begin{proof}
Notice that a weak Cauchy sequence $(x_n)$ corresponds to a to a sequence $(g_{x_n})$ in $X''$ that is pointwise Cauchy;
that is, for every $f \in X'$, $(g_{x_n}(f))$ is Cauchy.
By Theorem 4.6-4 and the fact that $X$ is reflexive, then $X$ (and $X''$) is complete.
Hence, we can use exercise 4.8.12 to conclude that $f(x_n) = g_{x_n}(f) \to g_x(f) = f(x)$, so that $f(x_n) \to f(x)$, meaning that $x_n \wto x$ in $X$ and as a result, $X$ is weakly complete.
\end{proof} 
