\section{Application to summability of sequences}

1
2
3
4 Cesaro summability is quite important in harmonic analysis. It turns out that the Fourier series of a continuous function need not converge to this function, which is bad. But the good thing is that it always is Cesaro summable to the continuous function. This is often good enough to ensure things like uniqueness of Fourier coefficients, etc. 
6 Abel summability is another cool method used in Fourier series. Of note here is Abel's theorem which gives conditions when Abel summability implies actual convergence of the series. 
8 I met Euler's method first in an exercise in SICP. There the method is used to accelerate convergence of series. We can see that happen here. The given series for log(2) converges slowly, but the transformed series converges waaay more rapidly. If it is still too slow for your taste, you can Euler transform it again and again!

\begin{exercise}{1}
fill
\end{exercise}
\begin{proof}
fill
\end{proof} 

\begin{exercise}{2}
fill
\end{exercise}
\begin{proof}
fill
\end{proof} 

\begin{exercise}{3}
fill
\end{exercise}
\begin{proof}
fill
\end{proof} 

\begin{exercise}{4}
fill
\end{exercise}
\begin{proof}
fill
\end{proof} 

\begin{exercise}{6}
fi
\end{exercise}
\begin{proof}
fill
\end{proof} 

\begin{exercise}{8}
fill
\end{exercise}
\begin{proof}
fill
\end{proof} 
