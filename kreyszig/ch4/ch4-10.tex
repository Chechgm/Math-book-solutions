\section{Application to summability of sequences}


\begin{exercise}{1 (Cesaro's summability method $C_1$)}
$C_1$ is defined by $y_n = (1/n)(x_1 + \dots + x_n)$, for $n = 1,2,\dots$, that is, one takes arithmetic means.
Find the corresponding matrix $A$.
\end{exercise}
\begin{proof}
The matrix would be $a_{n,k} = 1/n$ for $k \leq n$ and 0 otherwise.
\end{proof} 

\begin{exercise}{2}
Apply the method $C_1$ in exercise 1 to the sequences $x = (1,0,1,0,1,0,\dots)$ and $x' = (1, 0, -1/4, -2/8, -3/16, -4/32, \dots)$.
\end{exercise}
\begin{proof}
For $x$ we get the sequence $(1, 1/2, 2/3, 1/4, 3/5, 1/2, \dots)$.
For $x'$ we get the sequence $(1, 1/2, 1/3-1/12, 1/4-1/16-2/32, 1/5-1/20-2/40-3/80, \dots)$.
\end{proof} 

\begin{exercise}{3}
In exercise 1, express $(x_n)$ in terms of $(y_n)$.
Find $(x_n)$ such that $y_n = 1/n$.
\end{exercise}
\begin{proof}
we have $x_n = ny_n - x_1 - \dots - x_{n-1}$.
The sequence $x = (1, 0, 0, \dots)$ gives us $y_n = 1/n$ for all $n$.
\end{proof} 

\begin{exercise}{4}
Use the formula in exercise 3 for obtaining a sequence which is not $C_1$ summable.
\end{exercise}
\begin{proof}
Without using the sequence in the previous exercise, the sequence $(1,2,3,4,\dots)$ is not $C_1$ summable, given that its $C_1$ sequence is $(1, 1/2 + 1, 1/3 + 2/3 + 1, \dots)$ which does not converge.
\end{proof} 

\begin{exercise}{6 (Series)}
An infinite series is said to be A-summable if the sequence of its partial sums is A-summable, and the A-limit of that sequence is called the A-sum of the series.
Show that $1 + z + z^2 + \dots$ is $C_1$-summable for $\absoluteValue{z} = 1$, $z \neq 1$ and the $C_1$-sum is $1/(1-z)$
\end{exercise}
\begin{proof}
Notice that the sequence of partial $C_1$-sums is given by $s_n = \sum^n_{i=0} z^i/(i+1)$.
To see this series converges we can use Dirichlet's test.
Thus we write the series $\sum a_n b_n$, where $a_n = 1/(n-1)$ is a sequence of real numbers and $s'_n = \sum b_n = \sum z^n$ is complex.
We need to prove that $a_n \to 0$ and $s'_n$ is bounded to guarantee convergence of $\sum a_n b_n$.
It is clear that $a_n \to 0$, so that the first condition for the Dirichlet test is satisfied. 
To see $s'_n$ is bounded, notice that 
\begin{align*}
    \absoluteValue{s'_n}
    = \absoluteValue{\sum_{i=0}^n z^n}
    = \absoluteValue{\frac{1-z^{n+1}}{1-z}}
    \leq \frac{1+\absoluteValue{z^{n+1}}}{\absoluteValue{1-z}}
    = \frac{1+\absoluteValue{z}^{n+1}}{\absoluteValue{1-z}}
    = 2/\absoluteValue{1-z}.
\end{align*}
Thus, by the Dirichlet's test, the series converges and by an application of the geometric series we get the resulting sum.
\end{proof} 

\begin{exercise}{8 (Euler's method)}
Euler's method for series associates with a given series $\sum_{j=0}^\infty (-1)^j a_j$ the transformed series $\sum_{n=0}^\infty \Delta^n a_0/2^{n+1}$, where $\Delta^0 a_j = a_j$, $\Delta^n a_j = \Delta^{n-1}a_j - \Delta^{n-1} a_{j+1}$, for $j=1,2,\dots$, and $(-1)^j$ is written for convenience (hence the $a_j$ need not be positive).
It can be shown that the method is regular, so that the convergence of the given series implies that of the transformed series, the sum being the same.
Show that the method gives: $\ln 2 = 1 - 1/2 + 1/3 - 1/4 + \dots = 1/(1\cdot 2^1) + 1/(2 \cdot 2^2) + 1/(3 \cdot 2^3) + 1/(4 \cdot 2^4) + \dots$.
\end{exercise}
\begin{proof}
We have 
\begin{align*}
    \Delta^0 a_0 = 1,
\end{align*}
moreover,
\begin{align*}
    \Delta^1 a_0 = \Delta^0 a_0 - \Delta^0 a_1,
\end{align*}
so that if 
\begin{align*}
    \Delta^0 a_1 = a_1
\end{align*}
then $\Delta^1 a_0 = 1-1/2 = 1/2$.
Next, 
\begin{align*}
    \Delta^2 a_0 = \Delta^1 a_0 - \Delta^1 a_1,
\end{align*}
so that if 
\begin{align*}
    \Delta^1 a_1 
    = \Delta^0 a_1 - \Delta^0 a_2
    = 1/2 - 1/3 
    = 1/6
\end{align*}
and
\begin{align*}
    \Delta^2 a_0 
    = 1/2 - 1/6
    = 1/3.
\end{align*}
Finally,
\begin{align*}
    \Delta^3 a_0
    = \Delta^2 a_0 - \Delta^2 a_1
    = \Delta^2 a_0 - [\Delta^1 a_1 - \Delta^1 a_2],
\end{align*}
where
\begin{align*}
    \Delta^1 a_2 
    = \Delta^0 a_2 - \Delta^0 a_3
    = 1/3 - 1/4
    = 1/12,
\end{align*}
giving us that
\begin{align*}
    \Delta^3 a_0 
    = 1/3 - [1/6 - 1/12]
    = 1/3 - 1/12
    = 1/4.
\end{align*}
Replacing these values in the formula for Euler's method gives us the desired result.
\end{proof} 
