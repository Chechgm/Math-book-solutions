\section{Reflexive spaces}


\begin{exercise}{3}
If a normed space $X$ is reflexive, show that $X'$ is reflexive.
\end{exercise}
\begin{proof}
We will begin with notation, since this exercise requires a lot of bookkeeping.
Let $x \in X, f \in X', g \in X''$, and $h \in X'''$.
Let $C: X \to X''$ and $C': X' \to X''$ the canonical mappings in their defined spaces;
that is, $Cx = g_x$ and $C'(f) = h_f$, where the elements on the right are evaluation functionals.

Lemma 4.6-7 tells us that a space is reflexive if its canonical map is surjective.
This gives us that, for all $g \in X''$, there exists an $x \in X$, such that $g = Cx = g_x$ (slightly overloading the notation by stating $g = g_x$).
To get the desired result we take an arbitrary $h \in X'''$ and we need to find an $f \in X'$ such that $h(g_x) = C'(f)(g_x)$, for all $x \in X$ (we consider $g_x$ given the reflexivity of $X$).
But $f$ is precisely the required element, as $C'(f)(g_x) = h_f(g_x) = g_x(f) = f(x)$.
Thus $X'$ is reflexive, as required.
\end{proof} 

\begin{exercise}{4}
Show that a Banach space $X$ is reflexive if and only if its dual space $X'$ is reflexive.
[Hint: it can be shown that a closed subspace of a reflexive Banach space is reflexive.
Use this fact without proving it].
\end{exercise}
\begin{proof}
We only need to prove the converse, since we proved the forward direction in the previous exercise.
Suppose then that $X'$ is reflexive.
By the forward condition proved in the previous exercise, we have that $X''$ is reflexive too.
We have that $C(X) = \cR(C)$ is a subspace of $X''$, where $C$ is the canonical map from $X$ to its second dual.
We have that $\cR(C)$ is itself a Banach space, because it is isomorphic to $X$, and hence it is closed.
By the result given in the exercise hint, we have that $\cR(C)$ is reflexive and so (by the same isomorphism), we have that $X$ itself is reflexive, giving us the desired result.
\end{proof} 

\begin{exercise}{6}
Show that different closed subspaces $Y_1$ and $Y_2$ of a normed space $X$ have different annihilators.
\end{exercise}
\begin{proof}
Let $x_0 \in Y_2 \subseteq X \setminus Y_1$ and and define $\delta = \inf_{y \in Y_1}\norm{x_0 - y} > 0$, the last inequality following because $Y_1$ is closed.
By Lemma 4.6-7, there exists $f \in X'$ such that for all $y \in Y_1$ it holds that $f(y) = 0$ but $f(x_0) = \delta$, so that $f$ is an annihilator of $Y_1$ but not of $Y_2$.
Using the same technique, mutatis mutandis, we can find an annihilator of $Y_2$, which is not an annihilator of $Y_1$ so that these two closed subspaces have different annihilators, as required.
\end{proof} 

\begin{exercise}{7}
Let $Y$ be a closed subspace of a normed space $X$ such that every $f \in X'$ which is zero everywhere on $Y$ is zero everywhere on the whole space $X$.
Show that then $Y = X$.
\end{exercise}
\begin{proof}
Suppose, for the sake of contradiction that $Y \neq X$.
Let $x_0 \in X \setminus Y$, and $\delta = \inf_{y\in Y} \norm{x-y} > 0$, because $Y$ is a closed subspace of $X$, so that $x$ is not a limit point of $Y$.
By Lemma 4.6-7, there exists an $f \in X'$ such that $\norm{f} = 1$, $f(y) = 0$ for all $y \in Y$, but critically, $f(x) = \delta$, which contradicts our hypothesis that $f(x) = 0$, as required.
\end{proof} 

\begin{exercise}{8}
Let $M$ be any subset of a normed space $X$.
Show that an $x_0 \in X$ is an element of $A = \overline{\vecspan(M)}$ if and only if $f(x_0) = 0$ for every $f \in X'$ such that $f|_M = 0$.
\end{exercise}
\begin{proof}
($\Rightarrow$)
Suppose $x_0 \in A$, then there exists a sequence of linear combinations in $M$, say $(y_n)$ such that $y_n \to x_0$.
But $f(y_n) = 0$ for all $y_n$, so that by the continuity of $f$, it must be the case that $f(x_0) = 0$.

($\Leftarrow$)
Suppose $f(x_0) = 0$ for every $f \in X'$ such that $f|_M = 0$.
If $x_0 \notin A$, we can use the same Lemma 4.6-7 as in exercise 7, mutatis mutandis, to produce a contradiction. 
Thus, $x_0$ must be in $A$.
\end{proof} 

\begin{exercise}{9 (Total set)}
Show that a subset $M$ of a normed space $X$ is total in $X$ if and only if every $f \in X'$ which is zero everywhere on $M$ is zero everywhere on $X$.
\end{exercise}
\begin{proof}
This is a Corollary to the previous exercise, where $A=X$.
\end{proof} 

\begin{exercise}{10}
Show that if a normed space $X$ has a linearly independent subset of $n$ elements, so does the dual space $X'$.
\end{exercise}
\begin{proof}
Let $x_1, \dots, x_n$ be linearly independent vectors in $X$.
Consider the subspace given by $Y = \vecspan(x_1, \dots, x_n)$ and $n$ linear functionals defined on $Y$ given by $f_i(x_j) = 1$ if $i=j$ and 0 otherwise.
Using 4.3-2, the Hahn-Banach Theorem for normed spaces, extend $f_i$ to a functional $\tilde{f_i}$ in $X$.

We will now prove that $\tilde{f_1}, \dots, \tilde{f_n}$ is linearly independent in $X'$.
Consider $(\sum a_i \tilde{f_i})(x) = 0$.
For $x=x_i$, we get that $a_i = 0$, and since this is true for all $i$, we get that $f_1, \dots, f_n$ are linearly independent, as required.
\end{proof} 
