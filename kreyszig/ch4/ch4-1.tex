\section{Zorn's Lemma}


\begin{exercise}{2}
Let $X$ be the set of all real-valued functions $x$ on the interval $[0,1]$, and let $x \leq y$ mean that $x(t) \leq y(t)$ for all $t \in [0,1]$.
Show that this defines a partial ordering.
Is it a total ordering?
Does $X$ have maximal elements?
\end{exercise}
\begin{proof}
We will first prove that $X$ is a partial ordering.

(PO1) Reflexivity: We have that $x(t) = x(t)$ so that $x \leq x$ for all $x \in X$.

(PO2) Antisymmetry: Suppose $x \leq y$ and $y \leq x$.
Then $x(t) \leq y(t)$ and $y(t) \leq x(t)$ for all $t$ and thus $x = y$.

(PO3) Transitivity: Suppose $x \leq y$ and $y \leq z$.
Then $x(t) \leq y(t)$ and $y(t) \leq z(t)$ so that $x(t) \leq y(t) \leq z(t)$ for all $t$ so that $x \leq z$.

$X$ is not a total order.
Consider the function $x(t)=1$ for $t \in [0, 1/2)$ and $x(t) = 0$ otherwise, and the function $y(t) = 0$ for $t \in [1/2,1]$ and $y(t) = 0$ otherwise.
The functions $x$ and $y$ are not comparable.

Notice that $X$ does not have any maximal elements.
To see this, suppose $m$ is a candidate maximal element, we have that the function $m'(t) = m(t) + 1 \geq m(t)$ for all $t$ and $m'(t) \neq m(t)$, so that $m(t)$ is not a maximal element.
\end{proof} 

\begin{exercise}{5}
Prove that a finite partially ordered set $A$ has at least one maximal element.
\end{exercise}
\begin{proof}
We will give a procedure to find such element.
Take any element $a \in A$, and try to compare it to the rest of the elements in $A$.
If we cannot compare $a$ to any other element, then $a$ is maximal, as $a \leq a$ implies $a = a$.
Likewise, if $a \geq x$ for all $x$, then $a$ is maximal.
If neither of this two is the case, consider the subset, $A'$, given by all of the elements of $A$ comparable with $a$.
If all of the elements of $A$ are comparable with $a$, by transitivity there must be a maximal element.
If not all the elements of $A$ are comparable with $a$, then $A'$ is a proper subset of $A$ and we can repeat the process until we find a maximal element.
\end{proof} 

\begin{exercise}{6 (Least element, greatest element)}
Show that a partially ordered set can have at most one element $a$ such that $a \leq x$ for all $x \in M$ and at most one element $b$ such that $x \leq b$ for all $x \in M$.
[If such an $a$ (or $b$) exists, it is called the least element (greatest element, respectively) of $M$].
\end{exercise}
\begin{proof}
Suppose there are two such elements, say $a, a'$, then $a \leq a'$ and $a' \leq a$, in which case by antisymmetry we would have $a = a'$.
The proof for $b$ follows using the same technique mutatis mutandis.
\end{proof} 
