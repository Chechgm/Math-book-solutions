\section{Convergence of sequences of operators and functionals}

1
2
3
4
5
6 This is a nice exercise motivating the word uniform and showing it actually is the uniform convergence we're used to 
7
8
9
10 This is kind of an alternative to something like compactness or the Bolzano-Weierstrass theorem

\begin{exercise}{1}
Show that uniform operator convergence $T_n \to T, T_n \in B(X,Y)$, implies strong operator convergence with the same limit $T$.
\end{exercise}
\begin{proof}
fill
\end{proof} 

\begin{exercise}{2}
If $S_n, T_n \in B(X,Y)$ and $(S_n)$ and $(T_n)$ are strongly operator convergent with limits $S$ and $T$, show that $(S_n + T_n)$ is strongly operator convergent with the limit $(S+T)$.
\end{exercise}
\begin{proof}
fill
\end{proof} 

\begin{exercise}{3}
Show that strong operator convergence in $B(X,Y)$ implies weak operator convergence with the same limit.
\end{exercise}
\begin{proof}
fill
\end{proof} 

\begin{exercise}{4}
Show that weak operator
\end{exercise}
\begin{proof}
fill
\end{proof} 

\begin{exercise}{5}
Strong operator convergence does not imply uniform operator convergence.
Illustrate this by considering $T_n = f_n: l^1 \to \R$, where $f_n(x) = x_n$ and $x = (x_n)$.
\end{exercise}
\begin{proof}
fill
\end{proof} 

\begin{exercise}{6}
fill
\end{exercise}
\begin{proof}
fill
\end{proof} 

\begin{exercise}{7}
fill
\end{exercise}
\begin{proof}
fill
\end{proof} 

\begin{exercise}{8}
fill
\end{exercise}
\begin{proof}
fill
\end{proof} 

\begin{exercise}{9}
fill
\end{exercise}
\begin{proof}
fill
\end{proof} 

\begin{exercise}{10}
fill
\end{exercise}
\begin{proof}
fill
\end{proof} 
