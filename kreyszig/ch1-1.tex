\subsection{Metric space}

1 This is of course the most important metric space there is
2 Not so useful counterexample of a metric on R
4 There's infinitely many, but they're all kind of the "same"
5 Shows we can always multiply a metric with a number
6 Very important metric which ends up encoding uniform convergence 
8 Another very useful example of a metric leading eventually to the $L^1$ spaces 
9 The discrete metric is not so important except for counterexamples
11 Useful consequence of the triangle inequality
12 Useful consequence of the triangle inequality
13 Useful consequence of the triangle inequality
15 This one is pretty neat to see how some of the axioms are actually superfluous. Cool but not so important

\begin{exercise}{1}
Show that the real line is a metric space
\end{exercise}
\begin{proof}
fill
\end{proof}

\begin{exercise}{2}
Does $d(x,y)=(x-y)^2$ define a metric on the set of all real numbers?
\end{exercise}
\begin{proof}
fill
\end{proof}

\begin{exercise}{4}
Find all metrics on a set $X$ consisting of two points. Consisting of one point.
\end{exercise}
\begin{proof}
fill
\end{proof}

\begin{exercise}{5}
Let $d$ be a metric of $X$. Determine all constants $k$ such that (i) $kd$, (ii) $d+k$ is a metric on $X$
\end{exercise}
\begin{proof}
fill
\end{proof}

\begin{exercise}{6}
Show that $d$ in 1.1-6 satisfies the triangle inequality.
\end{exercise}
\begin{proof}
fill
\end{proof}

\begin{exercise}{8}
Show that another metric $\tilde{d}$ on the set $X$ in 1.1-7 is defined by $\tilde{d}(x,y)=\int_a^b\absoluteValue{x(t)-y(t)}dt$.
\end{exercise}
\begin{proof}
fill
\end{proof}

\begin{exercise}{9}
Show that $d$ in 1.1-8 is a metric.
\end{exercise}
\begin{proof}
fill
\end{proof}

\begin{exercise}{11}
Prove (1).
\end{exercise}
\begin{proof}
fill
\end{proof}

\begin{exercise}{12 (Triangle inequality)}
The triangle inequality has several useful consequences. For instance, using (1), show that $\absoluteValue{d(x,y)-d(z,w)}\leq d(x,z)+d(y,w)$
\end{exercise}
\begin{proof}
fill
\end{proof}

\begin{exercise}{13}
Using the triangle inequality show that $\absoluteValue{d(x,z)-d(y,z)}\leq d(x,y)$
\end{exercise}
\begin{proof}
fill
\end{proof}

\begin{exercise}{15}
Show that nonnegativity of a metric follows from (M2) to (M4).
\end{exercise}
\begin{proof}
fill
\end{proof}
