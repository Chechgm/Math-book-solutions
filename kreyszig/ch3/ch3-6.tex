\section{Total orthonormal sets and sequences}

1 The answer is yes for finite dimensions, but you should find a counterexample in infinite dimensions. This makes finite linear combinations not particularly useful in functional analysis
2 So all the different notions of dimension all equal eachother in finite dimensions. Things are much more complicated in infinite dimensions. 
3
4 Another equivalent form of the Parseval theorem, this time dealing with inner product instead of norm.
5 So we see that total <-> Parseval theorem <-> Parseval relation
6
7 So any separable Hilbert space admits an orthonormal basis, and it can be constructed explicitly, on the other hand, if the Hilbert space is not separable, the axiom of choice becomes a necessary tool. The orthonormal basis cannot be given explicitly anymore
8 A bit of a generalization, saying that an orthonormal sequence can be completed to an orthonormal basis. This holds in general Hilbert spaces without separable, but the separable case allows for an explicit basis
9 Useful to show when two vectors are equal.

\begin{exercise}{1}
fill
\end{exercise}
\begin{proof}
fill
\end{proof} 

\begin{exercise}{2}
fill
\end{exercise}
\begin{proof}
fill
\end{proof} 

\begin{exercise}{3}
fill
\end{exercise}
\begin{proof}
fill
\end{proof} 

\begin{exercise}{4}
fill
\end{exercise}
\begin{proof}
fill
\end{proof} 

\begin{exercise}{5}
fill
\end{exercise}
\begin{proof}
fill
\end{proof} 

\begin{exercise}{6}
fill
\end{exercise}
\begin{proof}
fill
\end{proof} 

\begin{exercise}{7}
fill
\end{exercise}
\begin{proof}
fill
\end{proof} 

\begin{exercise}{8}
fill
\end{exercise}
\begin{proof}
fill
\end{proof} 

\begin{exercise}{9}
fill
\end{exercise}
\begin{proof}
fill
\end{proof} 
