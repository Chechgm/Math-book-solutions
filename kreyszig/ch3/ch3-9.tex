\section{Hilbert-adjoint operator}


\begin{exercise}{1}
Show that $0^\ast = 0$ and $I^\ast = I$.
\end{exercise}
\begin{proof}
We have 
\begin{align*}
    \brackets{I^\ast y, x} = \brackets{y, Ix} = \brackets{y,x} =  \brackets{Iy, x},
\end{align*}
so that $\brackets{(I^\ast - I)y, x} = 0$.
We can then apply Lemma 3.9-3.a with $Q = I^\ast - I$ to get that $I^\ast = I$.

Furthermore, 
\begin{align*}
    \brackets{0^\ast y, x} = \brackets{y, 0x} = \brackets{y, 0} = 0,
\end{align*}
so that we can apply Lemma 3.9-3.a with $Q = 0^\ast$, concluding that $0^\ast = 0$.
\end{proof} 

\begin{exercise}{2}
Let $H$ be a Hilbert space and $T:H \to H$a bijective bounded linear operator whose inverse is bounded.
Show that $(T^\ast)^{-1}$ exists and $(T^\ast)^{-1} = (T^{-1})^\ast$.
\end{exercise}
\begin{proof}
We will follow a slightly different approach to what is suggested in the exercise.
We will prove that $(T^{-1})^\ast$ exists and that it is equal to $(T^\ast)^{-1}$, which guarantees the existence of the latter.
Since we assume $(T^{-1})^\ast$ is bounded, then Theorem 3.9-2 guarantees the existence of $(T^{-1})^\ast$.
We have $I = I^\ast = (TT^{-1})^\ast = (T^{-1})^\ast T^\ast$, multiplying both sides of the equality by $(T^\ast)^{-1}$ gives us $(T^\ast)^{-1} = (T^{-1})^\ast$.
Likewise, $I = I^\ast = (T^{-1}T)^\ast = T^\ast (T^{-1})^\ast$, so that multiplying both sides of the equality by $(T^\ast)^{-1}$, gives us $(T^\ast)^{-1} = (T^{-1})^\ast$.
This means $T^\ast$ is invertible with $(T^\ast)^{-1} = (T^{-1})^\ast$.
\end{proof} 

\begin{exercise}{3}
If $(T_n)$ is a sequence of bounded linear operators on a Hilbert space and $T_n \to T$, show that $T^\ast_n \to T^\ast$.
\end{exercise}
\begin{proof}
Consider the function $f: B(H, H) \to B(H,H)$, given by $f(T) = T^\ast$.
We will prove $f$ is continuous, from which we can conclude the exercise as a Corollary.
Let $S, T\in B(H,H)$.
We have $\norm{S^\ast - T^\ast} = \norm{(S-T)^\ast} = \norm{S-T}$, where the last equality follows from Theorem 3.9-2. 
Thus, by taking $\norm{S-T} < \epsilon$, we have $\norm{S^\ast - T^\ast} < \epsilon$.
As a Corollary, we have $\lim f(T_n) = \lim T^\ast_n = f(\lim T_n) = T^\ast$.
\end{proof} 

\begin{exercise}{4}
Let $H_1$ and $H_2$ be Hilbert spaces and $T:H_1 \to H_2$ be a bounded linear operator.
If $M_1 \subseteq H_1$ and $M_2 \subseteq H_2$ are such that $T(M_1) \subseteq M_2$, show that $T^\ast(M_2^\perp) \subseteq M_1^\perp$.
\end{exercise}
\begin{proof}
Let $y \in T^\ast(M_2^\perp)$, so that there exists an $x \in M^\perp_2$ with $y = T^\ast(x)$.
This implies $\brackets{x,v} = 0$ for all $v \in M_2$.
Furthermore, since $T(M_1) \subseteq M_2$, we have that for all $x' \in M_1$, $Tx_1 \in M_2$.
Thus, we have that $0 = \brackets{x, Tx'} = \brackets{T^\ast x, x'}$, so that $T^\ast x \perp M$ and $T^\ast(M_2^\perp) \subseteq M_1^\perp$, as required.
\end{proof} 

\begin{exercise}{5}
Let $M_1$ and $M_2$ in exercise 4 be closed subspaces.
Show that then $T(M_1) \subseteq M_2$ if and only if $T^\ast(M_2^\perp) \subseteq M_1^\perp$.
\end{exercise}
\begin{proof}
The forward direction is the content of exercise 4.
To see the converse holds, let $T^\ast(M_2^\perp) \subseteq M_1^\perp$, then $(T^\ast)^\ast((M_1^\perp)^\perp) \subseteq (M_2^\perp)^\perp$, by applying the result of exercise 4 to the hypothesis.
Now $(T^\ast)^\ast = T$ and $(M_i^\perp)^\perp = M_i$, since $M_i$ is a closed subspace (Lemma 3.3-6).
Putting these results together, we have $T(M_1) \subseteq M_2$, as required.
\end{proof} 

\begin{exercise}{6}
If $M_1 = \cN(T) = \set{x: Tx = 0}$ in exercise 4, show that
\begin{enumerate}
    \item $T^\ast(H_2) \subseteq M_1^\perp$,
    \item $[T(H_1)]^\perp \subseteq \cN(T^\ast)$,
    \item $M_1 = [T^\ast(H_2)]^\perp$.
\end{enumerate}
\end{exercise}
\begin{proof}
\begin{enumerate}
    \item Suppose $T(M_1) \subseteq M_2$.
    Let $y \in T^\ast(H_2)$ so that there exists $x \in H_2$ with $y = T^\ast x$.
    For any $x' \in M_1$ we have by definition $Tx' = 0 \in M_2 \subseteq H_2$.
    It is the case that $y \in M_1^\perp$.
    To see this, notice
    \begin{align*}
        \brackets{x', y} = \brackets{x', T^\ast x} = \brackets{Tx', x} = \brackets{0, x} = 0,
    \end{align*}
    as required.
    \item Suppose $x \in [T(H_1)]^\perp$.
    Thus, for all $y \in T(H_1)$, we have $\brackets{x,y} = 0$.
    For all $y \in T(H_1)$, there exists $x' \in H_1$ with $y = Tx'$.
    Putting these statements together, we have $0 = \brackets{x, Tx'} = \brackets{T^\ast x, x'}$.
    Since $0 = \brackets{T^\ast x, x'}$ is also true for all $x \in \cN(T^\ast)$, then $x \in \cN(T^\ast)$.
    \item In the first point we proved that $T^\ast(H_2) \subseteq M_1^\perp$.
    Using exercise 3.3.7.b, we conclude that $M_1 = (M_1^\perp)^\perp \subseteq [T^\ast(H_2)]^\perp$.
    To prove the containment in the other direction, suppose $x \in [T^\ast(H_2)]^\perp$, so that for all $y \in T^\ast(H_2)$, it holds that $\brackets{x, y} = 0$.
    Now, there exists $x' \in H_2$ such that $y = T^\ast x'$.
    Putting these together, we have $0 = \brackets{x, T^\ast x'} = \brackets{Tx,  x'}$.
    Since $\brackets{Tx,  x'} = 0$ also holds for all $x \in \cN(T)$, then $x \in M_1$, as required.
\end{enumerate}
\end{proof} 

\begin{exercise}{7}
Let $T_1$ and $T_2$ be bounded linear operators on a complex Hilbert space $H$ into itself.
If $\brackets{T_1 x, x} = \brackets{T_2 x, x}$ for all $x \in H$, show that $T_1 = T_2$.
\end{exercise}
\begin{proof}
We have that $\brackets{(T_1 - T_2)x, x} = 0$ for all $x$.
By Lemma 3.9-3.b, with $Q = T_1 - T_2$, we have that $T_1 = T_2$.
\end{proof} 

\begin{exercise}{9}
Show that a bounded linear operator $T: H \to H$ on a Hilbert space $H$ has a finite dimensional range if and only if $T$ can be represented in the form $Tx = \sum^n \brackets{x, v_j}w_j$, where $v_j, w_j \in H$.
\end{exercise}
\begin{proof}
($\Rightarrow$)
Let $w_1,\dots, w_n$ be an orthonormal basis for the range of $T$.
Let $x\in H$ and $Tx = \sum a_iw_i$.
Then $\brackets{Tx, w_i} = a_i =\brackets{x, T^\ast w_i}$.
Thus we get the desired form by replacing $v_i = T^\ast w_i$.

($\Leftarrow$)
Assuming $v_j, w_j \in H$ are fixed, then we can think of $w_j$ as a basis of the range of $T$ and $\brackets{x, v_j}$ as the coefficients of $X$ corresponding to each basis element.
\end{proof} 

\begin{exercise}{10 (Right shift operator)}
Let $(e_n)$ be a total orthonormal sequence in a separable Hilbert space $H$ and define the right shift operator to be the linear operator $T: H \to H$ such that $Te_n = e_{n+1}$ for $n = 1,2,\dots$.
Explain the name.
Find the range, null space, norm and Hilbert-adjoint operator of $T$.
\end{exercise}
\begin{proof}
The name is self-explanatory. 
The range of $T$ is $\vecspan{(e_2, e_3, \dots)}$.
The null space is 0.
To find the norm of $T$, notice that $\norm{Tx} = \norm{x}$, so that $\norm{T} = 1$.
The self-adjoint of $T$ is the left shift.
Let $Se_1 = 0$ and for all $e_i$ with $i>1$, $Se_i = e_{i-1}$.
Let $x, y \in H$, so that $x = \sum \brackets{x, e_i}e_i$ and $y = \sum \brackets{y, e_i}e_i$.
We have 
\begin{align*}
    \brackets{Tx, y}
    =& \brackets{T(\sum \brackets{x, e_i}e_)i, \sum \brackets{y,e_j}e_j}\\
    =& \brackets{\sum \brackets{x, e_i}Te_i, \sum \brackets{y,e_j}e_j}\\
    =& \brackets{\sum \brackets{x, e_i}e_{i+1}, \sum \brackets{y,e_j}e_j}\\
    =& \sum_i \sum_j \brackets{x, e_i} \brackets{y,e_j} \brackets{e_{i+1}, e_j}\\
    =& \sum_{i=2} \brackets{x, e_i} \brackets{y,e_{j-1}},
\end{align*}
the last equality following from the orthonormality of $(e_n)$.
Likewise,
\begin{align*}
    \brackets{x, Sy}
    =& \brackets{\sum \brackets{x, e_i}e_i, S(\sum \brackets{y,e_j}e_j)}\\
    =& \brackets{\sum \brackets{x, e_i}e_i, \sum \brackets{y,e_j}Se_j}\\
    =& \brackets{\sum \brackets{x, e_i}e_i, \sum \brackets{y,e_j}e_{j-1}}\\
    =& \sum_i \sum_j \brackets{x, e_i} \brackets{y,e_j} \brackets{e_{i+1}, e_{j-1}}\\
    =& \sum_{i=2} \brackets{x, e_i} \brackets{y,e_{j-1}},
\end{align*}
so that $S = T^\ast$.
\end{proof} 
