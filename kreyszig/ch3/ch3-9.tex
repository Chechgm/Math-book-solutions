\section{Hilbert-adjoint operator}

1 Easy property but obviously important
2 One thing you should keep in mind is that the hilbert adjoint is very close to complex conjugation. That is if a+bi is a complex number, we can take the complex conjugate a-bi. The Hilbert adjoint kind of mimicks this (and using C* algebras, this analogy can be made rigorously). So compare this property with the same property in complex numbers!
3 Likewise, in complex numbers this says that if zn->z then the conjugates converge too, which is just saying that the complex conjugate is continuous
4 
5
6 These three problems kind of show how the adjoint relates to orthogonal complement 
7 Interesting way to show when two linear maps are equal. We have done many such properties in the past 
9 Operators with finite dimensional range will be rather important in the future, not for their own sake though, but because they form a good approximation for important classes of operators 
10 Fun application of the theory

\begin{exercise}{1}
fill
\end{exercise}
\begin{proof}
fill
\end{proof} 

\begin{exercise}{2}
fill
\end{exercise}
\begin{proof}
fill
\end{proof} 

\begin{exercise}{3}
fill
\end{exercise}
\begin{proof}
fill
\end{proof} 

\begin{exercise}{4}
fill
\end{exercise}
\begin{proof}
fill
\end{proof} 

\begin{exercise}{5}
fill
\end{exercise}
\begin{proof}
fill
\end{proof} 

\begin{exercise}{6}
fill
\end{exercise}
\begin{proof}
fill
\end{proof} 

\begin{exercise}{7}
fill
\end{exercise}
\begin{proof}
fill
\end{proof} 

\begin{exercise}{9}
fill
\end{exercise}
\begin{proof}
fill
\end{proof} 

\begin{exercise}{10}
fill
\end{exercise}
\begin{proof}
fill
\end{proof} 
