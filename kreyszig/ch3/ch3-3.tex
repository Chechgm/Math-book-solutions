\section{Orthogonal complements and direct sums}


\begin{exercise}{1}
Let $H$ be a Hilbert space, $M \subseteq H$ a convex subset, and $(x_n)$ a sequence in $M$ such that $\norm{x_n} \to d$, where $d = \inf_{x \in M} \norm{x}$.
Show that $(x_n)$ converges in $H$.
Give an illustrative example in $\R^2$ or $\R^3$.
\end{exercise}
\begin{proof}
We will follow a strategy similar to Theorem 3.3-1.
To do that, we will prove $(x_n)$ is Cauchy, and then it will follow by the completeness of $H$, that $(x_n)$ converges in $H$.
Let $\norm{x_n} = d_n$.
We have that $\norm{x_n + x_m} = 2\norm{1/2(x_n + x_m)} \geq 2d$, because $x_n/2 + x_m/2$ is in $M$ due to its convexity.
Now by the parallelogram equality, we have that
\begin{align*}
    \norm{x_n - x_m} 
    =& -\norm{x_n + x_m}^2 + 2(\norm{x_n} + \norm{x_m})\\
    \leq& -(2d)^2 + 2(d_n + d_m) \to 0,
\end{align*}
where the convergence to 0 follows from the fact that $d_n,d_m \to d$.
Thus $(x_n)$ is Cauchy and by the completeness of $H$ it must converge to a point in $H$.
\end{proof}

\begin{exercise}{2}
Show that the subset $M=\set{y=(y_i): \sum y_i=1)}$ of complex space $\C^n$ (see 3.1-4) is complete and convex.
Find the vector of minimum norm in $M$.
\end{exercise}
\begin{proof}
(Complete)
Let $(x^n)$ be a sequence (of sequences) in $M$, so that for each sequence, $\sum_i x^n_i=1$.
Assume $x^n\to x$, so that for every $\epsilon>0$, we can find an $n\in\N$ such that $\norm{x^n-x}_1<\epsilon$.
We have
\begin{align*}
    \absoluteValue{1-\sum_i x_i}
    =& \absoluteValue{\sum_i x^n_i - \sum_i x_i}\\
    =& \absoluteValue{\sum_i x^n_i - x_i}\\
    \leq& \sum_i\absoluteValue{x^n_i - x_i}
    <\epsilon.
\end{align*}
That is, $x\in M$
Theorem 1.4-6 states that a closed subset of a complete space is complete.
Since $\C^n$ is complete, and $M$ is closed, then $M$ is complete.

(Convex)
For $x,y\in M$ and $a\in[0,1]$, we have
\begin{align*}
    \sum_i (ax_i+(1-a)y_i)
    =& \sum_i ax_i + \sum_i (1-a)y_i\\
    =& a\sum_i x_i + (1-a)\sum_i y_i\\
    =& a + (1-a) = 1,
\end{align*}
so that $ax+(1-a)y\in M$.

(Vector of minimum norm)
The norm of a vector $x$ in $\C^n$ is $\sum_i x_i\bar{x}_i$.
The vector that minimises such norm is $x_i=1/n$ for all $i$.
\end{proof}

\begin{exercise}{3}
\begin{enumerate}
    \item Show that the vector space $X$ of all real-valued continuous functions on $[-1,1]$ is the direct sum of of the set of all even continuous functions and the set of all odd functions on $[-1,1]$.
    \item Give examples of representations of $\R^3$ as a direct sum (i) of a subspace and its orthogonal complement, (ii) of any complementary pair of subspaces.
\end{enumerate}
\end{exercise}
\begin{proof}
\begin{enumerate}
    \item Recall that a function is even if $f(-x) = f(x)$ and odd if $f(-x) = -f(x)$.
    Let $f:[-1,1] \to \R$ be continuous.
    Let $o(x) = 1/2(f(x) - f(-x))$ and $e(x) = 1/2(f(x)+f(-x))$.
    We can verify using the above definitions that $o(x)$ is odd and $e(x)$ is even.
    We have 
    \begin{align*}
        f(x)
        = o(x) + e(x)
        = 1/2(f(x) - f(-x)) + 1/2(f(x) + f(-x))
        = f(x),
    \end{align*}
    so that an arbitrary continuous function in $[-1,1]$ is the sum of an even and an odd function.
    To finish the proof that the sum is direct, we prove that the intersection between the set of odd functions and even functions is 0.
    So let $f$ be odd and even.
    We then have that $f(-x) = f(x)$ and $f(-x) = -f(x)$, so that $f(x) = - f(x)$, so that $f(x) = 0$ for all $x$.
    \item $\R^3$ can be written as the direct sum of $\vecspan{(e_1,e_2)}$ and $\vecspan{(e_3)}$, these subspaces are orthogonal to each other.
    Furthermore, consider the subspaces given by $\vecspan{((1,1,1), (1,2,3))}$ and $\vecspan{((3,1,1))}$, these subspaces are not orthogonal, but their sum is $\R^3$ given that they are not linearly independent, and their dimension equals that of the dimension of $\R^3$.
\end{enumerate}
\end{proof}

\begin{exercise}{5}
Let $X=\R^2$.
find $M^\perp$ if $M$ is
\begin{enumerate}
    \item $\set{x}$, where $x = (x_1,x_2) \neq 0$,
    \item a linearly independent set $\set{x_1,x_2}\subseteq X$.
\end{enumerate}
\end{exercise}
\begin{proof}
\begin{enumerate}
    \item We can use the Gram-Schmidt process to find the vector orthogonal to $x$.
    In general, take any $y = (y_1,y_2)\in\R^2$.
    Then the vector
    \begin{align*}
        x' = y -\text{proj}_x(y) = y - \frac{\brackets{x,y}}{\brackets{y,y}}y
    \end{align*}
    is orthogonal to $x$.
    So that $M^\perp = \vecspan{(x)}$.
    \item Since the set is linearly independent, its span constitutes a basis of $\R^2$. 
    Thus, the only vector orthogonal to both of them simultaneously is 0.
\end{enumerate}
\end{proof}

\begin{exercise}{6}
\begin{enumerate}
    \item Show that $Y = \set{x \mid x = (x_n) \in l^2, x_{2n}=0, n\in\N}$ is a closed subspace of $l^2$ and find $Y^\perp$.
    \item What is $Y^\perp$ if $y = \vecspan{e_1, e_2, \dots, e_n} \subseteq l^2$, where $e_j = (\delta_{jk})$
\end{enumerate}
\end{exercise}
\begin{proof}
\begin{enumerate}
    \item To see that $Y$ is closed, consider any convergent sequence in $Y$, $(x_n^k) \to (x_n)$. 
    We have that for all $k$, $x_{2n}^k = 0$ for all $n\in\N$.
    Since convergence in $l^2$ implies convergence of each one of the elements of the sequence, it must be the case that $x_{2n} = 0$ for all $n\in\N$.
    Thus $(x_n) \in Y$ and $Y$ is closed.
    
    We have that $Y^\perp = \set{x \mid x = (x_n) \in l^2, x_{2n+1}=0, n\in\N}$.
    We can see this from two perspectives.
    The first is that $l^2 = Y \oplus Y^\perp$.
    The second is that any element from $Y^\perp$ (and only from $Y^\perp$) has zero inner product with $Y$.
    \item As in the previous numeral, $Y^\perp$ is $\overline{\vecspan{(e_{n+1}, e_{n+2},\dots)}}$ given that the inner product between $Y$ and $Y^\perp$ is 0.
\end{enumerate}
\end{proof}

\begin{exercise}{7}
Let $A$ and $B$ with $A \subseteq B$ be nonempty subsets of an inner product space $X$.
Show that
\begin{enumerate}
    \item $A \subseteq A^{\perp \perp}$.
    \item $B^\perp \subseteq A^\perp$.
    \item $A^{\perp \perp \perp} = A^\perp$.
\end{enumerate}
\end{exercise}
\begin{proof}
\begin{enumerate}
    \item Let $a \in A$, we have that $\brackets{a,a'} = 0$ for all $a'\in A^\perp$.
    We also have that $\brackets{a', a''} = 0$ for all $a'' \in A^{\perp \perp}$, that is, $a \in A^{\perp \perp}$ so that $A \subseteq A^{\perp \perp}$.
    \item Let $b' \in B^\perp$.
    We have that $\brackets{b', b} = 0$ for all $b \in B$, but this implies $\brackets{b', a} = 0$ for all $a \in A$, because $A \subseteq B$;
    that is, $b' \in A^\perp$, so that $B^\perp \subseteq A^\perp$.
    \item To prove this, we will use a combination of the previous results.
    
    ($\subseteq$)
    We proved that $A \subseteq A^{\perp \perp}$.
    Using this in conjunction with the second result, we have that $(A^{\perp \perp})^\perp \subseteq A^\perp$.

    ($\supseteq$)
    We proved that $A \subseteq A^{\perp \perp}$, replacing $A$ for $A^\perp$, we have that $A^\perp \subseteq (A^\perp)^{\perp \perp}$.
\end{enumerate}
\end{proof}

\begin{exercise}{8}
Show that the annihilator $M^\perp$ of a set $M\neq \emptyset$ in an inner product space $X$ is a closed subspace of $X$.
\end{exercise}
\begin{proof}
Let $(y_n)$ be a sequence in $M^\perp$ so that $y_n \to y$.
We want to prove $y\in M^\perp$
For all $n$ we have that $\brackets{y_n, x} = 0$ for all $x \in M$.
Hence, $\brackets{y,x} = \brackets{\lim y_n, x} = \lim \brackets{y_n, x} = 0$, for all $x \in M$, where the second equality follows from the continuity of the inner product (Lemma 3.2-2).
Thus $M^\perp$ is closed.
\end{proof}

\begin{exercise}{9}
Show that a subspace $Y$ of a Hilbert space $H$ is closed in $H$ if and only if $Y = Y^{\perp \perp}$.
\end{exercise}
\begin{proof}
The forward direction is the content of Lemma 3.3-6.
To prove the converse, suppose $Y = Y^{\perp \perp}$.
From exercise 8, we know that $Y^{\perp \perp}$ is a closed subspace of $H$ because $Y^{\perp \perp}$ is the annihilator of $Y^\perp$.
Thus, by the hypothesis, $Y$ is closed too, as desired.
\end{proof}

\begin{exercise}{10}
If $M \neq \emptyset$ is any subset of a Hilbert space $H$, show that $M^{\perp \perp}$ is the smallest closed subspace of $H$ which contains $M$, that is, $M^{\perp \perp}$ is contained in any closed subspace $Y \subseteq H$ such that $M \subseteq Y$.
\end{exercise}
\begin{proof}
Notice that by exercise 8, $M^{\perp \perp}$ is closed in $H$.
By 3.2-4, $M^{\perp \perp}$ is  complete, so that $M^{\perp \perp}$ is itself a Hilbert space, with $M \subseteq M^{\perp \perp}$.
We will prove that $M$ is dense in $M^{\perp \perp}$.
To do this, notice that by definition $M^\perp \perp M^{\perp \perp}$, so that the only element of $M^\perp$ in $M^{\perp \perp}$ is $0$;
that is $M^\perp = \set{0}$ as a subset of the Hilbert space $M^{\perp \perp}$.
Thus, by exercise 9, $M$ is dense in $M^{\perp \perp}$, which implies that any closed subspace containing $M^{\perp \perp}$ must also contain $M$ itself.
\end{proof}