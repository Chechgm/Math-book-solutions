\section{Metric space}


\begin{exercise}{1}
Show that the real line is a metric space.
\end{exercise}
\begin{proof}
We have that $d(x,y)=\absoluteValue{x-y}$ for $x,y,z\in\R$.

(M1): Certainly $d(x,y)$ is real valued, finite and nonnegative for all $x,y\in\R$.

(M2): We have $x=y$ if and only if $d(x,y)=\absoluteValue{x-y}=\absoluteValue{0}=0$.

(M3): Suppose, without loss of generality, that $x<y$. We have $d(x,y) =\absoluteValue{x-y} =-(x-y) =y-x$ and $d(y,x) =\absoluteValue{y-x} =y-x$, so that $d(x,y)=d(y,x)$.

(M4): We have $\absoluteValue{x-y} =\absoluteValue{(x-z)+(z-y)}\leq \absoluteValue{\absoluteValue{(x-z)}+\absoluteValue{(z-y)}} =\absoluteValue{(x-z)}+\absoluteValue{(z-y)}$. the inequality follows from the fact that $x-y \leq\absoluteValue{x-y}$ for all $x,y\in\R$, and the last equality follows from the fact that $\absoluteValue{(x-z)}, \absoluteValue{(z-y)}$ are both positive, so the absolute value of their sum is simply their sum.
\end{proof}

\begin{exercise}{2}
Does $d(x,y)=(x-y)^2$ define a metric on the set of all real numbers?
\end{exercise}
\begin{proof}
Although this metric candidate fulfills M1, M2 and M3, it doesn't fulfill M4. To see this, consider the points $x=1,y=2,z=1.5$. We have $d(x,y)=1$, $d(x,z)=0.25$ and $d(z,y)=0.25$, so that $1=d(x,y)\not\leq d(x,z)+d(z,y) =0.5$.
\end{proof}

\begin{exercise}{4}
Find all metrics on a set $X$ consisting of two points. Consisting of one point.
\end{exercise}
\begin{proof}
Let $d$ be the metric in both cases.

One point: Let $x\in X$. From M2, we must have that $d(x,x) =0$ so that the metric is the "zero" metric.

Two points: Let $x,y\in X$. From M1 we must have that $d$ is real valued, finite and nonnegative. Then $d(x,y)=c$ for some $c\in\R$ with $c>0$ and 0 otherwise.
\end{proof}

\begin{exercise}{5}
Let $d$ be a metric of $X$. Determine all constants $k$ such that (i) $kd$, (ii) $d+k$ is a metric on $X$
\end{exercise}
\begin{proof}
(i) Notice that if $k$ is 0 or negative it would violate M1. If $k>0$, then M1, M2 and M3 hold. To see M4 also holds, consider $x,y,z\in X$. Then we have that $d(x,y)\leq d(x,z)+d(z,y)$. Multiplying both sides of the inequality by $k$ we obtain $kd(x,y)\leq kd(x,z)+kd(z,y)$, as required.

(ii) If $k\neq 0$, then M2 doesn't hold because $d(x,x)+k=k$ but in a metric it should be 0. Hence we cannot add a real number to a metric.
\end{proof}

\begin{exercise}{6}
Show that $d$ in 1.1-6 satisfies the triangle inequality. Let $x=(x_1,x_2,\dots)$ be a bounded complex sequence, so that there exists a $c\in\R$ with $\absoluteValue{x_j}\leq c$ for all $j$. Let $x,y\in X$, and define $d(x,y)=\sup_{j\in\N}\absoluteValue{x_j-y_j}$.
\end{exercise}
\begin{proof}
We have $\absoluteValue{x_j-y_j} \leq \absoluteValue{x_j-z_j}+\absoluteValue{z_j-y_j}$. Since this holds for all $j$, we can take the supremum on both sides of the inequality to obtain $\sup_{j\in\N}\absoluteValue{x_j-y_j} \leq \sup_{j\in\N}(\absoluteValue{x_j-z_j}+\absoluteValue{z_j-y_j})$. Notice, however, that\\ $\absoluteValue{x_j-z_j}+\absoluteValue{z_j-y_j}\leq \sup_{j\in\N}\absoluteValue{x_j-z_j}+\sup_{j\in\N}\absoluteValue{z_j-y_j}$ for all $j$,\\ so that $\sup_{j\in\N}\absoluteValue{x_j-z_j}+\sup_{j\in\N}\absoluteValue{z_j-y_j}$ is an upper bound of $\absoluteValue{x_j-z_j}+\absoluteValue{z_j-y_j}$ and, by definition, greater than or equal to its supremum.
\end{proof}

\begin{exercise}{8}
Show that another metric $\tilde{d}$ on the set $X$ in 1.1-7 is defined by $\tilde{d}(f,g)=\int_a^b\absoluteValue{f(t)-g(t)}dt$.
\end{exercise}
\begin{proof}
Let $f,g,h\in C[a,b]$.

(M1): The integral of a nonnegative function ($\absoluteValue{f(x)-g(x)}$, in this case) is nonnegative. Integrals are real valued and because $f$ and $g$ are continuous, the image of a compact set (in this case $[a,b]$) is compact (Theorem 4.4.1. in Abbott's understanding analysis), so that the integral of $\absoluteValue{f(x)-g(x)}$ in $[a,b]$ is finite.

(M2): If $f=g$, clearly $d(f,g)=0$, on the other hand, if $\int_a^b\absoluteValue{f(x)-g(x)}dx =0$, it means that $f-g=0$. To see this, remember that $f-g$ is a continuous function, and that if there exists an $x\in [a,b]$ with $f(x)-g(x)>0$, then $f(x)-g(x)>0$ in an open interval around $x$ (see Carothers exercise 46), and then the integral would not be 0. Hence, $f=g$.

(M3): This follows from the symmetry of subtraction under absolute value.

(M4): We have 
\begin{align*}
    d(f,g) =& \int_a^b\absoluteValue{f(x)-g(x)}dx\\
    =&\int_a^b\absoluteValue{f(x)-h(x)+h(x)-g(x)}dx\\
    \leq& \int_a^b\absoluteValue{f(x)-h(x)}+\absoluteValue{h(x)-g(x)}dx\\
    =& \int_a^b\absoluteValue{f(x)-h(x)}dx+\int_a^b\absoluteValue{h(x)-g(x)}dx\\
    =& d(f,h) + d(h,g).
\end{align*}
\end{proof}

\begin{exercise}{9}
Show that $d$ in 1.1-8 is a metric. For any set $X$, and $x,y\in X$, we define $d(x,x)=0$ and $d(x,y)=1$ whenever $x\neq y$.
\end{exercise}
\begin{proof}
M1, M2 and M3 follow directly from the definition of the metric itself. To see that M4 holds, consider $x,y,z\in X$, then $1=d(x,y)\leq d(x,z)+d(z,y)=2$, as required.
\end{proof}

\begin{exercise}{11}
Prove (1): the generalized triangle inequality
\[
d(x_1,x_n)\leq d(x_1,x_2)+\dots+d(x_{n-1},x_n).
\]
\end{exercise}
\begin{proof}
We will prove this by induction.

Base case. If $n=3$, then $d(x_1,x_3)\leq d(x_1,x_2)+d(x_2,x_3)$.

Induction hypothesis. Suppose the inequality holds for $n-1$, so that\\ $d(x_1,x_{n-1})\leq d(x_1,x_2)+\dots+d(x_{n-2},x_{n-1})$.

Induction step. We have $d(x_1,x_n)\leq d(x_1,x_{n-1})+d(x_{n-1},x_n)\leq d(x_1,x_2)+\dots+d(x_{n-2},x_{n-1})+d(x_{n-1},x_n)$, as required.
\end{proof}

\begin{exercise}{12 (Triangle inequality)}
The triangle inequality has several useful consequences. For instance, using (1), show that $\absoluteValue{d(x,y)-d(z,w)}\leq d(x,z)+d(y,w)$
\end{exercise}
\begin{proof}
First, we have
\begin{align*}
    d(x,y) \leq d(x,z)+d(z,y) \leq d(x,z)+d(z,w)+d(y,z),
\end{align*}
so that $d(x,y)-d(z,w)\leq d(x,z)+d(y,w)$. On the other hand,
\begin{align*}
    d(z,w) \leq d(z,y)+d(y,w) \leq d(y,w)+d(z,x)+d(x,y),
\end{align*}
so that $-(d(y,w)+d(x,z))\leq d(x,y)-d(z,w)$. Putting these together we obtain the desired result.
\end{proof}

\begin{exercise}{13}
Using the triangle inequality show that $\absoluteValue{d(x,z)-d(y,z)}\leq d(x,y)$
\end{exercise}
\begin{proof}
We have $d(x,z)\leq d(x,y)+d(y,z)$, so that $d(x,z)-d(y,z)\leq d(x,y)$. On the other hand, $d(y,z)\leq d(x,y)+d(x,z)$, so that $-d(x,y)\leq d(x,z)-d(x,y)$. Combining these results together, we obtain that $\absoluteValue{d(x,z)-d(y,z)}\leq d(x,y)$.
\end{proof}

\begin{exercise}{15}
Show that nonnegativity of a metric follows from (M2) to (M4).
\end{exercise}
\begin{proof}
Let $x,y\in X$. We have $0=d(x,x)\leq d(x,y)+d(y,x)=2d(x,y)$, thus $0\leq d(x,y)$.
\end{proof}
