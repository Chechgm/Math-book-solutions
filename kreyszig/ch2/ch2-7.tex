\subsection{Bounded and continuous linear operators}


\begin{exercise}{1}
Let $X,Y,Z$ be normed spaces, and $T_1:Y\to Z$ and $T_2:Y\to Z$ and $T:X\to X$ bounded linear operators. Then:
\begin{enumerate}
    \item $\norm{T_1T_2}\leq\norm{T_1}\norm{T_2}$, and
    \item $\norm{T^n}\leq \norm{T}^n$, for all $n\in\N$.
\end{enumerate}
\end{exercise}
\begin{proof}
Notice that for all $x\in\DDD(T_1,T_2)$ we have
\[
\norm{T_1T_2x}\leq\norm{T_1}\norm{T_2x}\leq\norm{T_1}\norm{T_2}\norm{x},
\]
so that dividing both sides of the inequality by $\norm{x}$ we get that $\norm{T_1T_2}\leq\norm{T_1}\norm{T_2}$, as required.

The second inequality follows from induction, by using $T=T_1=T_2$ in the first step, and $T=T_1$ and $T^{n-1}=T_2$ on the induction step.
\end{proof}

\begin{exercise}{2}
Let $X$ and $Y$ be normed spaces. 
Show that a linear operator $T:X\to Y$ is bounded if and only if $T$ maps bounded sets in $X$ into bounded sets in $Y$.
\end{exercise}
\begin{proof}
($\Rightarrow$) Let $T$ be bounded. Then there exists a constant $c$ so that for all $x\in X$, we have $\norm{Tx}\leq c\norm{x}$. Let $A\subseteq X$ be bounded. Hence, $\sup_{x,y\in A}\norm{x-y}=M<\infty$. Thus, $\norm{T(x-y)}=\norm{Tx-Ty}\leq c\norm{x-y}<\infty$ so that the set $T(A)$ is bounded too.

($\Leftarrow$) Let $T$ map bounded sets to bounded sets, so that if $A\subseteq X$ is a bounded set, then $T(A)\subseteq Y$ is bounded too. Let $A=\set{x\in X:\norm{x}=1}$. To see this is a bounded set, notice that if $x,y\in A$, then $d(x,y)\leq d(x,0)+d(0,y) =\norm{x}+\norm{y}=2$. By exercise 2.2.15, we have that $A$ is bounded if and only if there is a positive number $c$ such that $\norm{x}\leq c$ for all $x\in A$. Thus, since $T(A)$ is bounded, such $c$ exists (notice this is an alternative proof that $A$ itself is bounded, by taking $c=1$). Finally, by definition, we have that $\norm{T} =\sup_{x\in\DDD(T),\norm{x}=1}\norm{Tx} =\sup_{x\in A}\norm{Tx}<\infty$, so that $T$ is bounded, as required.
\end{proof}

\begin{exercise}{5}
Show that the operator $T:l^\infty\to l^\infty$ defined by $x=(x^n)$, $y^n=x^n/n$, $y =(y^n) =Tx$, is linear and bounded.
\end{exercise}
\begin{proof}
We claim that for $\norm{Tx}\leq c\norm{x}$ to hold for all $x\in l_\infty$ we need to set $c=1$, and thus $T$ is bounded. 
To see this, recall that $l_\infty$ consists of all bounded sequences, so that for all $n$, $\absoluteValue{x^n}\leq M$ for some constant. 
Furthermore, $\norm{x}=\sup_{n\in\N}\absoluteValue{x^n}$. 
We also see that for all $n$, $\absoluteValue{y^n}=\absoluteValue{x^n/n}\leq M$. 
Thus $\sup_{n\in\N}\absoluteValue{x^n/n} =\norm{Tx} \leq\norm{x} =\sup_{n\in\N}\absoluteValue{x^n}$, as required.
\end{proof}

\begin{exercise}{6 (Range)}
Show that the range $\RRR(T)$ of a bounded linear operator $T:X\to Y$ need not be closed in $Y$. 
Hint: Use $T$ in exercise 5.
\end{exercise}
\begin{proof}
Consider the sequence $(x^n)$ given by $x^n=(1,2,\dots,n,0,0,\dots)$. 
Every element of this sequence is in $l^\infty$ (because it is bounded), and the image of every element of the sequence, $T(x^n)=(1,1,\dots,1,0,0,\dots)$ is in $l^\infty$. 
However, the limit of the sequence, the constant sequence of 1s is not in $T(l^\infty)$ because it is the image of the sequence of naturals, $(1,2,\dots,n,n+1,\dots)$ which is not in $l^\infty$ because it is not a bounded sequence.
Hence, $T(l^\infty)$ is not closed.
\end{proof}

\begin{exercise}{7 (Inverse operator)}
Let $T$ be a bounded linear operator from a normed space $X$ onto a normed space $Y$. 
If there is a positive $b$ such that $\norm{Tx}\geq b\norm{x}$ for all $x\in X$, show that then $T^{-1}:Y\to X$ exists and is bounded.
\end{exercise}
\begin{proof}
To see $T$ is injective (and thus has an inverse, given that we assume it is onto), suppose $Tx=Ty$.
Then $0 =\norm{Tx-Ty} \geq b\norm{x-y}\geq 0$, so that $\norm{x-y}=0$ and $x=y$.

To see $T^{-1}$ is bounded, notice that $\norm{x} =\norm{T(T^{-1}x)} \geq b\norm{T^{-1}x}$, so that $(1/b)\norm{x} =c\norm{x} \geq \norm{T^{-1}x}$, as desired.
\end{proof}

\begin{exercise}{8}
Show that the inverse $T^{-1}:\RRR(T)\to X$ of a bounded linear operator $T:X\to Y$ need not be bounded. 
Hint: Use $T$ in exercise 5.
\end{exercise}
\begin{proof}
We argued on exercise 6 that any finite sequence of $n$ sequential 1s is the image of the sequence $(1,2,\dots,n,0,0,\dots)\in l^\infty$.
Now consider the sequence (of sequences), where $x^n$ is the sequence of $n$ 1s.
For each $x^n$, we have $\norm{x^n}_\infty=\sup_j\absoluteValue{x^n_j}=1$.
Furthermore, for all $n$ we have $\norm{T^{-1}x^n} =\sup_j\absoluteValue{nx^n_j} =n$ so
$\norm{T^{-1}}=\sup_{x\in\DDD(T^{-1}),\norm{x}=1}\norm{T^{-1}x}=\infty$, so that $T^{-1}$ is not bounded.
\end{proof}

\begin{exercise}{9}
Let $T:C[0,1]\to C[0,1]$ be defined by $y(t)=\int_0^t x(\tau)d\tau$. 
Find $\RRR(T)$ and $T^{-1}:\RRR(T)\to C[0,1]$. Is $T^{-1}$ linear and bounded?
\end{exercise}
\begin{proof}
We have that $\RRR(T)$ is the set of functions in $[0,1]$ that are differentiable at least once and whose derivative is continuous (that is, in $C^1[0,1]$), and for which $f(0)=0$ (since whenever we integrate with predetermined limits we don't get any constant).
$T^{-1}$ is the differentiation operator, which is linear but not bounded.
To see that $T^{-1}$ is not bounded, notice that $x=t^n$ for an arbitrary $n$, which is in $\RRR(T)$, and use the argument in 2.7-5.
\end{proof}

\begin{exercise}{10}
On $C[0,1]$ define $S$ and $T$ by $y(s)=s\int_0^1x(t)dt$, $y(s)=sx(s)$, respectively. 
Do $S$ and $T$ commute? 
Find $\norm{S}, \norm{T}, \norm{ST}$ and $\norm{TS}$.
\end{exercise}
\begin{proof}
We have 
\begin{align*}
    STx 
    =& S(Tx) = S(sx(s)) = s\int_0^1 tx(t) dt\\
    TSx =& T(Sx) = T(s\int_0^1 x(t) dt) = s^2\int_0^1 x(t)dt.
\end{align*}
So that, in general, $T$ and $S$ do not commute.

Now,
\begin{align*}
    \norm{T} 
    =& \sup_{x\in C[0,1], \norm{x}=1} \norm{Tx}\\
    =& \sup_{x\in C[0,1], \norm{x}=1} \max_{s\in [0,1]}\absoluteValue{sx(s)}\\
    =& \sup_{x\in C[0,1], \norm{x}=1} \max_{s\in [0,1]}\absoluteValue{s}\absoluteValue{x(s)}\\
    \leq& \sup_{x\in C[0,1], \norm{x}=1} \max_{s\in [0,1]}\absoluteValue{s}\max_{s\in [0,1]}\absoluteValue{x(s)}\\
    =& \sup_{x\in C[0,1], \norm{x}=1}\norm{x} =1,
\end{align*}
furthermore, if we take $x(t)=1$ for all $t$, then using equation 3,
\begin{align*}
    1 
    =& \max_{s\in [0,1]}\absoluteValue{sx(s)}
    = \norm{Tx}\\
    \leq& \norm{T}\norm{x}\\
    =& \norm{T}\max_{s\in[0,1]}\absoluteValue{x(s)} =\norm{T}, 
\end{align*}
hence $\norm{T}=1$.

Now for the norm of $S$, we have
\begin{align*}
    \norm{S} 
    =& \sup_{x\in C[0,1], \norm{x}=1}\norm{Sx}\\
    =& \sup_{x\in C[0,1], \norm{x}=1} \max_{s\in[0,1]}\absoluteValue{s\int_0^1x(t)dt}\\
    \leq& \sup_{x\in C[0,1], \norm{x}=1} \max_{s\in[0,1]}\absoluteValue{s} \max_{s\in[0,1]}\absoluteValue{\int_0^1x(t)dt}\\
    =& \sup_{x\in C[0,1], \norm{x}=1} \max_{s\in[0,1]}\absoluteValue{\int_0^1x(t)dt}\\
    \leq& 1.
\end{align*}
The last inequality is true because $\absoluteValue{\int_0^1 x}\leq \int_0^1 \absoluteValue{x}$ and if $x(t)\leq 1$ for all $t$, then $\int_0^1x\leq 1(1-0)$ (Theorem 7.4.2.iii and v in Abbott).
Furthermore, choosing $x(t)=1$ for all $t$ and equation 3,
\begin{align*}
    1 =& \max_{s\in[0,1]}s\int_0^1x(t)dt
    = \max_{s\in[0,1]}\norm{Sx}\\
    \leq& \norm{S}\norm{x}\\
    =& \norm{S}\max_{s\in[0,1]}\int_0^1x(t)dt =\norm{S},
\end{align*}
so that $\norm{S}=1$.

For $\norm{TS}$ and $\norm{ST}$ we can use a similar technique as for $S$ to conclude that $\norm{TS}=\norm{ST}=1$.

\end{proof}

\begin{exercise}{12 (Matrices)}
From 2.7-7 we know that an $r\times n$ matrix $A=(a_{jk})$ defines a linear operator from the vector space $X$ of all ordered $n-$tuples of numbers into the vector space $Y$ of all ordered $r-$tuples of numbers. 
Suppose that any norm $\norm{\cdot}_1$ is given on $X$ and any norm $\norm{\cdot}_2$ is given on $Y$. 
Remember from exercise 2.4.10, that there are various norms on the space $Z$ of all those matrices ($r$ and $n$ fixed). 
A norm $\norm{\cdot}$ on $Z$ is said to be compatible with $\norm{\cdot}_1$ and $\norm{\cdot}_2$ if $\norm{Ax}_2 \leq\norm{A}\norm{x}_1$.

Show that the norm defined by
\[
\norm{A} = \sup_{x\in X, x\neq 0}\frac{\norm{Ax}_2}{\norm{x}_1}
\]
is compatible with $\norm{\cdot}_1$ and $\norm{\cdot}_2$. 
This norm is often called the natural norm defined by $\norm{\cdot}_1$ and $\norm{\cdot}_2$. 
If we choose $\norm{x}_1=\max_j\absoluteValue{x_j}$ and $\norm{y}=\max_j\absoluteValue{y_j}$, show that the natural norm is
\[
\norm{A}=\max_j\sum^n_{k=1}\absoluteValue{a_{jk}}.
\]
\end{exercise}
\begin{proof}
(Compatibility)
We have 
\begin{align*}
    \norm{A}\norm{x'}_1
    =& \sup_{x\in X, x\neq 0}\left\{\frac{\norm{Ax}_2}{\norm{x}_1}\right\}\norm{x'}_1\\
    \geq& \sup_{x\in X, x\neq 0}\left\{\norm{Ax}_2\right\}\frac{\norm{x'}_1}{{\norm{x'}_1}}\\
    =& \sup_{x\in X, x\neq 0}\norm{Ax}_2
    \geq \norm{Ax'}_2,
\end{align*}
as required.

(Natural norm)
We have
\begin{align*}
    \norm{A}
    =& \sup_{x\in X, x\neq 0}\left\{\frac{\max_k \absoluteValue{\sum_l a_{k,l}x_l}}{\max_j \absoluteValue{x_j}}\right\}\\
    \leq& \sup_{x\in X, x\neq 0}\left\{\frac{\max_k \absoluteValue{\sum_l a_{k,l}\max_l \absoluteValue{x_l}}}{\max_j \absoluteValue{x_j}}\right\}\\
    =& \sup_{x\in X, x\neq 0}\left\{\frac{\max_k \absoluteValue{\max_l \absoluteValue{x_l}\sum_l a_{k,l}}}{\max_j \absoluteValue{x_j}}\right\}\\
    =& \sup_{x\in X, x\neq 0}\left\{\frac{\max_l  \absoluteValue{x_l}\max_k\absoluteValue{\sum_l a_{k,l}}}{\max_j \absoluteValue{x_j}}\right\}\\
    =& \sup_{x\in X, x\neq 0}\left\{\max_k\absoluteValue{\sum_l a_{k,l}}\right\}\\
    =& \max_k\absoluteValue{\sum_l a_{k,l}}
\end{align*}
The second inequality would hold be an equality if $x$ is a constant vector.
Choosing $x=(1,\dots,1)$ is enough to get us the desired result.
\end{proof}

\begin{exercise}{13}
Show that in 2.7-7 with $r=n$, a compatible norm is defined by 
\[
\norm{A} = \parens{\sum^n_{j=1}\sum^n_{k=1}a^2_{jk}}^{1/2},
\]
but for $n>1$ this is not the natural norm defined by the Euclidean norm on $R^n$.
\end{exercise}
\begin{proof}
We have 
\begin{align*}
    \norm{Ax}^2
    =& \sum_{j=1}^n \parens{\sum_{k=1}^n a_{jk}x_k}^2\\
    \leq& \sum_{j=1}^n \parens{\sum_{k=1}^n a_{jk}^2}\parens{\sum_{k=1}^n x_k^2}\\
    =& \parens{\sum_{k=1}^n x_k^2}\parens{\sum_{j=1}^n\sum_{k=1}^n a_{jk}^2}\\
    =& \norm{x}^2\norm{A}^2.
\end{align*}
Where the inequality is given by Cauchy-Schwarz applied to every $j$ separately and summing them again. 
Thus, taking square root on both sides of the inequality tells us that $\norm{A}$ is compatible.

To see the second part, consider $A=I$.
Then $\norm{A}=\sqrt{n}$, and the natural norm is 1.
Since they are not the same, then $\norm{A}$ is not the natural norm.
\end{proof}

\begin{exercise}{14}
If in exercise 12 we choose 
\[
\norm{x}_1=\sum^n_{k=1}\absoluteValue{x_k},\quad \norm{y}_2=\sum^r_{j=1}\absoluteValue{y_j},
\]
show that a compatible norm is defined by $\norm{A}=\max_k\sum^r_{j=1}\absoluteValue{a_{jk}}$.
\end{exercise}
\begin{proof}
We have 
\begin{align*}
    \norm{Ax}_2
    =& \sum_j^r \absoluteValue{(Ax)_j}\\
    =& \sum_j^r \absoluteValue{\sum_k^n a_{jk}x_k}\\
    \leq& \sum_j^r \sum_k^n \absoluteValue{a_{jk}x_k}\\
    =& \sum_k^n \absoluteValue{x_k}\sum_j^r \absoluteValue{a_{jk}}\\
    \leq& \sum_k^n \absoluteValue{x_k} \max_k\sum_j^r \absoluteValue{a_{jk}}\\
    =& \sum_k^n \absoluteValue{x_k} \norm{A}\\
    =& \norm{A}\sum_k^n \absoluteValue{x_k} =\norm{A}\norm{x}_1
\end{align*}
Where the first inequality follows from the triangle inequality.
\end{proof}

\begin{exercise}{15}
Show that for $r=n$, the norm in exercise 14 is the natural norm corresponding to $\norm{\cdot}_1$ and $\norm{\cdot}_2$ as defined in that problem.
\end{exercise}
\begin{proof}
Starting from the second equality of exercise 14, we have
\begin{align*}
    \norm{Ax}_2
    =& \sum_j^n \absoluteValue{\sum_k^n a_{jk}x_k}.
\end{align*}
Let $x=(0,\dots,1,\dots,0)$, where $x_k=1$ when $\max_k \sum_j^n \absoluteValue{a_{jk}}$.
Then we have
\begin{align*}
    \sum_j^n \absoluteValue{\sum_k^n a_{jk}x_k}
    &= \sum_j^n \max_k \absoluteValue{a_{jk}}
    = \max_k \sum_j^n\absoluteValue{a_{jk}} = \norm{A}.
\end{align*}
Since $\norm{x}_1=1$, then this is consistent with the natural norm.
If $x_k$ was different from a constant in more than one $k$ then applying the triangle inequality (as the first inequality in exercise 14) would give us that our proposed solution is larger than with the alternative $x$.
\end{proof}