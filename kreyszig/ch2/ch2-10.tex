\subsection{Normed spaces of operators. Dual space}

Kreyszig 2.10 Normed Spaces of Operators. Dual Space 
1 Obviously important and elementary 
2 
3 I don't know if you're interested in doing unbounded operators since they're mostly important for QM, if so, this result is important 
4 An alternative characterization of convergence
6 Good practice, and important for intuition on dual spaces.
8 This is rather difficult, and not such an important result, but it's good to go through the details of finding the dual space yourself at some point
9 Should be known from LA
10
11 So we are in fact considering a different thing than the Axler dual space (in infinite dimensions)
12 Some applications of dual spaces

\begin{exercise}{1}
What is the zero element of the vector space $B(X,Y)$?
The inverse of a $T\in B(X,Y)$ in the sense of Definition 2.1-1?
\end{exercise}
\begin{proof}
fill
\end{proof}

\begin{exercise}{2}
The operators and functionals considered in the text are defined on the entire space $X$.
Show that without that assumption, in the case of functionals we still have the following theorem:
If $f$ and $g$ are bounded linear functionals with domains in a normed space $X$ then for any nonzero scalars $\alpha$ and $\beta$ the linear combination $h=\alpha f+\beta g$ if a bounded linear functional with domain $\DDD(f)\cap\DDD(g)$.
\end{exercise}
\begin{proof}
fill
\end{proof}

\begin{exercise}{3}
Extend the Theorem in exercise 2 to bounded linear operators $T_1$ and $T_2$.
\end{exercise}
\begin{proof}
fill
\end{proof}

\begin{exercise}{4}
Let $X$ and $Y$ be normed spaces and $T_n:X\to Y\ (n=1,2\dots)$ bounded linear operators.
Show that convergence $T_n\to T$ implies that for every $\epsilon>0$ there is an $N$ such that for all $n>N$ and all $x$ in any given closed ball we have $\norm{T_nx-tTx}<\epsilon$.
\end{exercise}
\begin{proof}
fill
\end{proof}

\begin{exercise}{6}
If $X$ is the space of ordered $n$-tuples of real numbers and $\norm{x}=\max_i \absoluteValue{x_i}$, what is the corresponding norm on the dual space $X'$?
\end{exercise}
\begin{proof}
fill
\end{proof}

\begin{exercise}{8}
Show that the dual space of the space $c_0$ is $l^1$. 
(See exercise 2.3.1).
\end{exercise}
\begin{proof}
fill
\end{proof}

\begin{exercise}{9}
Show that a linear functional $f$ on a vector space $X$ is uniquely determined by its values on a Hamel basis for $X$.
(See section 2.1).
\end{exercise}
\begin{proof}
fill
\end{proof}

\begin{exercise}{10}
Let $X$ and $Y\neq\set{0}$ be normed spaces, where $\dim X=\infty $.
Show that there is at least one unbounded linear operator $T:X\to Y$.
(Use a Hamel basis).
\end{exercise}
\begin{proof}
fill
\end{proof}

\begin{exercise}{11}
If $X$ is a normed space and $\dim X=\infty$, show that the dual space $X'$ is not identical with the algebraic dual space $X^\ast$.
\end{exercise}
\begin{proof}
fill
\end{proof}

\begin{exercise}{12 (Completeness)}
The examples in the text can be used to prove completeness of certain spaces.
How?
For what spaces?
\end{exercise}
\begin{proof}
fill
\end{proof}
