\section{Linear operators}


\begin{exercise}{5}
Let $T:X\to Y$ be a linear operator. Show that the image of a subspace $V$ of $X$ is a vector space, and so is the inverse image of a subspace $W$ of $Y$.
\end{exercise}
\begin{proof}
\textbf{Image of the subspace}:

Let $V$ be a subspace of $X$. Let $v,u,w\in V$ and $a,b\in K$. Then we have:

Commutativity: we have $Tv+Tu =T(v+u) = T(u+v) =Tu+Tv$.

Associativity: we have $(Tv+Tu)+Tw =T(v+u)+Tw =T((v+u)+w) =T(v+(u+w)) =Tv+T(u+w) =Tv+(Tu+Tw)$.

Additive identity: we have $Tv+T0 =T(v+0) =Tv$.

Additive inverse: $Tv+T(-v) =T(v-v) =T0 =0$ so that $Tv$ has $T(-v)$ as additive inverse.

Multiplicative identity: we have $1Tv= T(1v) =Tv$ so that 1 is a multiplicative identity.

Distributive properties: we have $(a+b)Tv =T((a+b)v) =T(av+bv) =T(av)+T(bv) =aTv+bTv$. Furthermore, $a(Tv+Tu) =aT(v+u) =T(a(v+u)) =T(av+au) =T(av)+T(au) =aTv+aTu$.

\textbf{Inverse image of a subspace}:

Let $W$ be a subspace of $Y$. Let $v,u,w\in W$, so that $v=Tx, u=Ty, w=Tz$ for $x,y,z\in X$, and $a,b\in K$. Furthermore, notice that because $T$ is a linear operator, it holds that $Tx+Ty =T(x+y)$ and $T(ax) =aTx$. Then we have:

Commutativity: we have $v+u =Tx+Ty =T(x+y) =T(y+x) =Ty+Tx =u+v$.

Associativity: we have $(v+u)+w =(Tx+Ty)+Tz =T(x+y)+Tz =T(x+y+z) =Tx+T(y+z) =Tx+(Ty+Tz) =v+(u+w)$.

Additive identity: since $W$ is a subspace of $Y$, then $0\in W$. Thus, $v+0 =Tx+T0 =T(x+0) =Tx =v$.

Additive inverse:We have $v-v =Tx-Tx =T(x-x) =T0 =0$. So that $v$ has $-v$ as additive inverse.

Multiplicative identity: We have $1v =1Tx =T(1x) =Tx =v$. So that 1 is a multiplicative identity.

Distributive properties: we have $(a+b)v =(a+b)Tx =T[(a+b)x] =T(ax+bx) =T(ax)+T(bx) =aTx +bTx =av+bv$. Likewise, $a(v+u) =a(Tx+Ty) =aT(x+y) =T(a(x+y)) =T(ax+ay) =T(ax)+T(ay) =aTx+aTy =av+au$. 
\end{proof}

\begin{exercise}{6}
If the product (the composite) of two linear operators exist, show that it is linear.
\end{exercise}
\begin{proof}
Let $T:X\to Y$ and $S:Y\to Z$ be linear operators. Then it holds for $TS$:

(i) The domain, $X$, of $TS$ is a vector space and the range $\cR(TS)\subseteq Z$ is a vector space too. Furthermore;

(ii) if $x,x'\in X$, then $T(S(x+x')) =T(Sx+Sx') =TSx+TSx'$, so that $TS$ is additive; and if $a\in K$, then $TS(ax) =T(aSx) =aTSx$ so that $TS$ is homogenous.
\end{proof}

\begin{exercise}{10}
Formulate the conditions in 2.6-10(a) in terms of the null space of $T$.

2.6-10(a): Let $X$ and $Y$ be vector spaces, both real or both complex. Let $T:\cD(T)\to Y$ be a linear operator with domain $\cD(T)\subseteq X$ and range $\cR(T)\subseteq Y$. Then:

(a) The inverse $T^{-1}:\cR(T)\to\cD(T)$ exists if and only if $Tx=0$ implies $x=0$.
\end{exercise}
\begin{proof}
There are two equivalent ways in which we can write this condition: First, the inverse exists, if and only if $\dim\cN(T)=0$. Second, the inverse exists, if and only if $\cN(T)=\set{0}$.
\end{proof}

\begin{exercise}{11}
Let $X$ be the vector space of all $2\times 2$ matrices and define $T:X\to X$ by $Tx=bx$, where $b\in X$ is fixed and $bx$ denotes the usual product of matrices. Show that $T$ is linear. Under what condition does $T^{-1}$ exist?
\end{exercise}
\begin{proof}
Let $x,y\in X$ and $\alpha,\beta\in K$. Then, $T(\alpha x+\beta y) =b(\alpha x+\beta y) =\alpha bx+\beta by =\alpha T(x)+\beta T(y)$, so that $T$ is linear. 

From exercise 10, $T^{-1}$ exists if and only if $\cN(T)=\set{0}$, which is true if and only if $\det b\neq 0$, which is the same as invertible $b$.
\end{proof}

\begin{exercise}{13}
Let $T:\cD(T)\to Y$ be a linear operator whose inverse exists. If $\set{x_1,\dots,x_n}$ is a linearly independent set in $\cD(T)$, show that the set $\set{Tx_1,\dots,Tx_n}$ is linearly independent.
\end{exercise}
\begin{proof}
Consider $0 =a_1Tx_1+\dots+a_nTx_n =T(a_1x_1)+\dots+T(a_nx_n) =T(a_1x_1+\dots+a_nx_n)$. Then we have that $T(a_1x_1+\dots+a_nx_n)=0$, so that $a_1x_1+\dots+a_nx_n=0$. Since $\set{x_1,\dots,x_n}$ is linearly independent, then $a_1=\dots=a_n=0$, so that $\set{Tx_1,\dots,Tx_n}$ is independent too.
\end{proof}

\begin{exercise}{14}
Let $T:X\to Y$ be a linear operator and $\dim X =\dim Y =n <\infty$. Show that $\cR(T)=Y$ if and only if $T^{-1}$ exists.
\end{exercise}
\begin{proof}
($\Rightarrow$) Suppose $\cR(T)=Y$, so that $\dim\cR(T)=\dim Y=n$. Since we assume $X$ and $Y$ are finite dimensional, then the fundamental Theorem of linear algebra which asserts that $\dim X=\dim\cR(T)+\dim\cN(T)$ applies. That is, $n=n+\dim\cN(T)$. Since $\dim\cN(T)=0$, then $T$ is injective, and by 2.6-10.a, then $T^{-1}$ exists.

($\Leftarrow$) Suppose $T^{-1}$ exists. Then by Theorem 2.6-10.c $\dim\cD(T)=\dim X=\dim\cR(T)=\dim Y$, so that $\cR(T)=Y$, because finite dimensional spaces of the same dimension are isomorphic to each other.
\end{proof}

\begin{exercise}{15}
Consider the vector space $X$ of all real-value functions which are defined on $\R$ and have derivatives of all orders everywhere on $\R$. Define $T:X\to X$ by $y(t)=Tx(t)=x'(t)$. Show that $\cR(T)$ is all of $X$ but $T^{-1}$ does not exist. Compare with exercise 14 and comment.
\end{exercise}
\begin{proof}
Since $X$ consists of all the real-valued functions that have derivatives of all orders everywhere on $\R$, then $X$ consists of continuous functions. Thus, we can apply the Fundamental Theorem of Calculus (see Theorem 7.5.1.ii of Abbott) so that if $F\in X$ is differentiable at $x$, then $F(x)=f'(x)$ for some continuous function in $X$. Thus, $\cR(T)=X$. 

To see that $T^{-1}$ does not exist, it suffices to find two functions that have the same preimage under $T$, so that $T$ is not injective. Consider $f(x)=x$ and $g(x)=x+c$ for a constant $c$. Both $f,g\in X$ and $Tf=Tg=1$, so that $T^{-1}1$ is not uniquely defined.

In exercise 14 we assumed the spaces were finite dimensional so we could use the fundamental Theorem of Linear algebra to conclude (starting from the range of $T$) that $T^{-1}$ exists. However, in infnite dimensions this relation between the dimensions of the subspaces does not hold.
\end{proof}