\subsection{Further properties of normed spaces}

1
2 
3 Some important subspaces of $l^\infty$
4
5 So similar to R
6 The closure respects the algebraic operations, which is something that sounds like we want to have
7
8
9 The notion of absolute convergent and conditionally convergent is rather important. This shows that Banach spaces are the only one where absolute => conditional, which is a very useful property to have
10 Not that easy to prove
11 What fails here for $l^\infty$?

\begin{exercise}{1}
Show that $c\subseteq l^\infty$ is a vector subspace of $l^\infty$ (cf. 1.5-3) and so is $c_0$, the space of all sequences of scalars converging to zero.
\end{exercise}
\begin{proof}
fill
\end{proof}

\begin{exercise}{2}
Show that $c_0$ in exercise 1 is a closed subspace of $l^\infty$, so that $c_0$ is complete by 1.5-2 and 1.4-7.
\end{exercise}
\begin{proof}
fill
\end{proof}

\begin{exercise}{3}
In $l^\infty$, let $Y$ be the subset of all sequences with only finitely many nonzero terms. Show that $Y$ is a subspace of $l^\infty$ but not a closed subspace.
\end{exercise}
\begin{proof}
fill
\end{proof}

\begin{exercise}{4 (Continuity of vector space operations)}
Show that in a normed space $X$, vector addition and multiplication by scalars are continuous operations with respect to the norm; that is, the mappings defined by $(x,y)\to x+y$ and $(\alpha, x)\to \alpha x$ are continuous.
\end{exercise}
\begin{proof}
fill
\end{proof}

\begin{exercise}{5}
Show that $x^n\to x$ and $y^n\to y$ implies $x^n+y^n\to x+y$. Show that $\alpha^n\to\alpha$ and $x^n\to x$ implies $\alpha^nx^n\to \alpha x$.
\end{exercise}
\begin{proof}
fill
\end{proof}

\begin{exercise}{6}
Show that the closure $\bar{Y}$ of a subspace $Y$ of a normed space $X$ is again a vector subspace.
\end{exercise}
\begin{proof}
fill
\end{proof}

\begin{exercise}{7 (Absolute convergence)}
Show that convergence of $\norm{y^1}+\norm{y^2}+\norm{y^3}+\dots$ may not imply convergence of $y^1+y^2+y^3+\dots$. Hint: Consider $Y$ in exercise 3 and $(y^n)_{n=1,\dots}$ (a sequence indexed by $n$), where $y^n=(y^n_m)_{m=1\dots,}$ (each particular $y^n$ is a sequence indexed by $m$), with $y^n_n=1/n^2$, and $y^n_j=0$ for all $j\neq n$.
\end{exercise}
\begin{proof}
fill
\end{proof}

\begin{exercise}{8}
If in a normed space $X$, absolute convergence of any series always implies convergence of that series, show that $X$ is complete.
\end{exercise}
\begin{proof}
fill
\end{proof}

\begin{exercise}{9}
Show that in a Banach space, an absolute convergent series is convergent.
\end{exercise}
\begin{proof}
fill
\end{proof}

\begin{exercise}{10 (Schauder basis)}
Show that if a normed space has a Schauder basis, it is separable.
\end{exercise}
\begin{proof}
fill
\end{proof}

\begin{exercise}{11}
Show that $(e^n)$, where $e^n=\delta^n_j$, is a Schauder basis for $l^p$ where $1\leq p<\infty$.
\end{exercise}
\begin{proof}
fill
\end{proof}
