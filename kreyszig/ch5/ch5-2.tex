\section{Applications of Banach's Theorem to linear equations}

1 Details from the theory
2 Applying the theory to a simple example. Don't do this one if you don't like programming it
4 One of my absolute favorite results in linear algebra. This is so beautiful and amazing. Notice that eigenvalues relate to roots of polynomials, so this also gives a theorem on where to find roots of polynomials!
5
6 These two questions tells us that both methods are not better than one another!
7 Some more criteria
8
9

\begin{exercise}{1}
Consider the system of $n$ linear equations in $n$ unknowns given by $Ax = c$, the Jacobi and the Gauss-Seidel iteration solves this system as the fixed point of the system given by $x = Cx + b$.
We can write $A = -L + D - U$, where $D$ is the matrix with the diagonal of $A$ and 0 everywhere else, $L$ is a lower triangular matrix (with zeros on the diagonal) and $U$ is an upper triangular matrix (with zeros on the diagonal).
Verify that in the Jacobi iteration with $C = -D^{-1}(A-D)$ and $b = D^{-1}c$ and the Gauss-Seidel iteration with $C = (D-L)^{-1}U$ and $b = (D-L)^{-1}c$ solve the original system.
\end{exercise}
\begin{proof}
(Jacobi iteration)
Replacing $C$ and $b$ in the iterative system, we have 
\begin{align*}
    x 
    = -D^{-1}(A - D)x + D^{-1}c
    = -D^{-1}Ax + x + D^{-1}c
\end{align*}
so that $0 = -D^{-1}Ax + D^{-1}c$, which implies $D^{-1}Ax = D^{-1}c$ and $Ax = c$, as required.

(Gauss-Seidel)
Replacing as above
\begin{align*}
    x
    = (D - L)^{-1}Ux + (D - L)^{-1}c
    = (D - L)^{-1}(Ux + c),
\end{align*}
so that $(D - L)x = Ux + c$, which gives us that $(D - L - U)x = c$, as required.
\end{proof} 

\begin{exercise}{2}
fill
\end{exercise}
\begin{proof}
fill
\end{proof} 

\begin{exercise}{4}
fill
\end{exercise}
\begin{proof}
fill
\end{proof} 

\begin{exercise}{5}
fill
\end{exercise}
\begin{proof}
fill
\end{proof} 

\begin{exercise}{6}
fill
\end{exercise}
\begin{proof}
fill
\end{proof} 

\begin{exercise}{7}
fill
\end{exercise}
\begin{proof}
fill
\end{proof} 

\begin{exercise}{8}
fill
\end{exercise}
\begin{proof}
fill
\end{proof} 

\begin{exercise}{9}
fill
\end{exercise}
\begin{proof}
fill
\end{proof} 
