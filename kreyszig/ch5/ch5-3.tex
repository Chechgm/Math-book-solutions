\section{Applications of Banach's Theorem to differential equations}

Kreyszig 5.3 Application of Banach's Theorem to Differential Equations
1 This is the case you'll use in practice since it's a lot easier to check than to check Lipschitz directly
7 So the initial value doesn't really matter
9 An example of a system that might have more than one solution. Interestingly, this gives rise to a system in classical mechanics called Norton's Dome, which is a nondeterministic system in Newtonian mechanics.

\begin{exercise}{1}
fill
\end{exercise}
\begin{proof}
fill
\end{proof} 

\begin{exercise}{7}
fill
\end{exercise}
\begin{proof}
fill
\end{proof} 

\begin{exercise}{9}
fill
\end{exercise}
\begin{proof}
fill
\end{proof} 
