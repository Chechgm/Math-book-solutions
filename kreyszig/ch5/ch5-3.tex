\section{Applications of Banach's Theorem to differential equations}

1 This is the case you'll use in practice since it's a lot easier to check than to check Lipschitz directly
7 So the initial value doesn't really matter
9 An example of a system that might have more than one solution. Interestingly, this gives rise to a system in classical mechanics called Norton's Dome, which is a nondeterministic system in Newtonian mechanics.

\begin{exercise}{1}
If the partial derivative $\partial f/ \partial x$ of $f$ exists and is continous on the rectangle $R$ (as in Picard's Theorem), show that $f$ satisfies a Lipschitz condition on $R$ with respect to the second argument.
\end{exercise}
\begin{proof}
fill
\end{proof} 

\begin{exercise}{5}
Explain the reasons for the restrictions $\beta < b/c$ and $\beta 1/k$ in Picard's Theorem.
\end{exercise}
\begin{proof}
fill
\end{proof} 

\begin{exercise}{6}
Show that in Picard's Theorem, the subspace $\tilde{C} \subseteq C(J)$, where $J = [t_0 - \beta, t_0 + \beta]$, given by the functions such that $\absoluteValue{x(t) - x_0} \leq c\beta$ is closed in $C(J)$.
\end{exercise}
\begin{proof}
fill
\end{proof} 

\begin{exercise}{7}
Show that in Picard's Theorem, instead of the constant $x_0$ we can take any other function $y_0 \in \tilde{C}$, with $y_0(t_0) = x_0$ as the initial function of the iteration.
\end{exercise}
\begin{proof}
fill
\end{proof} 

\begin{exercise}{9}
Show that $x' = 3x^{2/3}$, $x(0) = 0$, has infinitely many solutions $x$, given by $x(t) = 0$ if $t < c$, and $x(t) = (t - c)^3$ if $t \geq c$.
\end{exercise}
\begin{proof}
fill
\end{proof} 
